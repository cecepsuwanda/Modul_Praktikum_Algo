\documentclass[12pt,a4paper]{article}
\usepackage[utf8]{inputenc}
\usepackage[T1]{fontenc}
\usepackage[indonesian]{babel}
\usepackage{geometry}
\usepackage{longtable}
\usepackage{array}
\usepackage{booktabs}
\usepackage{enumitem}
\usepackage[hidelinks,breaklinks]{hyperref}
\usepackage{bookmark}
\geometry{margin=2.5cm}

\begin{document}

\begin{center}
\textbf{\Large RENCANA PEMBELAJARAN SEMESTER (RPS)}\\
\textbf{Berbasis Outcome-Based Education (OBE)}\\[0.5cm]
\textbf{PROGRAM STUDI TEKNIK INFORMATIKA}\\
\textbf{FAKULTAS TEKNIK}\\
\textbf{UNIVERSITAS LOREM IPSUM}\\
\end{center}

\vspace{1cm}

% ============================================================
% 1. IDENTITAS MATA KULIAH
% ============================================================
\section*{1. Identitas Mata Kuliah}
\begin{tabular}{ll}
Nama Program Studi & : Teknik Informatika \\
Nama Mata Kuliah & : Algoritma dan Pemrograman \\
Kode Mata Kuliah & : TIF-101 \\
Semester & : 1 (Satu) \\
SKS / Bobot Kredit & : 3 SKS (2 Teori, 1 Praktikum) \\
Bahasa Pemrograman & : C \\
Dosen Pengampu & : Dr. Lorem Ipsum, M.Kom. \\
Tanggal Penyusunan & : 12 Februari 2026 \\
\end{tabular}

\vspace{0.5cm}

% ============================================================
% 2. CAPAIAN PEMBELAJARAN LULUSAN (CPL)
% ============================================================
\section*{2. Capaian Pembelajaran Lulusan (CPL)}

CPL yang dibebankan pada mata kuliah ini mencakup kompetensi lulusan dalam aspek pengetahuan, keterampilan, dan sikap:

\begin{itemize}[leftmargin=*]
  \item \textbf{CPL-1 (Pengetahuan):} Menguasai konsep teoretis bidang pengetahuan tertentu secara umum dan konsep teoretis bagian khusus dalam bidang pengetahuan tersebut secara mendalam, serta mampu memformulasikan penyelesaian masalah prosedural.
  
  \item \textbf{CPL-2 (Keterampilan Umum):} Mampu menerapkan pemikiran logis, kritis, sistematis, dan inovatif dalam konteks pengembangan atau implementasi ilmu pengetahuan dan teknologi yang memperhatikan dan menerapkan nilai humaniora.
  
  \item \textbf{CPL-3 (Keterampilan Khusus):} Mampu merancang dan mengimplementasikan algoritma serta program komputer sederhana menggunakan bahasa pemrograman prosedural untuk menyelesaikan permasalahan komputasi.
  
  \item \textbf{CPL-4 (Sikap):} Menunjukkan sikap bertanggung jawab atas pekerjaan di bidang keahliannya secara mandiri dan mampu bekerja sama dalam tim.
\end{itemize}

\vspace{0.5cm}

% ============================================================
% 3. CAPAIAN PEMBELAJARAN MATA KULIAH (CPMK)
% ============================================================
\section*{3. Capaian Pembelajaran Mata Kuliah (CPMK)}

Kemampuan atau kompetensi spesifik yang diharapkan mahasiswa kuasai setelah menyelesaikan mata kuliah:

\begin{itemize}[leftmargin=*]
  \item \textbf{CPMK-1:} Mahasiswa mampu memahami dan menjelaskan konsep algoritma, flowchart, dan pseudocode beserta karakteristiknya.
  
  \item \textbf{CPMK-2:} Mahasiswa mampu merancang algoritma untuk menyelesaikan masalah sederhana menggunakan flowchart dan pseudocode.
  
  \item \textbf{CPMK-3:} Mahasiswa mampu mengimplementasikan algoritma dalam bahasa pemrograman C meliputi variabel, tipe data, I/O, operator, struktur percabangan, dan perulangan.
  
  \item \textbf{CPMK-4:} Mahasiswa mampu menggunakan fungsi/prosedur, array, dan struktur data dasar dalam bahasa C.
  
  \item \textbf{CPMK-5:} Mahasiswa mampu menerapkan algoritma pengurutan dan pencarian sederhana serta mengimplementasikannya dalam program.
\end{itemize}

\vspace{0.5cm}

% ============================================================
% 4. SUB-CPMK / INDIKATOR PENCAPAIAN
% ============================================================
\section*{4. Sub-CPMK / Indikator Pencapaian}

Penjabaran CPMK menjadi indikator yang lebih terukur dan dapat diuji:

\begin{itemize}[leftmargin=*]
  \item \textbf{Sub-CPMK 1.1:} Menjelaskan definisi dan karakteristik algoritma (input, output, definiteness, finiteness, effectiveness)
  \item \textbf{Sub-CPMK 1.2:} Membuat flowchart dan pseudocode untuk masalah sederhana
  \item \textbf{Sub-CPMK 2.1:} Merancang algoritma untuk masalah yang melibatkan percabangan dan perulangan
  \item \textbf{Sub-CPMK 3.1:} Mengimplementasikan struktur seleksi (if-else, switch) dalam bahasa C
  \item \textbf{Sub-CPMK 3.2:} Mengimplementasikan struktur perulangan (for, while, do-while) dalam bahasa C
  \item \textbf{Sub-CPMK 4.1:} Membuat fungsi dengan parameter dan nilai return dalam bahasa C
  \item \textbf{Sub-CPMK 4.2:} Menggunakan array satu dimensi dan dua dimensi dalam program
  \item \textbf{Sub-CPMK 5.1:} Mengimplementasikan algoritma pengurutan (bubble sort, selection sort) dan pencarian (linear search, binary search) sederhana
\end{itemize}

\vspace{0.5cm}

% ============================================================
% 5. MATERI PEMBELAJARAN (BAHAN KAJIAN)
% ============================================================
\section*{5. Materi Pembelajaran (Bahan Kajian)}

Daftar topik materi yang relevan dengan Sub-CPMK dan CPMK, disusun untuk 16 pertemuan:

\begin{enumerate}[leftmargin=*]
  \item Pengenalan algoritma dan pemrograman (definisi algoritma, karakteristik, peran dalam komputasi)
  \item Flowchart dan pseudocode (simbol, notasi, cara merancang)
  \item Variabel, tipe data, I/O dasar dalam C (scanf, printf, type casting)
  \item Operator (aritmatika, perbandingan, logika, penugasan)
  \item Struktur percabangan (if, if-else, if-else if)
  \item Struktur percabangan (switch-case)
  \item Struktur perulangan (for, while, do-while)
  \item Perulangan bersarang dan penerapannya
  \item Fungsi dan prosedur (definisi, parameter, return value, scope variabel)
  \item Array satu dimensi (deklarasi, akses elemen, operasi dasar)
  \item Array dua dimensi dan operasi matriks sederhana
  \item String dan manipulasi teks dalam C
  \item Struktur data sederhana (struct/record)
  \item Algoritma pengurutan dan pencarian
  \item Integrasi materi dan studi kasus
  \item Evaluasi capaian pembelajaran (UAS)
\end{enumerate}

\vspace{0.5cm}

% ============================================================
% MATRIKS 16 PERTEMUAN
% ============================================================
\subsection*{Matriks Pembelajaran 16 Pertemuan}

\setlength{\LTcapwidth}{\linewidth}
\begin{longtable}{|>{\centering\arraybackslash}p{1.2cm}|>{\raggedright\arraybackslash}p{8.2cm}|>{\raggedright\arraybackslash}p{3.5cm}|}
\hline
\textbf{Pert.} & \textbf{Materi Utama} & \textbf{Sub-CPMK} \\
\hline
\endfirsthead

\hline
\textbf{Pert.} & \textbf{Materi Utama} & \textbf{Sub-CPMK} \\
\hline
\endhead

\hline
\endfoot

1 & Pengenalan algoritma, flowchart, pseudocode & 1.1, 1.2 \\
\hline
2 & Variabel, tipe data, I/O, operator & CPMK-3 (dasar) \\
\hline
3 & Percabangan if-else & 2.1, 3.1 \\
\hline
4 & Percabangan switch-case & 3.1 \\
\hline
5 & Perulangan for, while, do-while & 2.1, 3.2 \\
\hline
6 & Perulangan bersarang & 3.2 \\
\hline
7 & Praktikum 1 (dasar pemrograman) & 3.1, 3.2 \\
\hline
8 & Fungsi dan prosedur & 4.1 \\
\hline
9 & UTS & CPMK-1, CPMK-2, CPMK-3 \\
\hline
10 & Array satu dimensi & 4.2 \\
\hline
11 & Array dua dimensi, matriks & 4.2 \\
\hline
12 & String dan manipulasi teks & 4.2 \\
\hline
13 & Struct/record, pengurutan dan pencarian & 4.2, 5.1 \\
\hline
14 & Algoritma pengurutan dan pencarian lanjutan & 5.1 \\
\hline
15 & Integrasi, proyek, review materi & Semua CPMK \\
\hline
16 & UAS & Semua CPMK \\
\hline
\end{longtable}

\vspace{0.5cm}

% ============================================================
% 6. METODE PEMBELAJARAN
% ============================================================
\section*{6. Metode Pembelajaran}

Strategi atau pendekatan pembelajaran yang dipilih sesuai OBE yang menekankan aktivitas mahasiswa:

\begin{itemize}[leftmargin=*]
  \item \textbf{Ceramah Interaktif:} Penjelasan konsep dengan diskusi tanya jawab
  \item \textbf{Problem-Based Learning (PBL):} Mahasiswa menyelesaikan permasalahan nyata dengan merancang dan mengimplementasikan algoritma
  \item \textbf{Praktikum Terbimbing:} Latihan coding di laboratorium dengan bimbingan asisten
  \item \textbf{Tugas Individu dan Kelompok:} Pengerjaan assignment dan proyek kecil
  \item \textbf{Diskusi dan Studi Kasus:} Analisis solusi algoritmik untuk masalah komputasi sederhana
\end{itemize}

\vspace{0.5cm}

% ============================================================
% 7. PENGALAMAN BELAJAR MAHASISWA
% ============================================================
\section*{7. Pengalaman Belajar Mahasiswa}

Deskripsi tugas, aktivitas, atau pengalaman belajar yang mendukung pencapaian Sub-CPMK:

\begin{itemize}[leftmargin=*]
  \item Merancang flowchart dan pseudocode untuk kasus nyata (misal: konversi suhu, penentuan bilangan ganjil/genap)
  \item Coding latihan di laboratorium menggunakan bahasa C
  \item Tugas coding dengan percabangan dan perulangan (misal: kalkulator sederhana, pola bilangan)
  \item Membuat program modular dengan fungsi untuk menyelesaikan masalah terstruktur
  \item Implementasi array dan operasi dasar (input/output, penjumlahan, pencarian)
  \item Proyek kecil: program aplikatif sederhana yang mengintegrasikan materi (misal: manajemen data siswa, sistem inventory sederhana)
  \item Mengerjakan latihan soal algoritma pengurutan dan pencarian
\end{itemize}

\vspace{0.5cm}

% ============================================================
% 8. KRITERIA, INDIKATOR, DAN BOBOT PENILAIAN
% ============================================================
\section*{8. Kriteria, Indikator, dan Bobot Penilaian}

Teknik/alat asesmen dipetakan ke Sub-CPMK/CPMK dengan bobot yang jelas:

\setlength{\LTcapwidth}{\linewidth}
\begin{longtable}{|>{\raggedright\arraybackslash}p{2.5cm}|>{\raggedright\arraybackslash}p{3.6cm}|>{\raggedright\arraybackslash}p{5cm}|>{\centering\arraybackslash}p{1.6cm}|}
\hline
\textbf{Komponen} & \textbf{Teknik Asesmen} & \textbf{Indikator/CPMK} & \textbf{Bobot (\%)} \\
\hline
\endfirsthead
\hline
\textbf{Komponen} & \textbf{Teknik Asesmen} & \textbf{Indikator/CPMK} & \textbf{Bobot (\%)} \\
\hline
\endhead
\hline
\endfoot

Tugas Individu & Coding Assignment & Sub-CPMK 1.2, 2.1, 3.1, 3.2, 4.1 & 15 \\
\hline
Kuis & Multiple Choice \& Essay & Sub-CPMK 1.1, 1.2, 3.1, 3.2 & 10 \\
\hline
Praktikum & Lab Exercise & Sub-CPMK 3.1, 3.2, 4.1, 4.2 & 15 \\
\hline
UTS & Written Exam \& Coding & CPMK-1, CPMK-2, CPMK-3 & 25 \\
\hline
Tugas Proyek & Application Development & CPMK-2, CPMK-3, CPMK-4, CPMK-5 & 15 \\
\hline
UAS & Comprehensive Exam & Semua CPMK & 20 \\
\hline
\textbf{Total} & & & \textbf{100} \\
\hline
\end{longtable}

\textbf{Kriteria Penilaian:}
\begin{itemize}[leftmargin=*]
  \item A (85-100): Menguasai semua CPMK dengan sangat baik, mampu menerapkan dalam kasus kompleks
  \item B (70-84): Menguasai sebagian besar CPMK dengan baik
  \item C (60-69): Menguasai CPMK dasar dengan cukup
  \item D (50-59): Menguasai sebagian kecil CPMK
  \item E (<50): Belum menguasai CPMK yang ditetapkan
\end{itemize}

\vspace{0.5cm}

% ============================================================
% 9. EVALUASI DAN REFLEKSI PEMBELAJARAN
% ============================================================
\section*{9. Evaluasi dan Refleksi Pembelajaran}

Penilaian sumatif/formatif untuk memantau ketercapaian outcome secara menyeluruh:

\begin{itemize}[leftmargin=*]
  \item \textbf{Evaluasi Formatif:} Kuis mingguan, latihan coding, feedback praktikum untuk memberikan umpan balik berkelanjutan
  \item \textbf{Evaluasi Sumatif:} UTS dan UAS untuk mengukur pencapaian CPMK secara komprehensif
  \item \textbf{Refleksi Mahasiswa:} Jurnal belajar mingguan untuk refleksi diri terhadap pemahaman materi
  \item \textbf{Evaluasi Dosen:} Survey kepuasan mahasiswa di tengah dan akhir semester
  \item \textbf{Continuous Improvement:} Analisis hasil penilaian untuk perbaikan RPS di semester berikutnya
\end{itemize}

\vspace{0.5cm}

% ============================================================
% 10. DAFTAR REFERENSI
% ============================================================
\section*{10. Daftar Referensi}

Sumber belajar utama yang digunakan dalam penyusunan materi dan asesmen:

\begin{enumerate}[leftmargin=*]
  \item Deitel, P. J., \& Deitel, H. M. (2016). \textit{C How to Program} (8th ed.). Pearson.
  
  \item Kernighan, B. W., \& Ritchie, D. M. (1988). \textit{The C Programming Language} (2nd ed.). Prentice Hall.
  
  \item Cormen, T. H., Leiserson, C. E., Rivest, R. L., \& Stein, C. (2009). \textit{Introduction to Algorithms} (3rd ed.). MIT Press. (Bagian dasar algoritma)
  
  \item King, K. N. (2008). \textit{C Programming: A Modern Approach} (2nd ed.). W. W. Norton \& Company.
  
  \item Sedgewick, R., \& Wayne, K. (2011). \textit{Algorithms} (4th ed.). Addison-Wesley.
  
  \item SPADA Indonesia. \textit{Course: Algoritma dan Pemrograman}. Kemdikbudristek. \url{https://lmsspada.kemdiktisaintek.go.id/}
  
  \item cppreference.com. \textit{C reference}. Retrieved from \url{https://en.cppreference.com/w/c}
  
  \item Widina, B. (2020). \textit{Algoritma \& Pemrograman}. Penerbit Widina.
\end{enumerate}

\vspace{1cm}

\begin{flushright}
\begin{tabular}{c}
Disusun oleh,\\[2cm]
\textbf{Dr. Lorem Ipsum, M.Kom.}\\
NIP. 123456789012345678
\end{tabular}
\end{flushright}

\end{document}
