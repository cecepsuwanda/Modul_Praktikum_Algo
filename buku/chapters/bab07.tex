\documentclass[../main.tex]{subfiles}
\begin{document}
\chapter{String dan Karakter}

\section*{Tujuan Praktikum}
Setelah menyelesaikan praktikum ini, mahasiswa diharapkan mampu:
\begin{itemize}
  \item Memahami konsep string sebagai urutan karakter dan perbedaan representasinya di Pascal, C, dan C++
  \item Mendeklarasikan dan menginisialisasi string dengan berbagai cara
  \item Melakukan operasi dasar string (panjang, concatenation, copy, compare)
  \item Menggunakan fungsi-fungsi manipulasi string built-in
  \item Mengakses dan memodifikasi karakter individual dalam string
  \item Memahami perbedaan antara null-terminated string (C) dan managed string (C++, Pascal)
  \item Membuat program pengolahan teks sederhana dengan manipulasi string
\end{itemize}

\section{Pengantar String}

\subsection{Konsep Dasar String}
String represents sequence of characters yang stored consecutively di memory, digunakan untuk represent text, words, sentences, atau textual data lainnya. Sebagai salah satu most fundamental data types, strings extremely common dipakai dalam programming untuk text manipulation, I/O operations, dan data processing \parencite{pascal-tutorial-wikibooks,iso-c-draft-n1570,cpp-strings,tutorialspoint-c-strings}.

\subsection{Representasi String di Berbagai Bahasa}

Masing-masing bahasa programming punya approach yang berbeda dalam representing dan managing strings:

\begin{itemize}
  \item \textbf{Pascal:} Offers built-in \texttt{string} type yang automatically manages length. Pascal also supports array-based representation dengan fixed atau dynamic length \parencite{pascal-tutorial-wikibooks,free-pascal-docs}.
  
  \item \textbf{C:} Strings are character arrays (\texttt{char[]}) yang terminated dengan null character (\texttt{'\textbackslash 0'}). Programmers fully responsible untuk memory management dan ensuring null terminator exists \parencite{iso-c-draft-n1570,c-strings-h,tutorialspoint-c-strings}.
  
  \item \textbf{C++:} Provides dual approaches: C-style character arrays dan \texttt{std::\allowbreak string} class. String class handles automatic memory management, dynamic sizing, dan rich manipulation methods~\parencite{cpp-strings,cplusplus-string,yuliaagustin-string-cpp}.
\end{itemize}

\paragraph{Catatan Unicode dan UTF-8.} Encoding umum saat ini adalah UTF-8 yang kompatibel dengan ASCII dan efisien untuk teks multibahasa \parencite{rfc3629,utf8-everywhere}. Pada C++ modern tersedia tipe \texttt{char8\_t} (C++20) untuk byte UTF-8, serta literal \texttt{u8"..."} untuk menandai string literal UTF-8 \parencite{cpp-char8,cpp-string-literals}. Di C/Pascal, pastikan file sumber disimpan sebagai UTF-8 agar karakter non-ASCII muncul benar.

\subsection{Keuntungan dan Pertimbangan}

\textbf{Benefits dari high-level string abstractions:}
\begin{itemize}
  \item Automatic memory management (no worries tentang buffer overflows)
  \item Operations yang more intuitive dan safer
  \item Dynamic sizing yang adjusts to needs
  \item Rich API untuk various manipulations
\end{itemize}

\textbf{Considerations untuk C-style strings:}
\begin{itemize}
  \item Higher performance dalam certain scenarios
  \item Compatibility dengan system APIs dan C libraries
  \item Fine-grained control atas memory
  \item Requires extra caution untuk avoid bugs
\end{itemize}

\section{Deklarasi String}

\subsection{Deklarasi String di Pascal}

Pascal provides flexible \texttt{string} type dengan automatic memory management.

Contoh-contoh berikut illustrate various ways untuk declare strings di Pascal, baik dengan dynamic sizing maupun predetermined maximum length:

\begin{lstlisting}[language=Pascal, caption={Deklarasi string di Pascal}]
var
  nama: string;                    // String dengan panjang dinamis
  alamat: string[100];             // String dengan panjang maksimum 100
  kota: string[50];                // String dengan panjang maksimum 50
  
  // Array of string
  daftarNama: array[1..10] of string;
  
  // String kosong
  pesan: string = '';
\end{lstlisting}

\textbf{Karakteristik string Pascal:}
\begin{itemize}
  \item \texttt{string} tanpa ukuran: panjang dinamis hingga batas sistem
  \item \texttt{string[n]}: panjang maksimum \texttt{n} karakter
  \item Indeks dimulai dari 1 (karakter pertama pada \texttt{str[1]})
  \item Fungsi \texttt{Length()} memberikan panjang aktual string
  \item Tidak memerlukan null terminator
\end{itemize}

\subsection{Deklarasi String di C}

Di C, string adalah array karakter yang diakhiri dengan \texttt{'\textbackslash 0'}.

Program berikut mendemonstrasikan berbagai cara mendeklarasikan dan menginisialisasi string dalam C, termasuk array karakter dan pointer ke string literal:

\begin{lstlisting}[language=C, caption={Deklarasi string di C}]
#include <stdio.h>

// Deklarasi berbagai cara
char nama[50];                    // Array 50 karakter (49 + null)
char alamat[100] = "Jl. Merdeka"; // Inisialisasi langsung
char kota[] = "Jakarta";          // Ukuran otomatis (8 byte)

// Array of strings
char hari[7][10] = {
  "Senin", "Selasa", "Rabu", "Kamis",
  "Jumat", "Sabtu", "Minggu"
};

// Pointer ke string literal (immutable)
const char *pesan = "Hello World";

// String kosong
char buffer[256] = "";            // String kosong dengan buffer
\end{lstlisting}

\textbf{Poin penting string C:}
\begin{itemize}
  \item Selalu sediakan ruang untuk null terminator (\texttt{'\textbackslash 0'})
  \item Array berukuran \texttt{n} dapat menyimpan maksimal \texttt{n-1} karakter
  \item String literal disimpan di read-only memory
  \item Pointer ke string literal tidak boleh dimodifikasi
  \item Indeks dimulai dari 0
\end{itemize}

\subsection{Deklarasi String di C++}

C++ menyediakan \texttt{std::string} modern dan array karakter gaya C.

Contoh berikut menunjukkan berbagai cara mendeklarasikan string di C++, dari \texttt{std::string} modern hingga kompatibilitas dengan C-style string:

\begin{lstlisting}[language=C++, caption={Deklarasi string di C++}]
#include <string>
#include <vector>

// Deklarasi std::string
std::string nama;                      // String kosong
std::string alamat = "Jl. Merdeka";    // Inisialisasi langsung
std::string kota("Jakarta");           // Constructor syntax

// String dengan ukuran awal
std::string buffer(100, ' ');          // 100 spasi

// Array gaya C (kompatibilitas)
char legacyStr[50] = "Hello";

// Vector of strings
std::vector<std::string> daftarNama;

// String view (C++17, read-only, non-owning)
std::string_view pesan = "Hello World";

// UTF-8 literal (C++20): tipe char8_t const*
auto hello8 = u8"Halo"; // byte UTF-8
\end{lstlisting}

\textbf{Keuntungan \texttt{std::string}:}
\begin{itemize}
  \item Manajemen memori otomatis (RAII)
  \item Ukuran dinamis, tumbuh sesuai kebutuhan
  \item Aman dari buffer overflow
  \item Metode manipulasi yang kaya
  \item Operator overloading (\texttt{+}, \texttt{==}, \texttt{<}, dll)
\end{itemize}

\section{Inisialisasi dan Pengisian String}

\subsection{Inisialisasi String di Pascal}

Program berikut mendemonstrasikan berbagai cara untuk menginisialisasi dan memanipulasi string di Pascal, termasuk concatenation dan escape karakter:

\begin{lstlisting}[language=Pascal, caption={Inisialisasi string di Pascal}]
var
  nama: string;
  pesan: string;
  alamat: string;
begin
  // Inisialisasi langsung dengan assignment
  nama := 'John Doe';
  
  // Inisialisasi string kosong
  pesan := '';
  
  // Mengisi karakter per karakter
  alamat := 'J';
  alamat := alamat + 'a';
  alamat := alamat + 'k';
  alamat := alamat + 'a';
  alamat := alamat + 'r';
  alamat := alamat + 't';
  alamat := alamat + 'a';  // alamat = "Jakarta"
  
  // Menggunakan string literal dengan quotes
  pesan := 'Hello, World!';
  
  // Escape untuk single quote di dalam string
  pesan := 'It''s a nice day';  // Gunakan double single-quote
end.
\end{lstlisting}

\subsection{Inisialisasi String di C}

Contoh berikut menunjukkan berbagai metode inisialisasi string di C, termasuk pentingnya null terminator dan penggunaan fungsi yang aman:

\begin{lstlisting}[language=C, caption={Inisialisasi string di C}]
#include <string.h>
#include <stdio.h>

int main() {
  // Inisialisasi saat deklarasi
  char nama[] = "John Doe";
  
  // Inisialisasi array karakter
  char alamat[50] = "Jl. Merdeka No. 1";
  
  // Inisialisasi karakter per karakter
  char kota[20];
  kota[0] = 'J';
  kota[1] = 'a';
  kota[2] = 'k';
  kota[3] = 'a';
  kota[4] = 'r';
  kota[5] = 't';
  kota[6] = 'a';
  kota[7] = '\0';  // PENTING: null terminator
  
  // Menggunakan strcpy untuk copy string
  char pesan[100];
  strcpy(pesan, "Hello, World!");
  
  // Menggunakan strncpy (lebih aman)
  char buffer[50];
  strncpy(buffer, "Safe copy", sizeof(buffer) - 1);
  buffer[sizeof(buffer) - 1] = '\0';  // Pastikan null-terminated
  
  // Inisialisasi dengan sprintf
  char formatted[100];
  sprintf(formatted, "Umur: %d tahun", 25);
  
  return 0;
}
\end{lstlisting}

\subsection{Inisialisasi String di C++}

Program berikut mendemonstrasikan berbagai cara menginisialisasi \texttt{std::string} di C++, dari constructor hingga move semantics:

\begin{lstlisting}[language=C++, caption={Inisialisasi string di C++}]
#include <string>
#include <iostream>

int main() {
  // Inisialisasi dengan assignment
  std::string nama = "John Doe";
  
  // Inisialisasi dengan constructor
  std::string alamat("Jl. Merdeka No. 1");
  
  // Inisialisasi string kosong
  std::string pesan;
  
  // Assignment setelah deklarasi
  pesan = "Hello, World!";
  
  // Inisialisasi dengan karakter berulang
  std::string garis(50, '-');  // 50 karakter '-'
  
  // Inisialisasi dari substring
  std::string full = "Hello World";
  std::string sub(full, 0, 5);  // "Hello"
  
  // Inisialisasi dari array C
  char cstr[] = "Legacy string";
  std::string modern(cstr);
  
  // Inisialisasi dengan move semantics
  std::string temp = "Temporary";
  std::string moved = std::move(temp);
  
  // Uniform initialization (C++11)
  std::string uniform{"Modern C++"};
  
  return 0;
}
\end{lstlisting}

\section{Input String}

\subsection{Input String di Pascal}

Program berikut menunjukkan cara membaca string dari pengguna menggunakan \texttt{Readln} di Pascal:

\begin{lstlisting}[language=Pascal, caption={Input string di Pascal}]
var
  nama: string;
  alamat: string;
  umur: integer;
begin
  // Input dengan Readln (membaca satu baris)
  Write('Masukkan nama: ');
  Readln(nama);
  
  // Input string panjang
  Write('Masukkan alamat lengkap: ');
  Readln(alamat);
  
  // Input dengan validasi
  Write('Masukkan umur: ');
  Readln(umur);
  
  // Menampilkan hasil
  Writeln('Nama: ', nama);
  Writeln('Alamat: ', alamat);
  Writeln('Umur: ', umur, ' tahun');
end.
\end{lstlisting}

\subsection{Input String di C}

Contoh berikut mendemonstrasikan cara aman membaca string di C menggunakan \texttt{fgets}, termasuk teknik menghapus newline dan membersihkan buffer:

\begin{lstlisting}[language=C, caption={Input string di C}]
#include <stdio.h>
#include <string.h>

int main() {
  char nama[50];
  char alamat[100];
  int umur;
  
  // Menggunakan scanf untuk satu kata
  printf("Masukkan nama depan: ");
  scanf("%49s", nama);  // Batasi input, cegah overflow
  
  // Clear buffer
  while (getchar() != '\n');
  
  // Menggunakan fgets untuk satu baris (lebih aman)
  printf("Masukkan alamat: ");
  fgets(alamat, sizeof(alamat), stdin);
  
  // Hapus newline di akhir jika ada
  size_t len = strlen(alamat);
  if (len > 0 && alamat[len-1] == '\n') {
    alamat[len-1] = '\0';
  }
  
  // Input angka
  printf("Masukkan umur: ");
  scanf("%d", &umur);
  
  // Tampilkan hasil
  printf("\nData yang dimasukkan:\n");
  printf("Nama: %s\n", nama);
  printf("Alamat: %s\n", alamat);
  printf("Umur: %d tahun\n", umur);
  
  return 0;
}
\end{lstlisting}

\textbf{Catatan penting untuk input string di C:}
\begin{itemize}
  \item \texttt{scanf("\%s")} berhenti di whitespace (spasi, tab, newline)
  \item \texttt{fgets()} membaca seluruh baris termasuk newline
  \item Selalu batasi input untuk menghindari buffer overflow
  \item \texttt{gets()} TIDAK aman dan sudah deprecated, jangan gunakan
  \item Perlu membersihkan buffer input setelah \texttt{scanf()}
\end{itemize}

\subsection{Input String di C++}

Program berikut menunjukkan cara membaca input string di C++ dengan \texttt{std::getline} dan validasi input:

\begin{lstlisting}[language=C++, caption={Input string di C++}]
#include <iostream>
#include <string>
#include <limits>

int main() {
  std::string nama;
  std::string alamat;
  int umur;
  
  // Input satu kata dengan cin
  std::cout << "Masukkan nama depan: ";
  std::cin >> nama;
  
  // Clear buffer dan ignore newline
  std::cin.ignore(std::numeric_limits<std::streamsize>::max(), '\n');
  
  // Input satu baris dengan getline (lebih fleksibel)
  std::cout << "Masukkan alamat lengkap: ";
  std::getline(std::cin, alamat);
  
  // Input angka
  std::cout << "Masukkan umur: ";
  std::cin >> umur;
  
  // Validasi input
  if (std::cin.fail()) {
    std::cerr << "Input tidak valid!\n";
    return 1;
  }
  
  // Tampilkan hasil
  std::cout << "\nData yang dimasukkan:\n";
  std::cout << "Nama: " << nama << "\n";
  std::cout << "Alamat: " << alamat << "\n";
  std::cout << "Umur: " << umur << " tahun\n";
  
  return 0;
}
\end{lstlisting}

\textbf{Tips input string di C++:}
\begin{itemize}
  \item \texttt{std::cin >> str} membaca hingga whitespace
  \item \texttt{std::getline(std::cin, str)} membaca seluruh baris
  \item Gunakan \texttt{std::cin.ignore()} untuk membersihkan buffer
  \item \texttt{std::string} mengelola memori otomatis, tidak ada buffer overflow
  \item Periksa \texttt{std::cin.fail()} untuk validasi input
\end{itemize}

\section{Output String}

\subsection{Output String di Pascal}

Program berikut mendemonstrasikan berbagai cara output string dengan pemformatan di Pascal:

\begin{lstlisting}[language=Pascal, caption={Output string di Pascal}]
var
  nama: string;
  umur: integer;
  nilai: real;
begin
  nama := 'John Doe';
  umur := 25;
  nilai := 87.5;
  
  // Output sederhana
  Writeln('Nama: ', nama);
  
  // Output dengan format
  Writeln('Umur: ', umur, ' tahun');
  Writeln('Nilai: ', nilai:0:2);  // 2 desimal
  
  // Output tanpa newline
  Write('Nama: ');
  Write(nama);
  Writeln;
  
  // Concatenation dalam output
  Writeln('Halo, ' + nama + '!');
  
  // Output dengan lebar field
  Writeln(nama:20);        // Right-aligned, lebar 20
  Writeln('Total':10, nilai:10:2);
end.
\end{lstlisting}

\subsection{Output String di C}

Contoh berikut menunjukkan berbagai teknik output string di C menggunakan \texttt{printf}, \texttt{puts}, dan \texttt{sprintf}:

\begin{lstlisting}[language=C, caption={Output string di C}]
#include <stdio.h>

int main() {
  char nama[] = "John Doe";
  int umur = 25;
  double nilai = 87.5;
  
  // Output dengan printf
  printf("Nama: %s\n", nama);
  printf("Umur: %d tahun\n", umur);
  printf("Nilai: %.2f\n", nilai);
  
  // Output dengan format lebar
  printf("%-20s %3d %6.2f\n", nama, umur, nilai);
  //      |        |   |
  //      left     right aligned
  //      aligned  
  
  // Output dengan puts (otomatis newline)
  puts("=== Data Mahasiswa ===");
  puts(nama);
  
  // Output ke string (sprintf)
  char buffer[100];
  sprintf(buffer, "Nama: %s, Umur: %d", nama, umur);
  printf("%s\n", buffer);
  
  // Output karakter per karakter
  for (int i = 0; nama[i] != '\0'; i++) {
    putchar(nama[i]);
  }
  putchar('\n');
  
  return 0;
}
\end{lstlisting}

\textbf{Format specifier penting:}
\begin{itemize}
  \item \texttt{\%s}: string
  \item \texttt{\%d}: integer desimal
  \item \texttt{\%f}: float/double
  \item \texttt{\%.2f}: float dengan 2 desimal
  \item \texttt{\%10s}: string dengan lebar 10 (right-aligned)
  \item \texttt{\%-10s}: string dengan lebar 10 (left-aligned)
\end{itemize}

\subsection{Output String di C++}

Program berikut mendemonstrasikan output string di C++ dengan berbagai manipulator untuk mengontrol format:

\begin{lstlisting}[language=C++, caption={Output string di C++}]
#include <iostream>
#include <string>
#include <iomanip>

int main() {
  std::string nama = "John Doe";
  int umur = 25;
  double nilai = 87.5;
  
  // Output sederhana dengan cout
  std::cout << "Nama: " << nama << "\n";
  std::cout << "Umur: " << umur << " tahun\n";
  std::cout << "Nilai: " << nilai << "\n";
  
  // Output dengan manipulator format
  std::cout << std::fixed << std::setprecision(2);
  std::cout << "Nilai: " << nilai << "\n";
  
  // Output dengan width dan alignment
  std::cout << std::left << std::setw(20) << nama;
  std::cout << std::right << std::setw(5) << umur;
  std::cout << std::setw(10) << nilai << "\n";
  
  // Output dengan endl vs \n
  std::cout << "Baris 1" << std::endl;  // Flush buffer
  std::cout << "Baris 2" << "\n";       // Tidak flush
  
  // String concatenation dalam output
  std::cout << "Halo, " + nama + "!\n";
  
  // Output ke string (stringstream)
  std::ostringstream oss;
  oss << "Nama: " << nama << ", Umur: " << umur;
  std::string hasil = oss.str();
  std::cout << hasil << "\n";
  
  return 0;
}
\end{lstlisting}

\textbf{Manipulator I/O penting:}
\begin{itemize}
  \item \texttt{std::endl}: newline + flush buffer
  \item \texttt{std::setw(n)}: set field width
  \item \texttt{std::left}, \texttt{std::right}: alignment
  \item \texttt{std::setprecision(n)}: set decimal precision
  \item \texttt{std::fixed}: fixed-point notation
  \item \texttt{std::scientific}: scientific notation
\end{itemize}

\section{Operasi Dasar String}

\subsection{Panjang String}

Obtaining string length merupakan one of the most fundamental operations. Berikut bagaimana cara melakukannya di Pascal dengan memanfaatkan \texttt{Length} function:

\begin{lstlisting}[language=Pascal, caption={Panjang string di Pascal}]
var
  teks: string;
  panjang: integer;
begin
  teks := 'Hello World';
  panjang := Length(teks);  // panjang = 11
  Writeln('Panjang: ', panjang);
end.
\end{lstlisting}

Dalam C, function \texttt{strlen} dari header \texttt{<string.h>} calculates length dengan searching for null terminator:

\begin{lstlisting}[language=C, caption={Panjang string di C}]
#include <string.h>
#include <stdio.h>

int main() {
  char teks[] = "Hello World";
  size_t panjang = strlen(teks);  // panjang = 11
  printf("Panjang: %zu\n", panjang);
  return 0;
}
\end{lstlisting}

C++ offers equivalent methods \texttt{length()} atau \texttt{size()} pada \texttt{std::string} objects:

\begin{lstlisting}[language=C++, caption={Panjang string di C++}]
#include <string>
#include <iostream>

int main() {
  std::string teks = "Hello World";
  size_t panjang = teks.length();  // atau teks.size()
  std::cout << "Panjang: " << panjang << "\n";
  return 0;
}
\end{lstlisting}

\subsection{Konkatenasi (Penggabungan) String}

Program berikut menunjukkan cara menggabungkan dua atau lebih string menjadi satu di Pascal:

\begin{lstlisting}[language=Pascal, caption={Konkatenasi di Pascal}]
var
  str1, str2, hasil: string;
begin
  str1 := 'Hello';
  str2 := 'World';
  
  // Menggunakan operator +
  hasil := str1 + ' ' + str2;  // "Hello World"
  
  // Menggunakan Concat
  hasil := Concat(str1, ' ', str2);  // "Hello World"
  
  Writeln(hasil);
end.
\end{lstlisting}

Di C, konkatenasi dilakukan menggunakan \texttt{strcat} atau \texttt{strncat} (lebih aman) yang memodifikasi string tujuan:

\begin{lstlisting}[language=C, caption={Konkatenasi di C}]
#include <string.h>
#include <stdio.h>

int main() {
  char str1[50] = "Hello";
  char str2[] = "World";
  
  // Menggunakan strcat (modifikasi str1)
  strcat(str1, " ");
  strcat(str1, str2);  // str1 = "Hello World"
  
  // Menggunakan strncat (lebih aman)
  char hasil[100] = "Hello";
  strncat(hasil, " ", sizeof(hasil) - strlen(hasil) - 1);
  strncat(hasil, str2, sizeof(hasil) - strlen(hasil) - 1);
  
  printf("%s\n", hasil);
  return 0;
}
\end{lstlisting}

C++ menyediakan operator \texttt{+} dan \texttt{+=} untuk konkatenasi string yang lebih intuitif:

\begin{lstlisting}[language=C++, caption={Konkatenasi di C++}]
#include <string>
#include <iostream>

int main() {
  std::string str1 = "Hello";
  std::string str2 = "World";
  
  // Menggunakan operator +
  std::string hasil = str1 + " " + str2;
  
  // Menggunakan operator +=
  str1 += " ";
  str1 += str2;  // str1 = "Hello World"
  
  // Menggunakan append()
  std::string gabung = "Hello";
  gabung.append(" ");
  gabung.append(str2);
  
  std::cout << hasil << "\n";
  return 0;
}
\end{lstlisting}

\subsection{Perbandingan String}

Program berikut mendemonstrasikan cara membandingkan string di Pascal menggunakan operator relasional dan fungsi \texttt{CompareStr}:

\begin{lstlisting}[language=Pascal, caption={Perbandingan di Pascal}]
var
  str1, str2: string;
begin
  str1 := 'Apple';
  str2 := 'Banana';
  
  // Perbandingan menggunakan operator
  if str1 = str2 then
    Writeln('String sama')
  else
    Writeln('String berbeda');
  
  // Perbandingan leksikografis
  if str1 < str2 then
    Writeln('str1 lebih kecil')
  else if str1 > str2 then
    Writeln('str1 lebih besar');
  
  // Menggunakan CompareStr (case-sensitive)
  case CompareStr(str1, str2) of
    -1: Writeln('str1 < str2');
     0: Writeln('str1 = str2');
     1: Writeln('str1 > str2');
  end;
end.
\end{lstlisting}

Di C, fungsi \texttt{strcmp} digunakan untuk membandingkan string secara leksikografis:

\begin{lstlisting}[language=C, caption={Perbandingan di C}]
#include <string.h>
#include <stdio.h>

int main() {
  char str1[] = "Apple";
  char str2[] = "Banana";
  
  // strcmp mengembalikan: 0 (sama), <0 (str1<str2), >0 (str1>str2)
  int hasil = strcmp(str1, str2);
  
  if (hasil == 0) {
    printf("String sama\n");
  } else if (hasil < 0) {
    printf("str1 lebih kecil\n");
  } else {
    printf("str1 lebih besar\n");
  }
  
  // strncmp untuk membandingkan n karakter pertama
  if (strncmp(str1, str2, 3) == 0) {
    printf("3 karakter pertama sama\n");
  }
  
  // strcasecmp untuk case-insensitive (non-standard)
  // if (strcasecmp(str1, str2) == 0) { ... }
  
  return 0;
}
\end{lstlisting}

C++ memungkinkan perbandingan string langsung menggunakan operator relasional, serta menyediakan metode \texttt{compare}:

\begin{lstlisting}[language=C++, caption={Perbandingan di C++}]
#include <string>
#include <iostream>
#include <algorithm>

int main() {
  std::string str1 = "Apple";
  std::string str2 = "Banana";
  
  // Menggunakan operator
  if (str1 == str2) {
    std::cout << "String sama\n";
  }
  
  if (str1 < str2) {
    std::cout << "str1 lebih kecil\n";
  }
  
  // Menggunakan compare()
  int hasil = str1.compare(str2);
  // hasil: 0 (sama), <0 (str1<str2), >0 (str1>str2)
  
  // Case-insensitive comparison (manual)
  std::string s1 = str1, s2 = str2;
  std::transform(s1.begin(), s1.end(), s1.begin(), ::tolower);
  std::transform(s2.begin(), s2.end(), s2.begin(), ::tolower);
  if (s1 == s2) {
    std::cout << "Sama (case-insensitive)\n";
  }
  
  return 0;
}
\end{lstlisting}

\section{Manipulasi String Lanjutan}

\subsection{Substring (Mengambil Bagian String)}

Program berikut menunjukkan cara mengambil sebagian dari string menggunakan fungsi \texttt{Copy} di Pascal:

\begin{lstlisting}[language=Pascal, caption={Substring di Pascal}]
var
  teks: string;
  bagian: string;
begin
  teks := 'Hello World';
  
  // Copy(str, start, length)
  bagian := Copy(teks, 1, 5);      // "Hello"
  bagian := Copy(teks, 7, 5);      // "World"
  bagian := Copy(teks, 1, 11);     // "Hello World"
  
  Writeln(bagian);
end.
\end{lstlisting}

Di C, substring dapat dibuat menggunakan \texttt{strncpy} dengan hati-hati menambahkan null terminator:

\begin{lstlisting}[language=C, caption={Substring di C}]
#include <string.h>
#include <stdio.h>

int main() {
  char teks[] = "Hello World";
  char bagian[20];
  
  // Menggunakan strncpy
  strncpy(bagian, teks, 5);      // Copy 5 karakter
  bagian[5] = '\0';               // Tambah null terminator
  printf("%s\n", bagian);         // "Hello"
  
  // Substring dari posisi tertentu
  strncpy(bagian, &teks[6], 5);   // Mulai dari index 6
  bagian[5] = '\0';
  printf("%s\n", bagian);         // "World"
  
  return 0;
}
\end{lstlisting}

C++ menyediakan metode \texttt{substr} yang aman dan mudah digunakan untuk mengekstrak bagian string:

\begin{lstlisting}[language=C++, caption={Substring di C++}]
#include <string>
#include <iostream>

int main() {
  std::string teks = "Hello World";
  
  // substr(pos, len)
  std::string bagian = teks.substr(0, 5);   // "Hello"
  std::cout << bagian << "\n";
  
  // substr dari posisi tertentu
  bagian = teks.substr(6, 5);               // "World"
  std::cout << bagian << "\n";
  
  // substr tanpa length (sampai akhir)
  bagian = teks.substr(6);                  // "World"
  std::cout << bagian << "\n";
  
  return 0;
}
\end{lstlisting}

\subsection{Pencarian dalam String}

Program berikut mendemonstrasikan cara mencari substring dalam string menggunakan fungsi \texttt{Pos} di Pascal:

\begin{lstlisting}[language=Pascal, caption={Pencarian di Pascal}]
var
  teks: string;
  posisi: integer;
begin
  teks := 'Hello World';
  
  // Pos(substring, string) - mengembalikan posisi (1-based)
  posisi := Pos('World', teks);   // posisi = 7
  posisi := Pos('xyz', teks);     // posisi = 0 (tidak ketemu)
  
  if posisi > 0 then
    Writeln('Ditemukan di posisi: ', posisi)
  else
    Writeln('Tidak ditemukan');
end.
\end{lstlisting}

Di C, fungsi \texttt{strstr} dan \texttt{strchr} digunakan untuk mencari substring dan karakter dalam string:

\begin{lstlisting}[language=C, caption={Pencarian di C}]
#include <string.h>
#include <stdio.h>

int main() {
  char teks[] = "Hello World";
  char *posisi;
  
  // strstr() - mengembalikan pointer ke substring
  posisi = strstr(teks, "World");
  if (posisi != NULL) {
    int index = posisi - teks;  // Hitung index
    printf("Ditemukan di index: %d\n", index);  // 6
  } else {
    printf("Tidak ditemukan\n");
  }
  
  // strchr() - mencari karakter tunggal
  posisi = strchr(teks, 'W');
  if (posisi != NULL) {
    printf("Karakter 'W' di index: %ld\n", posisi - teks);
  }
  
  return 0;
}
\end{lstlisting}

C++ menyediakan berbagai metode pencarian yang powerful termasuk \texttt{find}, \texttt{rfind}, dan \texttt{find\_first\_of}:

\begin{lstlisting}[language=C++, caption={Pencarian di C++}]
#include <string>
#include <iostream>

int main() {
  std::string teks = "Hello World";
  
  // find() - mengembalikan posisi atau string::npos
  size_t posisi = teks.find("World");
  if (posisi != std::string::npos) {
    std::cout << "Ditemukan di posisi: " << posisi << "\n";
  }
  
  // find karakter tunggal
  posisi = teks.find('W');
  std::cout << "Karakter 'W' di: " << posisi << "\n";
  
  // rfind() - cari dari belakang
  posisi = teks.rfind('o');
  std::cout << "'o' terakhir di: " << posisi << "\n";
  
  // find_first_of() - cari salah satu karakter
  posisi = teks.find_first_of("aeiou");
  std::cout << "Vokal pertama di: " << posisi << "\n";
  
  return 0;
}
\end{lstlisting}

\subsection{Penggantian Substring}

\textbf{Pascal:}
\begin{lstlisting}[language=Pascal, caption={Replace di Pascal}]
var
  teks: string;
  hasil: string;
  posisi: integer;
begin
  teks := 'Hello World';
  
  // Manual replacement
  posisi := Pos('World', teks);
  if posisi > 0 then
  begin
    Delete(teks, posisi, 5);  // Hapus "World"
    Insert('Pascal', teks, posisi);  // Sisipkan "Pascal"
  end;
  Writeln(teks);  // "Hello Pascal"
  
  // Menggunakan StringReplace (Free Pascal)
  teks := 'Hello World World';
  hasil := StringReplace(teks, 'World', 'Pascal', [rfReplaceAll]);
  Writeln(hasil);  // "Hello Pascal Pascal"
end.
\end{lstlisting}

\textbf{C:}
\begin{lstlisting}[language=C, caption={Replace di C (manual)}]
#include <string.h>
#include <stdio.h>

int main() {
  char teks[100] = "Hello World";
  char buffer[1000];
  char *pos;
  const char *old = "World";
  const char *new = "C";
  
  // Cari substring
  pos = strstr(teks, old);
  
  if (pos) {
    // Copy sebelum substring
    strncpy(buffer, teks, pos - teks);
    buffer[pos - teks] = '\0';
    
    // Tambah replacement
    strcat(buffer, new);
    
    // Tambah sisanya
    strcat(buffer, pos + strlen(old));
    
    // Copy kembali
    strcpy(teks, buffer);
  }
  
  printf("%s\n", teks);  // "Hello C"
  return 0;
}
\end{lstlisting}

\textbf{C++:}
\begin{lstlisting}[language=C++, caption={Replace di C++}]
#include <string>
#include <iostream>

int main() {
  std::string teks = "Hello World";
  
  // replace(pos, len, newstr)
  size_t posisi = teks.find("World");
  if (posisi != std::string::npos) {
    teks.replace(posisi, 5, "C++");
  }
  std::cout << teks << "\n";  // "Hello C++"
  
  // Replace all occurrences
  std::string str = "abc abc abc";
  std::string dari = "abc";
  std::string ke = "xyz";
  
  size_t pos = 0;
  while ((pos = str.find(dari, pos)) != std::string::npos) {
    str.replace(pos, dari.length(), ke);
    pos += ke.length();
  }
  std::cout << str << "\n";  // "xyz xyz xyz"
  
  return 0;
}
\end{lstlisting}

\subsection{Konversi Case (Upper/Lower)}

\textbf{Pascal:}
\begin{lstlisting}[language=Pascal, caption={Case conversion di Pascal}]
uses SysUtils;

var
  teks: string;
begin
  teks := 'Hello World';
  
  // Uppercase
  Writeln(UpperCase(teks));  // "HELLO WORLD"
  
  // Lowercase
  Writeln(LowerCase(teks));  // "hello world"
end.
\end{lstlisting}

\textbf{C:}
\begin{lstlisting}[language=C, caption={Case conversion di C}]
#include <ctype.h>
#include <stdio.h>
#include <string.h>

int main() {
  char teks1[50] = "Hello World";
  char teks2[50] = "Hello World";
  
  // Konversi ke uppercase
  for (int i = 0; teks1[i]; i++) {
    teks1[i] = toupper(teks1[i]);
  }
  printf("%s\n", teks1);  // "HELLO WORLD"
  
  // Konversi ke lowercase
  for (int i = 0; teks2[i]; i++) {
    teks2[i] = tolower(teks2[i]);
  }
  printf("%s\n", teks2);  // "hello world"
  
  return 0;
}
\end{lstlisting}

\textbf{C++:}
\begin{lstlisting}[language=C++, caption={Case conversion di C++}]
#include <string>
#include <algorithm>
#include <iostream>

int main() {
  std::string teks = "Hello World";
  
  // To uppercase
  std::transform(teks.begin(), teks.end(), teks.begin(), 
                 ::toupper);
  std::cout << teks << "\n";  // "HELLO WORLD"
  
  // To lowercase
  teks = "Hello World";
  std::transform(teks.begin(), teks.end(), teks.begin(), 
                 ::tolower);
  std::cout << teks << "\n";  // "hello world"
  
  return 0;
}
\end{lstlisting}

\section{Pengolahan Karakter \& Escape Sequences}

\subsection{Karakter Khusus dan Escape Sequences}

Karakter khusus diwakili dengan escape sequences menggunakan backslash (\textbackslash).

\begin{table}[H]
  \centering
  \caption{Escape sequences umum}
  \begin{tabular}{@{}lll@{}}
    \toprule
    Notasi & Arti & Deskripsi \\
    \midrule
    \texttt{\textbackslash n} & Newline & Baris baru (Line Feed) \\
    \texttt{\textbackslash r} & Carriage Return & Kembali ke awal baris \\
    \texttt{\textbackslash t} & Tab & Horizontal tab \\
    \texttt{\textbackslash\textbackslash} & Backslash & Karakter backslash literal \\
    \texttt{\textbackslash'} & Single quote & Karakter single quote \\
    \texttt{\textbackslash"} & Double quote & Karakter double quote \\
    \texttt{\textbackslash 0} & Null & Karakter null (string terminator di C) \\
    \texttt{\textbackslash b} & Backspace & Mundur satu karakter \\
    \texttt{\textbackslash f} & Form feed & Ganti halaman \\
    \bottomrule
  \end{tabular}
\end{table}

\subsection{Contoh Penggunaan Escape Sequences}

\textbf{Pascal:}
\begin{lstlisting}[language=Pascal, caption={Escape sequences di Pascal}]
begin
  // Single quote dalam string (double single quote)
  Writeln('It''s a nice day');
  
  // Menggunakan #10 untuk newline, #13 untuk CR
  Writeln('Baris 1'#10'Baris 2');
  
  // Tab dengan #9
  Writeln('Kolom1'#9'Kolom2'#9'Kolom3');
  
  // Kombinasi
  Writeln('Nama:'#9'John Doe'#10'Umur:'#9'25');
end.
\end{lstlisting}

\textbf{C/C++:}
\begin{lstlisting}[language=C, caption={Escape sequences di C/C++}]
#include <stdio.h>

int main() {
  // Newline
  printf("Baris 1\nBaris 2\n");
  
  // Tab
  printf("Kolom1\tKolom2\tKolom3\n");
  
  // Quote marks
  printf("He said, \"Hello!\"\n");
  printf("It's a nice day\n");
  
  // Backslash
  printf("Path: C:\\Users\\Admin\n");
  
  // Kombinasi
  printf("Nama:\tJohn Doe\nUmur:\t25\n");
  
  return 0;
}
\end{lstlisting}

\subsection{Klasifikasi Karakter}

\textbf{Fungsi \texttt{<ctype.h>} di C:}
\begin{lstlisting}[language=C, caption={Klasifikasi karakter di C}]
#include <ctype.h>
#include <stdio.h>

int main() {
  char ch = 'A';
  
  // Cek tipe karakter
  if (isalpha(ch))  printf("%c adalah huruf\n", ch);
  if (isupper(ch))  printf("%c adalah huruf besar\n", ch);
  if (islower(ch))  printf("%c adalah huruf kecil\n", ch);
  if (isdigit(ch))  printf("%c adalah digit\n", ch);
  if (isalnum(ch))  printf("%c adalah alphanumeric\n", ch);
  if (isspace(ch))  printf("%c adalah whitespace\n", ch);
  if (ispunct(ch))  printf("%c adalah punctuation\n", ch);
  
  // Konversi
  printf("Uppercase: %c\n", toupper('a'));  // 'A'
  printf("Lowercase: %c\n", tolower('A'));  // 'a'
  
  return 0;
}
\end{lstlisting}

\section{Fungsi/Prosedur String Library}

\subsection{Fungsi String di C (\texttt{<string.h>})}

\begin{table}[H]
  \centering
  \caption{Fungsi-fungsi penting di \texttt{<string.h>}}
  \begin{tabular}{@{}lp{8cm}@{}}
    \toprule
    Fungsi & Deskripsi \\
    \midrule
    \texttt{strlen(s)} & Menghitung panjang string \\
    \texttt{strcpy(dest, src)} & Copy string (unsafe) \\
    \texttt{strncpy(dest, src, n)} & Copy n karakter (safer) \\
    \texttt{strcat(dest, src)} & Concatenate string (unsafe) \\
    \texttt{strncat(dest, src, n)} & Concatenate n karakter (safer) \\
    \texttt{strcmp(s1, s2)} & Bandingkan string \\
    \texttt{strncmp(s1, s2, n)} & Bandingkan n karakter \\
    \texttt{strchr(s, c)} & Cari karakter pertama \\
    \texttt{strrchr(s, c)} & Cari karakter terakhir \\
    \texttt{strstr(haystack, needle)} & Cari substring \\
    \texttt{strtok(s, delim)} & Tokenize string \\
    \texttt{memset(s, c, n)} & Set n bytes dengan nilai c \\
    \texttt{memcpy(dest, src, n)} & Copy n bytes \\
    \bottomrule
  \end{tabular}
\end{table}

\subsection{Contoh Penggunaan Library C}

\begin{lstlisting}[language=C, caption={Contoh fungsi string C}]
#include <string.h>
#include <stdio.h>

int main() {
  char str1[50] = "Hello";
  char str2[50];
  char str3[100];
  
  // Copy string
  strcpy(str2, str1);
  printf("Copied: %s\n", str2);
  
  // Concatenate
  strcpy(str3, str1);
  strcat(str3, " World");
  printf("Concatenated: %s\n", str3);
  
  // Compare
  if (strcmp(str1, str2) == 0) {
    printf("Strings are equal\n");
  }
  
  // Search
  char *pos = strstr(str3, "World");
  if (pos != NULL) {
    printf("Found at position: %ld\n", pos - str3);
  }
  
  // Tokenize
  char sentence[] = "This is a test";
  char *token = strtok(sentence, " ");
  while (token != NULL) {
    printf("Token: %s\n", token);
    token = strtok(NULL, " ");
  }
  
  return 0;
}
\end{lstlisting}

\subsection{Metode \texttt{std::string} di C++}

\begin{table}[H]
  \centering
  \caption{Metode penting \texttt{std::string}}
  \begin{tabular}{@{}lp{8cm}@{}}
    \toprule
    Metode & Deskripsi \\
    \midrule
    \texttt{length()}, \texttt{size()} & Ukuran string \\
    \texttt{empty()} & Cek apakah string kosong \\
    \texttt{clear()} & Kosongkan string \\
    \texttt{append(s)} & Tambah string di akhir \\
    \texttt{insert(pos, s)} & Sisipkan string di posisi \\
    \texttt{erase(pos, len)} & Hapus substring \\
    \texttt{replace(pos, len, s)} & Ganti substring \\
    \texttt{substr(pos, len)} & Ambil substring \\
    \texttt{find(s)} & Cari substring \\
    \texttt{rfind(s)} & Cari dari belakang \\
    \texttt{compare(s)} & Bandingkan string \\
    \texttt{c\_str()} & Konversi ke C-string \\
    \texttt{at(i)} & Akses karakter dengan bounds checking \\
    \texttt{front()}, \texttt{back()} & Karakter pertama/terakhir \\
    \bottomrule
  \end{tabular}
\end{table}

\subsection{Contoh Penggunaan \texttt{std::string}}

\begin{lstlisting}[language=C++, caption={Contoh metode std::string}]
#include <string>
#include <iostream>

int main() {
  std::string str = "Hello World";
  
  // Size operations
  std::cout << "Length: " << str.length() << "\n";
  std::cout << "Empty: " << str.empty() << "\n";
  
  // Modification
  str.append(" C++");
  std::cout << str << "\n";  // "Hello World C++"
  
  str.insert(5, ",");
  std::cout << str << "\n";  // "Hello, World C++"
  
  str.erase(5, 1);  // Hapus koma
  std::cout << str << "\n";  // "Hello World C++"
  
  // Substring
  std::string sub = str.substr(0, 5);
  std::cout << "Substring: " << sub << "\n";  // "Hello"
  
  // Search
  size_t pos = str.find("World");
  if (pos != std::string::npos) {
    std::cout << "Found at: " << pos << "\n";
  }
  
  // Replace
  str.replace(pos, 5, "C++");
  std::cout << str << "\n";  // "Hello C++ C++"
  
  // Access
  std::cout << "First char: " << str.front() << "\n";
  std::cout << "Last char: " << str.back() << "\n";
  std::cout << "At position 0: " << str.at(0) << "\n";
  
  // Convert to C-string
  const char* cstr = str.c_str();
  printf("As C-string: %s\n", cstr);
  
  return 0;
}
\end{lstlisting}

\section{Contoh Program Lengkap}

\subsection{Program Pengolahan Teks di Pascal}

Program lengkap berikut menganalisis teks dengan menghitung jumlah huruf besar, huruf kecil, angka, dan spasi, serta menampilkan versi uppercase dan lowercase:

\begin{lstlisting}[language=Pascal, caption={Program analisis teks di Pascal}]
program AnalisTeks;
uses SysUtils;

var
  teks: string;
  kata: string;
  i, panjang: integer;
  hurufBesar, hurufKecil, angka, spasi: integer;

begin
  hurufBesar := 0;
  hurufKecil := 0;
  angka := 0;
  spasi := 0;
  
  // Input teks
  Write('Masukkan teks: ');
  Readln(teks);
  
  // Analisis karakter
  panjang := Length(teks);
  for i := 1 to panjang do
  begin
    if teks[i] in ['A'..'Z'] then
      Inc(hurufBesar)
    else if teks[i] in ['a'..'z'] then
      Inc(hurufKecil)
    else if teks[i] in ['0'..'9'] then
      Inc(angka)
    else if teks[i] = ' ' then
      Inc(spasi);
  end;
  
  // Tampilkan hasil
  Writeln;
  Writeln('=== Hasil Analisis ===');
  Writeln('Panjang teks: ', panjang);
  Writeln('Huruf besar: ', hurufBesar);
  Writeln('Huruf kecil: ', hurufKecil);
  Writeln('Angka: ', angka);
  Writeln('Spasi: ', spasi);
  Writeln;
  Writeln('Uppercase: ', UpperCase(teks));
  Writeln('Lowercase: ', LowerCase(teks));
end.
\end{lstlisting}

\subsection{Program Manipulasi String di C}

Program C berikut melakukan analisis karakter dan konversi case menggunakan fungsi dari \texttt{<ctype.h>}:

\begin{lstlisting}[language=C, caption={Program manipulasi string di C}]
#include <stdio.h>
#include <string.h>
#include <ctype.h>

int main() {
  char teks[256];
  char upper[256], lower[256];
  int nUpper, nLower, nDigit, nSpace;
  int i;
  
  // Input
  printf("Masukkan teks: ");
  fgets(teks, sizeof(teks), stdin);
  
  // Hapus newline
  size_t len = strlen(teks);
  if (len > 0 && teks[len-1] == '\n') {
    teks[len-1] = '\0';
  }
  
  // Analisis karakter
  nUpper = nLower = nDigit = nSpace = 0;
  for (i = 0; teks[i]; i++) {
    if (isupper(teks[i])) nUpper++;
    else if (islower(teks[i])) nLower++;
    else if (isdigit(teks[i])) nDigit++;
    else if (isspace(teks[i])) nSpace++;
  }
  
  // Buat versi uppercase
  strcpy(upper, teks);
  for (i = 0; upper[i]; i++) {
    upper[i] = toupper(upper[i]);
  }
  
  // Buat versi lowercase
  strcpy(lower, teks);
  for (i = 0; lower[i]; i++) {
    lower[i] = tolower(lower[i]);
  }
  
  // Output
  printf("\n=== Hasil Analisis ===\n");
  printf("Panjang teks: %zu\n", strlen(teks));
  printf("Huruf besar: %d\n", nUpper);
  printf("Huruf kecil: %d\n", nLower);
  printf("Angka: %d\n", nDigit);
  printf("Spasi: %d\n", nSpace);
  printf("\nUppercase: %s\n", upper);
  printf("Lowercase: %s\n", lower);
  
  return 0;
}
\end{lstlisting}

\subsection{Program Pengolahan String di C++}

Program C++ lengkap berikut menggunakan range-based for loop dan algoritme STL untuk analisis dan manipulasi string:

\begin{lstlisting}[language=C++, caption={Program pengolahan string di C++}]
#include <iostream>
#include <string>
#include <algorithm>
#include <cctype>
using namespace std;

int main() {
  string teks;
  int panjang, hurufBesar, hurufKecil, angka, spasi;
  
  // Input
  cout << "Masukkan teks: ";
  getline(cin, teks);
  
  // Analisis dengan loop
  panjang = teks.length();
  hurufBesar = 0;
  hurufKecil = 0;
  angka = 0;
  spasi = 0;
  
  for (char ch : teks) {
    if (isupper(ch)) hurufBesar++;
    else if (islower(ch)) hurufKecil++;
    else if (isdigit(ch)) angka++;
    else if (isspace(ch)) spasi++;
  }
  
  // Output
  cout << "\n=== Hasil Analisis ===\n";
  cout << "Panjang teks: " << panjang << "\n";
  cout << "Huruf besar: " << hurufBesar << "\n";
  cout << "Huruf kecil: " << hurufKecil << "\n";
  cout << "Angka: " << angka << "\n";
  cout << "Spasi: " << spasi << "\n";
  
  // Konversi ke uppercase
  string upper = teks;
  transform(upper.begin(), upper.end(), upper.begin(), ::toupper);
  cout << "\nUppercase: " << upper << "\n";
  
  // Konversi ke lowercase
  string lower = teks;
  transform(lower.begin(), lower.end(), lower.begin(), ::tolower);
  cout << "Lowercase: " << lower << "\n";
  
  // Demonstrasi operasi tambahan
  cout << "\n=== Operasi Tambahan ===\n";
  
  // Substring
  if (teks.length() >= 5) {
    cout << "5 karakter pertama: " << teks.substr(0, 5) << "\n";
  }
  
  // Find
  size_t pos = teks.find("a");
  if (pos != string::npos) {
    cout << "Huruf 'a' pertama di posisi: " << pos << "\n";
  }
  
  // Replace
  string modified = teks;
  pos = modified.find(" ");
  if (pos != string::npos) {
    modified.replace(pos, 1, "_");
    cout << "Spasi pertama diganti underscore: " << modified << "\n";
  }
  
  return 0;
}
\end{lstlisting}

\section{Best Practices dan Peringatan}

\subsection{Keamanan String di C}

\textbf{Hindari fungsi unsafe:}
\begin{itemize}
  \item \textbf{JANGAN} gunakan \texttt{gets()} -- sangat berbahaya, sudah deprecated
  \item \textbf{JANGAN} gunakan \texttt{strcpy()} tanpa cek ukuran
  \item \textbf{JANGAN} gunakan \texttt{strcat()} tanpa cek ukuran
  \item \textbf{JANGAN} gunakan \texttt{sprintf()} tanpa batas
\end{itemize}

\textbf{Gunakan alternatif aman:}
\begin{itemize}
  \item Gunakan \texttt{fgets()} bukan \texttt{gets()}
  \item Gunakan \texttt{strncpy()} bukan \texttt{strcpy()}
  \item Gunakan \texttt{strncat()} bukan \texttt{strcat()}
  \item Gunakan \texttt{snprintf()} bukan \texttt{sprintf()}
\end{itemize}

\subsection{Tips Umum}

\textbf{Pascal:}
\begin{itemize}
  \item Gunakan \texttt{SetLength()} untuk resize string dinamis
  \item Indeks string dimulai dari 1, bukan 0
  \item Single quote dalam string dengan double single-quote (\texttt{''})
  \item Manfaatkan fungsi bawaan seperti \texttt{Trim()}, \texttt{Pos()}, \texttt{Copy()}
\end{itemize}

\textbf{C:}
\begin{itemize}
  \item Selalu sediakan ruang untuk null terminator
  \item Cek batas buffer sebelum operasi
  \item Gunakan \texttt{sizeof()} untuk ukuran buffer
  \item Validasi pointer sebelum dereference
  \item Gunakan fungsi dengan suffix 'n' (strncpy, strncat) untuk keamanan
\end{itemize}

\textbf{C++:}
\begin{itemize}
  \item Prefer \texttt{std::string} daripada C-string untuk code baru
  \item Gunakan \texttt{std::string\_view} untuk parameter read-only
  \item Gunakan \texttt{at()} bukan \texttt{[]} untuk bounds checking
  \item Manfaatkan range-based for loop untuk iterasi
  \item Gunakan \texttt{std::getline()} untuk input multi-kata
\end{itemize}

\subsection{Masalah Umum Encoding}
\begin{itemize}
  \item Pastikan file sumber disimpan dengan encoding UTF-8 tanpa BOM agar literal string non-ASCII konsisten.
  \item Pada C/C++, literal seperti \texttt{u8"teks"} menghasilkan byte UTF-8; gunakan API I/O yang kompatibel.
  \item Pada platform Windows lawas, konversi antara UTF-16 (wide) dan UTF-8 mungkin diperlukan sebelum I/O.
\end{itemize}

\subsection{Kesalahan Umum}

\begin{enumerate}
  \item \textbf{Buffer overflow di C}: Menulis melewati batas array
  \item \textbf{Missing null terminator}: Lupa menambah \texttt{'\textbackslash 0'}
  \item \textbf{Off-by-one error}: Salah hitung panjang dengan terminator
  \item \textbf{Modifikasi string literal}: Undefined behavior di C/C++
  \item \textbf{Tidak membersihkan buffer input}: Karakter tersisa di stdin
  \item \textbf{Perbandingan pointer}: Menggunakan \texttt{==} bukan \texttt{strcmp()}
  \item \textbf{Memory leak}: Lupa free string yang dialokasi dinamis
\end{enumerate}

\section{Rangkuman Materi}

Strings constitute fundamental data type untuk text representation dalam programming. Solid understanding tentang representation, operations, dan string manipulation critically important untuk application development.

\subsection{Poin-Poin Penting}

\textbf{String Representation:}
\begin{itemize}
  \item Pascal: Built-in \texttt{string} type dengan automatic management
  \item C: \texttt{char} arrays dengan null terminator requirement
  \item C++: \texttt{std::string} featuring automatic memory management
  \item Unicode/UTF-8 considerations, \texttt{u8"..."} literals, \texttt{char8\_t} types, dan encoding issues
\end{itemize}

\textbf{Basic Operations:}
\begin{itemize}
  \item Declaration dan initialization of strings
  \item String input/output dengan multiple methods
  \item Length calculation, concatenation, comparison
  \item Substring extraction, searching, replacement
  \item Case conversion (uppercase/lowercase)
  \item String/character I/O; escape sequence handling
\end{itemize}

\textbf{Security Considerations:}
\begin{itemize}
  \item Always validate buffer bounds di C
  \item Employ safe functions (strncpy, fgets, snprintf)
  \item Prefer \texttt{std::string} di C++ untuk automatic security
  \item Avoid deprecated functions seperti \texttt{gets()}
  \item Utilize C library functions dan string methods untuk safe manipulation
\end{itemize}

\subsection{Perbandingan Antar Bahasa}

\begin{table}[H]
\centering
\begin{tabular}{|l|c|c|c|}
\hline
\textbf{Aspek} & \textbf{Pascal} & \textbf{C} & \textbf{C++} \\
\hline
Manajemen memori & Otomatis & Manual & Otomatis \\
\hline
Null terminator & Tidak perlu & Wajib & Tidak (std::string) \\
\hline
Ukuran dinamis & Ya & Tidak & Ya \\
\hline
Keamanan tipe & Kuat & Lemah & Kuat \\
\hline
Konkatenasi & Operator + & strcat/strncat & Operator + \\
\hline
Indeks mulai & 1 & 0 & 0 \\
\hline
Bounds checking & Ya & Tidak & at() method \\
\hline
\end{tabular}
\caption{Perbandingan string di Pascal, C, dan C++}
\end{table}

\subsection{Kapan Menggunakan Apa}

\textbf{Use Pascal strings ketika:}
\begin{itemize}
  \item Developing Pascal applications
  \item Need index flexibility
  \item Want automatic memory management
\end{itemize}

\textbf{Use C strings ketika:}
\begin{itemize}
  \item Interfacing dengan system APIs atau C libraries
  \item Require low-level memory control
  \item Performance-critical optimization scenarios
  \item Embedded systems dengan limited resources
\end{itemize}

\textbf{Use C++ \texttt{std::string} ketika:}
\begin{itemize}
  \item Building modern C++ applications
  \item Safety dan convenience outweigh granular control
  \item Need rich string operations
  \item Want to avoid manual memory management
\end{itemize}

Strings form the foundation dari text processing dalam programming. Mastering string operations akan significantly facilitate development dari various applications—from data parsing, input validation, hingga complex text processing tasks. Always prioritize safety dan code clarity!

\end{document}
