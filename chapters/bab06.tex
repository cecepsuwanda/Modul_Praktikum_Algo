\documentclass[../main.tex]{subfiles}
\begin{document}
\chapter{Array \& Multidimensi}
\section{Array Satu Dimensi}
Array satu dimensi menyimpan elemen bertipe sama secara kontigu, dengan akses indeks \(O(1)\) dan iterasi efisien. Pascal memakai batas eksplisit, C memakai ukuran sebagai bagian dari tipe, dan C++ menyediakan \texttt{std::array} (ukuran tetap) \parencite{pascal-tutorial-wikibooks,iso-c-draft-n1570,cpp-reference,cpp-std-array}.

Pada C, nama array sering terdegradasi menjadi pointer sehingga memengaruhi cara parameter fungsi. Di C++, \texttt{std::array} menyimpan ukuran di tingkat tipe (bantu validasi kompilasi). Pada Pascal, rentang indeks eksplisit memudahkan penalaran batas \parencite{free-pascal-docs,cpp-std-array}.

Representasi kontigu memaksimalkan locality cache dan performa operasi berulang. Namun, hal ini juga menuntut kebijakan alokasi yang jelas untuk koleksi besar agar tidak menimbulkan fragmentasi memori. Uji unit untuk indeks pertama dan terakhir penting untuk mencegah kesalahan off-by-one.

\section{Array Multidimensi}
Array multidimensi memperluas konsep ke dua dimensi atau lebih untuk memodelkan matriks/tensor. Pascal dan C mendukung array bertingkat (row-major), sedangkan C++ umum memakai \texttt{std::vector<std::vector<T>>} atau buffer datar + indeks terhitung \parencite{iso-c-draft-n1570,cpp-reference,cpp-vector}.

Pada representasi datar, hitung indeks linier menggunakan rumus yang mempertimbangkan lebar baris dan kolom. Pendekatan ini mengurangi alokasi berulang dan meningkatkan locality, terutama untuk operasi numerik intensif. Sertakan fungsi pembantu untuk menerjemahkan koordinat ke indeks linier agar kesalahan penghitungan dapat diminimalkan.

\section{Operasi Dasar (iterasi, traversal)}
Operasi umum: iterasi, pencarian, transformasi. Gunakan bentuk loop yang jelas; pada C++ pertimbangkan algoritme standar \texttt{<algorithm>} untuk menggantikan loop manual. Pada Pascal dan C, gunakan idiom traversal yang aman \parencite{free-pascal-docs,gnu-c-manual}.

\subsection{Contoh Lintas Bahasa}
\begin{lstlisting}[language=Pascal, caption={Array tetap di Pascal}]
var a: array[1..5] of integer; i: integer; begin
  for i := 1 to 5 do a[i] := i*i; Writeln(a[5]);
end.
\end{lstlisting}

\begin{lstlisting}[language=C, caption={Array C dan degradasi ke pointer}]
void fill(int *p, int n){ for (int i=0;i<n;++i) p[i]=i*i; }
int a[5]; fill(a, 5); // a terdegradasi menjadi int*
\end{lstlisting}

\begin{lstlisting}[language=C++, caption={std::array dan algoritme}]
#include <array>
#include <algorithm>
#include <iostream>
int main(){
  std::array<int,5> a{}; int v=1;
  std::generate(a.begin(), a.end(), [&]{ return v++ * v; });
  std::cout << a.back() << "\n";
}
\end{lstlisting}

Validasi batas sebelum akses merupakan kebiasaan yang tidak boleh diabaikan, terutama ketika indeks berasal dari input pengguna. Untuk operasi transformasi, pertimbangkan penggunaan buffer sementara untuk menghindari konflik baca-tulis yang mengakibatkan hasil tak terduga. Uji unit berfokus pada nilai batas dan koleksi kosong sangat membantu.

\section{Alokasi Dinamis pada Array}
Ketika ukuran tidak diketahui di awal, gunakan alokasi dinamis. Di C, \texttt{malloc}/\texttt{free}/\texttt{realloc}; di C++ \texttt{std::vector} mengelola kapasitas otomatis; pada Pascal modern, array dinamis didukung \parencite{iso-c-draft-n1570,cpp-reference,free-pascal-docs,cpp-vector,pascal-dynarray}.

\subsection{Contoh Dinamis}
\begin{lstlisting}[language=C, caption={malloc/realloc/free di C}]
#include <stdlib.h>
int *a = malloc(n * sizeof *a);
// ... isi a ...
int *tmp = realloc(a, m * sizeof *a); if (tmp) a = tmp;
free(a);
\end{lstlisting}

\begin{lstlisting}[language=C++, caption={std::vector di C++}]
#include <vector>
#include <iostream>
int main(){
  std::vector<int> v; v.reserve(100);
  for (int i=0;i<200;++i) v.push_back(i);
  std::cout << v.size() << " " << v.capacity() << "\n";
}
\end{lstlisting}

\begin{lstlisting}[language=Pascal, caption={Array dinamis di Free Pascal}]
var a: array of integer; i: integer; begin
  SetLength(a, 10);
  for i := 0 to High(a) do a[i] := i;
  SetLength(a, 20);
end.
\end{lstlisting}

Pastikan invarian kepemilikan memori dipertahankan pada semua jalur eksekusi, termasuk cabang kesalahan. Pada C++, gunakan RAII untuk memastikan pembebasan sumber daya meskipun terjadi pengecualian. Hindari penyalinan yang tidak perlu dengan menerapkan strategi \emph{move} dan peminjaman referensi yang jelas.

\subsection{Catatan Eksekusi (OnlineGDB, Lazarus, Code::Blocks)}
\begin{itemize}
  \item \textbf{OnlineGDB}: \url{https://www.onlinegdb.com/} \textrightarrow{} pilih Pascal/C/C++, tempel contoh satu berkas, Run.
  \item \textbf{Lazarus (Pascal)}: Console Application, tempel contoh Pascal, Run.
  \item \textbf{Code::Blocks (C/C++)}: Console application, pilih C/C++, tempel ke \texttt{main.c}/\texttt{main.cpp}, Build \& Run.
\end{itemize}
\end{document}
