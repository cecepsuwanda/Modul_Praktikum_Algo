\documentclass[../main.tex]{subfiles}
\begin{document}
\chapter{Array \& Multidimensi}
\section{Array Satu Dimensi}
Array satu dimensi menyimpan elemen bertipe sama dalam susunan kontigu, sehingga mendukung akses indeks \(O(1)\) dan iterasi yang efisien. Pascal menggunakan deklarasi batas eksplisit, C memanfaatkan ukuran sebagai bagian dari tipe, dan C++ menyediakan \texttt{std::array} untuk ukuran tetap. Pertimbangan utama meliputi pemilihan ukuran, kebijakan kepemilikan, dan strategi validasi indeks \parencite{pascal-tutorial-wikibooks,iso-c-draft-n1570,cpp-reference}.

Pada C, nama array sering terdegradasi menjadi pointer pada konteks tertentu sehingga memengaruhi cara parameter diteruskan ke fungsi. Di C++, \texttt{std::array} menjaga informasi ukuran di tingkat tipe yang membantu validasi pada waktu kompilasi. Pada Pascal, rentang indeks yang eksplisit memudahkan penalaran formal terhadap batas dan invarian \parencite{free-pascal-docs}.

Representasi kontigu memaksimalkan locality cache dan performa operasi berulang. Namun, hal ini juga menuntut kebijakan alokasi yang jelas untuk koleksi besar agar tidak menimbulkan fragmentasi memori. Uji unit untuk indeks pertama dan terakhir penting untuk mencegah kesalahan off-by-one.

\section{Array Multidimensi}
Array multidimensi memperluas konsep ke dua dimensi atau lebih untuk memodelkan matriks dan tensor. Pascal dan C mendukung array bertingkat dengan tata letak baris-utama, sementara C++ menyediakan \texttt{std::vector<std::vector<T>>} atau penampung datar dengan indeks terhitung. Pilihan representasi memengaruhi efisiensi akses dan kesederhanaan API \parencite{iso-c-draft-n1570,cpp-reference}.

Pada representasi datar, hitung indeks linier menggunakan rumus yang mempertimbangkan lebar baris dan kolom. Pendekatan ini mengurangi alokasi berulang dan meningkatkan locality, terutama untuk operasi numerik intensif. Sertakan fungsi pembantu untuk menerjemahkan koordinat ke indeks linier agar kesalahan penghitungan dapat diminimalkan.

\section{Operasi Dasar (iterasi, traversal)}
Operasi umum pada array meliputi iterasi, pencarian, dan transformasi elemen. Gunakan bentuk loop yang menyatakan niat dengan jelas dan pertimbangkan algoritme standar pada C++ untuk menggantikan loop manual ketika sesuai. Pada Pascal dan C, dokumentasi referensi menyediakan idiom yang aman dan portabel untuk traversal \parencite{free-pascal-docs,gnu-c-manual}.

Validasi batas sebelum akses merupakan kebiasaan yang tidak boleh diabaikan, terutama ketika indeks berasal dari input pengguna. Untuk operasi transformasi, pertimbangkan penggunaan buffer sementara untuk menghindari konflik baca-tulis yang mengakibatkan hasil tak terduga. Uji unit berfokus pada nilai batas dan koleksi kosong sangat membantu.

\section{Alokasi Dinamis pada Array}
Ketika ukuran tidak diketahui di awal, gunakan alokasi dinamis untuk menyesuaikan kapasitas saat runtime. Di C, \texttt{malloc}/\texttt{free} dan keluarga \texttt{realloc} memberikan kontrol tingkat rendah, sedangkan di C++ \texttt{std::vector} mengelola kapasitas dan pertumbuhan secara otomatis. Pada Pascal modern, dukungan array dinamis memungkinkan strategi serupa dengan antarmuka yang lebih aman \parencite{iso-c-draft-n1570,cpp-reference,free-pascal-docs}.

Pastikan invarian kepemilikan memori dipertahankan pada semua jalur eksekusi, termasuk cabang kesalahan. Pada C++, gunakan RAII untuk memastikan pembebasan sumber daya meskipun terjadi pengecualian. Hindari penyalinan yang tidak perlu dengan menerapkan strategi \emph{move} dan peminjaman referensi yang jelas.
\end{document}
