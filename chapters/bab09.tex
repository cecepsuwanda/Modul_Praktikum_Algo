\documentclass[../main.tex]{subfiles}
\begin{document}
\chapter{Pointer dan Alamat Memori}

\section*{Tujuan Praktikum}
Setelah menyelesaikan praktikum ini, mahasiswa diharapkan mampu:
\begin{itemize}
  \item Memahami konsep pointer sebagai variabel yang menyimpan alamat memori
  \item Mendeklarasikan dan menginisialisasi pointer dengan benar
  \item Melakukan operasi dereferencing untuk mengakses nilai di alamat yang ditunjuk
  \item Memahami hubungan antara pointer dan array
  \item Menggunakan pointer untuk parameter passing by reference
  \item Memahami pointer aritmetika dan operasinya
  \item Menghindari kesalahan umum pointer (null pointer, dangling pointer, memory leak)
\end{itemize}

\section{Pengantar Pointer}

\subsection{Konsep Dasar Pointer}
Pointer adalah variabel khusus yang menyimpan alamat memori dari variabel lain atau objek. Berbeda dengan variabel biasa yang menyimpan nilai data, pointer menyimpan lokasi di mana data tersebut berada di memori \parencite{gnu-c-manual,cpp-reference,tutorialspoint-c-pointers,geeksforgeeks-pointers}.

\textbf{Mengapa pointer penting?}
\begin{itemize}
  \item \textbf{Efisiensi:} Menghindari copy data besar dengan passing address
  \item \textbf{Dynamic memory:} Alokasi memori runtime untuk ukuran yang tidak diketahui
  \item \textbf{Data structures:} Membangun linked list, tree, graph, dan struktur kompleks
  \item \textbf{Function modification:} Mengubah variabel di luar fungsi
  \item \textbf{Array handling:} Pointer dan array sangat terkait erat
  \item \textbf{Low-level access:} Akses langsung ke hardware dan memory-mapped I/O
\end{itemize}

\subsection{Visualisasi Memori dan Pointer}

\begin{figure}[H]
  \centering
  \begin{tikzpicture}[node distance=0.5cm]
    \tikzstyle{mem}=[rectangle, draw, minimum width=2cm, minimum height=0.8cm, align=center]
    
    % Variable
    \node[mem] (var) {5};
    \node[left=0.1cm of var] {var:};
    \node[right=0.1cm of var, anchor=west] (addr1) {\texttt{0x1000}};
    
    % Pointer
    \node[mem, below=of var] (ptr) {\texttt{0x1000}};
    \node[left=0.1cm of ptr] {ptr:};
    \node[right=0.1cm of ptr, anchor=west] (addr2) {\texttt{0x2000}};
    
    % Arrow from pointer to variable
    \draw[->, thick, blue] (ptr.north) to[out=90,in=0] (var.east);
  \end{tikzpicture}
  \caption{Pointer menyimpan alamat variabel}
\end{figure}

\subsection{Analogi Pointer}

Bayangkan alamat rumah:
\begin{itemize}
  \item \textbf{Variabel} = Rumah (berisi data)
  \item \textbf{Alamat memori} = Alamat rumah (misalnya "Jl. Merdeka No. 10")
  \item \textbf{Pointer} = Kertas yang menulis alamat rumah
  \item \textbf{Dereferensi} = Pergi ke alamat tersebut dan masuk ke rumah
\end{itemize}

\section{Deklarasi dan Inisialisasi Pointer}

\subsection{Pointer di Pascal}

\begin{lstlisting}[language=Pascal, caption={Deklarasi pointer di Pascal}]
var
  x: integer;
  p: ^integer;        // Pointer to integer
  
  str: string;
  pStr: ^string;      // Pointer to string
  
  arr: array[1..10] of real;
  pArr: ^real;        // Pointer to real
  
  pNil: ^integer;     // Uninitialized pointer

begin
  x := 42;
  p := @x;            // @ adalah address-of operator
  
  // Dereferencing menggunakan ^
  Writeln('Nilai x: ', x);           // 42
  Writeln('Nilai via pointer: ', p^); // 42
  
  // Mengubah nilai melalui pointer
  p^ := 100;
  Writeln('Nilai x sekarang: ', x);  // 100
  
  // Nil pointer (equivalent to NULL)
  pNil := nil;
  if pNil = nil then
    Writeln('Pointer is nil');
end.
\end{lstlisting}

\subsection{Pointer di C}

\begin{lstlisting}[language=C, caption={Deklarasi pointer di C}]
#include <stdio.h>

int main() {
  int x = 42;
  int *p;             // Pointer to int (uninitialized!)
  int *q = NULL;      // Null pointer (good practice)
  
  // Inisialisasi pointer dengan address
  p = &x;             // & adalah address-of operator
  
  // Dereferencing menggunakan *
  printf("Nilai x: %d\n", x);          // 42
  printf("Alamat x: %p\n", (void*)&x); // Address
  printf("Nilai p: %p\n", (void*)p);   // Same address
  printf("Nilai via pointer: %d\n", *p); // 42
  
  // Mengubah nilai melalui pointer
  *p = 100;
  printf("Nilai x sekarang: %d\n", x);  // 100
  
  // Multiple pointers to same variable
  int *r = &x;
  *r = 200;
  printf("x = %d, *p = %d, *r = %d\n", x, *p, *r);
  
  // NULL pointer check
  if (q == NULL) {
    printf("q is NULL\n");
  }
  
  return 0;
}
\end{lstlisting}

\textbf{Operator penting:}
\begin{itemize}
  \item \texttt{\&} --- address-of operator: mengambil alamat variabel
  \item \texttt{*} --- dereference operator: mengakses nilai yang ditunjuk
  \item \texttt{*} dalam deklarasi: menandakan tipe pointer
\end{itemize}

\subsection{Pointer di C++}

\begin{lstlisting}[language=C++, caption={Pointer dan reference di C++}]
#include <iostream>

int main() {
  int x = 42;
  
  // Raw pointer (seperti C)
  int *p = &x;
  std::cout << "Nilai x: " << x << "\n";
  std::cout << "Nilai via pointer: " << *p << "\n";
  
  // Reference (alias untuk variabel)
  int &ref = x;  // HARUS diinisialisasi saat deklarasi
  std::cout << "Nilai via reference: " << ref << "\n";
  
  // Mengubah nilai
  *p = 100;
  std::cout << "x = " << x << "\n";  // 100
  
  ref = 200;
  std::cout << "x = " << x << "\n";  // 200
  
  // Nullptr (C++11, lebih aman dari NULL)
  int *q = nullptr;
  if (q == nullptr) {
    std::cout << "q is nullptr\n";
  }
  
  // Const pointer dan pointer to const
  const int *pc = &x;       // Pointer to const int
  // *pc = 5;               // ERROR! Tidak bisa ubah nilai
  pc = &x;                  // OK, bisa ubah pointer
  
  int * const cp = &x;      // Const pointer to int
  *cp = 5;                  // OK, bisa ubah nilai
  // cp = &x;               // ERROR! Tidak bisa ubah pointer
  
  const int * const cpc = &x;  // Const pointer to const int
  // *cpc = 5;             // ERROR!
  // cpc = &x;             // ERROR!
  
  return 0;
}
\end{lstlisting}

\textbf{Reference vs Pointer di C++:}
\begin{itemize}
  \item Reference harus diinisialisasi, pointer boleh NULL/nullptr
  \item Reference tidak bisa di-reassign, pointer bisa
  \item Reference lebih aman (tidak ada NULL reference)
  \item Syntax reference lebih bersih (tidak perlu \texttt{*} dan \texttt{\&})
\end{itemize}

\section{Pointer Arithmetic}

Pointer arithmetic memungkinkan navigasi melalui memori menggunakan operasi aritmatika.

\subsection{Operasi Pointer di C}

\begin{lstlisting}[language=C, caption={Pointer arithmetic di C}]
#include <stdio.h>

int main() {
  int arr[] = {10, 20, 30, 40, 50};
  int *p = arr;  // arr is pointer to first element
  
  // Akses elemen dengan pointer arithmetic
  printf("*p = %d\n", *p);       // 10
  printf("*(p+1) = %d\n", *(p+1)); // 20
  printf("*(p+2) = %d\n", *(p+2)); // 30
  
  // Increment pointer
  p++;  // Pindah ke elemen berikutnya
  printf("*p setelah p++: %d\n", *p);  // 20
  
  // Decrement pointer
  p--;
  printf("*p setelah p--: %d\n", *p);  // 10
  
  // Pointer subtraction
  int *p1 = &arr[1];
  int *p2 = &arr[4];
  printf("Jarak p2 - p1: %ld\n", p2 - p1);  // 3
  
  // Comparison
  if (p < p1) {
    printf("p sebelum p1\n");
  }
  
  // Array notation vs pointer notation
  for (int i = 0; i < 5; i++) {
    printf("arr[%d] = %d, *(arr+%d) = %d\n", 
           i, arr[i], i, *(arr+i));
  }
  
  return 0;
}
\end{lstlisting}

\textbf{Operasi yang valid:}
\begin{itemize}
  \item \texttt{ptr + n}, \texttt{ptr - n}: geser pointer
  \item \texttt{ptr++}, \texttt{ptr--}: increment/decrement
  \item \texttt{ptr2 - ptr1}: jarak antar pointer (dalam elemen)
  \item \texttt{ptr1 < ptr2}, \texttt{ptr1 == ptr2}: comparison
\end{itemize}

\textbf{Catatan penting:}
\begin{itemize}
  \item Pointer increment: \texttt{ptr + 1} menambah \texttt{sizeof(type)} bytes
  \item \texttt{int} 4 bytes: \texttt{ptr + 1} skip 4 bytes
  \item \texttt{double} 8 bytes: \texttt{ptr + 1} skip 8 bytes
  \item Jangan lakukan arithmetic pada pointer yang tidak menunjuk array!
\end{itemize}

\section{Pointer dan Array}

Array dan pointer sangat terkait erat dalam C/C++.

\subsection{Hubungan Array dan Pointer}

\begin{lstlisting}[language=C, caption={Array dan pointer di C}]
#include <stdio.h>

int main() {
  int arr[5] = {10, 20, 30, 40, 50};
  
  // arr adalah pointer to first element
  printf("arr = %p\n", (void*)arr);
  printf("&arr[0] = %p\n", (void*)&arr[0]);
  // arr == &arr[0]
  
  // Array notation vs pointer notation
  printf("arr[0] = %d\n", arr[0]);     // 10
  printf("*arr = %d\n", *arr);         // 10
  printf("arr[2] = %d\n", arr[2]);     // 30
  printf("*(arr+2) = %d\n", *(arr+2)); // 30
  
  // Pointer dapat digunakan seperti array
  int *p = arr;
  printf("p[0] = %d\n", p[0]);         // 10
  printf("p[2] = %d\n", p[2]);         // 30
  
  // Iterasi dengan pointer
  printf("Iterasi dengan pointer:\n");
  for (int *ptr = arr; ptr < arr + 5; ptr++) {
    printf("%d ", *ptr);
  }
  printf("\n");
  
  return 0;
}
\end{lstlisting}

\textbf{Perbedaan array dan pointer:}
\begin{itemize}
  \item \texttt{sizeof(arr)} memberikan ukuran seluruh array
  \item \texttt{sizeof(ptr)} memberikan ukuran pointer (8 bytes di 64-bit)
  \item Array name adalah constant pointer (tidak bisa di-reassign)
  \item Pointer variable bisa di-reassign
\end{itemize}

\subsection{Passing Array ke Fungsi}

\begin{lstlisting}[language=C, caption={Array sebagai parameter fungsi}]
#include <stdio.h>

// Array decay to pointer
void printArray1(int arr[], int size) {
  for (int i = 0; i < size; i++) {
    printf("%d ", arr[i]);
  }
  printf("\n");
}

// Equivalent: explicit pointer
void printArray2(int *arr, int size) {
  for (int i = 0; i < size; i++) {
    printf("%d ", *(arr + i));
  }
  printf("\n");
}

// Modify array through pointer
void doubleValues(int *arr, int size) {
  for (int i = 0; i < size; i++) {
    arr[i] *= 2;
  }
}

int main() {
  int numbers[] = {1, 2, 3, 4, 5};
  int size = sizeof(numbers) / sizeof(numbers[0]);
  
  printArray1(numbers, size);
  printArray2(numbers, size);
  
  doubleValues(numbers, size);
  printf("After doubling: ");
  printArray1(numbers, size);
  
  return 0;
}
\end{lstlisting}

\section{Pointer ke Struct}

Pointer sangat berguna untuk bekerja dengan struct, terutama untuk struktur data dinamis.

\subsection{Pointer ke Struct di C}

\begin{lstlisting}[language=C, caption={Pointer to struct di C}]
#include <stdio.h>
#include <string.h>

typedef struct {
  char nama[50];
  int umur;
  float ipk;
} Mahasiswa;

void printMahasiswa(const Mahasiswa *m) {
  // Akses dengan ->
  printf("Nama: %s\n", m->nama);
  printf("Umur: %d\n", m->umur);
  printf("IPK: %.2f\n", m->ipk);
  
  // Equivalent dengan (*m).field
  printf("IPK (syntax alt): %.2f\n", (*m).ipk);
}

void updateIPK(Mahasiswa *m, float ipkBaru) {
  m->ipk = (m->ipk + ipkBaru) / 2.0;
}

int main() {
  Mahasiswa mhs = {"John Doe", 20, 3.5};
  Mahasiswa *ptr = &mhs;
  
  // Akses melalui pointer
  printf("Nama via pointer: %s\n", ptr->nama);
  
  // Modifikasi melalui pointer
  ptr->umur = 21;
  ptr->ipk = 3.7;
  
  // Passing pointer ke fungsi
  printMahasiswa(ptr);
  updateIPK(ptr, 3.9);
  printMahasiswa(ptr);
  
  return 0;
}
\end{lstlisting}

\textbf{Operator \texttt{->}:}
\begin{itemize}
  \item \texttt{ptr->field} adalah shorthand untuk \texttt{(*ptr).field}
  \item Lebih ringkas dan lebih mudah dibaca
  \item Standard practice dalam C programming
\end{itemize}

\subsection{Array of Pointers}

\begin{lstlisting}[language=C, caption={Array of pointers di C}]
#include <stdio.h>
#include <string.h>

int main() {
  // Array of string pointers
  const char *hari[] = {
    "Senin", "Selasa", "Rabu", "Kamis",
    "Jumat", "Sabtu", "Minggu"
  };
  
  printf("Hari ke-3: %s\n", hari[2]);  // Rabu
  
  // Iterate
  for (int i = 0; i < 7; i++) {
    printf("%d. %s\n", i+1, hari[i]);
  }
  
  // Array of int pointers
  int a = 10, b = 20, c = 30;
  int *arr[] = {&a, &b, &c};
  
  printf("Nilai via array of pointers:\n");
  for (int i = 0; i < 3; i++) {
    printf("*arr[%d] = %d\n", i, *arr[i]);
  }
  
  // Modify through pointer
  *arr[0] = 100;
  printf("a = %d\n", a);  // 100
  
  return 0;
}
\end{lstlisting}

\section{Dynamic Memory Allocation}

Alokasi memori dinamis memungkinkan program mengalokasikan memori saat runtime.

\subsection{Alokasi Dinamis di Pascal}

\begin{lstlisting}[language=Pascal, caption={New dan Dispose di Pascal}]
type
  PInteger = ^integer;
  
  PMahasiswa = ^TMahasiswa;
  TMahasiswa = record
    nama: string;
    nim: string;
    ipk: real;
  end;

var
  p: PInteger;
  mhs: PMahasiswa;

begin
  // Alokasi memori
  New(p);
  p^ := 42;
  Writeln('Nilai: ', p^);
  
  // Dealokasi
  Dispose(p);
  
  // Alokasi untuk struct
  New(mhs);
  mhs^.nama := 'John Doe';
  mhs^.nim := '123';
  mhs^.ipk := 3.75;
  
  Writeln('Mahasiswa: ', mhs^.nama);
  
  // Jangan lupa dealokasi!
  Dispose(mhs);
end.
\end{lstlisting}

\subsection{Alokasi Dinamis di C}

\begin{lstlisting}[language=C, caption={malloc, calloc, realloc, free di C}]
#include <stdio.h>
#include <stdlib.h>
#include <string.h>

int main() {
  // malloc - allocate uninitialized memory
  int *p = (int*)malloc(sizeof(int));
  if (p == NULL) {
    printf("Alokasi gagal!\n");
    return 1;
  }
  *p = 42;
  printf("*p = %d\n", *p);
  free(p);  // PENTING: dealokasi
  
  // calloc - allocate zero-initialized memory
  int *arr = (int*)calloc(5, sizeof(int));
  if (arr == NULL) {
    printf("Alokasi gagal!\n");
    return 1;
  }
  for (int i = 0; i < 5; i++) {
    printf("arr[%d] = %d\n", i, arr[i]);  // Semua 0
  }
  
  // realloc - resize memory
  arr = (int*)realloc(arr, 10 * sizeof(int));
  if (arr == NULL) {
    printf("Realloc gagal!\n");
    return 1;
  }
  for (int i = 5; i < 10; i++) {
    arr[i] = i;
  }
  
  free(arr);
  
  // Alokasi struct
  typedef struct {
    char nama[50];
    int umur;
  } Person;
  
  Person *person = (Person*)malloc(sizeof(Person));
  if (person != NULL) {
    strcpy(person->nama, "John");
    person->umur = 25;
    printf("Person: %s, %d\n", person->nama, person->umur);
    free(person);
  }
  
  return 0;
}
\end{lstlisting}

\textbf{Fungsi alokasi memori:}
\begin{itemize}
  \item \texttt{malloc(size)}: alokasi \texttt{size} bytes (uninitialized)
  \item \texttt{calloc(n, size)}: alokasi \texttt{n*size} bytes (zero-initialized)
  \item \texttt{realloc(ptr, size)}: resize memori yang sudah dialokasi
  \item \texttt{free(ptr)}: dealokasi memori (WAJIB!)
\end{itemize}

\textbf{Aturan penting:}
\begin{itemize}
  \item Selalu cek return value (NULL = gagal)
  \item Setiap \texttt{malloc}/\texttt{calloc} harus paired dengan \texttt{free}
  \item Jangan akses memory setelah \texttt{free} (dangling pointer)
  \item Jangan \texttt{free} dua kali (double free)
  \item Jangan \texttt{free} pointer yang tidak dari \texttt{malloc/calloc}
\end{itemize}

\subsection{Smart Pointers di C++}

\begin{lstlisting}[language=C++, caption={Smart pointers di C++}]
#include <iostream>
#include <memory>
#include <string>

struct Person {
  std::string nama;
  int umur;
  
  Person(const std::string& n, int u) : nama(n), umur(u) {
    std::cout << "Person " << nama << " created\n";
  }
  
  ~Person() {
    std::cout << "Person " << nama << " destroyed\n";
  }
};

int main() {
  // unique_ptr - exclusive ownership
  {
    std::unique_ptr<int> p1 = std::make_unique<int>(42);
    std::cout << "*p1 = " << *p1 << "\n";
    
    // std::unique_ptr<int> p2 = p1;  // ERROR! Tidak bisa copy
    std::unique_ptr<int> p2 = std::move(p1);  // OK, transfer ownership
    // p1 is now nullptr
    
    std::unique_ptr<Person> person = 
      std::make_unique<Person>("John", 25);
    std::cout << person->nama << ", " << person->umur << "\n";
  }  // Automatic deallocation!
  
  // shared_ptr - shared ownership
  {
    std::shared_ptr<Person> p1 = 
      std::make_shared<Person>("Jane", 30);
    std::cout << "Use count: " << p1.use_count() << "\n";  // 1
    
    {
      std::shared_ptr<Person> p2 = p1;  // Copy OK
      std::cout << "Use count: " << p1.use_count() << "\n";  // 2
      // Both p1 and p2 point to same Person
    }  // p2 destroyed, count decreases
    
    std::cout << "Use count: " << p1.use_count() << "\n";  // 1
  }  // p1 destroyed, count = 0, Person deallocated
  
  // weak_ptr - non-owning observer
  std::weak_ptr<Person> weak;
  {
    std::shared_ptr<Person> shared = 
      std::make_shared<Person>("Bob", 35);
    weak = shared;
    
    if (auto locked = weak.lock()) {
      std::cout << "Observed: " << locked->nama << "\n";
    }
  }  // shared destroyed, Person deallocated
  
  if (weak.expired()) {
    std::cout << "weak pointer expired\n";
  }
  
  return 0;
}
\end{lstlisting}

\textbf{Smart pointer types:}
\begin{itemize}
  \item \texttt{unique\_ptr}: exclusive ownership, tidak bisa copy, otomatis dealokasi
  \item \texttt{shared\_ptr}: shared ownership dengan reference counting
  \item \texttt{weak\_ptr}: non-owning observer untuk mencegah circular reference
\end{itemize}

\textbf{Kapan menggunakan:}
\begin{itemize}
  \item Prefer \texttt{unique\_ptr} untuk single ownership
  \item Gunakan \texttt{shared\_ptr} jika ownership shared diperlukan
  \item Gunakan \texttt{weak\_ptr} untuk break circular dependencies
  \item Avoid raw pointers di modern C++ kecuali untuk observing
\end{itemize}

\section{Pointer to Pointer}

Pointer dapat menunjuk ke pointer lain, membentuk multiple levels of indirection.

\subsection{Double Pointer di C}

\begin{lstlisting}[language=C, caption={Pointer to pointer di C}]
#include <stdio.h>

void changeValue(int **pp, int newValue) {
  **pp = newValue;  // Mengubah nilai yang ditunjuk
}

void changePointer(int **pp, int *newTarget) {
  *pp = newTarget;  // Mengubah ke mana pointer menunjuk
}

int main() {
  int x = 10, y = 20;
  int *p = &x;
  int **pp = &p;  // Pointer to pointer
  
  printf("x = %d\n", x);         // 10
  printf("*p = %d\n", *p);       // 10
  printf("**pp = %d\n", **pp);   // 10
  
  // Mengubah nilai melalui double pointer
  **pp = 30;
  printf("x = %d\n", x);         // 30
  
  // Mengubah pointer melalui double pointer
  *pp = &y;
  printf("*p = %d\n", *p);       // 20 (now points to y)
  
  // Melalui fungsi
  p = &x;
  changeValue(&p, 100);
  printf("x = %d\n", x);         // 100
  
  changePointer(&p, &y);
  printf("*p = %d\n", *p);       // 20
  
  return 0;
}
\end{lstlisting}

\subsection{Aplikasi Double Pointer}

\begin{lstlisting}[language=C, caption={Double pointer untuk 2D array}]
#include <stdio.h>
#include <stdlib.h>

// Alokasi 2D array dinamis
int** allocate2DArray(int rows, int cols) {
  int **arr = (int**)malloc(rows * sizeof(int*));
  for (int i = 0; i < rows; i++) {
    arr[i] = (int*)malloc(cols * sizeof(int));
  }
  return arr;
}

void free2DArray(int **arr, int rows) {
  for (int i = 0; i < rows; i++) {
    free(arr[i]);
  }
  free(arr);
}

int main() {
  int rows = 3, cols = 4;
  int **matrix = allocate2DArray(rows, cols);
  
  // Isi matrix
  int value = 1;
  for (int i = 0; i < rows; i++) {
    for (int j = 0; j < cols; j++) {
      matrix[i][j] = value++;
    }
  }
  
  // Print matrix
  for (int i = 0; i < rows; i++) {
    for (int j = 0; j < cols; j++) {
      printf("%3d ", matrix[i][j]);
    }
    printf("\n");
  }
  
  free2DArray(matrix, rows);
  return 0;
}
\end{lstlisting}

\section{Linked List - Aplikasi Pointer}

Linked list adalah struktur data dinamis yang menggunakan pointer untuk menghubungkan node.

\subsection{Singly Linked List di C}

\begin{lstlisting}[language=C, caption={Linked list implementation di C}]
#include <stdio.h>
#include <stdlib.h>

typedef struct Node {
  int data;
  struct Node *next;
} Node;

// Create new node
Node* createNode(int data) {
  Node *newNode = (Node*)malloc(sizeof(Node));
  if (newNode != NULL) {
    newNode->data = data;
    newNode->next = NULL;
  }
  return newNode;
}

// Insert at beginning
void pushFront(Node **head, int data) {
  Node *newNode = createNode(data);
  if (newNode != NULL) {
    newNode->next = *head;
    *head = newNode;
  }
}

// Insert at end
void pushBack(Node **head, int data) {
  Node *newNode = createNode(data);
  if (newNode == NULL) return;
  
  if (*head == NULL) {
    *head = newNode;
    return;
  }
  
  Node *current = *head;
  while (current->next != NULL) {
    current = current->next;
  }
  current->next = newNode;
}

// Print list
void printList(Node *head) {
  Node *current = head;
  printf("List: ");
  while (current != NULL) {
    printf("%d ", current->data);
    current = current->next;
  }
  printf("\n");
}

// Free list
void freeList(Node **head) {
  Node *current = *head;
  while (current != NULL) {
    Node *next = current->next;
    free(current);
    current = next;
  }
  *head = NULL;
}

// Delete node with value
void deleteValue(Node **head, int value) {
  Node *current = *head;
  Node *prev = NULL;
  
  // If head needs to be deleted
  if (current != NULL && current->data == value) {
    *head = current->next;
    free(current);
    return;
  }
  
  // Search for the node
  while (current != NULL && current->data != value) {
    prev = current;
    current = current->next;
  }
  
  // If not found
  if (current == NULL) return;
  
  // Delete node
  prev->next = current->next;
  free(current);
}

int main() {
  Node *head = NULL;
  
  // Build list
  pushFront(&head, 3);
  pushFront(&head, 2);
  pushFront(&head, 1);
  pushBack(&head, 4);
  pushBack(&head, 5);
  
  printList(head);  // 1 2 3 4 5
  
  // Delete a value
  deleteValue(&head, 3);
  printList(head);  // 1 2 4 5
  
  // Clean up
  freeList(&head);
  
  return 0;
}
\end{lstlisting}

\subsection{Linked List di C++}

\begin{lstlisting}[language=C++, caption={Linked list dengan smart pointers di C++}]
#include <iostream>
#include <memory>

struct Node {
  int data;
  std::unique_ptr<Node> next;
  
  Node(int d) : data(d), next(nullptr) {}
};

class LinkedList {
private:
  std::unique_ptr<Node> head;
  
public:
  LinkedList() : head(nullptr) {}
  
  void pushFront(int data) {
    auto newNode = std::make_unique<Node>(data);
    newNode->next = std::move(head);
    head = std::move(newNode);
  }
  
  void pushBack(int data) {
    auto newNode = std::make_unique<Node>(data);
    
    if (!head) {
      head = std::move(newNode);
      return;
    }
    
    Node *current = head.get();
    while (current->next) {
      current = current->next.get();
    }
    current->next = std::move(newNode);
  }
  
  void print() const {
    std::cout << "List: ";
    Node *current = head.get();
    while (current) {
      std::cout << current->data << " ";
      current = current->next.get();
    }
    std::cout << "\n";
  }
  
  void deleteValue(int value) {
    if (!head) return;
    
    // If head needs to be deleted
    if (head->data == value) {
      head = std::move(head->next);
      return;
    }
    
    Node *current = head.get();
    while (current->next && current->next->data != value) {
      current = current->next.get();
    }
    
    if (current->next) {
      current->next = std::move(current->next->next);
    }
  }
  
  // Destructor automatic with unique_ptr!
};

int main() {
  LinkedList list;
  
  list.pushFront(3);
  list.pushFront(2);
  list.pushFront(1);
  list.pushBack(4);
  list.pushBack(5);
  
  list.print();  // 1 2 3 4 5
  
  list.deleteValue(3);
  list.print();  // 1 2 4 5
  
  // No manual cleanup needed!
  return 0;
}
\end{lstlisting}

\section{Common Pointer Errors}

\subsection{Dangling Pointer}

Pointer yang menunjuk ke memori yang sudah didealokasi.

\begin{lstlisting}[language=C, caption={Dangling pointer problem}]
int* createDanglingPointer() {
  int x = 42;
  return &x;  // BAD! x destroyed after return
}

int main() {
  int *p = createDanglingPointer();
  // *p is undefined behavior!
  
  // Correct version
  int *q = (int*)malloc(sizeof(int));
  *q = 42;
  // Use q...
  free(q);
  q = NULL;  // Good practice to avoid dangling pointer
  
  return 0;
}
\end{lstlisting}

\subsection{Memory Leak}

Alokasi memori yang tidak pernah didealokasi.

\begin{lstlisting}[language=C, caption={Memory leak}]
void memoryLeak() {
  int *p = (int*)malloc(sizeof(int));
  *p = 42;
  // Forgot to free(p)!
}  // Memory leak!

void correct() {
  int *p = (int*)malloc(sizeof(int));
  *p = 42;
  free(p);  // Properly deallocated
}
\end{lstlisting}

\subsection{Null Pointer Dereference}

Dereferensi pointer NULL/nullptr.

\begin{lstlisting}[language=C, caption={Null pointer dereference}]
int main() {
  int *p = NULL;
  // *p = 42;  // CRASH! Null pointer dereference
  
  // Always check before dereference
  if (p != NULL) {
    *p = 42;
  }
  
  return 0;
}
\end{lstlisting}

\subsection{Wild Pointer}

Pointer yang tidak diinisialisasi.

\begin{lstlisting}[language=C, caption={Wild pointer}]
int main() {
  int *p;  // Uninitialized! Contains garbage address
  // *p = 42;  // DANGEROUS! Writing to random memory
  
  // Always initialize
  int *q = NULL;
  // Or
  int x;
  int *r = &x;
  
  return 0;
}
\end{lstlisting}

\section{Best Practices}

\subsection{Pointer Safety Guidelines}

\begin{enumerate}
  \item \textbf{Initialize pointers:} Selalu inisialisasi ke NULL/nullptr atau valid address
  \item \textbf{Check before dereference:} Validasi pointer bukan NULL sebelum akses
  \item \textbf{Free memory:} Setiap malloc harus paired dengan free
  \item \textbf{Set to NULL after free:} Hindari dangling pointer
  \item \textbf{Avoid returning local address:} Jangan return alamat variabel lokal
  \item \textbf{Use const:} Gunakan const untuk pointer read-only
  \item \textbf{Document ownership:} Jelaskan siapa yang bertanggung jawab free
\end{enumerate}

\subsection{Modern C++ Recommendations}

\begin{itemize}
  \item Prefer smart pointers (\texttt{unique\_ptr}, \texttt{shared\_ptr}) over raw pointers
  \item Use references untuk parameter pass-by-reference
  \item Use \texttt{nullptr} bukan \texttt{NULL}
  \item Avoid manual \texttt{new}/\texttt{delete}, use smart pointers
  \item Use containers (\texttt{vector}, \texttt{list}) bukan manual array
  \item Follow RAII principle
\end{itemize}

\subsection{Debugging Tools}

\begin{itemize}
  \item \textbf{Valgrind:} Detect memory leaks dan invalid access (Linux)
  \item \textbf{AddressSanitizer:} Compiler-based memory error detection
  \item \textbf{Visual Studio Debugger:} Memory windows dan breakpoints
  \item \textbf{GDB:} Command-line debugger dengan pointer inspection
  \item \textbf{Static analyzers:} Cppcheck, Clang-Tidy untuk deteksi dini
\end{itemize}

\section{Rangkuman}

Pointer adalah konsep fundamental yang memberikan kontrol langsung atas memori dan memungkinkan struktur data dinamis.

\subsection{Poin-Poin Penting}

\begin{itemize}
  \item Pointer menyimpan alamat memori variabel lain
  \item Operator \texttt{\&} mengambil alamat, \texttt{*} untuk dereferensi
  \item Pointer arithmetic untuk navigasi memori
  \item Array dan pointer sangat terkait (array name adalah pointer)
  \item Dynamic memory allocation dengan malloc/new, dealokasi dengan free/delete
  \item Pointer ke struct menggunakan operator \texttt{->}
  \item Smart pointers di C++ untuk manajemen memori otomatis
  \item Hati-hati dengan dangling pointer, memory leak, null dereference
\end{itemize}

\subsection{Perbandingan Antar Bahasa}

\begin{table}[H]
\centering
\begin{tabular}{|l|c|c|c|}
\hline
\textbf{Aspek} & \textbf{Pascal} & \textbf{C} & \textbf{C++} \\
\hline
Declaration & \texttt{\^{}type} & \texttt{type*} & \texttt{type*} \\
\hline
Address-of & \texttt{@} & \texttt{\&} & \texttt{\&} \\
\hline
Dereference & \texttt{\^{}} & \texttt{*} & \texttt{*} \\
\hline
NULL value & \texttt{nil} & \texttt{NULL} & \texttt{nullptr} \\
\hline
Allocation & \texttt{New} & \texttt{malloc} & \texttt{new}/smart ptrs \\
\hline
Deallocation & \texttt{Dispose} & \texttt{free} & \texttt{delete}/automatic \\
\hline
Reference & Tidak & Tidak & Ya (\texttt{\&}) \\
\hline
Smart pointers & Tidak & Tidak & Ya \\
\hline
\end{tabular}
\caption{Perbandingan pointer di Pascal, C, dan C++}
\end{table}

\subsection{Kapan Menggunakan Pointer}

\textbf{Gunakan pointer untuk:}
\begin{itemize}
  \item Dynamic memory allocation
  \item Building data structures (linked list, tree, graph)
  \item Efficient passing of large objects
  \item Implementing callbacks dan function pointers
  \item Low-level system programming
\end{itemize}

\textbf{Hindari pointer bila:}
\begin{itemize}
  \item Bisa menggunakan reference (C++)
  \item Bisa menggunakan value semantics
  \item Bisa menggunakan container standar (vector, list)
  \item Tidak perlu dynamic lifetime
\end{itemize}

Pointer adalah tool yang powerful namun memerlukan kehati-hatian. Dengan pemahaman yang baik tentang konsep dan best practices, Anda dapat menggunakan pointer dengan aman dan efektif untuk membangun program yang efisien dan fleksibel!

\end{document}

