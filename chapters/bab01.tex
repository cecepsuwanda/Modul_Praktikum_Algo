\documentclass[../main.tex]{subfiles}
\begin{document}
\chapter{Pengantar \& Persiapan Lingkungan}
\section{Sejarah dan Posisi Pascal, C, C++}
Pascal, C, dan C++ merupakan bahasa pemrograman prosedural dan sistem yang membentuk fondasi banyak perangkat lunak modern. Pascal diperkenalkan untuk pendidikan dengan penekanan pada struktur dan keterbacaan, sedangkan C dirancang untuk pemrograman sistem dengan kontrol memori yang presisi. C++ berevolusi dari C dengan menambahkan paradigma berorientasi objek dan generik untuk membangun sistem berskala besar. Lihat rangkuman historis dan konsep pada \textcite{pascal-tutorial-wikibooks,gnu-c-manual,cpp-reference}.

Ketiga bahasa ini menempati posisi strategis dalam kurikulum Algoritma dan Pemrograman karena keseimbangan antara kedekatan ke perangkat keras dan kekayaan abstraksi. Mahasiswa yang memahami idiom dan batasan masing-masing bahasa akan lebih siap mempelajari struktur data, algoritme, dan rekayasa perangkat lunak lanjutan. Fokus praktikum diarahkan pada kompetensi menulis program yang benar, efisien, dan mudah diuji di tiga ekosistem tersebut.

Di ekosistem industri, C tetap dominan untuk firmware dan sistem tertanam, sedangkan C++ mendominasi perangkat lunak performa tinggi seperti gim dan mesin basis data. Pascal melalui Free Pascal/Lazarus masih relevan untuk pembelajaran dan aplikasi desktop tertentu. Referensi terbuka menyediakan dokumentasi resmi dan contoh yang konsisten untuk eksperimen \parencite{free-pascal-docs,gnu-c-manual,cpp-reference}.

\section{Menyiapkan Lingkungan Pengembangan}
Lingkungan pengembangan minimal meliputi compiler, editor/IDE, dan alat pembangunan (build tools). Untuk Pascal, gunakan Free Pascal Compiler (FPC) dengan atau tanpa Lazarus IDE; untuk C/C++, gunakan GCC atau Clang yang tersedia pada Linux, macOS, dan Windows melalui distribusi seperti MSYS2 atau WSL. Editor modern seperti Visual Studio Code menyediakan ekstensi linting, debugging, dan integrasi kompilasi lintas platform.

Pastikan variabel lingkungan telah dikonfigurasi sehingga perintah \texttt{fpc}, \texttt{gcc}, dan \texttt{g++} dapat diakses dari terminal. Susun struktur proyek sederhana dengan direktori untuk kode sumber, berkas header/unit, dan artefak build agar proses kompilasi dan pembersihan konsisten. Dokumentasi resmi memberikan petunjuk instalasi, opsi kompilasi, dan praktik terbaik lintas sistem operasi \parencite{free-pascal-docs,gnu-c-manual,cpp-reference}.

Pengaturan kompilasi yang eksplisit sangat dianjurkan untuk reproducibility, misalnya menentukan standar bahasa (\texttt{-std=c11} atau \texttt{-std=c++17}) dan level peringatan (\texttt{-Wall -Wextra}). Pada tahap awal, gunakan kompilasi satu berkas untuk mempercepat siklus umpan balik; selanjutnya, pelajari sistem build seperti \texttt{make} atau CMake untuk proyek multi-berkas. Pendekatan bertahap ini menjaga fokus pada konsep algoritmik tanpa mengorbankan kualitas teknis lingkungan kerja.

\section{Hello, World! — Program Pertama}
Program pertama berfungsi memverifikasi instalasi compiler dan alur kompilasi-\emph{run}. Contoh di bawah memperlihatkan cara menghasilkan keluaran teks sederhana pada ketiga bahasa. Perhatikan perbedaan model I/O: Pascal menggunakan prosedur \texttt{Writeln}, C menggunakan \texttt{printf}, dan C++ menggunakan \texttt{std::cout} dengan operator \texttt{<<} \parencite{w3pascal-io,gnu-c-manual,cpp-reference}.

\begin{lstlisting}[language=Pascal, caption={Hello World pada Pascal}]
program Hello;
begin
  Writeln('Hello, World!');
end.
\end{lstlisting}

\begin{lstlisting}[language=C, caption={Hello World pada C}]
#include <stdio.h>
int main(void) {
  printf("Hello, World!\n");
  return 0;
}
\end{lstlisting}

\begin{lstlisting}[language=C++, caption={Hello World pada C++}]
#include <iostream>
int main() {
  std::cout << "Hello, World!\n";
}
\end{lstlisting}

Uji setiap program dengan cara kompilasi dan eksekusi pada terminal untuk memastikan seluruh rantai alat berfungsi. Jika terjadi kegagalan, baca pesan kesalahan kompilator secara teliti karena sering kali menunjukkan lokasi dan jenis masalah. Dokumentasi referensi membantu memetakan fungsi I/O dasar dan konvensi lingkungan eksekusi \parencite{w3pascal-io,gnu-c-manual,cpp-reference}.
\end{document}
