\documentclass[../main.tex]{subfiles}
\begin{document}
\chapter{Pengantar \& Persiapan Lingkungan}
\section{Sejarah dan Posisi Pascal, C, C++}
Pascal, C, dan C++ adalah bahasa dasar yang banyak dipakai untuk belajar algoritma dan membangun perangkat lunak performa tinggi. Pascal awalnya dibuat untuk pendidikan dengan sintaks yang jelas; C dipakai luas untuk pemrograman sistem karena kontrol memori yang ketat; C++ menambah konsep berorientasi objek dan generik di atas C untuk proyek besar \parencite{pascal-tutorial-wikibooks,gnu-c-manual,cpp-reference}.

Ketiganya penting dalam kurikulum karena memberi keseimbangan antara kedekatan ke perangkat keras dan kemampuan abstraksi. Menguasai ketiga perspektif ini memudahkan mahasiswa melanjutkan ke struktur data, algoritme, dan rekayasa perangkat lunak. Fokus praktikum kita adalah menulis program yang benar, mudah diuji, dan cukup efisien \parencite{free-pascal-docs,gnu-c-manual,cpp-reference}.

Dalam praktik industri, C umum untuk firmware dan sistem tertanam, C++ untuk aplikasi performa tinggi (gim, mesin basis data), sedangkan Pascal (melalui Free Pascal/Lazarus) tetap relevan untuk pembelajaran dan sebagian aplikasi desktop \parencite{free-pascal-docs}.

\section{Menyiapkan Lingkungan Pengembangan}
Lingkungan minimal: kompiler, editor/IDE, dan alat build.
\begin{itemize}
  \item \textbf{Kompiler}: FPC untuk Pascal; GCC/Clang untuk C/C++ \parencite{free-pascal-docs,gcc-docs,clang-docs}.
  \item \textbf{Editor}: Visual Studio Code dengan ekstensi C/C++ untuk linting dan debugging \parencite{vscode-cpp}.
  \item \textbf{Alat build (opsional)}: \texttt{make} atau CMake untuk proyek multi-berkas \parencite{gnu-make,cmake-docs}.
\end{itemize}

\subsection{Instalasi Cepat}
Contoh perintah di Linux/macOS/Windows (sesuaikan distro):
\begin{lstlisting}[language=bash]
# Debian/Ubuntu
sudo apt update
sudo apt install fpc gcc g++ make
\end{lstlisting}
\begin{lstlisting}[language=bash]
# Fedora
sudo dnf install fpc gcc gcc-c++ make
\end{lstlisting}
\begin{lstlisting}[language=bash]
# macOS (Homebrew)
/bin/bash -c "$(curl -fsSL https://raw.githubusercontent.com/Homebrew/install/HEAD/install.sh)"
brew install fpc gcc make cmake
\end{lstlisting}
\begin{lstlisting}[language=bash]
# Windows (Chocolatey; jalankan PowerShell sebagai Administrator)
Set-ExecutionPolicy Bypass -Scope Process -Force
Set-ExecutionPolicy RemoteSigned -Scope CurrentUser -Force
choco install mingw make cmake -y
# Alternatif: MSYS2 (kompiler & tools)
# https://www.msys2.org/
\end{lstlisting}
Untuk pengembangan C/C++ di Windows, alternatif lain adalah WSL \parencite{wsl} atau MSYS2 \parencite{msys2}. Untuk Pascal dengan GUI, Lazarus IDE tersedia lintas platform \parencite{lazarus}.

\subsection{Verifikasi dan Struktur Proyek}
Pastikan perintah tersedia:
\begin{lstlisting}[language=bash]
fpc -v
gcc --version
g++ --version
\end{lstlisting}
Gunakan struktur direktori sederhana: \texttt{src/} untuk kode, \texttt{build/} untuk output. Tetapkan standar bahasa dan peringatan saat kompilasi (mis. \texttt{-std=c11}, \texttt{-std=c++17}, \texttt{-Wall -Wextra}) \parencite{gnu-c-manual,cpp-reference}.

\subsection{Arsitektur Kompilasi dan Eksekusi}
\begin{figure}[h]
  \centering
  \begin{tikzpicture}[node distance=1.8cm, >=Stealth]
    \tikzstyle{box}=[rectangle, draw, rounded corners, align=center, minimum width=3.2cm, minimum height=1cm]
    \node[box] (src) {Kode Sumber\\(\texttt{.pas}, \texttt{.c}, \texttt{.cpp})};
    \node[box, right=of src] (compiler) {Kompiler\\FPC / GCC / Clang};
    \node[box, right=of compiler] (exe) {Berkas Eksekusi\\(\texttt{.out}/\texttt{.exe})};
    \node[box, right=of exe] (run) {Proses Berjalan\\Output ke Terminal};
    \draw[->] (src) -- (compiler);
    \draw[->] (compiler) -- node[above]{compile/link} (exe);
    \draw[->] (exe) -- node[above]{run} (run);
  \end{tikzpicture}
  \caption{Alur kompilasi dan eksekusi program}
  \label{fig:compile-run}
\end{figure}

\section{Hello, World! — Program Pertama}
Program pertama memastikan instalasi dan alur kompilasi-pengejalan (\emph{run}) bekerja. Perbedaan I/O dasar: Pascal memakai \texttt{Writeln}, C memakai \texttt{printf}, C++ memakai \texttt{std::cout} \parencite{w3pascal-io,gnu-c-manual,cpp-reference}.

\begin{lstlisting}[language=Pascal, caption={Hello World pada Pascal}]
program Hello;
begin
  Writeln('Hello, World!');
end.
\end{lstlisting}

\begin{lstlisting}[language=C, caption={Hello World pada C}]
#include <stdio.h>
int main(void) {
  printf("Hello, World!\n");
  return 0;
}
\end{lstlisting}

\begin{lstlisting}[language=C++, caption={Hello World pada C++}]
#include <iostream>
int main() {
  std::cout << "Hello, World!\n";
}
\end{lstlisting}

Kompilasi dan jalankan dari terminal:
\begin{lstlisting}[language=bash]
# Pascal
fpc Hello.pas && ./Hello

# C
gcc -std=c11 -Wall -Wextra hello.c -o hello && ./hello

# C++
g++ -std=c++17 -Wall -Wextra hello.cpp -o hello && ./hello
\end{lstlisting}

\subsection{Program Interaktif Sederhana}
\begin{lstlisting}[language=Pascal, caption={Input nama pada Pascal}]
program Greet;
var name: string;
begin
  Write('Nama Anda? '); Readln(name);
  Writeln('Halo, ', name, '!');
end.
\end{lstlisting}

\begin{lstlisting}[language=C, caption={Input nama pada C}]
#include <stdio.h>
int main(void) {
  char name[64];
  printf("Nama Anda? ");
  if (fgets(name, sizeof name, stdin)) {
    printf("Halo, %s", name); // fgets menyertakan newline
  }
  return 0;
}
\end{lstlisting}

\begin{lstlisting}[language=C++, caption={Input nama pada C++}]
#include <iostream>
#include <string>
int main() {
  std::string name;
  std::cout << "Nama Anda? ";
  std::getline(std::cin, name);
  std::cout << "Halo, " << name << "!\n";
}
\end{lstlisting}

\subsection{Ringkasan Peralatan}
\begin{table}[h]
  \centering
  \caption{Ringkasan peralatan dan perintah dasar}
  \begin{tabular}{@{}llll@{}}
    \toprule
    Bahasa & Ekstensi & Kompiler & Perintah contoh \\
    \midrule
    Pascal & \texttt{.pas} & FPC & \texttt{fpc file.pas} \\
    C & \texttt{.c} & GCC/Clang & \texttt{gcc -std=c11 file.c -o app} \\
    C++ & \texttt{.cpp} & G++/Clang++ & \texttt{g++ -std=c++17 file.cpp -o app} \\
    \bottomrule
  \end{tabular}
\end{table}

\subsection{Troubleshooting Cepat}
\begin{itemize}
  \item \textbf{Perintah tidak ditemukan}: pastikan instalasi selesai dan folder \texttt{bin} ada di \texttt{PATH} \parencite{gcc-docs,clang-docs,free-pascal-docs}.
  \item \textbf{Hak eksekusi}: pada Linux/macOS gunakan \texttt{chmod +x app} bila perlu.
  \item \textbf{Linker error}: tambahkan file sumber yang benar, atau opsi standar bahasa yang sesuai.
\end{itemize}
\end{document}
