\documentclass[../main.tex]{subfiles}
\begin{document}
\chapter{Pengantar \& Persiapan Lingkungan}

\section*{Tujuan Praktikum}
Setelah menyelesaikan praktikum ini, mahasiswa diharapkan mampu:
\begin{itemize}
  \item Memahami sejarah dan peran Pascal, C, dan C++ dalam pembelajaran pemrograman
  \item Menginstal dan mengkonfigurasi lingkungan pengembangan (IDE) untuk Pascal, C, dan C++
  \item Menggunakan IDE baik online (OnlineGDB) maupun offline (Lazarus, Code::Blocks)
  \item Menulis, mengkompilasi, dan menjalankan program "Hello World" dalam ketiga bahasa
  \item Memahami perbedaan dasar sintaks input/output antara Pascal, C, dan C++
  \item Memahami alur kompilasi, linking, hingga eksekusi program \parencite{gnu-c-manual,free-pascal-docs}
  \item Mengatasi kesalahan kompilasi/eksekusi umum pada lingkungan praktikum
\end{itemize}

\section{Sejarah dan Posisi Pascal, C, C++}
Pascal, C, dan C++ adalah bahasa dasar yang banyak dipakai untuk belajar algoritma dan membangun perangkat lunak performa tinggi. Pascal awalnya dibuat untuk pendidikan dengan sintaks yang jelas; C dipakai luas untuk pemrograman sistem karena kontrol memori yang ketat; C++ menambah konsep berorientasi objek dan generik di atas C untuk proyek besar \parencite{pascal-tutorial-wikibooks,gnu-c-manual,cpp-reference}.

Ketiganya penting dalam kurikulum karena memberi keseimbangan antara kedekatan ke perangkat keras dan kemampuan abstraksi. Menguasai ketiga perspektif ini memudahkan mahasiswa melanjutkan ke struktur data, algoritme, dan rekayasa perangkat lunak. Fokus praktikum kita adalah menulis program yang benar, mudah diuji, dan cukup efisien \parencite{free-pascal-docs,gnu-c-manual,cpp-reference}.

Dalam praktik industri, C umum untuk firmware dan sistem tertanam, C++ untuk aplikasi performa tinggi (gim, mesin basis data), sedangkan Pascal (melalui Free Pascal/Lazarus) tetap relevan untuk pembelajaran dan sebagian aplikasi desktop \parencite{free-pascal-docs}.

\section{Lingkungan Pengembangan}
Untuk praktikum ini, kita akan menggunakan lingkungan pengembangan terintegrasi (IDE) yang menyediakan editor kode, kompiler, debugger, dan alat bantu lainnya dalam satu antarmuka. IDE mempermudah proses belajar karena mahasiswa tidak perlu mengatur kompiler secara manual via terminal.

Berikut penjelasan lengkap untuk masing-masing IDE yang direkomendasikan. Mahasiswa dapat memilih salah satu sesuai kebutuhan dan preferensi.

\subsection{Lingkungan Online}
\textbf{OnlineGDB} \parencite{onlinegdb} — platform berbasis web untuk menjalankan Pascal, C, dan C++ tanpa instalasi. Cocok untuk latihan cepat dan berbagi kode.

Langkah penggunaan:
\begin{enumerate}
  \item Buka \url{https://www.onlinegdb.com}
  \item Pilih bahasa (Pascal, C, atau C++)
  \item Tulis atau tempel kode pada editor
  \item Klik tombol \emph{Run} untuk kompilasi dan eksekusi
  \item Simpan berkas untuk mendapatkan tautan berbagi
\end{enumerate}

\subsection{Lingkungan Offline untuk Pascal}
\textbf{Lazarus IDE} \parencite{lazarus} — IDE lengkap berbasis Free Pascal dengan dukungan GUI dan konsol. Tersedia untuk Windows, Linux, dan macOS.

\subsubsection{Instalasi Lazarus IDE}
Langkah instalasi:
\begin{enumerate}
  \item Kunjungi situs resmi Lazarus di \url{https://www.lazarus-ide.org}
  \item Unduh versi yang sesuai dengan sistem operasi Anda (Windows, Linux, atau macOS)
  \item Untuk Windows, unduh file installer (misalnya \texttt{lazarus-}\\\texttt{X.X.X-fpc-X.X.X-win64.exe})
  \item Jalankan file instalasi dan ikuti petunjuk instalasi
  \item Paket instalasi Lazarus biasanya sudah menyertakan Free Pascal Compiler (FPC). Jika tidak, unduh FPC dari \url{https://www.freepascal.org/download.html}
  \item Setelah instalasi selesai, buka Lazarus IDE untuk memverifikasi bahwa instalasi berjalan dengan baik
\end{enumerate}

Langkah penggunaan:
\begin{enumerate}
  \item Buat proyek baru: \texttt{Project \textrightarrow{} New Project \textrightarrow{} Console Application}
  \item Tulis kode Pascal pada editor
  \item Klik \texttt{Run} atau tekan F9 untuk kompilasi dan eksekusi
  \item Tambahkan unit bila diperlukan untuk modularisasi
\end{enumerate}

\subsection{Lingkungan Offline untuk C/C++}
\textbf{Code::Blocks} \parencite{codeblocks} — IDE lintas platform untuk C dan C++. Mendukung berbagai kompiler (GCC, Clang, MSVC).

\subsubsection{Instalasi Code::Blocks}
Langkah instalasi:
\begin{enumerate}
  \item Kunjungi situs resmi Code::Blocks di \url{http://www.codeblocks.org/downloads/binaries}
  \item Untuk Windows, unduh versi yang sudah menyertakan compiler MinGW (misalnya \texttt{codeblocks-20.03mingw-setup.exe})
  \item Untuk Linux, gunakan package manager sistem Anda (misalnya \texttt{sudo apt-get install codeblocks} pada Ubuntu/Debian)
  \item Untuk macOS, unduh file DMG atau gunakan Homebrew
  \item Jalankan file instalasi yang telah diunduh
  \item Pada instalasi Windows: klik \texttt{Next}, setujui lisensi (\texttt{I Agree}), biarkan komponen default, pilih lokasi instalasi, lalu klik \texttt{Install}
  \item Setelah instalasi selesai, buka Code::Blocks
  \item Saat pertama kali membuka, pilih \texttt{GNU GCC Compiler} jika diminta memilih compiler
  \item Verifikasi pengaturan compiler melalui menu \texttt{Settings \textrightarrow{} Compiler \textrightarrow{} Toolchain executables}
\end{enumerate}

Langkah penggunaan:
\begin{enumerate}
  \item Buat proyek baru: \texttt{File \textrightarrow{} New \textrightarrow{} Project \textrightarrow{} Console Application}
  \item Pilih bahasa C atau C++
  \item Tulis kode di \texttt{main.c} atau \texttt{main.cpp}
  \item Klik \texttt{Build \& Run} atau tekan F9
\end{enumerate}

\section{Program Pertama}
Program pertama memastikan instalasi dan alur kompilasi-pengejalan (\emph{run}) bekerja. Perbedaan I/O dasar: Pascal memakai \texttt{Writeln}, C memakai \texttt{printf}, C++ memakai \texttt{std::cout} \parencite{w3pascal-io,gnu-c-manual,cpp-reference}.

\begin{lstlisting}[language=Pascal, caption={Hello World pada Pascal}]
program Hello;
begin
  Writeln('Hello, World!');
end.
\end{lstlisting}

\begin{lstlisting}[language=C, caption={Hello World pada C}]
#include <stdio.h>
int main(void) {
  printf("Hello, World!\n");
  return 0;
}
\end{lstlisting}

\begin{lstlisting}[language=C++, caption={Hello World pada C++}]
#include <iostream>
using namespace std;

int main() {
  cout << "Hello, World!\n";
}
\end{lstlisting}

Untuk menjalankan program di atas, gunakan salah satu IDE yang direkomendasikan (OnlineGDB, Lazarus, atau Code::Blocks). Cukup salin kode ke editor IDE, lalu klik tombol Run atau tekan F9.

\section{Rangkuman Materi}
\begin{itemize}
  \item Peran Pascal, C, C++ dalam pembelajaran dan konteks industri.
  \item Menyiapkan lingkungan: OnlineGDB, Lazarus (Pascal), Code::Blocks (C/C++).
  \item Alur build dasar: tulis sumber \textrightarrow{} kompilasi \textrightarrow{} linking \textrightarrow{} eksekusi (\Cref{fig:build-run-flow}).
  \item Kerangka program minimal di Pascal, C, C++ dan contoh \emph{Hello World}.
  \item Perbandingan I/O dasar (Writeln/printf/cout) dan contoh program interaktif sederhana.
  \item Tabel ringkas perintah I/O dan kiat troubleshooting umum (kompilator, path, newline).
\end{itemize}

\section{Konsep Dasar: Kompilasi, Linking, dan Eksekusi}
Secara garis besar, proses membangun program terdiri dari: menulis \emph{source code}, kompilasi menjadi \emph{object file}, menggabungkannya lewat linker menjadi \emph{binary}, lalu menjalankannya melalui sistem operasi. Pada Pascal prosesnya ditangani oleh FPC/Lazarus; pada C/C++ umumnya GCC atau Clang \parencite{free-pascal-docs,gnu-c-manual,cpp-reference}.

\begin{figure}[H]
  \centering
  \begin{tikzpicture}[node distance=1.8cm, >=Stealth]
    \tikzstyle{block}=[draw, rounded corners, minimum width=3.2cm, minimum height=1cm, align=center]
    \node[block] (src) {Source code \\\small(.pas, .c, .cpp)};
    \node[block, right=of src] (cc) {Compiler \\\small(FPC/GCC/Clang)};
    \node[block, right=of cc] (ld) {Linker};
    \node[block, right=of ld] (bin) {Program \\\small(EXE/ELF)};
    \node[block, below=of bin] (rt) {OS + Runtime};
    \draw[->] (src) -- (cc);
    \draw[->] (cc) -- (ld);
    \draw[->] (ld) -- (bin);
    \draw[->] (bin) -- (rt);
  \end{tikzpicture}
  \caption{Alur singkat build dan eksekusi}
  \label{fig:build-run-flow}
\end{figure}

\paragraph{Implikasi Praktis.} Jika kompilasi gagal, periksa pesan kesalahan pada baris dan kolom yang disebutkan. Jika gagal saat menjalankan, periksa input, akses file, atau izin eksekusi. Dokumentasi resmi sangat membantu untuk memahami pesan kesalahan dan opsi kompilator \parencite{free-pascal-docs,gnu-c-manual}.

\section{Struktur Program Dasar}
Sebelum menulis program yang lebih kompleks, pahami kerangka minimal tiap bahasa.

\begin{lstlisting}[language=Pascal, caption={Kerangka program Pascal}]
program Main;
begin
  { kode dieksekusi di sini }
end.
\end{lstlisting}

\begin{lstlisting}[language=C, caption={Kerangka program C}]
#include <stdio.h>

int main(void) {
  /* kode dieksekusi di sini */
  return 0;
}
\end{lstlisting}

\begin{lstlisting}[language=C++, caption={Kerangka program C++}]
#include <iostream>
using namespace std;

int main() {
  // kode dieksekusi di sini
  return 0;
}
\end{lstlisting}

\section{Contoh Input/Output Dasar}
Berikut contoh membaca dua bilangan bulat dan menampilkan hasil penjumlahannya. Perhatikan perbedaan I/O \parencite{w3pascal-io,gnu-c-manual,cpp-reference}.

\begin{lstlisting}[language=Pascal, caption={Menjumlah dua bilangan pada Pascal}]
program SumTwo;
var
  a, b, total: integer;
begin
  Readln(a);
  Readln(b);
  total := a + b;
  Writeln('Jumlah = ', total);
end.
\end{lstlisting}

\begin{lstlisting}[language=C, caption={Menjumlah dua bilangan pada C}]
#include <stdio.h>

int main(void) {
  int a, b;
  if (scanf("%d %d", &a, &b) != 2) {
    return 1; // input tidak valid
  }
  printf("Jumlah = %d\n", a + b);
  return 0;
}
\end{lstlisting}

\begin{lstlisting}[language=C++, caption={Menjumlah dua bilangan pada C++}]
#include <iostream>
using namespace std;

int main() {
  int a, b;
  if (!(cin >> a >> b)) return 1; // input tidak valid
  cout << "Jumlah = " << (a + b) << '\n';
}
\end{lstlisting}

\section{Perbandingan Sintaks I/O Dasar}
\begin{table}[H]
  \centering
  \caption{Perbandingan ringkas perintah I/O}
  \label{tab:io-basic}
  \begin{tabular}{@{}lll@{}}
    \toprule
    Bahasa & Output & Input \\
    \midrule
    Pascal & \texttt{Writeln('...')} & \texttt{Readln(x)} \\
    C      & \texttt{printf("...")} & \texttt{scanf("\%d", \&x)} \\
    C++    & \texttt{std::cout << "..."} & \texttt{std::cin >> x} \\
    \bottomrule
  \end{tabular}
\end{table}

\section{Troubleshooting Umum}
\begin{itemize}
  \item \textbf{Kompilator tidak ditemukan}: pastikan FPC/GCC terpasang dan terdeteksi IDE.
  \item \textbf{Karakter non-ASCII}: simpan berkas sebagai UTF-8 (lihat pengaturan editor).
  \item \textbf{Path dengan spasi}: hindari spasi pada nama folder proyek.
  \item \textbf{Kesalahan linker}: cek ulang nama fungsi/berkas sumber yang disertakan.
  \item \textbf{I/O tidak tampil}: tambahkan karakter baris baru (\texttt{\n}) pada C/C++ atau gunakan \texttt{Writeln} pada Pascal.
\end{itemize}
\end{document}

