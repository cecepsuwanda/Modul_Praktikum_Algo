\documentclass[../main.tex]{subfiles}
\begin{document}
\chapter{Fungsi dan Prosedur}

\section*{Tujuan Praktikum}
Setelah menyelesaikan praktikum ini, mahasiswa diharapkan mampu:
\begin{itemize}
  \item Memahami perbedaan antara fungsi (mengembalikan nilai) dan prosedur (tanpa nilai kembali)
  \item Mendeklarasikan dan mendefinisikan fungsi/prosedur dengan benar
  \item Menggunakan parameter untuk komunikasi data dengan fungsi/prosedur
  \item Memahami perbedaan passing by value, by reference, dan by pointer
  \item Membuat fungsi rekursif untuk menyelesaikan masalah tertentu
  \item Menerapkan prinsip modularitas dengan memecah program menjadi fungsi-fungsi kecil
  \item Memahami scope dan lifetime variabel lokal vs global
\end{itemize}

Abstraksi fungsi dan prosedur memungkinkan dekomposisi masalah ke unit yang dapat diuji dan digunakan ulang. Pascal membedakan dengan jelas antara \texttt{procedure} (tidak mengembalikan nilai) dan \texttt{function} (mengembalikan nilai), sementara C/C++ menggunakan fungsi sebagai konstruksi utama dengan tipe kembalian \texttt{void} untuk prosedur. Pemahaman mendalam tentang kedua konsep ini sangat penting dalam desain program terstruktur \parencite{pascal-procedure-function,c-functions,cpp-functions}.

Bab ini dibagi menjadi dua bagian utama: pertama membahas \textbf{Fungsi} yang mengembalikan nilai, kedua membahas \textbf{Prosedur} yang melakukan aksi tanpa pengembalian nilai eksplisit. Setiap bagian mencakup deklarasi, definisi, cara penggunaan, strategi passing parameter, serta contoh implementasi lintas bahasa pemrograman.

\section{Fungsi}

\subsection{Konsep Dasar Fungsi}
Fungsi adalah blok kode independen yang dirancang untuk melakukan tugas tertentu dan \textbf{mengembalikan nilai} sebagai hasil komputasinya. Fungsi menerima input (parameter), memproses data, dan menghasilkan output melalui pernyataan \texttt{return}. Karakteristik utama fungsi adalah adanya nilai kembalian yang dapat digunakan dalam ekspresi atau disimpan dalam variabel \parencite{cpp-reference,gnu-c-manual}.

Keuntungan menggunakan fungsi meliputi:
\begin{itemize}
  \item \textbf{Reusabilitas}: Kode dapat dipanggil berulang kali dari berbagai lokasi
  \item \textbf{Modularitas}: Memecah program kompleks menjadi unit-unit kecil yang mudah dipahami
  \item \textbf{Testabilitas}: Setiap fungsi dapat diuji secara independen
  \item \textbf{Abstraksi}: Menyembunyikan detail implementasi dari pemanggil
\end{itemize}

\subsection{Deklarasi dan Definisi Fungsi}
\subsubsection{Sintaks Umum}
Deklarasi fungsi menyatakan eksistensi fungsi kepada kompilator, sedangkan definisi menyediakan implementasi aktual. Dalam beberapa kasus, deklarasi dan definisi dapat digabung.

\textbf{Bentuk umum:}
\begin{verbatim}
tipe_kembalian nama_fungsi(tipe_param1 param1, tipe_param2 param2, ...);
\end{verbatim}

\subsubsection{Contoh dalam Pascal}
\begin{lstlisting}[language=Pascal, caption={Deklarasi dan definisi fungsi (Pascal)}]
// Fungsi yang mengembalikan jumlah dua bilangan
function Tambah(a, b: integer): integer;
begin
  Tambah := a + b;  // Penugasan ke nama fungsi
end;

// Fungsi yang menghitung kuadrat
function Kuadrat(x: real): real;
begin
  Kuadrat := x * x;
end;

// Fungsi dengan beberapa pernyataan
function Faktorial(n: integer): integer;
var
  i, hasil: integer;
begin
  hasil := 1;
  for i := 2 to n do
    hasil := hasil * i;
  Faktorial := hasil;
end;

// Pemanggilan fungsi
var
  hasil, angka: integer;
  kuadratNilai: real;
begin
  hasil := Tambah(5, 3);           // hasil = 8
  angka := Faktorial(5);            // angka = 120
  kuadratNilai := Kuadrat(3.5);     // kuadratNilai = 12.25
  Writeln('Hasil: ', hasil);
end.
\end{lstlisting}

\subsubsection{Contoh dalam C}
\begin{lstlisting}[language=C, caption={Deklarasi dan definisi fungsi (C)}]
#include <stdio.h>

// Deklarasi fungsi (function prototype)
int tambah(int a, int b);
double kuadrat(double x);
unsigned int faktorial(unsigned int n);

// Definisi fungsi
int tambah(int a, int b) {
  return a + b;
}

double kuadrat(double x) {
  return x * x;
}

unsigned int faktorial(unsigned int n) {
  unsigned int hasil = 1;
  for (unsigned int i = 2; i <= n; i++) {
    hasil *= i;
  }
  return hasil;
}

int main() {
  int hasil = tambah(5, 3);            // hasil = 8
  unsigned int angka = faktorial(5);   // angka = 120
  double kuadratNilai = kuadrat(3.5);  // kuadratNilai = 12.25
  
  printf("Hasil: %d\n", hasil);
  printf("Faktorial: %u\n", angka);
  printf("Kuadrat: %.2f\n", kuadratNilai);
  
  return 0;
}
\end{lstlisting}

\subsubsection{Contoh dalam C++}
\begin{lstlisting}[language=C++, caption={Deklarasi dan definisi fungsi (C++)}]
#include <iostream>
#include <string>
using namespace std;

// Fungsi sederhana
int tambah(int a, int b) {
  return a + b;
}

// Fungsi dengan tipe data kompleks
double hitungRataRata(double nilai[], int jumlah) {
  double total = 0.0;
  for (int i = 0; i < jumlah; i++) {
    total += nilai[i];
  }
  return total / jumlah;
}

// Fungsi dengan string
string gabungkanString(const string &s1, const string &s2) {
  return s1 + " " + s2;
}

// Fungsi dengan return type auto (C++11)
auto kali(int x, int y) -> int {
  return x * y;
}

int main() {
  cout << "Tambah: " << tambah(10, 20) << endl;
  
  double nilai[] = {85.5, 90.0, 78.5, 92.0};
  cout << "Rata-rata: " << hitungRataRata(nilai, 4) << endl;
  
  string hasil = gabungkanString("Hello", "World");
  cout << hasil << endl;
  
  cout << "Kali: " << kali(6, 7) << endl;
  
  return 0;
}
\end{lstlisting}

\subsection{Pass by Value pada Fungsi}
Pada mekanisme \emph{pass by value}, nilai argumen disalin ke parameter fungsi. Fungsi bekerja dengan salinan data, sehingga perubahan pada parameter \textbf{tidak mempengaruhi} variabel asli di pemanggil.

\subsubsection{Karakteristik Pass by Value}
\begin{itemize}
  \item Data asli terlindungi dari modifikasi
  \item Overhead penyalinan untuk objek besar
  \item Cocok untuk tipe data primitif dan objek kecil
  \item Default di C dan C++
\end{itemize}

\subsubsection{Contoh Implementasi}
\begin{lstlisting}[language=Pascal, caption={Pass by value (Pascal)}]
function TambahSepuluh(x: integer): integer;
begin
  x := x + 10;  // Mengubah salinan lokal
  TambahSepuluh := x;
end;

var
  angka: integer;
begin
  angka := 5;
  Writeln('Hasil fungsi: ', TambahSepuluh(angka));  // 15
  Writeln('Nilai asli: ', angka);                    // 5 (tidak berubah)
end.
\end{lstlisting}

\begin{lstlisting}[language=C, caption={Pass by value (C)}]
#include <stdio.h>

int kuadratkanDanTambah(int x, int penambah) {
  x = x * x;  // Mengubah salinan lokal
  return x + penambah;
}

int main() {
  int angka = 5;
  int hasil = kuadratkanDanTambah(angka, 10);
  
  printf("Hasil: %d\n", hasil);     // 35 (5*5 + 10)
  printf("Angka asli: %d\n", angka); // 5 (tidak berubah)
  
  return 0;
}
\end{lstlisting}

\begin{lstlisting}[language=C++, caption={Pass by value dengan objek (C++)}]
#include <iostream>
#include <string>
using namespace std;

string ubahString(string teks) {
  teks += " - Dimodifikasi";  // Mengubah salinan
  return teks;
}

int main() {
  string asli = "Teks Asli";
  string hasil = ubahString(asli);
  
  cout << "Hasil: " << hasil << endl;    // Teks Asli - Dimodifikasi
  cout << "Asli: " << asli << endl;      // Teks Asli (tidak berubah)
  
  return 0;
}
\end{lstlisting}

\subsection{Pass by Reference pada Fungsi}
Pada mekanisme \emph{pass by reference}, fungsi menerima referensi (alamat) ke variabel asli. Perubahan pada parameter \textbf{akan mempengaruhi} variabel asli di pemanggil. Ini efisien untuk objek besar dan memungkinkan fungsi menghasilkan beberapa output.

\subsubsection{Karakteristik Pass by Reference}
\begin{itemize}
  \item Tidak ada penyalinan data (efisien untuk objek besar)
  \item Fungsi dapat memodifikasi argumen asli
  \item Di Pascal: gunakan kata kunci \texttt{var}
  \item Di C: gunakan pointer dengan operator \texttt{\&} dan \texttt{*}
  \item Di C++: gunakan referensi dengan tanda \texttt{\&}
\end{itemize}

\subsubsection{Contoh Implementasi}
\begin{lstlisting}[language=Pascal, caption={Pass by reference (Pascal)}]
function TukarDanJumlah(var a, b: integer): integer;
var
  temp: integer;
begin
  temp := a;
  a := b;
  b := temp;
  TukarDanJumlah := a + b;
end;

var
  x, y, jumlah: integer;
begin
  x := 10;
  y := 20;
  jumlah := TukarDanJumlah(x, y);
  Writeln('x = ', x, ', y = ', y);  // x = 20, y = 10
  Writeln('Jumlah = ', jumlah);     // 30
end.
\end{lstlisting}

\begin{lstlisting}[language=C, caption={Pass by reference via pointer (C)}]
#include <stdio.h>

int tukarDanJumlah(int *a, int *b) {
  int temp = *a;
  *a = *b;
  *b = temp;
  return *a + *b;
}

int main() {
  int x = 10, y = 20;
  int jumlah = tukarDanJumlah(&x, &y);  // Kirim alamat
  
  printf("x = %d, y = %d\n", x, y);      // x = 20, y = 10
  printf("Jumlah = %d\n", jumlah);       // 30
  
  return 0;
}
\end{lstlisting}

\begin{lstlisting}[language=C++, caption={Pass by reference (C++)}]
#include <iostream>
#include <string>
using namespace std;

// Referensi non-const untuk modifikasi
int tukarDanJumlah(int &a, int &b) {
  int temp = a;
  a = b;
  b = temp;
  return a + b;
}

// Referensi const untuk efisiensi tanpa modifikasi
int hitungPanjangTotal(const string &s1, const string &s2) {
  return s1.length() + s2.length();
}

int main() {
  int x = 10, y = 20;
  int jumlah = tukarDanJumlah(x, y);
  
  cout << "x = " << x << ", y = " << y << endl;  // x = 20, y = 10
  cout << "Jumlah = " << jumlah << endl;         // 30
  
  string kata1 = "Hello", kata2 = "World";
  cout << "Total panjang: " << hitungPanjangTotal(kata1, kata2) << endl;
  
  return 0;
}
\end{lstlisting}

\subsection{Fungsi Rekursif}
Fungsi rekursif adalah fungsi yang memanggil dirinya sendiri untuk memecahkan masalah menjadi submasalah yang lebih kecil. Setiap fungsi rekursif harus memiliki \textbf{basis} (kondisi berhenti) dan \textbf{langkah rekursif} yang mengarah ke basis.

\subsubsection{Diagram Tumpukan Panggilan}
\begin{figure}[H]
  \centering
  \begin{tikzpicture}[node distance=1.2cm]
    \tikzstyle{f}=[rectangle, draw, rounded corners, minimum width=3.0cm, align=center, fill=blue!10]
    \node[f] (f1) {fact(4) = 4 × fact(3)};
    \node[f, below=of f1, fill=green!10] (f2) {fact(3) = 3 × fact(2)};
    \node[f, below=of f2, fill=yellow!10] (f3) {fact(2) = 2 × fact(1)};
    \node[f, below=of f3, fill=red!10] (f4) {fact(1) = 1 (basis)};
    \draw[->, thick] (f1) -- (f2);
    \draw[->, thick] (f2) -- (f3);
    \draw[->, thick] (f3) -- (f4);
  \end{tikzpicture}
  \caption{Tumpukan panggilan saat menghitung faktorial secara rekursif}
\end{figure}

\subsubsection{Contoh Rekursi}
\begin{lstlisting}[language=Pascal, caption={Fungsi rekursif (Pascal)}]
function Faktorial(n: integer): integer;
begin
  if n <= 1 then
    Faktorial := 1    // Basis
  else
    Faktorial := n * Faktorial(n - 1);  // Rekursi
end;

function Fibonacci(n: integer): integer;
begin
  if n <= 1 then
    Fibonacci := n
  else
    Fibonacci := Fibonacci(n - 1) + Fibonacci(n - 2);
end;
\end{lstlisting}

\begin{lstlisting}[language=C, caption={Fungsi rekursif vs iteratif (C)}]
// Faktorial rekursif
unsigned int fact_rec(unsigned int n) {
  return (n <= 1) ? 1 : n * fact_rec(n - 1);
}

// Faktorial iteratif (lebih efisien)
unsigned int fact_it(unsigned int n) {
  unsigned int hasil = 1;
  for (unsigned int i = 2; i <= n; i++)
    hasil *= i;
  return hasil;
}

// Pangkat rekursif
double pangkat(double base, int exp) {
  if (exp == 0)
    return 1.0;
  if (exp < 0)
    return 1.0 / pangkat(base, -exp);
  return base * pangkat(base, exp - 1);
}
\end{lstlisting}

\begin{lstlisting}[language=C++, caption={Rekursi dengan optimasi (C++)}]
#include <iostream>
using namespace std;

// Tail recursion (dapat dioptimasi kompilator)
int sumTailRec(int n, int akumulator = 0) {
  if (n == 0)
    return akumulator;
  return sumTailRec(n - 1, akumulator + n);
}

// Non-tail recursion
int sumRec(int n) {
  return (n == 0) ? 0 : n + sumRec(n - 1);
}

// Pencarian biner rekursif
int binarySearch(int arr[], int left, int right, int target) {
  if (left > right)
    return -1;
  
  int mid = left + (right - left) / 2;
  
  if (arr[mid] == target)
    return mid;
  else if (arr[mid] > target)
    return binarySearch(arr, left, mid - 1, target);
  else
    return binarySearch(arr, mid + 1, right, target);
}
\end{lstlisting}

Rekursi efektif untuk masalah yang memiliki struktur rekursif alami seperti traversal pohon, divide-and-conquer, dan backtracking. Namun, perhatikan overhead memori stack dan pertimbangkan solusi iteratif untuk input besar \parencite{cpp-reference,gnu-c-manual}.

\subsection{Fitur Lanjutan Fungsi C++}
\subsubsection{Function Overloading}
\begin{lstlisting}[language=C++, caption={Overloading fungsi}]
#include <iostream>
#include <string>
using namespace std;

// Overload berdasarkan jumlah parameter
int maksimum(int a, int b) {
  return (a > b) ? a : b;
}

int maksimum(int a, int b, int c) {
  return maksimum(maksimum(a, b), c);
}

// Overload berdasarkan tipe parameter
double maksimum(double a, double b) {
  return (a > b) ? a : b;
}

string maksimum(const string &a, const string &b) {
  return (a > b) ? a : b;
}

int main() {
  cout << maksimum(5, 10) << endl;          // int (2 param)
  cout << maksimum(5, 10, 15) << endl;      // int (3 param)
  cout << maksimum(5.5, 3.2) << endl;       // double
  cout << maksimum("abc", "xyz") << endl;   // string
  return 0;
}
\end{lstlisting}

\subsubsection{Default Arguments}
\begin{lstlisting}[language=C++, caption={Default argument}]
#include <iostream>
using namespace std;

int pangkat(int base, int eksponen = 2) {
  int hasil = 1;
  for (int i = 0; i < eksponen; i++)
    hasil *= base;
  return hasil;
}

void cetakPesan(string pesan, int ulang = 1) {
  for (int i = 0; i < ulang; i++)
    cout << pesan << endl;
}

int main() {
  cout << pangkat(5) << endl;        // 25 (5^2, default)
  cout << pangkat(5, 3) << endl;     // 125 (5^3)
  
  cetakPesan("Hello");               // Cetak 1x
  cetakPesan("World", 3);            // Cetak 3x
  
  return 0;
}
\end{lstlisting}

\subsubsection{Lambda Expressions dan std::function (C++)}
Lambda memudahkan mendefinisikan fungsi anonim in-line dengan capture variabel \parencite{cpp-lambda,cpp-std-function}.
\begin{lstlisting}[language=C++, caption={Lambda dan std::function}]
#include <iostream>
#include <vector>
#include <algorithm>
#include <functional>
using namespace std;

int main() {
  vector<int> v{5,1,4,2,3};
  int pivot = 3;
  // Capture by value (pivot), by reference (vsum)
  int vsum = 0;
  auto lessThanPivot = [pivot,&vsum](int x){ vsum += x; return x < pivot; };
  auto it = partition(v.begin(), v.end(), lessThanPivot);
  cout << "sum=" << vsum << " first part size=" << distance(v.begin(), it) << "\n";

  function<int(int,int)> op = [](int a,int b){ return a + b; };
  cout << op(2,3) << "\n";
}
\end{lstlisting}

\subsection{Pointer ke Fungsi dan qsort (C)}
Fungsi sebagai nilai pertama kelas di C direpresentasikan melalui pointer ke fungsi; \texttt{qsort} menerima pointer pembanding \parencite{c-function-pointers,c-qsort}.
\begin{lstlisting}[language=C, caption={qsort dengan pointer ke fungsi (C)}]
#include <stdio.h>
#include <stdlib.h>

int cmpInt(const void *a, const void *b) {
  int ia = *(const int*)a;
  int ib = *(const int*)b;
  return (ia > ib) - (ia < ib); // -1,0,1
}

int main() {
  int arr[] = {5,1,4,2,3};
  size_t n = sizeof arr / sizeof *arr;
  qsort(arr, n, sizeof *arr, cmpInt);
  for (size_t i = 0; i < n; ++i) printf("%d ", arr[i]);
  printf("\n");
}
\end{lstlisting}

\section{Prosedur}

\subsection{Konsep Dasar Prosedur}
Prosedur adalah blok kode yang dirancang untuk melakukan serangkaian aksi atau operasi tertentu \textbf{tanpa mengembalikan nilai eksplisit}. Dalam Pascal, prosedur dideklarasikan dengan kata kunci \texttt{procedure}, sedangkan dalam C/C++, prosedur diimplementasikan sebagai fungsi dengan tipe kembalian \texttt{void} \parencite{pascal-procedure-function,c-functions}.

Perbedaan utama prosedur dengan fungsi:
\begin{itemize}
  \item Prosedur tidak memiliki nilai kembalian (atau \texttt{void})
  \item Prosedur dipanggil sebagai pernyataan mandiri, bukan sebagai bagian dari ekspresi
  \item Prosedur biasanya digunakan untuk efek samping: mencetak output, memodifikasi parameter, mengubah state global
  \item Prosedur fokus pada \emph{aksi} bukan pada \emph{komputasi nilai}
\end{itemize}

\subsection{Deklarasi dan Definisi Prosedur}
\subsubsection{Sintaks Umum}
\textbf{Pascal:}
\begin{verbatim}
procedure nama_prosedur(parameter1: tipe1; parameter2: tipe2; ...);
begin
  // Blok kode
end;
\end{verbatim}

\textbf{C/C++:}
\begin{verbatim}
void nama_prosedur(tipe1 param1, tipe2 param2, ...) {
  // Blok kode
}
\end{verbatim}

\subsubsection{Contoh dalam Pascal}
\begin{lstlisting}[language=Pascal, caption={Deklarasi dan definisi prosedur (Pascal)}]
// Prosedur tanpa parameter
procedure CetakGaris;
begin
  Writeln('=========================');
end;

// Prosedur dengan parameter
procedure CetakPesan(pesan: string; ulang: integer);
var
  i: integer;
begin
  for i := 1 to ulang do
    Writeln(pesan);
end;

// Prosedur dengan parameter rujukan
procedure TukarNilai(var a, b: integer);
var
  temp: integer;
begin
  temp := a;
  a := b;
  b := temp;
end;

// Prosedur kompleks
procedure CetakInfo(nama: string; umur: integer; var status: string);
begin
  Writeln('Nama: ', nama);
  Writeln('Umur: ', umur);
  if umur >= 18 then
    status := 'Dewasa'
  else
    status := 'Anak-anak';
end;

// Penggunaan
var
  x, y: integer;
  statusOrang: string;
begin
  CetakGaris;
  CetakPesan('Hello World', 3);
  
  x := 10;
  y := 20;
  TukarNilai(x, y);
  Writeln('x = ', x, ', y = ', y);  // x = 20, y = 10
  
  CetakInfo('Budi', 25, statusOrang);
  Writeln('Status: ', statusOrang);  // Dewasa
end.
\end{lstlisting}

\subsubsection{Contoh dalam C}
\begin{lstlisting}[language=C, caption={Definisi prosedur (C)}]
#include <stdio.h>
#include <string.h>

// Prosedur sederhana
void cetakGaris(void) {
  printf("=========================\n");
}

// Prosedur dengan parameter
void cetakPesan(const char *pesan, int ulang) {
  for (int i = 0; i < ulang; i++) {
    printf("%s\n", pesan);
  }
}

// Prosedur dengan pointer (pass by reference)
void tukarNilai(int *a, int *b) {
  int temp = *a;
  *a = *b;
  *b = temp;
}

// Prosedur kompleks dengan multiple outputs via pointer
void hitungStatistik(int arr[], int size, int *min, int *max, double *avg) {
  *min = arr[0];
  *max = arr[0];
  int sum = 0;
  
  for (int i = 0; i < size; i++) {
    if (arr[i] < *min) *min = arr[i];
    if (arr[i] > *max) *max = arr[i];
    sum += arr[i];
  }
  
  *avg = (double)sum / size;
}

int main() {
  cetakGaris();
  cetakPesan("Hello C", 2);
  
  int x = 10, y = 20;
  tukarNilai(&x, &y);
  printf("x = %d, y = %d\n", x, y);
  
  int data[] = {5, 2, 8, 1, 9};
  int min, max;
  double avg;
  hitungStatistik(data, 5, &min, &max, &avg);
  printf("Min: %d, Max: %d, Avg: %.2f\n", min, max, avg);
  
  return 0;
}
\end{lstlisting}

\subsubsection{Contoh dalam C++}
\begin{lstlisting}[language=C++, caption={Definisi prosedur (C++)}]
#include <iostream>
#include <vector>
#include <string>
using namespace std;

// Prosedur sederhana
void cetakGaris() {
  cout << "=========================" << endl;
}

// Prosedur dengan parameter
void cetakArray(const vector<int> &arr) {
  cout << "[ ";
  for (int val : arr) {
    cout << val << " ";
  }
  cout << "]" << endl;
}

// Prosedur dengan referensi untuk modifikasi
void tukarNilai(int &a, int &b) {
  int temp = a;
  a = b;
  b = temp;
}

// Prosedur dengan multiple outputs
void pisahkanNama(const string &namaLengkap, string &namaDepan, 
                  string &namaBelakang) {
  size_t pos = namaLengkap.find(' ');
  if (pos != string::npos) {
    namaDepan = namaLengkap.substr(0, pos);
    namaBelakang = namaLengkap.substr(pos + 1);
  } else {
    namaDepan = namaLengkap;
    namaBelakang = "";
  }
}

// Prosedur dengan objek
void sortirVector(vector<int> &vec) {
  // Bubble sort sederhana
  for (size_t i = 0; i < vec.size(); i++) {
    for (size_t j = 0; j < vec.size() - 1 - i; j++) {
      if (vec[j] > vec[j + 1]) {
        tukarNilai(vec[j], vec[j + 1]);
      }
    }
  }
}

int main() {
  cetakGaris();
  
  vector<int> angka = {5, 2, 8, 1, 9};
  cout << "Sebelum sortir: ";
  cetakArray(angka);
  
  sortirVector(angka);
  cout << "Setelah sortir: ";
  cetakArray(angka);
  
  string depan, belakang;
  pisahkanNama("John Doe", depan, belakang);
  cout << "Nama depan: " << depan << endl;
  cout << "Nama belakang: " << belakang << endl;
  
  return 0;
}
\end{lstlisting}

\subsection{Pass by Value pada Prosedur}
Pada prosedur dengan \emph{pass by value}, parameter menerima salinan nilai argumen. Modifikasi pada parameter tidak mempengaruhi variabel asli. Mekanisme ini cocok ketika prosedur hanya perlu membaca data tanpa mengubahnya.

\subsubsection{Contoh Implementasi}
\begin{lstlisting}[language=Pascal, caption={Pass by value pada prosedur (Pascal)}]
procedure CetakDanTambah(x: integer; penambah: integer);
begin
  x := x + penambah;  // Mengubah salinan lokal
  Writeln('Dalam prosedur: ', x);
end;

var
  angka: integer;
begin
  angka := 10;
  CetakDanTambah(angka, 5);
  Writeln('Setelah prosedur: ', angka);  // Tetap 10
end.
\end{lstlisting}

\begin{lstlisting}[language=C, caption={Pass by value pada prosedur (C)}]
#include <stdio.h>

void cetakDanKali(int x, int pengali) {
  x = x * pengali;  // Mengubah salinan
  printf("Dalam prosedur: %d\n", x);
}

int main() {
  int angka = 10;
  cetakDanKali(angka, 5);
  printf("Setelah prosedur: %d\n", angka);  // Tetap 10
  return 0;
}
\end{lstlisting}

\begin{lstlisting}[language=C++, caption={Pass by value pada prosedur (C++)}]
#include <iostream>
#include <string>
using namespace std;

void ubahDanCetak(string teks, string tambahan) {
  teks += tambahan;  // Mengubah salinan
  cout << "Dalam prosedur: " << teks << endl;
}

int main() {
  string asli = "Hello";
  ubahDanCetak(asli, " World");
  cout << "Setelah prosedur: " << asli << endl;  // Tetap "Hello"
  return 0;
}
\end{lstlisting}

\subsection{Pass by Reference pada Prosedur}
Mekanisme \emph{pass by reference} memungkinkan prosedur memodifikasi variabel asli yang dilewatkan sebagai argumen. Ini adalah cara utama prosedur menghasilkan output atau efek samping.

\subsubsection{Karakteristik}
\begin{itemize}
  \item Prosedur dapat mengubah nilai variabel pemanggil
  \item Efisien untuk objek besar (tidak ada penyalinan)
  \item Memungkinkan multiple outputs
  \item Harus hati-hati dengan efek samping yang tidak diinginkan
\end{itemize}

\subsubsection{Contoh Implementasi}
\begin{lstlisting}[language=Pascal, caption={Pass by reference pada prosedur (Pascal)}]
// Prosedur mengubah nilai melalui var parameter
procedure IncByOne(var x: integer);
begin
  x := x + 1;
end;

procedure ResetArray(var arr: array of integer; size: integer);
var
  i: integer;
begin
  for i := 0 to size - 1 do
    arr[i] := 0;
end;

procedure HitungLuasKeliling(panjang, lebar: real; 
                             var luas, keliling: real);
begin
  luas := panjang * lebar;
  keliling := 2 * (panjang + lebar);
end;

var
  n: integer;
  l, k: real;
begin
  n := 10;
  IncByOne(n);
  Writeln(n);  // 11
  
  HitungLuasKeliling(5.0, 3.0, l, k);
  Writeln('Luas: ', l:0:1, ', Keliling: ', k:0:1);
end.
\end{lstlisting}

\begin{lstlisting}[language=C, caption={Pass by reference via pointer (C)}]
#include <stdio.h>
#include <math.h>

void incByOne(int *p) {
  if (p)
    (*p)++;
}

void resetArray(int *arr, int size) {
  for (int i = 0; i < size; i++)
    arr[i] = 0;
}

void hitungLingkaran(double radius, double *luas, double *keliling) {
  *luas = M_PI * radius * radius;
  *keliling = 2 * M_PI * radius;
}

void pisahkanBilanganReal(double angka, int *bagianBulat, 
                          double *bagianDesimal) {
  *bagianBulat = (int)angka;
  *bagianDesimal = angka - *bagianBulat;
}

int main() {
  int n = 10;
  incByOne(&n);
  printf("n = %d\n", n);  // 11
  
  int data[] = {1, 2, 3, 4, 5};
  resetArray(data, 5);
  printf("data[0] = %d\n", data[0]);  // 0
  
  double luas, keliling;
  hitungLingkaran(7.0, &luas, &keliling);
  printf("Luas: %.2f, Keliling: %.2f\n", luas, keliling);
  
  int bulat;
  double desimal;
  pisahkanBilanganReal(3.14159, &bulat, &desimal);
  printf("Bulat: %d, Desimal: %.5f\n", bulat, desimal);
  
  return 0;
}
\end{lstlisting}

\begin{lstlisting}[language=C++, caption={Pass by reference (C++)}]
#include <iostream>
#include <vector>
#include <algorithm>
using namespace std;

void inc(int &x) {
  x++;
}

void resetVector(vector<int> &vec) {
  for (int &val : vec)
    val = 0;
}

void hitungStatistik(const vector<int> &data, int &min, int &max, 
                     double &avg) {
  min = *min_element(data.begin(), data.end());
  max = *max_element(data.begin(), data.end());
  
  int sum = 0;
  for (int val : data)
    sum += val;
  avg = (double)sum / data.size();
}

void kaliMatriks(int &a, int &b, int &c, int &d, int skalar) {
  a *= skalar;
  b *= skalar;
  c *= skalar;
  d *= skalar;
}

int main() {
  int n = 10;
  inc(n);
  cout << "n = " << n << endl;  // 11
  
  vector<int> angka = {5, 2, 8, 1, 9};
  int minVal, maxVal;
  double avgVal;
  hitungStatistik(angka, minVal, maxVal, avgVal);
  cout << "Min: " << minVal << ", Max: " << maxVal 
       << ", Avg: " << avgVal << endl;
  
  int a = 1, b = 2, c = 3, d = 4;
  kaliMatriks(a, b, c, d, 10);
  cout << "Matriks: [" << a << ", " << b << "; " 
       << c << ", " << d << "]" << endl;
  
  return 0;
}
\end{lstlisting}

\subsection{Prosedur Rekursif}
Prosedur juga dapat memanggil dirinya sendiri untuk menyelesaikan masalah secara rekursif. Prosedur rekursif sering digunakan untuk traversal struktur data atau menghasilkan efek samping berulang.

\begin{lstlisting}[language=Pascal, caption={Prosedur rekursif (Pascal)}]
procedure CetakMundur(n: integer);
begin
  if n > 0 then
  begin
    Writeln(n);
    CetakMundur(n - 1);
  end;
end;

procedure HanoiTower(n: integer; asal, tujuan, bantu: char);
begin
  if n = 1 then
    Writeln('Pindah disk dari ', asal, ' ke ', tujuan)
  else
  begin
    HanoiTower(n - 1, asal, bantu, tujuan);
    Writeln('Pindah disk dari ', asal, ' ke ', tujuan);
    HanoiTower(n - 1, bantu, tujuan, asal);
  end;
end;
\end{lstlisting}

\begin{lstlisting}[language=C, caption={Prosedur rekursif (C)}]
#include <stdio.h>

void cetakMundur(int n) {
  if (n > 0) {
    printf("%d ", n);
    cetakMundur(n - 1);
  }
}

void cetakBiner(unsigned int n) {
  if (n > 1)
    cetakBiner(n / 2);
  printf("%d", n % 2);
}

void hanoiTower(int n, char asal, char tujuan, char bantu) {
  if (n == 1) {
    printf("Pindah disk dari %c ke %c\n", asal, tujuan);
  } else {
    hanoiTower(n - 1, asal, bantu, tujuan);
    printf("Pindah disk dari %c ke %c\n", asal, tujuan);
    hanoiTower(n - 1, bantu, tujuan, asal);
  }
}

int main() {
  cetakMundur(5);
  printf("\n");
  
  printf("15 dalam biner: ");
  cetakBiner(15);
  printf("\n");
  
  printf("Solusi Hanoi Tower (3 disk):\n");
  hanoiTower(3, 'A', 'C', 'B');
  
  return 0;
}
\end{lstlisting}

\begin{lstlisting}[language=C++, caption={Prosedur rekursif (C++)}]
#include <iostream>
#include <vector>
using namespace std;

void cetakMundur(int n) {
  if (n > 0) {
    cout << n << " ";
    cetakMundur(n - 1);
  }
}

void generatePermutasi(vector<int> &arr, int l, int r) {
  if (l == r) {
    for (int val : arr)
      cout << val << " ";
    cout << endl;
  } else {
    for (int i = l; i <= r; i++) {
      swap(arr[l], arr[i]);
      generatePermutasi(arr, l + 1, r);
      swap(arr[l], arr[i]);  // Backtrack
    }
  }
}

void cetakPohonAngka(int n, int level = 0) {
  if (n > 0) {
    for (int i = 0; i < level; i++)
      cout << "  ";
    cout << n << endl;
    cetakPohonAngka(n - 1, level + 1);
    cetakPohonAngka(n - 1, level + 1);
  }
}

int main() {
  cout << "Mundur: ";
  cetakMundur(5);
  cout << endl;
  
  vector<int> arr = {1, 2, 3};
  cout << "Permutasi {1,2,3}:" << endl;
  generatePermutasi(arr, 0, arr.size() - 1);
  
  return 0;
}
\end{lstlisting}

\subsection{Perbandingan Fungsi dan Prosedur}
\begin{table}[H]
  \centering
  \caption{Perbandingan fungsi dan prosedur}
  \begin{tabular}{@{}lll@{}}
    \toprule
    Aspek & Fungsi & Prosedur \\
    \midrule
    Nilai kembalian & Ya (tipe tertentu) & Tidak (void) \\
    Pemanggilan & Dalam ekspresi & Sebagai pernyataan \\
    Tujuan utama & Komputasi nilai & Efek samping/aksi \\
    Return statement & Wajib (kecuali void) & Opsional \\
    Contoh Pascal & function & procedure \\
    Contoh C/C++ & int, double, dll. & void \\
    \bottomrule
  \end{tabular}
\end{table}

\subsection{Ringkasan Strategi Parameter}
\begin{table}[H]
  \centering
  \caption{Ringkasan strategi parameter untuk fungsi dan prosedur}
  \small
  \begin{tabular}{@{}p{4cm}p{5cm}p{4.5cm}@{}}
    \toprule
    Strategi & Kapan dipakai & Catatan \\
    \midrule
    Pass by value & Argumen kecil, tidak perlu diubah & Aman, data terlindungi \\
    Pass by reference & Perlu modifikasi atau objek besar & Efisien, hati-hati efek samping \\
    Const reference (C++) & Baca saja, objek besar & Optimal: aman dan efisien \\
    Pointer (C) & Opsional/nullable, array & Jelaskan ownership dengan jelas \\
    Var (Pascal) & Parameter output/inout & Eksplisit, jelas intent \\
    \bottomrule
  \end{tabular}
\end{table}

Pemilihan strategi parameter memengaruhi performa dan kebenaran program. Gunakan \emph{pass by value} untuk keamanan dan \emph{pass by reference} untuk efisiensi atau multiple outputs. Pada C++, preferensi \texttt{const T\&} untuk parameter baca-saja dan \texttt{T\&} untuk parameter yang dimodifikasi. Dokumentasikan kontrak parameter dengan jelas untuk maintainability jangka panjang \parencite{cpp-functions,gnu-c-manual,free-pascal-docs}.

\end{document}
