\documentclass[../main.tex]{subfiles}
\begin{document}
\chapter{Perulangan}

\section*{Tujuan Praktikum}
Setelah menyelesaikan praktikum ini, mahasiswa diharapkan mampu:
\begin{itemize}
  \item Memahami konsep dan jenis-jenis perulangan (for, while, repeat-until, do-while)
  \item Memilih bentuk loop yang sesuai berdasarkan kondisi masalah
  \item Mengimplementasikan nested loop untuk masalah multidimensi
  \item Menggunakan break dan continue untuk kontrol alur loop
  \item Menghindari infinite loop dan off-by-one error
  \item Membuat program dengan pola output menggunakan nested loop
  \item Mengoptimalkan efisiensi loop untuk performa yang lebih baik
\end{itemize}

\section{Perulangan: \texttt{for}, \texttt{while}, \texttt{repeat} / \texttt{do\textendash while}}
Bentuk perulangan dipilih berdasarkan apakah jumlah iterasi diketahui dan kapan kondisi dievaluasi (awal/akhir). Pascal: \texttt{for}, \texttt{while}, \texttt{repeat\ldots until}; C/C++: \texttt{for}, \texttt{while}, \texttt{do\ldots while} \parencite{free-pascal-docs,gnu-c-manual,cpp-reference}.

\subsection{Ringkasan Pemilihan Bentuk Loop}
\begin{table}[H]
  \centering
  \caption{Kapan memakai bentuk loop}
  \begin{tabular}{@{}lll@{}}
    \toprule
    Bentuk & Kapan dipakai & Catatan \\
    \midrule
    while & Iterasi tak diketahui, cek di awal & Bisa 0 iterasi \\
    do-while/repeat-until & Harus jalan minimal sekali & Cek di akhir \\
    for & Indeks berhingga, batas jelas & Hati-hati off-by-one \\
    range-based for (C++) & Koleksi/iterator modern & Lebih aman dan ringkas \\
    \bottomrule
  \end{tabular}
\end{table}

\subsection{Diagram Alur Loop}
\begin{figure}[H]
  \centering
  \begin{tikzpicture}[node distance=1.7cm, >=Stealth]
    \tikzstyle{b}=[rectangle, draw, rounded corners, align=center, minimum width=3.0cm, minimum height=1cm]
    % while
    \node[b] (start) {Mulai};
    \node[b, below=of start] (cond) {kondisi?};
    \node[b, below=of cond] (body) {badan loop};
    \node[b, below=of body] (end) {Selesai};
    \draw[->] (start) -- (cond);
    \draw[->] (cond) -- node[right]{Ya} (body);
    \draw[->] (body) -- +( -3.0, 0) |- (cond);
    \draw[->] (cond.east) -- +(3.0,0) |- node[pos=0.25,right]{Tidak} (end);
  \end{tikzpicture}
  \caption{Skema \texttt{while}: cek di awal, mungkin 0 iterasi}
\end{figure}

\subsection{Diagram Alur \texttt{repeat-until} / \texttt{do-while}}
\begin{figure}[H]
  \centering
  \begin{tikzpicture}[node distance=1.7cm, >=Stealth]
    \tikzstyle{b}=[rectangle, draw, rounded corners, align=center, minimum width=3.0cm, minimum height=1cm]
    \node[b] (start) {Mulai};
    \node[b, below=of start] (body) {badan loop};
    \node[b, below=of body] (cond) {kondisi?};
    \node[b, below=of cond] (end) {Selesai};
    \draw[->] (start) -- (body);
    \draw[->] (body) -- (cond);
    \draw[->] (cond) -- node[right]{Tidak (C)} (end);
    \draw[->] (cond.west) -- +(-2.0, 0) |- node[pos=0.25,left]{Ya (C)} (body);
    \node[right=0.3cm of cond] {\footnotesize Pascal: until true keluar};
  \end{tikzpicture}
  \caption{Skema \texttt{do-while}/\texttt{repeat-until}: cek di akhir, minimal 1 iterasi}
\end{figure}

\subsection{Contoh Loop \texttt{for}}

Program berikut menggunakan loop \texttt{for} untuk mencetak tabel perkalian dari sebuah angka yang diinput oleh pengguna:

\begin{lstlisting}[language=Pascal, caption={Cetak tabel perkalian dengan for (Pascal)}]
program TabelPerkalian;
var
  i, angka, hasil: integer;
begin
  Write('Masukkan angka: ');
  Readln(angka);
  for i := 1 to 10 do begin
    hasil := angka * i;
    Writeln(angka, ' x ', i, ' = ', hasil);
  end;
end.
\end{lstlisting}

Di C, loop \texttt{for} digunakan untuk menghitung faktorial dengan iterasi dari 1 hingga n:

\begin{lstlisting}[language=C, caption={Hitung faktorial dengan for (C)}]
#include <stdio.h>
int main(void) {
  int n;
  unsigned long faktorial = 1;
  printf("Masukkan n: ");
  scanf("%d", &n);
  
  for (int i = 1; i <= n; ++i) {
    faktorial *= i;
  }
  printf("Faktorial %d = %lu\n", n, faktorial);
  return 0;
}
\end{lstlisting}

Program C++ berikut menggunakan loop untuk membaca beberapa nilai dan menghitung total serta rata-ratanya:

\begin{lstlisting}[language=C++, caption={Loop for dengan hitung total (C++)}]
#include <iostream>
using namespace std;

int main() {
  int total = 0;
  int jumlah = 5;
  
  cout << "Masukkan " << jumlah << " nilai:\n";
  for (int i = 0; i < jumlah; ++i) {
    int nilai;
    cout << "Nilai ke-" << (i+1) << ": ";
    cin >> nilai;
    total += nilai;
  }
  
  cout << "Total: " << total << "\n";
  cout << "Rata-rata: " << (double)total / jumlah << "\n";
}
\end{lstlisting}

C++11 memperkenalkan range-based for loop yang lebih ringkas untuk iterasi koleksi tanpa perlu indeks eksplisit:

\begin{lstlisting}[language=C++, caption={Range-based for pada koleksi (C++11+)}]
#include <iostream>
#include <vector>
using namespace std;

int main() {
  vector<int> data{1,2,3,4,5};
  int total = 0;
  for (int v : data) { // tidak perlu indeks eksplisit
    total += v;
  }
  cout << total << "\n";
}
\end{lstlisting}

\subsection{Contoh Loop \texttt{while}}

Program berikut menggunakan loop \texttt{while} untuk validasi input, meminta pengguna memasukkan angka positif hingga input valid diberikan:

\begin{lstlisting}[language=Pascal, caption={Input validasi dengan while (Pascal)}]
program ValidasiInput;
var
  angka: integer;
begin
  angka := -1;
  while angka < 0 do begin
    Write('Masukkan angka positif: ');
    Readln(angka);
    if angka < 0 then
      Writeln('Angka harus positif!');
  end;
  Writeln('Angka valid: ', angka);
end.
\end{lstlisting}

Program C berikut menggunakan \texttt{while} untuk menghitung jumlah digit dalam sebuah bilangan dengan membagi berulang kali dengan 10:

\begin{lstlisting}[language=C, caption={Hitung digit dengan while (C)}]
#include <stdio.h>
int main(void) {
  int n, digit = 0;
  printf("Masukkan angka: ");
  scanf("%d", &n);
  
  int temp = n;
  while (temp != 0) {
    temp /= 10;
    digit++;
  }
  printf("Jumlah digit: %d\n", digit);
  return 0;
}
\end{lstlisting}

Contoh membaca data dari input hingga mencapai end-of-file (EOF), berguna untuk pemrosesan batch data:

\begin{lstlisting}[language=Pascal, caption={Baca hingga EOF (Pascal)}]
program ReadUntilEOF;
var
  x: integer;
begin
  while not EOF do begin
    Readln(x);
    Writeln('Baca: ', x);
  end;
end.
\end{lstlisting}

Program C++ berikut menggunakan \texttt{while} untuk mengecek apakah suatu bilangan prima dengan mencoba membaginya dengan semua pembagi potensial:

\begin{lstlisting}[language=C++, caption={Hitung bilangan prima dengan while (C++)}]
#include <iostream>
using namespace std;

int main() {
  int n;
  cout << "Masukkan angka: ";
  cin >> n;
  
  int pembagi = 2;
  bool prima = (n > 1);
  
  while (pembagi * pembagi <= n && prima) {
    if (n % pembagi == 0)
      prima = false;
    pembagi++;
  }
  
  if (prima)
    cout << n << " adalah bilangan prima\n";
  else
    cout << n << " bukan bilangan prima\n";
}
\end{lstlisting}

\subsection{Contoh Loop \texttt{repeat-until} / \texttt{do-while}}

Program menu berikut menggunakan \texttt{repeat-until} untuk memastikan menu ditampilkan minimal satu kali dan terus berjalan hingga pengguna memilih keluar:

\begin{lstlisting}[language=Pascal, caption={Menu dengan repeat-until (Pascal)}]
program MenuLoop;
var
  pilihan: integer;
begin
  repeat
    Writeln('=== Menu ===');
    Writeln('1. Opsi A');
    Writeln('2. Opsi B');
    Writeln('0. Keluar');
    Write('Pilih: ');
    Readln(pilihan);
    
    case pilihan of
      1: Writeln('Anda pilih A');
      2: Writeln('Anda pilih B');
      0: Writeln('Keluar...');
    else
      Writeln('Pilihan tidak valid');
    end;
  until pilihan = 0;
end.
\end{lstlisting}

Di C, \texttt{do-while} memastikan blok kode dieksekusi minimal sekali sebelum kondisi diperiksa. Program berikut memproses angka dan menanyakan apakah pengguna ingin mengulang:

\begin{lstlisting}[language=C, caption={Input ulang dengan do-while (C)}]
#include <stdio.h>
int main(void) {
  int angka;
  char lagi;
  
  do {
    printf("Masukkan angka: ");
    scanf("%d", &angka);
    printf("Kuadrat: %d\n", angka * angka);
    
    printf("Ulangi? (y/n): ");
    scanf(" %c", &lagi);
  } while (lagi == 'y' || lagi == 'Y');
  
  printf("Selesai.\n");
  return 0;
}
\end{lstlisting}

Program kalkulator sederhana berikut mendemonstrasikan \texttt{do-while} untuk menu yang berulang, mengkombinasikan loop dengan percabangan \texttt{switch}:

\begin{lstlisting}[language=C++, caption={Validasi input dengan do-while (C++)}]
#include <iostream>
using namespace std;

int main() {
  int pilihan;
  
  do {
    cout << "\n=== Kalkulator Sederhana ===\n";
    cout << "1. Tambah\n";
    cout << "2. Kurang\n";
    cout << "3. Kali\n";
    cout << "4. Bagi\n";
    cout << "0. Keluar\n";
    cout << "Pilih operasi: ";
    cin >> pilihan;
    
    if (pilihan >= 1 && pilihan <= 4) {
      int a, b;
      cout << "Angka 1: "; cin >> a;
      cout << "Angka 2: "; cin >> b;
      
      switch(pilihan) {
        case 1: cout << "Hasil: " << a + b << "\n"; break;
        case 2: cout << "Hasil: " << a - b << "\n"; break;
        case 3: cout << "Hasil: " << a * b << "\n"; break;
        case 4: 
          if (b != 0) cout << "Hasil: " << (double)a / b << "\n";
          else cout << "Error: pembagi nol\n";
          break;
      }
    }
  } while (pilihan != 0);
  
  cout << "Terima kasih!\n";
}
\end{lstlisting}

Ketelitian terhadap \emph{off-by-one} penting pada indeks. Pilih batas eksplisit dan nama variabel penghitung yang jelas.

\subsection{Pola Umum dan Anti-Pattern}
\begin{itemize}
  \item \textbf{Sentinel loop}: baca dulu, cek kondisi, lalu proses; hindari duplikasi kode.
  \item \textbf{Loop kosong berbahaya}: hindari \texttt{while(...);}; pastikan ada badan loop atau komentar jelas.
  \item \textbf{Mutasi koleksi saat iterasi}: pada C++, hati-hati menghapus elemen dari \texttt{vector} saat \texttt{for-range}; gunakan indeks atau iterator aman.
  \item \textbf{Overflow penghitung}: gunakan tipe cukup besar untuk jumlah iterasi.
\end{itemize}

\section{Perulangan Bersarang (Nested Loop)}
Perulangan bersarang adalah loop di dalam loop lainnya. Loop luar dieksekusi sekali, sementara loop dalam dieksekusi sepenuhnya untuk setiap iterasi loop luar. Nested loop umum digunakan untuk memproses struktur data multidimensi seperti matriks atau untuk menghasilkan kombinasi.

\subsection{Diagram Nested Loop}
\begin{figure}[H]
  \centering
  \begin{tikzpicture}[node distance=1.4cm, >=Stealth]
    \tikzstyle{b}=[rectangle, draw, rounded corners, align=center, minimum width=2.6cm, minimum height=0.9cm]
    \node[b] (start) {Mulai};
    \node[b, below=of start] (outer) {Loop Luar};
    \node[b, below=of outer] (inner) {Loop Dalam};
    \node[b, below=of inner] (body) {Proses};
    \node[b, below=of body] (end) {Selesai};
    \draw[->] (start) -- (outer);
    \draw[->] (outer) -- (inner);
    \draw[->] (inner) -- (body);
    \draw[->] (body.east) -- +(1.5,0) |- (inner);
    \draw[->] (inner.west) -- +(-1.5,0) |- (outer);
    \draw[->] (outer) -- +(0,3.5) -| (end);
  \end{tikzpicture}
  \caption{Skema nested loop: loop dalam dieksekusi penuh untuk setiap iterasi loop luar}
\end{figure}

\subsection{Contoh Nested Loop}

Program berikut menggunakan dua loop bersarang untuk mencetak pola segitiga bintang. Loop luar mengontrol jumlah baris, loop dalam mengontrol jumlah bintang per baris:

\begin{lstlisting}[language=Pascal, caption={Pola segitiga dengan nested loop (Pascal)}]
program PolaBintang;
var
  i, j, tinggi: integer;
begin
  Write('Masukkan tinggi segitiga: ');
  Readln(tinggi);
  
  for i := 1 to tinggi do begin
    for j := 1 to i do
      Write('* ');
    Writeln;
  end;
end.
\end{lstlisting}

Program C berikut menghasilkan tabel perkalian menggunakan nested loop untuk iterasi baris dan kolom:

\begin{lstlisting}[language=C, caption={Tabel perkalian dengan nested loop (C)}]
#include <stdio.h>
int main(void) {
  int baris = 5, kolom = 5;
  
  printf("Tabel Perkalian %dx%d:\n", baris, kolom);
  for (int i = 1; i <= baris; ++i) {
    for (int j = 1; j <= kolom; ++j) {
      printf("%4d", i * j);
    }
    printf("\n");
  }
  return 0;
}
\end{lstlisting}

Versi C++ yang serupa untuk membuat pola segitiga dengan nested loop:

\begin{lstlisting}[language=C++, caption={Pola bintang dengan nested loop (C++)}]
#include <iostream>
using namespace std;

int main() {
  int tinggi;
  cout << "Masukkan tinggi segitiga: ";
  cin >> tinggi;
  
  cout << "Pola segitiga:\n";
  for (int i = 1; i <= tinggi; ++i) {
    for (int j = 1; j <= i; ++j) {
      cout << "* ";
    }
    cout << "\n";
  }
}
\end{lstlisting}

Pada nested loop, kompleksitas waktu menjadi perkalian iterasi loop luar dan dalam. Loop 3 tingkat atau lebih harus dievaluasi untuk efisiensi, terutama pada dataset besar. Pertimbangkan algoritme alternatif atau optimisasi bila diperlukan.

\section{Penggunaan \texttt{break}, \texttt{continue}}
\texttt{break} menghentikan loop sepenuhnya dan keluar ke pernyataan setelah loop. \texttt{continue} melompati sisa iterasi saat ini dan langsung ke iterasi berikutnya. Gunakan seperlunya agar alur tetap jelas; terlalu banyak \texttt{break}/\texttt{continue} dapat membingungkan \parencite{gnu-c-manual,cpp-reference}.

\subsection{Contoh \texttt{break} dan \texttt{continue}}

Program berikut mendemonstrasikan penggunaan \texttt{continue} untuk melewati iterasi tertentu dan \texttt{break} untuk keluar dari loop lebih awal:

\begin{lstlisting}[language=Pascal, caption={break dan continue di Pascal}]
program BreakContinue;
var
  i: integer;
begin
  { Continue: lewati bilangan ganjil }
  for i := 1 to 10 do begin
    if i mod 2 = 1 then
      continue;
    Writeln(i);  { hanya cetak genap }
  end;
  
  { Break: hentikan loop saat menemukan kondisi }
  i := 0;
  while true do begin
    i := i + 1;
    if i > 5 then
      break;
    Writeln('Iterasi: ', i);
  end;
end.
\end{lstlisting}

Di C, contoh berikut menunjukkan bagaimana \texttt{continue} melewati bilangan ganjil dan \texttt{break} membatasi proses, serta aplikasinya pada nested loop:

\begin{lstlisting}[language=C, caption={break dan continue di C}]
#include <stdio.h>
int main(void) {
  // Continue: cetak bilangan genap saja
  for (int i = 0; i < 100; ++i) {
    if (i % 2 == 1)
      continue;  // lewati bilangan ganjil
    if (i > 10)
      break;     // batasi proses
    printf("%d\n", i);
  }
  
  // Break pada nested loop: keluar dari loop dalam
  for (int i = 0; i < 5; ++i) {
    for (int j = 0; j < 5; ++j) {
      if (i * j > 6)
        break;  // keluar dari loop j
      printf("(%d,%d) ", i, j);
    }
    printf("\n");
  }
  return 0;
}
\end{lstlisting}

Program berikut mendemonstrasikan penggunaan \texttt{break} untuk menghentikan pencarian saat elemen yang dicari ditemukan:

\begin{lstlisting}[language=C++, caption={break saat menemukan kondisi (C++)}]
#include <iostream>
using namespace std;

int main() {
  int n;
  cout << "Masukkan batas: ";
  cin >> n;
  
  cout << "Bilangan kelipatan 7:\n";
  for (int i = 1; i <= n; ++i) {
    if (i % 7 == 0) {
      cout << "Ditemukan: " << i << "\n";
      cout << "Berhenti di kelipatan 7 pertama\n";
      break;  // keluar loop setelah ditemukan
    }
    cout << i << " ";
  }
  cout << "\n";
}
\end{lstlisting}

Ekstrak predikat kompleks ke fungsi terpisah untuk menghindari banyak \texttt{break} atau \texttt{continue}. Pada nested loop, \texttt{break} hanya keluar dari loop terdalamnya; untuk keluar dari semua loop, gunakan flag boolean atau label (di bahasa yang mendukung). Pada C++, pastikan iterator tetap valid saat keluar dini.
\section{Rangkuman Materi}
\begin{itemize}
  \item Pemilihan bentuk loop: \texttt{while}, \texttt{do-while}/\texttt{repeat-until}, \texttt{for}, dan range-based for.
  \item Diagram alur, pola umum (sentinel), dan anti-pattern (loop kosong, mutasi saat iterasi, overflow penghitung).
  \item I/O dalam loop, validasi input, dan penggunaan \texttt{break}/\texttt{continue} secara bijak.
  \item Nested loop untuk pola/matriks dan implikasi kompleksitas waktu.
  \item Contoh lintas Pascal/C/C++ untuk setiap bentuk loop.
\end{itemize}
\end{document}
