\documentclass[../main.tex]{subfiles}
\begin{document}
\chapter{Struktur Kontrol Perulangan \& Lainnya}
\section{Perulangan: \texttt{for}, \texttt{while}, \texttt{repeat} / \texttt{do\textendash while}}
Bentuk perulangan dipilih berdasarkan apakah jumlah iterasi diketahui dan kapan kondisi dievaluasi (awal/akhir). Pascal: \texttt{for}, \texttt{while}, \texttt{repeat\ldots until}; C/C++: \texttt{for}, \texttt{while}, \texttt{do\ldots while} \parencite{free-pascal-docs,gnu-c-manual,cpp-reference}.

\subsection{Diagram Alur Loop}
\begin{figure}[h]
  \centering
  \begin{tikzpicture}[node distance=1.7cm, >=Stealth]
    \tikzstyle{b}=[rectangle, draw, rounded corners, align=center, minimum width=3.0cm, minimum height=1cm]
    % while
    \node[b] (start) {Mulai};
    \node[b, below=of start] (cond) {kondisi?};
    \node[b, below=of cond] (body) {badan loop};
    \node[b, below=of body] (end) {Selesai};
    \draw[->] (start) -- (cond);
    \draw[->] (cond) -- node[right]{Ya} (body);
    \draw[->] (body) -- +( -3.0, 0) |- (cond);
    \draw[->] (cond.east) -- +(3.0,0) |- node[pos=0.25,right]{Tidak} (end);
  \end{tikzpicture}
  \caption{Skema \texttt{while}: cek di awal, mungkin 0 iterasi}
\end{figure}

\subsection{Contoh Lintas Bahasa}
\begin{lstlisting}[language=Pascal, caption={for dan repeat\ldots until (Pascal)}]
var i: integer; begin
  for i := 1 to 5 do Writeln(i);
  repeat
    Writeln('jalankan minimal sekali');
  until SomeCondition;
end.
\end{lstlisting}

\begin{lstlisting}[language=C, caption={for, while, do\ldots while (C)}]
for (int i=1; i<=5; ++i) printf("%d\n", i);
int j=1; while (j<=5) { printf("%d\n", j++); }
int k=0; do { ++k; } while (k<1); // jalan sekali
\end{lstlisting}

\begin{lstlisting}[language=C++, caption={Range-based for (C++)}]
#include <vector>
#include <iostream>
int main(){
  std::vector<int> v{1,2,3,4,5};
  for (int x : v) std::cout << x << "\n"; // aman untuk koleksi
}
\end{lstlisting}

Ketelitian terhadap \emph{off-by-one} penting pada indeks. Pilih batas eksplisit dan nama variabel penghitung yang jelas.

\section{Penggunaan \texttt{break}, \texttt{continue}}
\texttt{break} menghentikan loop, \texttt{continue} melompati ke iterasi berikutnya. Gunakan seperlunya agar alur tetap jelas \parencite{gnu-c-manual,cpp-reference}.

\begin{lstlisting}[language=C, caption={break dan continue di C}]
for (int i=0; i<100; ++i) {
  if (i%2==1) continue; // lewati bilangan ganjil
  if (i>10) break;      // batasi proses
  printf("%d\n", i);
}
\end{lstlisting}

Ekstrak predikat kompleks ke fungsi terpisah untuk menghindari banyak \texttt{break}/\texttt{continue}. Pada C++, pastikan iterator tetap valid saat keluar dini.

\section{Operator Ternary / Conditional (C / C++)}
Operator \texttt{?:} memilih nilai berbasis kondisi dalam satu ekspresi. Gunakan hemat; jika bercabang kompleks, kembali ke \texttt{if/else} \parencite{cpp-reference}.

\begin{lstlisting}[language=C++]
int absVal = (x >= 0) ? x : -x;
auto label = (score >= 90) ? "A" : (score >= 80) ? "B" : "C"; // hati-hati keterbacaan
\end{lstlisting}

Pastikan tipe kedua cabang kompatibel untuk mencegah konversi mengejutkan.

\subsection{Ringkasan Pemilihan Bentuk Loop}
\begin{table}[h]
  \centering
  \caption{Kapan memakai bentuk loop}
  \begin{tabular}{@{}lll@{}}
    \toprule
    Bentuk & Kapan dipakai & Catatan \\
    \midrule
    while & Iterasi tak diketahui, cek di awal & Bisa 0 iterasi \\
    do-while/repeat-until & Harus jalan minimal sekali & Cek di akhir \\
    for & Indeks berhingga, batas jelas & Hati-hati off-by-one \\
    range-based for (C++) & Koleksi/iterator modern & Lebih aman dan ringkas \\
    \bottomrule
  \end{tabular}
\end{table}

\subsection{Catatan Eksekusi (OnlineGDB, Lazarus, Code::Blocks)}
\begin{itemize}
  \item \textbf{OnlineGDB}: \url{https://www.onlinegdb.com/} \textrightarrow{} pilih bahasa, tempel contoh, Run.
  \item \textbf{Lazarus (Pascal)}: Console Application, tempel contoh Pascal, Run.
  \item \textbf{Code::Blocks (C/C++)}: Console application, pilih C/C++, tempel ke \texttt{main.c}/\texttt{main.cpp}, Build \& Run.
\end{itemize}
\end{document}
