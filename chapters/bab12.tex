\documentclass[../main.tex]{subfiles}
\begin{document}
\chapter{File dan I/O}

\section*{Tujuan Praktikum}
Setelah menyelesaikan praktikum ini, mahasiswa diharapkan mampu:
\begin{itemize}
  \item Memahami perbedaan antara file teks (ASCII) dan file binary
  \item Membuka, membaca, menulis, dan menutup file dengan benar
  \item Menggunakan mode file yang tepat (read, write, append, binary)
  \item Melakukan operasi seek dan tell untuk akses random pada file
  \item Menangani error file operation dengan tepat
  \item Mengimplementasikan atomic write untuk keamanan data
  \item Membuat program yang dapat menyimpan dan membaca data dari file
\end{itemize}

Berinteraksi dengan berkas memungkinkan program menyimpan dan memuat data secara persisten. Terdapat dua jenis berkas utama dalam pemrograman: file ASCII/teks dan file binary. Keduanya memiliki perbedaan fundamental dalam cara deklarasi, mode pembukaan, fungsi manipulasi, dan representasi data. Memahami perbedaan ini penting untuk memilih pendekatan yang sesuai dengan kebutuhan aplikasi \parencite{gnu-c-manual,cpp-reference,free-pascal-docs}.

\section{File ASCII/Teks}

File teks adalah berkas yang berisi karakter ASCII atau Unicode yang dapat dibaca langsung oleh manusia. Data disimpan sebagai rangkaian karakter dengan encoding tertentu (UTF-8, ASCII, dll). File teks cocok untuk menyimpan konfigurasi, log, data terstruktur sederhana, dan dokumen yang perlu diedit manual.

\subsection{Karakteristik File Teks}
\begin{itemize}
  \item Data disimpan dalam bentuk karakter yang dapat dibaca manusia
  \item Karakter khusus seperti newline (\textbackslash n) diterjemahkan secara berbeda antar sistem operasi (Windows: CR+LF, Unix/Linux: LF)
  \item Ukuran file cenderung lebih besar karena representasi numerik sebagai string
  \item Dapat dibuka dan diedit dengan editor teks biasa
  \item Portabel lintas platform dengan konversi line ending yang sesuai
\end{itemize}

\subsection{Deklarasi dan Mode Pembukaan}

\subsubsection{Pascal}
Dalam Pascal, tipe \texttt{text} digunakan untuk file teks. Deklarasi dan operasi dasar:

Program berikut mendemonstrasikan berbagai mode pembukaan file teks di Pascal (write, read, append):

\begin{lstlisting}[language=Pascal, caption={Deklarasi dan mode file teks Pascal}]
var
  f: text;           // Deklarasi variabel file teks
  s: string;
  angka: integer;
begin
  // Mode tulis (w): membuat file baru atau overwrite
  Assign(f, 'output.txt');
  Rewrite(f);                    // Buka untuk menulis
  Writeln(f, 'Baris pertama');
  Writeln(f, 'Angka: ', 42);
  Close(f);
  
  // Mode baca (r): membaca file yang sudah ada
  Assign(f, 'output.txt');
  Reset(f);                      // Buka untuk membaca
  Readln(f, s);
  Writeln('Dibaca: ', s);
  Close(f);
  
  // Mode append (a): menambah di akhir file
  Assign(f, 'output.txt');
  Append(f);                     // Buka untuk menambah
  Writeln(f, 'Baris tambahan');
  Close(f);
end.
\end{lstlisting}

\subsubsection{C}
C menggunakan pointer \texttt{FILE*} dengan mode string untuk file teks:

Program C berikut menunjukkan berbagai mode file (r, w, a, r+) dan penggunaan \texttt{fopen}, \texttt{fprintf}, \texttt{fgets}:

\begin{lstlisting}[language=C, caption={Mode file teks di C}]
#include <stdio.h>

int main() {
  FILE *fp;
  char buffer[256];
  
  // Mode "r": baca (read), file harus ada
  fp = fopen("data.txt", "r");
  if (fp == NULL) {
    perror("Gagal membuka file");
    return 1;
  }
  fgets(buffer, sizeof(buffer), fp);
  printf("Dibaca: %s", buffer);
  fclose(fp);
  
  // Mode "w": tulis (write), file baru atau overwrite
  fp = fopen("output.txt", "w");
  fprintf(fp, "Halo dunia\n");
  fprintf(fp, "Angka: %d\n", 100);
  fclose(fp);
  
  // Mode "a": append, menambah di akhir
  fp = fopen("output.txt", "a");
  fprintf(fp, "Baris tambahan\n");
  fclose(fp);
  
  // Mode "r+": baca dan tulis, file harus ada
  fp = fopen("data.txt", "r+");
  // dapat fread dan fwrite
  fclose(fp);
  
  return 0;
}
\end{lstlisting}

\subsubsection{C++}
C++ menyediakan stream object dengan manipulator yang lebih type-safe:

Program C++ berikut menggunakan \texttt{ifstream}, \texttt{ofstream}, dan \texttt{fstream} dengan berbagai mode untuk operasi file:

\begin{lstlisting}[language=C++, caption={Mode file teks di C++}]
#include <fstream>
#include <string>
#include <iostream>
using namespace std;

int main() {
  // ofstream: output (write)
  ofstream ofs("output.txt");
  if (!ofs.is_open()) {
    cerr << "Gagal membuka file\n";
    return 1;
  }
  ofs << "Baris pertama\n";
  ofs << "Angka: " << 42 << "\n";
  ofs.close();
  
  // ifstream: input (read)
  ifstream ifs("output.txt");
  string line;
  while (getline(ifs, line)) {
    cout << line << "\n";
  }
  ifs.close();
  
  // Mode append: ios::app
  ofstream ofs_append("output.txt", ios::app);
  ofs_append << "Baris tambahan\n";
  ofs_append.close();
  
  // Mode read+write: ios::in | ios::out
  fstream fs("data.txt", ios::in | ios::out);
  // dapat baca dan tulis
  fs.close();
  
  return 0;
}
\end{lstlisting}

\subsection{Fungsi Manipulasi File Teks}

\begin{table}[H]
\centering
\begin{tabular}{|l|l|l|l|}
\hline
\textbf{Operasi} & \textbf{Pascal} & \textbf{C} & \textbf{C++} \\ \hline
Baca baris & \texttt{Readln(f, s)} & \texttt{fgets(buf, n, fp)} & \texttt{getline(ifs, str)} \\ \hline
Tulis baris & \texttt{Writeln(f, s)} & \texttt{fprintf(fp, "\%s\textbackslash n", s)} & \texttt{ofs << s << "\textbackslash n"} \\ \hline
Baca format & \texttt{Read(f, i)} & \texttt{fscanf(fp, "\%d", \&i)} & \texttt{ifs >> i} \\ \hline
Tulis format & \texttt{Write(f, i)} & \texttt{fprintf(fp, "\%d", i)} & \texttt{ofs << i} \\ \hline
Cek EOF & \texttt{Eof(f)} & \texttt{feof(fp)} & \texttt{ifs.eof()} \\ \hline
\end{tabular}
\caption{Perbandingan fungsi manipulasi file teks}
\end{table}

\subsection{Contoh Aplikasi File Teks}

Program berikut mendemonstrasikan parsing file konfigurasi sederhana dengan format key=value, melewati baris komentar:

\begin{lstlisting}[language=C, caption={Membaca file konfigurasi sederhana}]
#include <stdio.h>
#include <string.h>

int read_config(const char *filename) {
  FILE *fp = fopen(filename, "r");
  if (!fp) return -1;
  
  char line[256];
  while (fgets(line, sizeof(line), fp)) {
    // Lewati komentar
    if (line[0] == '#') continue;
    
    // Parse key=value
    char key[64], value[64];
    if (sscanf(line, "%63[^=]=%63s", key, value) == 2) {
      printf("Config: %s = %s\n", key, value);
    }
  }
  
  fclose(fp);
  return 0;
}
\end{lstlisting}

\section{File Binary}

File binary menyimpan data dalam format biner mentah (raw bytes) yang tidak dapat dibaca langsung oleh manusia. Data disimpan dalam representasi internal mesin, seperti integer 32-bit disimpan sebagai 4 byte secara langsung. File binary cocok untuk menyimpan struktur data kompleks, gambar, audio, executable, dan data yang memerlukan efisiensi ruang dan kecepatan akses.

\subsection{Karakteristik File Binary}
\begin{itemize}
  \item Data disimpan dalam format biner mentah (byte sequence)
  \item Tidak ada translasi karakter khusus (newline tetap sebagai byte mentah)
  \item Ukuran file lebih efisien untuk data numerik
  \item Tidak dapat dibaca/diedit dengan editor teks biasa
  \item Perlu perhatian terhadap endianness, alignment, dan portabilitas antar arsitektur
  \item Kecepatan baca/tulis lebih tinggi karena tidak ada konversi format
\end{itemize}

\subsection{Deklarasi dan Mode Pembukaan}

\subsubsection{Pascal}
Pascal menggunakan \texttt{file of <type>} untuk file binary dengan tipe tertentu:

Program Pascal berikut menunjukkan file binary bertipe dan untyped untuk akses byte-level:

\begin{lstlisting}[language=Pascal, caption={Deklarasi dan mode file binary Pascal}]
type
  TRecord = record
    id: integer;
    nilai: real;
    nama: string[50];
  end;

var
  fb: file of TRecord;    // File binary bertipe
  fu: file;               // File binary untyped (byte stream)
  rec: TRecord;
  buffer: array[1..1024] of byte;
  numRead: integer;
begin
  // File bertipe
  Assign(fb, 'data.dat');
  Rewrite(fb);                   // Buka untuk tulis binary
  rec.id := 1;
  rec.nilai := 95.5;
  rec.nama := 'Alice';
  Write(fb, rec);                // Tulis record
  Close(fb);
  
  Reset(fb);                     // Buka untuk baca binary
  Read(fb, rec);                 // Baca record
  Writeln('ID: ', rec.id);
  Close(fb);
  
  // File untyped untuk akses byte-level
  Assign(fu, 'binary.bin');
  Reset(fu, 1);                  // Ukuran record = 1 byte
  BlockRead(fu, buffer, 1024, numRead);
  Close(fu);
end.
\end{lstlisting}

\subsubsection{C}
C menggunakan mode dengan suffix \texttt{b} untuk binary:

Program C berikut mendemonstrasikan operasi file binary (rb, wb, ab, rb+) menggunakan \texttt{fread} dan \texttt{fwrite}:

\begin{lstlisting}[language=C, caption={Mode file binary di C}]
#include <stdio.h>

typedef struct {
  int id;
  double nilai;
  char nama[50];
} Record;

int main() {
  FILE *fp;
  Record rec;
  
  // Mode "rb": read binary
  fp = fopen("data.bin", "rb");
  if (fp) {
    fread(&rec, sizeof(Record), 1, fp);
    printf("ID: %d\n", rec.id);
    fclose(fp);
  }
  
  // Mode "wb": write binary (overwrite)
  fp = fopen("data.bin", "wb");
  rec.id = 1;
  rec.nilai = 95.5;
  snprintf(rec.nama, sizeof(rec.nama), "Alice");
  fwrite(&rec, sizeof(Record), 1, fp);
  fclose(fp);
  
  // Mode "ab": append binary
  fp = fopen("data.bin", "ab");
  rec.id = 2;
  fwrite(&rec, sizeof(Record), 1, fp);
  fclose(fp);
  
  // Mode "rb+": read+write binary
  fp = fopen("data.bin", "rb+");
  // dapat fread dan fwrite pada posisi mana saja
  fclose(fp);
  
  return 0;
}
\end{lstlisting}

\subsubsection{C++}
C++ menggunakan flag \texttt{ios::binary} untuk mode binary:

Program C++ berikut menggunakan stream dengan flag \texttt{ios::binary} untuk read/write data binary:

\begin{lstlisting}[language=C++, caption={Mode file binary di C++}]
#include <fstream>
#include <iostream>
using namespace std;

struct Record {
  int id;
  double nilai;
  char nama[50];
};

int main() {
  Record rec;
  
  // Tulis binary: ios::binary | ios::out
  ofstream ofs("data.bin", ios::binary);
  if (!ofs) {
    cerr << "Gagal buka file\n";
    return 1;
  }
  rec.id = 1;
  rec.nilai = 95.5;
  snprintf(rec.nama, sizeof(rec.nama), "Alice");
  ofs.write(reinterpret_cast<char*>(&rec), sizeof(Record));
  ofs.close();
  
  // Baca binary: ios::binary | ios::in
  ifstream ifs("data.bin", ios::binary);
  ifs.read(reinterpret_cast<char*>(&rec), sizeof(Record));
  cout << "ID: " << rec.id << "\n";
  ifs.close();
  
  // Read+write binary
  fstream fs("data.bin", ios::binary | ios::in | ios::out);
  // dapat seek, read, write
  fs.close();
  
  return 0;
}
\end{lstlisting}

\subsection{Fungsi Manipulasi File Binary}

\begin{table}[H]
\centering
\small
\begin{tabular}{|l|l|l|l|}
\hline
\textbf{Operasi} & \textbf{Pascal} & \textbf{C} & \textbf{C++} \\ \hline
Tulis data & \texttt{Write(f, v)} & \texttt{fwrite(\&v, sz, n, fp)} & \begin{tabular}[t]{@{}l@{}}\texttt{ofs.write}\\\texttt{((char*)\&v, sz)}\end{tabular} \\ \hline
Baca data & \texttt{Read(f, v)} & \texttt{fread(\&v, sz, n, fp)} & \begin{tabular}[t]{@{}l@{}}\texttt{ifs.read}\\\texttt{((char*)\&v, sz)}\end{tabular} \\ \hline
Tulis blok & \begin{tabular}[t]{@{}l@{}}\texttt{BlockWrite}\\\texttt{(f, buf, n)}\end{tabular} & \texttt{fwrite(buf, 1, n, fp)} & \texttt{ofs.write(buf, n)} \\ \hline
Baca blok & \begin{tabular}[t]{@{}l@{}}\texttt{BlockRead}\\\texttt{(f, buf, n)}\end{tabular} & \texttt{fread(buf, 1, n, fp)} & \texttt{ifs.read(buf, n)} \\ \hline
Seek posisi & \texttt{Seek(f, pos)} & \begin{tabular}[t]{@{}l@{}}\texttt{fseek}\\\texttt{(fp, pos, orig)}\end{tabular} & \texttt{fs.seekg(pos)} \\ \hline
Posisi saat ini & \texttt{FilePos(f)} & \texttt{ftell(fp)} & \texttt{fs.tellg()} \\ \hline
Ukuran file & \texttt{FileSize(f)} & seek+ftell & seekg+tellg \\ \hline
\end{tabular}
\caption{Perbandingan fungsi manipulasi file binary}
\end{table}

\subsection{Akses Acak dan Posisi dalam File Binary}

File binary mendukung akses acak (random access) yang memungkinkan membaca/menulis pada posisi arbitrary tanpa harus membaca seluruh file:

Program berikut mendemonstrasikan penggunaan \texttt{fseek} dan \texttt{ftell} untuk navigasi file dan update record pada posisi tertentu:

\begin{lstlisting}[language=C, caption={fseek/ftell untuk akses acak}]
#include <stdio.h>

typedef struct { int id; char nama[20]; } Record;

int main() {
  FILE *fp = fopen("data.bin", "rb+");
  Record rec;
  
  // Pergi ke akhir file untuk ukuran
  fseek(fp, 0, SEEK_END);
  long file_size = ftell(fp);
  int num_records = file_size / sizeof(Record);
  
  // Kembali ke awal
  rewind(fp);  // atau fseek(fp, 0, SEEK_SET);
  
  // Baca record ke-3 (indeks 2)
  fseek(fp, 2 * sizeof(Record), SEEK_SET);
  fread(&rec, sizeof(Record), 1, fp);
  
  // Update record tersebut
  rec.id = 999;
  fseek(fp, -sizeof(Record), SEEK_CUR); // mundur 1 record
  fwrite(&rec, sizeof(Record), 1, fp);
  
  fclose(fp);
  return 0;
}
\end{lstlisting}

Versi C++ menggunakan \texttt{seekg}/\texttt{seekp} dan \texttt{tellg}/\texttt{tellp} untuk operasi yang sama:

\begin{lstlisting}[language=C++, caption={seekg/tellg untuk akses acak}]
#include <fstream>
using namespace std;

struct Record { int id; char nama[20]; };

int main() {
  fstream fs("data.bin", ios::binary | ios::in | ios::out);
  Record rec;
  
  // Ukuran file
  fs.seekg(0, ios::end);
  auto file_size = fs.tellg();
  int num_records = file_size / sizeof(Record);
  
  // Baca record ke-3
  fs.seekg(2 * sizeof(Record), ios::beg);
  fs.read(reinterpret_cast<char*>(&rec), sizeof(Record));
  
  // Update pada posisi yang sama
  rec.id = 999;
  fs.seekp(2 * sizeof(Record), ios::beg);  // seekp untuk write
  fs.write(reinterpret_cast<char*>(&rec), sizeof(Record));
  
  fs.close();
  return 0;
}
\end{lstlisting}

Parameter origin untuk seek:
\begin{itemize}
  \item \texttt{SEEK\_SET} (C) / \texttt{ios::beg} (C++): dari awal file
  \item \texttt{SEEK\_CUR} (C) / \texttt{ios::cur} (C++): dari posisi saat ini
  \item \texttt{SEEK\_END} (C) / \texttt{ios::end} (C++): dari akhir file
\end{itemize}

\subsection{Pertimbangan Portabilitas File Binary}

File binary memerlukan perhatian khusus untuk portabilitas antar platform:

\begin{itemize}
  \item \textbf{Endianness}: Urutan byte integer berbeda antara big-endian dan little-endian. Gunakan fungsi konversi byte order atau format standar (network byte order)
  \item \textbf{Alignment dan padding}: Compiler dapat menambahkan padding pada struct. Gunakan \texttt{\#pragma pack} (C/C++) atau attribute untuk kontrol layout
  \item \textbf{Ukuran tipe data}: \texttt{int}, \texttt{long} dapat berbeda ukuran antar platform. Gunakan tipe fixed-width seperti \texttt{int32\_t}, \texttt{int64\_t}
  \item \textbf{Format serialisasi}: Untuk portabilitas maksimal, gunakan format standar seperti JSON, XML, Protocol Buffers, atau MessagePack
\end{itemize}

\begin{lstlisting}[language=C, caption={Contoh struct dengan padding control}]
#include <stdint.h>

// Tanpa pragma pack, compiler menambahkan padding
struct Unpacked {
  char c;      // 1 byte + 3 byte padding
  int32_t i;   // 4 byte
  char c2;     // 1 byte + 3 byte padding
};  // Total: 12 byte

// Dengan pragma pack, tidak ada padding
#pragma pack(push, 1)
struct Packed {
  char c;      // 1 byte
  int32_t i;   // 4 byte
  char c2;     // 1 byte
};  // Total: 6 byte
#pragma pack(pop)
\end{lstlisting}

\begin{figure}[H]
  \centering
  \begin{tikzpicture}[node distance=1.6cm]
    \tikzstyle{b}=[rectangle, draw, rounded corners, minimum width=2.6cm, minimum height=0.9cm]
    \node[b] (open) {buka file};
    \node[b, right=of open] (proc) {baca/olah/tulis};
    \node[b, right=of proc] (close) {tutup file};
    \draw[->] (open) -- (proc); \draw[->] (proc) -- (close);
  \end{tikzpicture}
  \caption{Alur sederhana pemrosesan berkas}
\end{figure}

\section{Penanganan Kesalahan File}

Operasi file dapat gagal karena berbagai alasan: file tidak ditemukan, izin ditolak, disk penuh, atau media rusak. Penanganan kesalahan yang tepat penting untuk mencegah kehilangan data dan memberikan pengalaman pengguna yang baik.

\subsection{Pemeriksaan Kesalahan}

Program berikut mendemonstrasikan cara memeriksa dan menangani error file operation di C menggunakan \texttt{errno}, \texttt{perror}, dan \texttt{ferror}:

\begin{lstlisting}[language=C, caption={Penanganan kesalahan di C}]
#include <stdio.h>
#include <errno.h>
#include <string.h>

int main() {
  FILE *fp = fopen("data.txt", "r");
  if (fp == NULL) {
    // Gunakan errno dan perror untuk detail error
    fprintf(stderr, "Error: %s\n", strerror(errno));
    perror("fopen failed");
    return 1;
  }
  
  char buffer[256];
  if (fgets(buffer, sizeof(buffer), fp) == NULL) {
    if (feof(fp)) {
      printf("End of file reached\n");
    } else if (ferror(fp)) {
      fprintf(stderr, "Read error occurred\n");
      clearerr(fp);  // clear error flag
    }
  }
  
  fclose(fp);
  return 0;
}
\end{lstlisting}

Di C++, exception dapat digunakan untuk error handling yang lebih terstruktur:

\begin{lstlisting}[language=C++, caption={Penanganan kesalahan di C++}]
#include <fstream>
#include <iostream>
using namespace std;

int main() {
  ifstream ifs("data.txt");
  
  // Cek apakah file berhasil dibuka
  if (!ifs.is_open()) {
    cerr << "Gagal membuka file\n";
    return 1;
  }
  
  // Aktifkan exception untuk error
  ifs.exceptions(ifstream::failbit | ifstream::badbit);
  
  try {
    string line;
    while (getline(ifs, line)) {
      cout << line << "\n";
    }
  } catch (ifstream::failure& e) {
    if (ifs.eof()) {
      cout << "End of file\n";
    } else {
      cerr << "Exception: " << e.what() << "\n";
    }
  }
  
  ifs.close();
  return 0;
}
\end{lstlisting}

\subsection{Tulis Atomik untuk Keamanan Data}

Untuk mencegah korupsi data saat terjadi kegagalan di tengah penulisan, gunakan strategi tulis ke file temporary kemudian rename:

Program berikut mengimplementasikan atomic write dengan menulis ke file temporary terlebih dahulu, kemudian rename untuk menggantikan file tujuan:

\begin{lstlisting}[language=C, caption={Menulis ke file sementara lalu rename (C, POSIX)}]
#include <stdio.h>

int atomic_write(const char* path, const char* data) {
  char temp_path[256];
  snprintf(temp_path, sizeof(temp_path), "%s.tmp", path);
  
  // Tulis ke file temporary
  FILE* f = fopen(temp_path, "w");
  if (!f) {
    perror("Failed to open temp file");
    return -1;
  }
  
  if (fprintf(f, "%s", data) < 0) {
    fclose(f);
    remove(temp_path);
    return -1;
  }
  
  // Flush buffer ke disk
  fflush(f);
  fclose(f);
  
  // Rename atomik (pada filesystem yang mendukung)
  if (rename(temp_path, path) != 0) {
    perror("Failed to rename");
    remove(temp_path);
    return -1;
  }
  
  return 0;
}
\end{lstlisting}

Operasi \texttt{rename} pada filesystem modern (POSIX, NTFS) bersifat atomik, artinya file target akan muncul lengkap atau tidak sama sekali, tidak pernah dalam keadaan setengah tertulis. Ini melindungi dari korupsi data jika program crash atau sistem mati mendadak \parencite{gnu-c-manual}.

\subsection{Best Practices}
\begin{itemize}
  \item Selalu periksa return value dari operasi file
  \item Gunakan \texttt{perror} atau \texttt{strerror(errno)} untuk pesan error informatif
  \item Tutup file dengan \texttt{fclose}/\texttt{close()} dalam semua jalur eksekusi
  \item Pada C++, gunakan RAII (Resource Acquisition Is Initialization) agar file otomatis tertutup
  \item Flush buffer secara eksplisit untuk data kritis
  \item Implementasikan retry logic untuk error transient (seperti \texttt{EINTR})
  \item Gunakan atomic write untuk mencegah korupsi data
\end{itemize}

Strategi buffering dapat meningkatkan throughput tetapi memerlukan sinkronisasi saat menggabungkan I/O dan perhitungan. Pada C++, nonaktifkan sinkronisasi dengan stdio (via \texttt{sync\_with\_stdio(false)}) bila tidak memerlukan interoperabilitas untuk meningkatkan kinerja. Uji integrasi harus mensimulasikan skenario kesalahan untuk memastikan program merespons dengan benar tanpa korupsi data \parencite{cplusplus-io,cpp-reference}.

\section{Rangkuman Materi}
\begin{itemize}
  \item Perbedaan file teks vs biner, karakteristik, dan kasus penggunaan umum.
  \item Mode pembukaan dan operasi dasar lintas Pascal/C/C++ untuk baca/tulis/append.
  \item Akses acak dengan seek/tell; teknik menghitung ukuran dan memodifikasi record di posisi tertentu.
  \item Portabilitas biner: endianness, padding/alignment, dan tipe lebar tetap untuk interoperabilitas.
  \item Penanganan kesalahan: pemeriksaan status, exception pada C++, dan pembersihan sumber daya.
  \item Tulis atomik menggunakan file sementara dan rename untuk mencegah korupsi data.
  \item Praktik kinerja: buffering, sinkronisasi I/O, dan pengujian skenario error.
\end{itemize}

\end{document}
