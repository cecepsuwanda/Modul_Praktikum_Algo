\documentclass[../main.tex]{subfiles}
\begin{document}
\chapter{String dan Karakter}

\section*{Tujuan Praktikum}
Setelah menyelesaikan praktikum ini, mahasiswa diharapkan mampu:
\begin{itemize}
  \item Memahami konsep string sebagai urutan karakter dan perbedaan representasinya di Pascal, C, dan C++
  \item Mendeklarasikan dan menginisialisasi string dengan berbagai cara
  \item Melakukan operasi dasar string (panjang, concatenation, copy, compare)
  \item Menggunakan fungsi-fungsi manipulasi string built-in
  \item Mengakses dan memodifikasi karakter individual dalam string
  \item Memahami perbedaan antara null-terminated string (C) dan managed string (C++, Pascal)
  \item Membuat program pengolahan teks sederhana dengan manipulasi string
\end{itemize}

\section{Pengantar String}

\subsection{Konsep Dasar String}
String adalah urutan karakter yang disimpan secara berurutan di memori, digunakan untuk merepresentasikan teks, kata, kalimat, atau data tekstual lainnya. String merupakan salah satu tipe data paling fundamental dan sering digunakan dalam pemrograman untuk manipulasi teks, input/output, dan pemrosesan data \parencite{pascal-tutorial-wikibooks,iso-c-draft-n1570,cpp-strings,tutorialspoint-c-strings}.

\subsection{Representasi String di Berbagai Bahasa}

Setiap bahasa pemrograman memiliki cara berbeda dalam merepresentasikan dan mengelola string:

\begin{itemize}
  \item \textbf{Pascal:} Menyediakan tipe \texttt{string} built-in yang mengelola panjang secara otomatis. Pascal juga mendukung representasi berbasis array dengan panjang tetap atau dinamis \parencite{pascal-tutorial-wikibooks,free-pascal-docs}.
  
  \item \textbf{C:} String adalah array karakter (\texttt{char[]}) yang diakhiri dengan karakter null (\texttt{'\textbackslash 0'}). Programmer bertanggung jawab penuh untuk manajemen memori dan memastikan null terminator ada \parencite{iso-c-draft-n1570,c-strings-h,tutorialspoint-c-strings}.
  
  \item \textbf{C++:} Menyediakan dua pendekatan: array karakter gaya C dan kelas \texttt{std::\allowbreak string}. Kelas string mengelola memori otomatis, ukuran dinamis, dan menyediakan metode manipulasi yang kaya~\parencite{cpp-strings,cplusplus-string,yuliaagustin-string-cpp}.
\end{itemize}

\subsection{Keuntungan dan Pertimbangan}

\textbf{Keuntungan menggunakan abstraksi string tingkat tinggi:}
\begin{itemize}
  \item Manajemen memori otomatis (tidak perlu khawatir buffer overflow)
  \item Operasi lebih intuitif dan aman
  \item Ukuran dinamis menyesuaikan dengan kebutuhan
  \item API yang kaya untuk manipulasi
\end{itemize}

\textbf{Pertimbangan untuk string gaya C:}
\begin{itemize}
  \item Performa lebih tinggi dalam situasi tertentu
  \item Kompatibilitas dengan API sistem dan library C
  \item Control yang lebih detail atas memori
  \item Memerlukan kehati-hatian ekstra untuk menghindari bug
\end{itemize}

\section{Deklarasi String}

\subsection{Deklarasi String di Pascal}

Pascal menyediakan tipe \texttt{string} yang fleksibel dengan manajemen memori otomatis.

\begin{lstlisting}[language=Pascal, caption={Deklarasi string di Pascal}]
var
  nama: string;                    // String dengan panjang dinamis
  alamat: string[100];             // String dengan panjang maksimum 100
  kota: string[50];                // String dengan panjang maksimum 50
  
  // Array of string
  daftarNama: array[1..10] of string;
  
  // String kosong
  pesan: string = '';
\end{lstlisting}

\textbf{Karakteristik string Pascal:}
\begin{itemize}
  \item \texttt{string} tanpa ukuran: panjang dinamis hingga batas sistem
  \item \texttt{string[n]}: panjang maksimum \texttt{n} karakter
  \item Indeks dimulai dari 1 (karakter pertama pada \texttt{str[1]})
  \item Fungsi \texttt{Length()} memberikan panjang aktual string
  \item Tidak memerlukan null terminator
\end{itemize}

\subsection{Deklarasi String di C}

Di C, string adalah array karakter yang diakhiri dengan \texttt{'\textbackslash 0'}.

\begin{lstlisting}[language=C, caption={Deklarasi string di C}]
#include <stdio.h>

// Deklarasi berbagai cara
char nama[50];                    // Array 50 karakter (49 + null)
char alamat[100] = "Jl. Merdeka"; // Inisialisasi langsung
char kota[] = "Jakarta";          // Ukuran otomatis (8 byte)

// Array of strings
char hari[7][10] = {
  "Senin", "Selasa", "Rabu", "Kamis",
  "Jumat", "Sabtu", "Minggu"
};

// Pointer ke string literal (immutable)
const char *pesan = "Hello World";

// String kosong
char buffer[256] = "";            // String kosong dengan buffer
\end{lstlisting}

\textbf{Poin penting string C:}
\begin{itemize}
  \item Selalu sediakan ruang untuk null terminator (\texttt{'\textbackslash 0'})
  \item Array berukuran \texttt{n} dapat menyimpan maksimal \texttt{n-1} karakter
  \item String literal disimpan di read-only memory
  \item Pointer ke string literal tidak boleh dimodifikasi
  \item Indeks dimulai dari 0
\end{itemize}

\subsection{Deklarasi String di C++}

C++ menyediakan \texttt{std::string} modern dan array karakter gaya C.

\begin{lstlisting}[language=C++, caption={Deklarasi string di C++}]
#include <string>
#include <vector>

// Deklarasi std::string
std::string nama;                      // String kosong
std::string alamat = "Jl. Merdeka";    // Inisialisasi langsung
std::string kota("Jakarta");           // Constructor syntax

// String dengan ukuran awal
std::string buffer(100, ' ');          // 100 spasi

// Array gaya C (kompatibilitas)
char legacyStr[50] = "Hello";

// Vector of strings
std::vector<std::string> daftarNama;

// String view (C++17, read-only)
std::string_view pesan = "Hello World";
\end{lstlisting}

\textbf{Keuntungan \texttt{std::string}:}
\begin{itemize}
  \item Manajemen memori otomatis (RAII)
  \item Ukuran dinamis, tumbuh sesuai kebutuhan
  \item Aman dari buffer overflow
  \item Metode manipulasi yang kaya
  \item Operator overloading (\texttt{+}, \texttt{==}, \texttt{<}, dll)
\end{itemize}

\section{Inisialisasi dan Pengisian String}

\subsection{Inisialisasi String di Pascal}

\begin{lstlisting}[language=Pascal, caption={Inisialisasi string di Pascal}]
var
  nama: string;
  pesan: string;
  alamat: string;
begin
  // Inisialisasi langsung dengan assignment
  nama := 'John Doe';
  
  // Inisialisasi string kosong
  pesan := '';
  
  // Mengisi karakter per karakter
  alamat := 'J';
  alamat := alamat + 'a';
  alamat := alamat + 'k';
  alamat := alamat + 'a';
  alamat := alamat + 'r';
  alamat := alamat + 't';
  alamat := alamat + 'a';  // alamat = "Jakarta"
  
  // Menggunakan string literal dengan quotes
  pesan := 'Hello, World!';
  
  // Escape untuk single quote di dalam string
  pesan := 'It''s a nice day';  // Gunakan double single-quote
end.
\end{lstlisting}

\subsection{Inisialisasi String di C}

\begin{lstlisting}[language=C, caption={Inisialisasi string di C}]
#include <string.h>
#include <stdio.h>

int main() {
  // Inisialisasi saat deklarasi
  char nama[] = "John Doe";
  
  // Inisialisasi array karakter
  char alamat[50] = "Jl. Merdeka No. 1";
  
  // Inisialisasi karakter per karakter
  char kota[20];
  kota[0] = 'J';
  kota[1] = 'a';
  kota[2] = 'k';
  kota[3] = 'a';
  kota[4] = 'r';
  kota[5] = 't';
  kota[6] = 'a';
  kota[7] = '\0';  // PENTING: null terminator
  
  // Menggunakan strcpy untuk copy string
  char pesan[100];
  strcpy(pesan, "Hello, World!");
  
  // Menggunakan strncpy (lebih aman)
  char buffer[50];
  strncpy(buffer, "Safe copy", sizeof(buffer) - 1);
  buffer[sizeof(buffer) - 1] = '\0';  // Pastikan null-terminated
  
  // Inisialisasi dengan sprintf
  char formatted[100];
  sprintf(formatted, "Umur: %d tahun", 25);
  
  return 0;
}
\end{lstlisting}

\subsection{Inisialisasi String di C++}

\begin{lstlisting}[language=C++, caption={Inisialisasi string di C++}]
#include <string>
#include <iostream>

int main() {
  // Inisialisasi dengan assignment
  std::string nama = "John Doe";
  
  // Inisialisasi dengan constructor
  std::string alamat("Jl. Merdeka No. 1");
  
  // Inisialisasi string kosong
  std::string pesan;
  
  // Assignment setelah deklarasi
  pesan = "Hello, World!";
  
  // Inisialisasi dengan karakter berulang
  std::string garis(50, '-');  // 50 karakter '-'
  
  // Inisialisasi dari substring
  std::string full = "Hello World";
  std::string sub(full, 0, 5);  // "Hello"
  
  // Inisialisasi dari array C
  char cstr[] = "Legacy string";
  std::string modern(cstr);
  
  // Inisialisasi dengan move semantics
  std::string temp = "Temporary";
  std::string moved = std::move(temp);
  
  // Uniform initialization (C++11)
  std::string uniform{"Modern C++"};
  
  return 0;
}
\end{lstlisting}

\section{Input String}

\subsection{Input String di Pascal}

\begin{lstlisting}[language=Pascal, caption={Input string di Pascal}]
var
  nama: string;
  alamat: string;
  umur: integer;
begin
  // Input dengan Readln (membaca satu baris)
  Write('Masukkan nama: ');
  Readln(nama);
  
  // Input string panjang
  Write('Masukkan alamat lengkap: ');
  Readln(alamat);
  
  // Input dengan validasi
  Write('Masukkan umur: ');
  Readln(umur);
  
  // Menampilkan hasil
  Writeln('Nama: ', nama);
  Writeln('Alamat: ', alamat);
  Writeln('Umur: ', umur, ' tahun');
end.
\end{lstlisting}

\subsection{Input String di C}

\begin{lstlisting}[language=C, caption={Input string di C}]
#include <stdio.h>
#include <string.h>

int main() {
  char nama[50];
  char alamat[100];
  int umur;
  
  // Menggunakan scanf untuk satu kata
  printf("Masukkan nama depan: ");
  scanf("%49s", nama);  // Batasi input, cegah overflow
  
  // Clear buffer
  while (getchar() != '\n');
  
  // Menggunakan fgets untuk satu baris (lebih aman)
  printf("Masukkan alamat: ");
  fgets(alamat, sizeof(alamat), stdin);
  
  // Hapus newline di akhir jika ada
  size_t len = strlen(alamat);
  if (len > 0 && alamat[len-1] == '\n') {
    alamat[len-1] = '\0';
  }
  
  // Input angka
  printf("Masukkan umur: ");
  scanf("%d", &umur);
  
  // Tampilkan hasil
  printf("\nData yang dimasukkan:\n");
  printf("Nama: %s\n", nama);
  printf("Alamat: %s\n", alamat);
  printf("Umur: %d tahun\n", umur);
  
  return 0;
}
\end{lstlisting}

\textbf{Catatan penting untuk input string di C:}
\begin{itemize}
  \item \texttt{scanf("\%s")} berhenti di whitespace (spasi, tab, newline)
  \item \texttt{fgets()} membaca seluruh baris termasuk newline
  \item Selalu batasi input untuk menghindari buffer overflow
  \item \texttt{gets()} TIDAK aman dan sudah deprecated, jangan gunakan
  \item Perlu membersihkan buffer input setelah \texttt{scanf()}
\end{itemize}

\subsection{Input String di C++}

\begin{lstlisting}[language=C++, caption={Input string di C++}]
#include <iostream>
#include <string>
#include <limits>

int main() {
  std::string nama;
  std::string alamat;
  int umur;
  
  // Input satu kata dengan cin
  std::cout << "Masukkan nama depan: ";
  std::cin >> nama;
  
  // Clear buffer dan ignore newline
  std::cin.ignore(std::numeric_limits<std::streamsize>::max(), '\n');
  
  // Input satu baris dengan getline (lebih fleksibel)
  std::cout << "Masukkan alamat lengkap: ";
  std::getline(std::cin, alamat);
  
  // Input angka
  std::cout << "Masukkan umur: ";
  std::cin >> umur;
  
  // Validasi input
  if (std::cin.fail()) {
    std::cerr << "Input tidak valid!\n";
    return 1;
  }
  
  // Tampilkan hasil
  std::cout << "\nData yang dimasukkan:\n";
  std::cout << "Nama: " << nama << "\n";
  std::cout << "Alamat: " << alamat << "\n";
  std::cout << "Umur: " << umur << " tahun\n";
  
  return 0;
}
\end{lstlisting}

\textbf{Tips input string di C++:}
\begin{itemize}
  \item \texttt{std::cin >> str} membaca hingga whitespace
  \item \texttt{std::getline(std::cin, str)} membaca seluruh baris
  \item Gunakan \texttt{std::cin.ignore()} untuk membersihkan buffer
  \item \texttt{std::string} mengelola memori otomatis, tidak ada buffer overflow
  \item Periksa \texttt{std::cin.fail()} untuk validasi input
\end{itemize}

\section{Output String}

\subsection{Output String di Pascal}

\begin{lstlisting}[language=Pascal, caption={Output string di Pascal}]
var
  nama: string;
  umur: integer;
  nilai: real;
begin
  nama := 'John Doe';
  umur := 25;
  nilai := 87.5;
  
  // Output sederhana
  Writeln('Nama: ', nama);
  
  // Output dengan format
  Writeln('Umur: ', umur, ' tahun');
  Writeln('Nilai: ', nilai:0:2);  // 2 desimal
  
  // Output tanpa newline
  Write('Nama: ');
  Write(nama);
  Writeln;
  
  // Concatenation dalam output
  Writeln('Halo, ' + nama + '!');
  
  // Output dengan lebar field
  Writeln(nama:20);        // Right-aligned, lebar 20
  Writeln('Total':10, nilai:10:2);
end.
\end{lstlisting}

\subsection{Output String di C}

\begin{lstlisting}[language=C, caption={Output string di C}]
#include <stdio.h>

int main() {
  char nama[] = "John Doe";
  int umur = 25;
  double nilai = 87.5;
  
  // Output dengan printf
  printf("Nama: %s\n", nama);
  printf("Umur: %d tahun\n", umur);
  printf("Nilai: %.2f\n", nilai);
  
  // Output dengan format lebar
  printf("%-20s %3d %6.2f\n", nama, umur, nilai);
  //      |        |   |
  //      left     right aligned
  //      aligned  
  
  // Output dengan puts (otomatis newline)
  puts("=== Data Mahasiswa ===");
  puts(nama);
  
  // Output ke string (sprintf)
  char buffer[100];
  sprintf(buffer, "Nama: %s, Umur: %d", nama, umur);
  printf("%s\n", buffer);
  
  // Output karakter per karakter
  for (int i = 0; nama[i] != '\0'; i++) {
    putchar(nama[i]);
  }
  putchar('\n');
  
  return 0;
}
\end{lstlisting}

\textbf{Format specifier penting:}
\begin{itemize}
  \item \texttt{\%s}: string
  \item \texttt{\%d}: integer desimal
  \item \texttt{\%f}: float/double
  \item \texttt{\%.2f}: float dengan 2 desimal
  \item \texttt{\%10s}: string dengan lebar 10 (right-aligned)
  \item \texttt{\%-10s}: string dengan lebar 10 (left-aligned)
\end{itemize}

\subsection{Output String di C++}

\begin{lstlisting}[language=C++, caption={Output string di C++}]
#include <iostream>
#include <string>
#include <iomanip>

int main() {
  std::string nama = "John Doe";
  int umur = 25;
  double nilai = 87.5;
  
  // Output sederhana dengan cout
  std::cout << "Nama: " << nama << "\n";
  std::cout << "Umur: " << umur << " tahun\n";
  std::cout << "Nilai: " << nilai << "\n";
  
  // Output dengan manipulator format
  std::cout << std::fixed << std::setprecision(2);
  std::cout << "Nilai: " << nilai << "\n";
  
  // Output dengan width dan alignment
  std::cout << std::left << std::setw(20) << nama;
  std::cout << std::right << std::setw(5) << umur;
  std::cout << std::setw(10) << nilai << "\n";
  
  // Output dengan endl vs \n
  std::cout << "Baris 1" << std::endl;  // Flush buffer
  std::cout << "Baris 2" << "\n";       // Tidak flush
  
  // String concatenation dalam output
  std::cout << "Halo, " + nama + "!\n";
  
  // Output ke string (stringstream)
  std::ostringstream oss;
  oss << "Nama: " << nama << ", Umur: " << umur;
  std::string hasil = oss.str();
  std::cout << hasil << "\n";
  
  return 0;
}
\end{lstlisting}

\textbf{Manipulator I/O penting:}
\begin{itemize}
  \item \texttt{std::endl}: newline + flush buffer
  \item \texttt{std::setw(n)}: set field width
  \item \texttt{std::left}, \texttt{std::right}: alignment
  \item \texttt{std::setprecision(n)}: set decimal precision
  \item \texttt{std::fixed}: fixed-point notation
  \item \texttt{std::scientific}: scientific notation
\end{itemize}

\section{Operasi Dasar String}

\subsection{Panjang String}

\textbf{Pascal:}
\begin{lstlisting}[language=Pascal, caption={Panjang string di Pascal}]
var
  teks: string;
  panjang: integer;
begin
  teks := 'Hello World';
  panjang := Length(teks);  // panjang = 11
  Writeln('Panjang: ', panjang);
end.
\end{lstlisting}

\textbf{C:}
\begin{lstlisting}[language=C, caption={Panjang string di C}]
#include <string.h>
#include <stdio.h>

int main() {
  char teks[] = "Hello World";
  size_t panjang = strlen(teks);  // panjang = 11
  printf("Panjang: %zu\n", panjang);
  return 0;
}
\end{lstlisting}

\textbf{C++:}
\begin{lstlisting}[language=C++, caption={Panjang string di C++}]
#include <string>
#include <iostream>

int main() {
  std::string teks = "Hello World";
  size_t panjang = teks.length();  // atau teks.size()
  std::cout << "Panjang: " << panjang << "\n";
  return 0;
}
\end{lstlisting}

\subsection{Konkatenasi (Penggabungan) String}

\textbf{Pascal:}
\begin{lstlisting}[language=Pascal, caption={Konkatenasi di Pascal}]
var
  str1, str2, hasil: string;
begin
  str1 := 'Hello';
  str2 := 'World';
  
  // Menggunakan operator +
  hasil := str1 + ' ' + str2;  // "Hello World"
  
  // Menggunakan Concat
  hasil := Concat(str1, ' ', str2);  // "Hello World"
  
  Writeln(hasil);
end.
\end{lstlisting}

\textbf{C:}
\begin{lstlisting}[language=C, caption={Konkatenasi di C}]
#include <string.h>
#include <stdio.h>

int main() {
  char str1[50] = "Hello";
  char str2[] = "World";
  
  // Menggunakan strcat (modifikasi str1)
  strcat(str1, " ");
  strcat(str1, str2);  // str1 = "Hello World"
  
  // Menggunakan strncat (lebih aman)
  char hasil[100] = "Hello";
  strncat(hasil, " ", sizeof(hasil) - strlen(hasil) - 1);
  strncat(hasil, str2, sizeof(hasil) - strlen(hasil) - 1);
  
  printf("%s\n", hasil);
  return 0;
}
\end{lstlisting}

\textbf{C++:}
\begin{lstlisting}[language=C++, caption={Konkatenasi di C++}]
#include <string>
#include <iostream>

int main() {
  std::string str1 = "Hello";
  std::string str2 = "World";
  
  // Menggunakan operator +
  std::string hasil = str1 + " " + str2;
  
  // Menggunakan operator +=
  str1 += " ";
  str1 += str2;  // str1 = "Hello World"
  
  // Menggunakan append()
  std::string gabung = "Hello";
  gabung.append(" ");
  gabung.append(str2);
  
  std::cout << hasil << "\n";
  return 0;
}
\end{lstlisting}

\subsection{Perbandingan String}

\textbf{Pascal:}
\begin{lstlisting}[language=Pascal, caption={Perbandingan di Pascal}]
var
  str1, str2: string;
begin
  str1 := 'Apple';
  str2 := 'Banana';
  
  // Perbandingan menggunakan operator
  if str1 = str2 then
    Writeln('String sama')
  else
    Writeln('String berbeda');
  
  // Perbandingan leksikografis
  if str1 < str2 then
    Writeln('str1 lebih kecil')
  else if str1 > str2 then
    Writeln('str1 lebih besar');
  
  // Menggunakan CompareStr (case-sensitive)
  case CompareStr(str1, str2) of
    -1: Writeln('str1 < str2');
     0: Writeln('str1 = str2');
     1: Writeln('str1 > str2');
  end;
end.
\end{lstlisting}

\textbf{C:}
\begin{lstlisting}[language=C, caption={Perbandingan di C}]
#include <string.h>
#include <stdio.h>

int main() {
  char str1[] = "Apple";
  char str2[] = "Banana";
  
  // strcmp mengembalikan: 0 (sama), <0 (str1<str2), >0 (str1>str2)
  int hasil = strcmp(str1, str2);
  
  if (hasil == 0) {
    printf("String sama\n");
  } else if (hasil < 0) {
    printf("str1 lebih kecil\n");
  } else {
    printf("str1 lebih besar\n");
  }
  
  // strncmp untuk membandingkan n karakter pertama
  if (strncmp(str1, str2, 3) == 0) {
    printf("3 karakter pertama sama\n");
  }
  
  // strcasecmp untuk case-insensitive (non-standard)
  // if (strcasecmp(str1, str2) == 0) { ... }
  
  return 0;
}
\end{lstlisting}

\textbf{C++:}
\begin{lstlisting}[language=C++, caption={Perbandingan di C++}]
#include <string>
#include <iostream>
#include <algorithm>

int main() {
  std::string str1 = "Apple";
  std::string str2 = "Banana";
  
  // Menggunakan operator
  if (str1 == str2) {
    std::cout << "String sama\n";
  }
  
  if (str1 < str2) {
    std::cout << "str1 lebih kecil\n";
  }
  
  // Menggunakan compare()
  int hasil = str1.compare(str2);
  // hasil: 0 (sama), <0 (str1<str2), >0 (str1>str2)
  
  // Case-insensitive comparison (manual)
  std::string s1 = str1, s2 = str2;
  std::transform(s1.begin(), s1.end(), s1.begin(), ::tolower);
  std::transform(s2.begin(), s2.end(), s2.begin(), ::tolower);
  if (s1 == s2) {
    std::cout << "Sama (case-insensitive)\n";
  }
  
  return 0;
}
\end{lstlisting}

\section{Manipulasi String Lanjutan}

\subsection{Substring (Mengambil Bagian String)}

\textbf{Pascal:}
\begin{lstlisting}[language=Pascal, caption={Substring di Pascal}]
var
  teks: string;
  bagian: string;
begin
  teks := 'Hello World';
  
  // Copy(str, start, length)
  bagian := Copy(teks, 1, 5);      // "Hello"
  bagian := Copy(teks, 7, 5);      // "World"
  bagian := Copy(teks, 1, 11);     // "Hello World"
  
  Writeln(bagian);
end.
\end{lstlisting}

\textbf{C:}
\begin{lstlisting}[language=C, caption={Substring di C}]
#include <string.h>
#include <stdio.h>

int main() {
  char teks[] = "Hello World";
  char bagian[20];
  
  // Menggunakan strncpy
  strncpy(bagian, teks, 5);      // Copy 5 karakter
  bagian[5] = '\0';               // Tambah null terminator
  printf("%s\n", bagian);         // "Hello"
  
  // Substring dari posisi tertentu
  strncpy(bagian, &teks[6], 5);   // Mulai dari index 6
  bagian[5] = '\0';
  printf("%s\n", bagian);         // "World"
  
  return 0;
}
\end{lstlisting}

\textbf{C++:}
\begin{lstlisting}[language=C++, caption={Substring di C++}]
#include <string>
#include <iostream>

int main() {
  std::string teks = "Hello World";
  
  // substr(pos, len)
  std::string bagian = teks.substr(0, 5);   // "Hello"
  std::cout << bagian << "\n";
  
  // substr dari posisi tertentu
  bagian = teks.substr(6, 5);               // "World"
  std::cout << bagian << "\n";
  
  // substr tanpa length (sampai akhir)
  bagian = teks.substr(6);                  // "World"
  std::cout << bagian << "\n";
  
  return 0;
}
\end{lstlisting}

\subsection{Pencarian dalam String}

\textbf{Pascal:}
\begin{lstlisting}[language=Pascal, caption={Pencarian di Pascal}]
var
  teks: string;
  posisi: integer;
begin
  teks := 'Hello World';
  
  // Pos(substring, string) - mengembalikan posisi (1-based)
  posisi := Pos('World', teks);   // posisi = 7
  posisi := Pos('xyz', teks);     // posisi = 0 (tidak ketemu)
  
  if posisi > 0 then
    Writeln('Ditemukan di posisi: ', posisi)
  else
    Writeln('Tidak ditemukan');
end.
\end{lstlisting}

\textbf{C:}
\begin{lstlisting}[language=C, caption={Pencarian di C}]
#include <string.h>
#include <stdio.h>

int main() {
  char teks[] = "Hello World";
  char *posisi;
  
  // strstr() - mengembalikan pointer ke substring
  posisi = strstr(teks, "World");
  if (posisi != NULL) {
    int index = posisi - teks;  // Hitung index
    printf("Ditemukan di index: %d\n", index);  // 6
  } else {
    printf("Tidak ditemukan\n");
  }
  
  // strchr() - mencari karakter tunggal
  posisi = strchr(teks, 'W');
  if (posisi != NULL) {
    printf("Karakter 'W' di index: %ld\n", posisi - teks);
  }
  
  return 0;
}
\end{lstlisting}

\textbf{C++:}
\begin{lstlisting}[language=C++, caption={Pencarian di C++}]
#include <string>
#include <iostream>

int main() {
  std::string teks = "Hello World";
  
  // find() - mengembalikan posisi atau string::npos
  size_t posisi = teks.find("World");
  if (posisi != std::string::npos) {
    std::cout << "Ditemukan di posisi: " << posisi << "\n";
  }
  
  // find karakter tunggal
  posisi = teks.find('W');
  std::cout << "Karakter 'W' di: " << posisi << "\n";
  
  // rfind() - cari dari belakang
  posisi = teks.rfind('o');
  std::cout << "'o' terakhir di: " << posisi << "\n";
  
  // find_first_of() - cari salah satu karakter
  posisi = teks.find_first_of("aeiou");
  std::cout << "Vokal pertama di: " << posisi << "\n";
  
  return 0;
}
\end{lstlisting}

\subsection{Penggantian Substring}

\textbf{Pascal:}
\begin{lstlisting}[language=Pascal, caption={Replace di Pascal}]
var
  teks: string;
  hasil: string;
  posisi: integer;
begin
  teks := 'Hello World';
  
  // Manual replacement
  posisi := Pos('World', teks);
  if posisi > 0 then
  begin
    Delete(teks, posisi, 5);  // Hapus "World"
    Insert('Pascal', teks, posisi);  // Sisipkan "Pascal"
  end;
  Writeln(teks);  // "Hello Pascal"
  
  // Menggunakan StringReplace (Free Pascal)
  teks := 'Hello World World';
  hasil := StringReplace(teks, 'World', 'Pascal', [rfReplaceAll]);
  Writeln(hasil);  // "Hello Pascal Pascal"
end.
\end{lstlisting}

\textbf{C:}
\begin{lstlisting}[language=C, caption={Replace di C (manual)}]
#include <string.h>
#include <stdio.h>

void replace(char *str, const char *old, const char *new) {
  char buffer[1000];
  char *pos;
  
  if (!(pos = strstr(str, old)))
    return;
  
  // Copy sebelum substring
  strncpy(buffer, str, pos - str);
  buffer[pos - str] = '\0';
  
  // Tambah replacement
  strcat(buffer, new);
  
  // Tambah sisanya
  strcat(buffer, pos + strlen(old));
  
  // Copy kembali
  strcpy(str, buffer);
}

int main() {
  char teks[100] = "Hello World";
  replace(teks, "World", "C");
  printf("%s\n", teks);  // "Hello C"
  return 0;
}
\end{lstlisting}

\textbf{C++:}
\begin{lstlisting}[language=C++, caption={Replace di C++}]
#include <string>
#include <iostream>

int main() {
  std::string teks = "Hello World";
  
  // replace(pos, len, newstr)
  size_t posisi = teks.find("World");
  if (posisi != std::string::npos) {
    teks.replace(posisi, 5, "C++");
  }
  std::cout << teks << "\n";  // "Hello C++"
  
  // Replace all occurrences
  std::string str = "abc abc abc";
  std::string dari = "abc";
  std::string ke = "xyz";
  
  size_t pos = 0;
  while ((pos = str.find(dari, pos)) != std::string::npos) {
    str.replace(pos, dari.length(), ke);
    pos += ke.length();
  }
  std::cout << str << "\n";  // "xyz xyz xyz"
  
  return 0;
}
\end{lstlisting}

\subsection{Konversi Case (Upper/Lower)}

\textbf{Pascal:}
\begin{lstlisting}[language=Pascal, caption={Case conversion di Pascal}]
uses SysUtils;

var
  teks: string;
begin
  teks := 'Hello World';
  
  // Uppercase
  Writeln(UpperCase(teks));  // "HELLO WORLD"
  
  // Lowercase
  Writeln(LowerCase(teks));  // "hello world"
end.
\end{lstlisting}

\textbf{C:}
\begin{lstlisting}[language=C, caption={Case conversion di C}]
#include <ctype.h>
#include <stdio.h>
#include <string.h>

void toUpper(char *str) {
  for (int i = 0; str[i]; i++) {
    str[i] = toupper(str[i]);
  }
}

void toLower(char *str) {
  for (int i = 0; str[i]; i++) {
    str[i] = tolower(str[i]);
  }
}

int main() {
  char teks1[50] = "Hello World";
  char teks2[50] = "Hello World";
  
  toUpper(teks1);
  printf("%s\n", teks1);  // "HELLO WORLD"
  
  toLower(teks2);
  printf("%s\n", teks2);  // "hello world"
  
  return 0;
}
\end{lstlisting}

\textbf{C++:}
\begin{lstlisting}[language=C++, caption={Case conversion di C++}]
#include <string>
#include <algorithm>
#include <iostream>

int main() {
  std::string teks = "Hello World";
  
  // To uppercase
  std::transform(teks.begin(), teks.end(), teks.begin(), 
                 ::toupper);
  std::cout << teks << "\n";  // "HELLO WORLD"
  
  // To lowercase
  teks = "Hello World";
  std::transform(teks.begin(), teks.end(), teks.begin(), 
                 ::tolower);
  std::cout << teks << "\n";  // "hello world"
  
  return 0;
}
\end{lstlisting}

\section{Pengolahan Karakter \& Escape Sequences}

\subsection{Karakter Khusus dan Escape Sequences}

Karakter khusus diwakili dengan escape sequences menggunakan backslash (\textbackslash).

\begin{table}[H]
  \centering
  \caption{Escape sequences umum}
  \begin{tabular}{@{}lll@{}}
    \toprule
    Notasi & Arti & Deskripsi \\
    \midrule
    \texttt{\textbackslash n} & Newline & Baris baru (Line Feed) \\
    \texttt{\textbackslash r} & Carriage Return & Kembali ke awal baris \\
    \texttt{\textbackslash t} & Tab & Horizontal tab \\
    \texttt{\textbackslash\textbackslash} & Backslash & Karakter backslash literal \\
    \texttt{\textbackslash'} & Single quote & Karakter single quote \\
    \texttt{\textbackslash"} & Double quote & Karakter double quote \\
    \texttt{\textbackslash 0} & Null & Karakter null (string terminator di C) \\
    \texttt{\textbackslash b} & Backspace & Mundur satu karakter \\
    \texttt{\textbackslash f} & Form feed & Ganti halaman \\
    \bottomrule
  \end{tabular}
\end{table}

\subsection{Contoh Penggunaan Escape Sequences}

\textbf{Pascal:}
\begin{lstlisting}[language=Pascal, caption={Escape sequences di Pascal}]
begin
  // Single quote dalam string (double single quote)
  Writeln('It''s a nice day');
  
  // Menggunakan #10 untuk newline, #13 untuk CR
  Writeln('Baris 1'#10'Baris 2');
  
  // Tab dengan #9
  Writeln('Kolom1'#9'Kolom2'#9'Kolom3');
  
  // Kombinasi
  Writeln('Nama:'#9'John Doe'#10'Umur:'#9'25');
end.
\end{lstlisting}

\textbf{C/C++:}
\begin{lstlisting}[language=C, caption={Escape sequences di C/C++}]
#include <stdio.h>

int main() {
  // Newline
  printf("Baris 1\nBaris 2\n");
  
  // Tab
  printf("Kolom1\tKolom2\tKolom3\n");
  
  // Quote marks
  printf("He said, \"Hello!\"\n");
  printf("It's a nice day\n");
  
  // Backslash
  printf("Path: C:\\Users\\Admin\n");
  
  // Kombinasi
  printf("Nama:\tJohn Doe\nUmur:\t25\n");
  
  return 0;
}
\end{lstlisting}

\subsection{Klasifikasi Karakter}

\textbf{Fungsi \texttt{<ctype.h>} di C:}
\begin{lstlisting}[language=C, caption={Klasifikasi karakter di C}]
#include <ctype.h>
#include <stdio.h>

int main() {
  char ch = 'A';
  
  // Cek tipe karakter
  if (isalpha(ch))  printf("%c adalah huruf\n", ch);
  if (isupper(ch))  printf("%c adalah huruf besar\n", ch);
  if (islower(ch))  printf("%c adalah huruf kecil\n", ch);
  if (isdigit(ch))  printf("%c adalah digit\n", ch);
  if (isalnum(ch))  printf("%c adalah alphanumeric\n", ch);
  if (isspace(ch))  printf("%c adalah whitespace\n", ch);
  if (ispunct(ch))  printf("%c adalah punctuation\n", ch);
  
  // Konversi
  printf("Uppercase: %c\n", toupper('a'));  // 'A'
  printf("Lowercase: %c\n", tolower('A'));  // 'a'
  
  return 0;
}
\end{lstlisting}

\section{Fungsi/Prosedur String Library}

\subsection{Fungsi String di C (\texttt{<string.h>})}

\begin{table}[H]
  \centering
  \caption{Fungsi-fungsi penting di \texttt{<string.h>}}
  \begin{tabular}{@{}lp{8cm}@{}}
    \toprule
    Fungsi & Deskripsi \\
    \midrule
    \texttt{strlen(s)} & Menghitung panjang string \\
    \texttt{strcpy(dest, src)} & Copy string (unsafe) \\
    \texttt{strncpy(dest, src, n)} & Copy n karakter (safer) \\
    \texttt{strcat(dest, src)} & Concatenate string (unsafe) \\
    \texttt{strncat(dest, src, n)} & Concatenate n karakter (safer) \\
    \texttt{strcmp(s1, s2)} & Bandingkan string \\
    \texttt{strncmp(s1, s2, n)} & Bandingkan n karakter \\
    \texttt{strchr(s, c)} & Cari karakter pertama \\
    \texttt{strrchr(s, c)} & Cari karakter terakhir \\
    \texttt{strstr(haystack, needle)} & Cari substring \\
    \texttt{strtok(s, delim)} & Tokenize string \\
    \texttt{memset(s, c, n)} & Set n bytes dengan nilai c \\
    \texttt{memcpy(dest, src, n)} & Copy n bytes \\
    \bottomrule
  \end{tabular}
\end{table}

\subsection{Contoh Penggunaan Library C}

\begin{lstlisting}[language=C, caption={Contoh fungsi string C}]
#include <string.h>
#include <stdio.h>

int main() {
  char str1[50] = "Hello";
  char str2[50];
  char str3[100];
  
  // Copy string
  strcpy(str2, str1);
  printf("Copied: %s\n", str2);
  
  // Concatenate
  strcpy(str3, str1);
  strcat(str3, " World");
  printf("Concatenated: %s\n", str3);
  
  // Compare
  if (strcmp(str1, str2) == 0) {
    printf("Strings are equal\n");
  }
  
  // Search
  char *pos = strstr(str3, "World");
  if (pos != NULL) {
    printf("Found at position: %ld\n", pos - str3);
  }
  
  // Tokenize
  char sentence[] = "This is a test";
  char *token = strtok(sentence, " ");
  while (token != NULL) {
    printf("Token: %s\n", token);
    token = strtok(NULL, " ");
  }
  
  return 0;
}
\end{lstlisting}

\subsection{Metode \texttt{std::string} di C++}

\begin{table}[H]
  \centering
  \caption{Metode penting \texttt{std::string}}
  \begin{tabular}{@{}lp{8cm}@{}}
    \toprule
    Metode & Deskripsi \\
    \midrule
    \texttt{length()}, \texttt{size()} & Ukuran string \\
    \texttt{empty()} & Cek apakah string kosong \\
    \texttt{clear()} & Kosongkan string \\
    \texttt{append(s)} & Tambah string di akhir \\
    \texttt{insert(pos, s)} & Sisipkan string di posisi \\
    \texttt{erase(pos, len)} & Hapus substring \\
    \texttt{replace(pos, len, s)} & Ganti substring \\
    \texttt{substr(pos, len)} & Ambil substring \\
    \texttt{find(s)} & Cari substring \\
    \texttt{rfind(s)} & Cari dari belakang \\
    \texttt{compare(s)} & Bandingkan string \\
    \texttt{c\_str()} & Konversi ke C-string \\
    \texttt{at(i)} & Akses karakter dengan bounds checking \\
    \texttt{front()}, \texttt{back()} & Karakter pertama/terakhir \\
    \bottomrule
  \end{tabular}
\end{table}

\subsection{Contoh Penggunaan \texttt{std::string}}

\begin{lstlisting}[language=C++, caption={Contoh metode std::string}]
#include <string>
#include <iostream>

int main() {
  std::string str = "Hello World";
  
  // Size operations
  std::cout << "Length: " << str.length() << "\n";
  std::cout << "Empty: " << str.empty() << "\n";
  
  // Modification
  str.append(" C++");
  std::cout << str << "\n";  // "Hello World C++"
  
  str.insert(5, ",");
  std::cout << str << "\n";  // "Hello, World C++"
  
  str.erase(5, 1);  // Hapus koma
  std::cout << str << "\n";  // "Hello World C++"
  
  // Substring
  std::string sub = str.substr(0, 5);
  std::cout << "Substring: " << sub << "\n";  // "Hello"
  
  // Search
  size_t pos = str.find("World");
  if (pos != std::string::npos) {
    std::cout << "Found at: " << pos << "\n";
  }
  
  // Replace
  str.replace(pos, 5, "C++");
  std::cout << str << "\n";  // "Hello C++ C++"
  
  // Access
  std::cout << "First char: " << str.front() << "\n";
  std::cout << "Last char: " << str.back() << "\n";
  std::cout << "At position 0: " << str.at(0) << "\n";
  
  // Convert to C-string
  const char* cstr = str.c_str();
  printf("As C-string: %s\n", cstr);
  
  return 0;
}
\end{lstlisting}

\section{Contoh Program Lengkap}

\subsection{Program Pengolahan Teks di Pascal}

\begin{lstlisting}[language=Pascal, caption={Program analisis teks di Pascal}]
program AnalisTeks;
uses SysUtils;

var
  teks: string;
  kata: string;
  i, panjang: integer;
  hurufBesar, hurufKecil, angka, spasi: integer;

begin
  hurufBesar := 0;
  hurufKecil := 0;
  angka := 0;
  spasi := 0;
  
  // Input teks
  Write('Masukkan teks: ');
  Readln(teks);
  
  // Analisis karakter
  panjang := Length(teks);
  for i := 1 to panjang do
  begin
    if teks[i] in ['A'..'Z'] then
      Inc(hurufBesar)
    else if teks[i] in ['a'..'z'] then
      Inc(hurufKecil)
    else if teks[i] in ['0'..'9'] then
      Inc(angka)
    else if teks[i] = ' ' then
      Inc(spasi);
  end;
  
  // Tampilkan hasil
  Writeln;
  Writeln('=== Hasil Analisis ===');
  Writeln('Panjang teks: ', panjang);
  Writeln('Huruf besar: ', hurufBesar);
  Writeln('Huruf kecil: ', hurufKecil);
  Writeln('Angka: ', angka);
  Writeln('Spasi: ', spasi);
  Writeln;
  Writeln('Uppercase: ', UpperCase(teks));
  Writeln('Lowercase: ', LowerCase(teks));
end.
\end{lstlisting}

\subsection{Program Manipulasi String di C}

\begin{lstlisting}[language=C, caption={Program manipulasi string di C}]
#include <stdio.h>
#include <string.h>
#include <ctype.h>

void countChars(const char *str, int *upper, int *lower, 
                int *digit, int *space) {
  *upper = *lower = *digit = *space = 0;
  
  for (int i = 0; str[i]; i++) {
    if (isupper(str[i])) (*upper)++;
    else if (islower(str[i])) (*lower)++;
    else if (isdigit(str[i])) (*digit)++;
    else if (isspace(str[i])) (*space)++;
  }
}

void toUpperStr(char *str) {
  for (int i = 0; str[i]; i++) {
    str[i] = toupper(str[i]);
  }
}

void toLowerStr(char *str) {
  for (int i = 0; str[i]; i++) {
    str[i] = tolower(str[i]);
  }
}

int main() {
  char teks[256];
  char upper[256], lower[256];
  int nUpper, nLower, nDigit, nSpace;
  
  // Input
  printf("Masukkan teks: ");
  fgets(teks, sizeof(teks), stdin);
  
  // Hapus newline
  size_t len = strlen(teks);
  if (len > 0 && teks[len-1] == '\n') {
    teks[len-1] = '\0';
  }
  
  // Analisis
  countChars(teks, &nUpper, &nLower, &nDigit, &nSpace);
  
  // Prepare uppercase dan lowercase versions
  strcpy(upper, teks);
  strcpy(lower, teks);
  toUpperStr(upper);
  toLowerStr(lower);
  
  // Output
  printf("\n=== Hasil Analisis ===\n");
  printf("Panjang teks: %zu\n", strlen(teks));
  printf("Huruf besar: %d\n", nUpper);
  printf("Huruf kecil: %d\n", nLower);
  printf("Angka: %d\n", nDigit);
  printf("Spasi: %d\n", nSpace);
  printf("\nUppercase: %s\n", upper);
  printf("Lowercase: %s\n", lower);
  
  return 0;
}
\end{lstlisting}

\subsection{Program Pengolahan String di C++}

\begin{lstlisting}[language=C++, caption={Program pengolahan string di C++}]
#include <iostream>
#include <string>
#include <algorithm>
#include <cctype>

struct StringAnalysis {
  int length;
  int uppercase;
  int lowercase;
  int digits;
  int spaces;
};

StringAnalysis analyzeString(const std::string& str) {
  StringAnalysis result = {0, 0, 0, 0, 0};
  result.length = str.length();
  
  for (char ch : str) {
    if (std::isupper(ch)) result.uppercase++;
    else if (std::islower(ch)) result.lowercase++;
    else if (std::isdigit(ch)) result.digits++;
    else if (std::isspace(ch)) result.spaces++;
  }
  
  return result;
}

std::string toUpper(std::string str) {
  std::transform(str.begin(), str.end(), str.begin(), 
                 ::toupper);
  return str;
}

std::string toLower(std::string str) {
  std::transform(str.begin(), str.end(), str.begin(), 
                 ::tolower);
  return str;
}

int main() {
  std::string teks;
  
  // Input
  std::cout << "Masukkan teks: ";
  std::getline(std::cin, teks);
  
  // Analisis
  StringAnalysis analysis = analyzeString(teks);
  
  // Output
  std::cout << "\n=== Hasil Analisis ===\n";
  std::cout << "Panjang teks: " << analysis.length << "\n";
  std::cout << "Huruf besar: " << analysis.uppercase << "\n";
  std::cout << "Huruf kecil: " << analysis.lowercase << "\n";
  std::cout << "Angka: " << analysis.digits << "\n";
  std::cout << "Spasi: " << analysis.spaces << "\n";
  std::cout << "\nUppercase: " << toUpper(teks) << "\n";
  std::cout << "Lowercase: " << toLower(teks) << "\n";
  
  // Demonstrasi operasi lain
  std::cout << "\n=== Operasi Tambahan ===\n";
  
  // Substring
  if (teks.length() >= 5) {
    std::cout << "5 karakter pertama: " 
              << teks.substr(0, 5) << "\n";
  }
  
  // Find
  size_t pos = teks.find("a");
  if (pos != std::string::npos) {
    std::cout << "Huruf 'a' pertama di posisi: " 
              << pos << "\n";
  }
  
  // Replace
  std::string modified = teks;
  pos = modified.find(" ");
  if (pos != std::string::npos) {
    modified.replace(pos, 1, "_");
    std::cout << "Spasi pertama diganti underscore: " 
              << modified << "\n";
  }
  
  return 0;
}
\end{lstlisting}

\section{Best Practices dan Peringatan}

\subsection{Keamanan String di C}

\textbf{Hindari fungsi unsafe:}
\begin{itemize}
  \item \textbf{JANGAN} gunakan \texttt{gets()} -- sangat berbahaya, sudah deprecated
  \item \textbf{JANGAN} gunakan \texttt{strcpy()} tanpa cek ukuran
  \item \textbf{JANGAN} gunakan \texttt{strcat()} tanpa cek ukuran
  \item \textbf{JANGAN} gunakan \texttt{sprintf()} tanpa batas
\end{itemize}

\textbf{Gunakan alternatif aman:}
\begin{itemize}
  \item Gunakan \texttt{fgets()} bukan \texttt{gets()}
  \item Gunakan \texttt{strncpy()} bukan \texttt{strcpy()}
  \item Gunakan \texttt{strncat()} bukan \texttt{strcat()}
  \item Gunakan \texttt{snprintf()} bukan \texttt{sprintf()}
\end{itemize}

\subsection{Tips Umum}

\textbf{Pascal:}
\begin{itemize}
  \item Gunakan \texttt{SetLength()} untuk resize string dinamis
  \item Indeks string dimulai dari 1, bukan 0
  \item Single quote dalam string dengan double single-quote (\texttt{''})
  \item Manfaatkan fungsi bawaan seperti \texttt{Trim()}, \texttt{Pos()}, \texttt{Copy()}
\end{itemize}

\textbf{C:}
\begin{itemize}
  \item Selalu sediakan ruang untuk null terminator
  \item Cek batas buffer sebelum operasi
  \item Gunakan \texttt{sizeof()} untuk ukuran buffer
  \item Validasi pointer sebelum dereference
  \item Gunakan fungsi dengan suffix 'n' (strncpy, strncat) untuk keamanan
\end{itemize}

\textbf{C++:}
\begin{itemize}
  \item Prefer \texttt{std::string} daripada C-string untuk code baru
  \item Gunakan \texttt{std::string\_view} untuk parameter read-only
  \item Gunakan \texttt{at()} bukan \texttt{[]} untuk bounds checking
  \item Manfaatkan range-based for loop untuk iterasi
  \item Gunakan \texttt{std::getline()} untuk input multi-kata
\end{itemize}

\subsection{Kesalahan Umum}

\begin{enumerate}
  \item \textbf{Buffer overflow di C}: Menulis melewati batas array
  \item \textbf{Missing null terminator}: Lupa menambah \texttt{'\textbackslash 0'}
  \item \textbf{Off-by-one error}: Salah hitung panjang dengan terminator
  \item \textbf{Modifikasi string literal}: Undefined behavior di C/C++
  \item \textbf{Tidak membersihkan buffer input}: Karakter tersisa di stdin
  \item \textbf{Perbandingan pointer}: Menggunakan \texttt{==} bukan \texttt{strcmp()}
  \item \textbf{Memory leak}: Lupa free string yang dialokasi dinamis
\end{enumerate}

\section{Rangkuman}

String adalah tipe data fundamental untuk merepresentasikan teks dalam pemrograman. Pemahaman yang baik tentang representasi, operasi, dan manipulasi string sangat penting untuk pengembangan aplikasi.

\subsection{Poin-Poin Penting}

\textbf{Representasi String:}
\begin{itemize}
  \item Pascal: Tipe \texttt{string} dengan manajemen otomatis
  \item C: Array \texttt{char} dengan null terminator
  \item C++: \texttt{std::string} dengan manajemen memori otomatis
\end{itemize}

\textbf{Operasi Dasar:}
\begin{itemize}
  \item Deklarasi dan inisialisasi string
  \item Input/output string dengan berbagai metode
  \item Panjang string, konkatenasi, perbandingan
  \item Substring, pencarian, penggantian
  \item Konversi case (upper/lower)
\end{itemize}

\textbf{Keamanan:}
\begin{itemize}
  \item Selalu validasi batas buffer di C
  \item Gunakan fungsi aman (strncpy, fgets, snprintf)
  \item Prefer \texttt{std::string} di C++ untuk keamanan otomatis
  \item Hindari fungsi deprecated seperti \texttt{gets()}
\end{itemize}

\subsection{Perbandingan Antar Bahasa}

\begin{table}[H]
\centering
\begin{tabular}{|l|c|c|c|}
\hline
\textbf{Aspek} & \textbf{Pascal} & \textbf{C} & \textbf{C++} \\
\hline
Manajemen memori & Otomatis & Manual & Otomatis \\
\hline
Null terminator & Tidak perlu & Wajib & Tidak (std::string) \\
\hline
Ukuran dinamis & Ya & Tidak & Ya \\
\hline
Keamanan tipe & Kuat & Lemah & Kuat \\
\hline
Konkatenasi & Operator + & strcat/strncat & Operator + \\
\hline
Indeks mulai & 1 & 0 & 0 \\
\hline
Bounds checking & Ya & Tidak & at() method \\
\hline
\end{tabular}
\caption{Perbandingan string di Pascal, C, dan C++}
\end{table}

\subsection{Kapan Menggunakan Apa}

\textbf{Gunakan Pascal string ketika:}
\begin{itemize}
  \item Mengembangkan aplikasi Pascal
  \item Memerlukan fleksibilitas indeks
  \item Ingin manajemen memori otomatis
\end{itemize}

\textbf{Gunakan C string ketika:}
\begin{itemize}
  \item Interfacing dengan API sistem atau library C
  \item Memerlukan kontrol memori tingkat rendah
  \item Optimasi performa kritis
  \item Embedded systems dengan resource terbatas
\end{itemize}

\textbf{Gunakan C++ \texttt{std::string} ketika:}
\begin{itemize}
  \item Mengembangkan aplikasi C++ modern
  \item Keamanan dan kemudahan lebih penting dari kontrol detail
  \item Memerlukan operasi string yang kaya
  \item Ingin menghindari manual memory management
\end{itemize}

String adalah fondasi pemrosesan teks dalam pemrograman. Penguasaan operasi string yang baik akan memudahkan pengembangan berbagai aplikasi, dari parsing data, validasi input, hingga pemrosesan teks kompleks. Selalu prioritaskan keamanan dan clarity dalam code Anda!

\end{document}
