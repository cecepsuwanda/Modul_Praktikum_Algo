\documentclass[../main.tex]{subfiles}
\begin{document}
\chapter{String / Karakter \& Operasi String}
\section{String sebagai Array Karakter}
String adalah urutan karakter yang disimpan bersebelahan. Pascal menyediakan tipe \texttt{string} modern (serta bentuk berbasis array), C memakai array \texttt{char} berakhir null, dan C++ menyediakan \texttt{std::string} yang mengelola memori otomatis \parencite{pascal-tutorial-wikibooks,iso-c-draft-n1570,cpp-strings}.

Model array memudahkan akses indeks dan iterasi, tetapi menuntut kehati-hatian terhadap batas dan terminator. Pada C, kelalaian menempatkan karakter null dapat menyebabkan pembacaan melewati batas yang berbahaya. Abstraksi tingkat lebih tinggi seperti \texttt{std::string} mengurangi risiko tersebut dengan menjaga invarian panjang dan kapasitas \parencite{cpp-strings}.

Pemilihan representasi sebaiknya mempertimbangkan kompatibilitas pustaka dan performa. Untuk interoperabilitas dengan API C, C++ menyediakan \texttt{c\_str()} guna memperoleh penunjuk ke buffer null-terminated. Pascal modern juga menyediakan konversi antar tipe string untuk berinteraksi dengan komponen sistem \parencite{free-pascal-docs}.

\section{Operasi Dasar String (konkatenasi, substring, panjang)}
Operasi umum: konkatenasi, substring, panjang. Pascal menyediakan operator/fungsi bawaan; C menggunakan \texttt{<string.h>}; C++ menyediakan antarmuka kaya pada \texttt{std::string} \parencite{c-strings-h,cpp-strings,free-pascal-docs}.

\subsection{Contoh Lintas Bahasa}
\begin{lstlisting}[language=Pascal, caption={Konkatenasi dan panjang di Pascal}]
var a,b,c: string; begin
  a := 'Halo'; b := ' Dunia'; c := a + b; Writeln(Length(c));
end.
\end{lstlisting}

\begin{lstlisting}[language=C, caption={Konkatenasi aman di C}]
#include <string.h>
char dst[64] = "Halo";
strncat(dst, " Dunia", sizeof(dst)-strlen(dst)-1);
size_t len = strlen(dst);
\end{lstlisting}

\begin{lstlisting}[language=C++, caption={std::string di C++}]
#include <string>
#include <iostream>
int main(){
  std::string a = "Halo", b = " Dunia";
  std::string c = a + b;
  std::cout << c.size() << "\n";
}
\end{lstlisting}

Dalam C, operasi berbasis buffer menuntut pengelolaan ukuran tujuan untuk mencegah luapan dan korupsi memori. Fungsi dengan varian panjang tetap atau yang memerlukan ukuran buffer eksplisit lebih disarankan pada konteks aman. Pada C++, metode seperti \texttt{append}, \texttt{substr}, dan \texttt{size} memberikan semantik yang lebih jelas dan aman terhadap batas \parencite{cpp-strings}.

Pengukuran panjang perlu membedakan antara jumlah byte dan jumlah karakter ketika menggunakan enkode multibita. Untuk pemrosesan Unicode yang benar, gunakan pustaka khusus atau tipe yang sesuai di ekosistem masing-masing bahasa. Dokumentasi referensi menguraikan batasan representasi dan operasi yang tersedia \parencite{cpp-strings,iso-c-draft-n1570}.

\section{Pengolahan Karakter \& escape sequences}
Karakter diwakili oleh kode numerik dengan escape sequence untuk simbol khusus. C/C++: \texttt{\textbackslash n}, \texttt{\textbackslash t}, \texttt{\textbackslash\textbackslash}; Pascal memiliki notasi serupa bergantung dialek \parencite{iso-c-draft-n1570,free-pascal-docs}.

\begin{table}[h]
  \centering
  \caption{Escape sequence umum}
  \begin{tabular}{@{}lll@{}}
    \toprule
    Notasi & Arti & Catatan \\
    \midrule
    \texttt{\textbackslash n} & baris baru & LF \\
    \texttt{\textbackslash t} & tab & horizontal tab \\
    \texttt{\\} & backslash & butuh escape \\
    \bottomrule
  \end{tabular}
\end{table}

Normalisasi baris baru dan pengkodean berkas mempengaruhi hasil I/O dan perbandingan string. Pada sistem yang berbeda, representasi baris baru dapat bervariasi sehingga perlu strategi konsisten saat membaca atau menulis file teks. Pengujian yang memverifikasi perilaku pada berbagai lingkungan sangat dianjurkan.

Fungsi utilitas untuk klasifikasi karakter, seperti \texttt{isalpha} dan \texttt{isdigit} di C, memudahkan validasi input tingkat rendah. Pada C++, \texttt{<locale>} dan fasilitas standar lain menyediakan cara yang lebih kaya untuk mempertimbangkan regionalisasi. Rujuk dokumentasi resmi untuk cakupan kelas karakter dan dampaknya pada portable code \parencite{iso-c-draft-n1570,cpp-reference}.

\section{Fungsi/prosedur string library}
Pustaka standar menyediakan fungsi untuk manipulasi string tingkat rendah/tinggi. C: \texttt{strlen}, \texttt{strcmp}, \texttt{memcpy}; Pascal: unit dan fungsi bawaan; C++: \texttt{std::string} dan algoritme standar \parencite{c-strings-h,free-pascal-docs,cpp-strings}.

\subsection{Contoh API}
\begin{lstlisting}[language=C, caption={Validasi digit dengan <ctype.h>}]
#include <ctype.h>
int all_digit(const char* s){ for (; *s; ++s) if (!isdigit((unsigned char)*s)) return 0; return 1; }
\end{lstlisting}

\begin{lstlisting}[language=C++, caption={Mencari substring di C++}]
#include <string>
bool contains(std::string_view s, std::string_view sub){ return s.find(sub) != std::string::npos; }
\end{lstlisting}

Pemilihan API harus mempertimbangkan keselamatan, performa, dan kejelasan. Gunakan fungsi yang memerlukan ukuran buffer eksplisit atau objek yang mengelola memori otomatis untuk mengurangi risiko. Kombinasikan dengan pengujian properti untuk memastikan invarian seperti tidak adanya pemotongan senyap atau korupsi memori.

Integrasi dengan I/O sering kali melibatkan konversi antara representasi internal dan bentuk terformat. Dokumentasi referensi memberikan contoh pola yang konsisten untuk pembentukan string, parsing, dan pemformatan. Sumber terbuka menyediakan contoh yang dapat direplikasi untuk setiap bahasa \parencite{c-strings-h,cpp-strings,free-pascal-docs}.
\end{document}
