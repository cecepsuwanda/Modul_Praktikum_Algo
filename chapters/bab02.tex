\documentclass[../main.tex]{subfiles}
\begin{document}
\chapter{Tipe Data, Variabel, Input / Output (Pascal, C, C++)}
\section{Tipe Data Dasar}
Tipe data menentukan nilai apa yang bisa disimpan dan operasi apa yang sah. Pascal menyediakan \texttt{integer}, \texttt{real}, \texttt{char}, \texttt{boolean}; C/C++ menyediakan \texttt{int}, \texttt{double}, \texttt{char}, dan \texttt{bool}. Ukuran aktual tipe dapat berbeda antar platform, sehingga penting memahami model data yang digunakan \parencite{pascal-tutorial-wikibooks,iso-c-draft-n1570,cpp-arithmetic-types,cpp-fundamental-types}.

Pemilihan tipe yang tepat memperjelas maksud program dan membantu optimasi. Pada C/C++, kualifikator \texttt{signed/unsigned} serta \texttt{short/long} mengubah rentang; pada Pascal, rentang indeks eksplisit memperjelas invarian. Dokumentasi resmi memberi ringkasan ukuran dan standar yang relevan \parencite{free-pascal-docs,iso-c-draft-n1570,cpp-reference}.

Representasi memori dan konversi implisit perlu dipahami agar tidak terjadi kehilangan presisi. Misalnya, promosi integral di C dapat mengubah hasil ekspresi ketika tipe berbeda dicampur. Prinsip konservatif: lakukan \emph{cast} eksplisit hanya bila perlu \parencite{gnu-c-manual,cpp-reference}.

\subsection{Ringkasan Kategori Tipe (Gambaran Singkat)}
\begin{table}[h]
  \centering
  \caption{Kategori tipe dan contoh (ukuran tipikal; dapat bervariasi)}
  \begin{tabular}{@{}llll@{}}
    \toprule
    Kategori & Pascal & C & C++ \\
    \midrule
    Integer & \texttt{integer}, \texttt{longint} & \texttt{int}, \texttt{long} & \texttt{int}, \texttt{long} \\
    Pecahan & \texttt{real}, \texttt{double} & \texttt{float}, \texttt{double} & \texttt{float}, \texttt{double} \\
    Karakter & \texttt{char} & \texttt{char} & \texttt{char}, \texttt{char8\_t} (C++20) \\
    Boolean & \texttt{boolean} & (\texttt{_Bool}/\texttt{bool} di C99) & \texttt{bool} \\
    \bottomrule
  \end{tabular}
  \\Sumber: \parencite{free-pascal-docs,iso-c-draft-n1570,cpp-fundamental-types}
\end{table}

\section{Deklarasi \& Inisialisasi Variabel}
Deklarasi menetapkan nama, tipe, cakupan; inisialisasi awal mencegah nilai sampah (khususnya di C). C++ mendukung inisialisasi terbraket yang lebih aman, sedangkan Pascal memiliki bagian \texttt{var} yang eksplisit \parencite{pascal-tutorial-wikibooks,gnu-c-manual,cpp-reference}.

\subsection{Contoh Deklarasi dan Inisialisasi}
\begin{lstlisting}[language=Pascal, caption={Deklarasi dan inisialisasi di Pascal}]
program Vars;
var
  count: integer = 0;
  ratio: double = 0.5;
  letter: char = 'A';
  flag: boolean = true;
begin
  Writeln(count, ' ', ratio:0:2, ' ', letter, ' ', flag);
end.
\end{lstlisting}

\begin{lstlisting}[language=C, caption={Deklarasi dan inisialisasi di C}]
#include <stdio.h>
int main(void) {
  int count = 0;
  double ratio = 0.5;
  char letter = 'A';
  _Bool flag = 1; // C99
  printf("%d %.2f %c %s\n", count, ratio, letter, flag?"true":"false");
  return 0;
}
\end{lstlisting}

\begin{lstlisting}[language=C++, caption={Deklarasi dan inisialisasi di C++}]
#include <iostream>
int main() {
  int count{0};
  double ratio{0.5};
  char letter{'A'};
  bool flag{true};
  std::cout << count << ' ' << ratio << ' ' << letter << ' ' << std::boolalpha << flag << '\n';
}
\end{lstlisting}

Disiplin penamaan yang konsisten meningkatkan keterbacaan. Bila lingkup variabel terlalu panjang, pecahlah fungsi menjadi unit lebih kecil. Gunakan \texttt{const} (atau setara) untuk nilai yang tidak boleh berubah.

Pada C/C++, pahami durasi penyimpanan (otomatis, statik, dinamis) untuk mengelola siklus hidup. Deklarasi dekat titik penggunaan menurunkan kompleksitas. Pada Pascal, bagian \texttt{var} menjaga struktur yang mudah diaudit \parencite{free-pascal-docs,gnu-c-manual}.

\section{Input / Output Dasar}
Masukan/keluaran (I/O) memungkinkan program berinteraksi dengan pengguna dan berkas. Pascal memakai \texttt{Read/Readln} dan \texttt{Write/Writeln}; C memakai \texttt{scanf/printf}; C++ memakai \texttt{std::cin/std::cout} dan manipulator \texttt{<iomanip>} \parencite{w3pascal-io,gnu-c-manual,cplusplus-io,cpp-iomanip}.

Pisahkan keluaran kesalahan (\texttt{stderr}) dari keluaran normal untuk memudahkan otomasi. Pada C, gunakan \texttt{snprintf} untuk mencegah luapan buffer. Pada C++, gunakan \texttt{std::setw}, \texttt{std::setprecision} untuk keluaran yang rapi \parencite{gnu-c-manual,cpp-reference,cpp-iomanip}.

Untuk masukan yang tangguh, baca satu baris lalu parsing eksplisit agar validasi lebih terkendali. Pada C/C++, gabungkan \texttt{fgets}/\texttt{std::getline} dengan pemeriksaan konversi numerik. Dokumentasi standar merinci format dan status stream untuk diagnosa \parencite{iso-c-draft-n1570,cpp-reference}.

\subsection{Diagram Arus I/O}
\begin{figure}[h]
  \centering
  \begin{tikzpicture}[node distance=1.6cm, >=Stealth]
    \tikzstyle{box}=[rectangle, draw, rounded corners, align=center, minimum width=3.1cm, minimum height=1cm]
    \node[box] (kbd) {Keyboard/\newline Berkas Masuk};
    \node[box, right=of kbd] (prog) {Program};
    \node[box, right=of prog] (term) {Terminal/\newline Berkas Keluar};
    \draw[->] (kbd) -- node[above]{input} (prog);
    \draw[->] (prog) -- node[above]{output} (term);
  \end{tikzpicture}
  \caption{Arus I/O sederhana antara pengguna/berkas dan program}
  \label{fig:io-flow}
\end{figure}

\subsection{Contoh I/O Angka dan Pemformatan}
\begin{lstlisting}[language=Pascal, caption={Baca integer dan format keluaran (Pascal)}]
program IOFmt;
var n: integer;
begin
  Write('Masukkan angka: '); Readln(n);
  Writeln('Kuadrat = ', n*n);
  Writeln('Pi ~ ', 3.14159:0:2); // 2 digit desimal
end.
\end{lstlisting}

\begin{lstlisting}[language=C, caption={Baca integer dan format keluaran (C)}]
#include <stdio.h>
int main(void) {
  int n;
  printf("Masukkan angka: ");
  if (scanf("%d", &n) == 1) {
    printf("Kuadrat = %d\n", n*n);
    printf("Pi ~ %.2f\n", 3.14159);
  } else {
    fprintf(stderr, "Input tidak valid\n");
  }
  return 0;
}
\end{lstlisting}

\begin{lstlisting}[language=C++, caption={Baca integer dan format keluaran (C++)}]
#include <iostream>
#include <iomanip>
int main() {
  int n;
  std::cout << "Masukkan angka: ";
  if (std::cin >> n) {
    std::cout << "Kuadrat = " << n*n << "\n";
    std::cout << std::fixed << std::setprecision(2) << "Pi ~ " << 3.14159 << "\n";
  } else {
    std::cerr << "Input tidak valid\n";
  }
}
\end{lstlisting}

\subsection{Ringkasan Pemformatan}
\begin{table}[h]
  \centering
  \caption{Ringkasan pemformatan angka}
  \begin{tabular}{@{}lll@{}}
    \toprule
    Bahasa & Contoh & Keterangan \\
    \midrule
    Pascal & \texttt{real:0:2} & 2 digit desimal \\
    C & \texttt{"\\%.2f"} & 2 digit desimal (\texttt{printf}) \\
    C++ & \texttt{std::fixed, std::setprecision(2)} & 2 digit desimal \\
    \bottomrule
  \end{tabular}
  \\\parencite{w3pascal-io,gnu-c-manual,cpp-iomanip}
\end{table}

\section{Konversi Tipe dan Operasi Terkait}
Konversi dapat implisit atau eksplisit dan berdampak pada presisi serta aturan evaluasi. Di C, promosi aritmetika menstandarkan tipe operan sebelum evaluasi; di C++ gunakan \texttt{static\_cast} untuk niat yang tegas. Pascal cenderung menolak pencampuran tipe ambigu \parencite{pascal-tutorial-wikibooks,gnu-c-manual,cpp-reference}.

Perhatikan biaya konversi pada jalur panas eksekusi. Persempit titik konversi dan gunakan tipe target sejak awal untuk menurunkan beban runtime. Uji unit membantu memverifikasi kesetaraan numerik.

Dalam konteks I/O, konversi format ke representasi internal harus diperlakukan sebagai batas kepercayaan data. Validasi dan sanitasi input mencegah perilaku tak terduga \parencite{iso-c-draft-n1570,cpp-reference}.

\subsection{Contoh Dampak Promosi (C)}
\begin{lstlisting}[language=C, caption={Promosi integral memengaruhi hasil di C}]
#include <stdio.h>
int main(void) {
  unsigned char a = 250, b = 10; // masing-masing 8-bit
  // Dipromosikan ke int sebelum dijumlahkan
  printf("%d\n", a + b); // 260 pada implementasi umum
  return 0;
}
\end{lstlisting}

\subsection{Catatan Eksekusi (OnlineGDB, Lazarus, Code::Blocks)}
\begin{itemize}
  \item \textbf{OnlineGDB}: buka \url{https://www.onlinegdb.com/}, pilih Pascal/C/C++, tempel contoh dari bab ini, klik Run. Untuk beberapa berkas, gabungkan terlebih dahulu menjadi satu berkas.
  \item \textbf{Lazarus (Pascal)}: File \textrightarrow{} New \textrightarrow{} Console Application, tempel kode ke program utama, tekan Run.
  \item \textbf{Code::Blocks (C/C++)}: File \textrightarrow{} New \textrightarrow{} Project \textrightarrow{} Console application, pilih C/C++, tempel ke \texttt{main.c}/\texttt{main.cpp}, Build \& Run.
\end{itemize}
\end{document}
