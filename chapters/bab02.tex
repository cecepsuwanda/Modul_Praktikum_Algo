\documentclass[../main.tex]{subfiles}
\begin{document}
\chapter{Tipe Data, Variabel, dan I/O}

\section*{Tujuan Praktikum}
Setelah menyelesaikan praktikum ini, mahasiswa diharapkan mampu:
\begin{itemize}
  \item Memahami berbagai tipe data dasar (integer, real, char, boolean) di Pascal, C, dan C++
  \item Mendeklarasikan dan menginisialisasi variabel dengan tipe data yang tepat
  \item Memahami batasan dan rentang nilai setiap tipe data
  \item Melakukan operasi input dan output data menggunakan fungsi/prosedur yang sesuai
  \item Memformat output data dengan presisi dan lebar field yang diinginkan
  \item Membuat program interaktif sederhana dengan input dari pengguna
\end{itemize}

\section{Tipe Data Dasar}
Tipe data menentukan nilai apa yang bisa disimpan dan operasi apa yang sah. Pascal menyediakan \texttt{integer}, \texttt{real}, \texttt{char}, \texttt{boolean}; C/C++ menyediakan \texttt{int}, \texttt{double}, \texttt{char}, dan \texttt{bool}. Ukuran aktual tipe dapat berbeda antar platform, sehingga penting memahami model data yang digunakan \parencite{pascal-tutorial-wikibooks,iso-c-draft-n1570,cpp-arithmetic-types,cpp-fundamental-types}.

Pemilihan tipe yang tepat memperjelas maksud program dan membantu optimasi. Pada C/C++, kualifikator \texttt{signed/unsigned} serta \texttt{short/long} mengubah rentang; pada Pascal, rentang indeks eksplisit memperjelas invarian. Dokumentasi resmi memberi ringkasan ukuran dan standar yang relevan \parencite{free-pascal-docs,iso-c-draft-n1570,cpp-reference}.

Representasi memori dan konversi implisit perlu dipahami agar tidak terjadi kehilangan presisi. Misalnya, promosi integral di C dapat mengubah hasil ekspresi ketika tipe berbeda dicampur. Prinsip konservatif: lakukan \emph{cast} eksplisit hanya bila perlu \parencite{gnu-c-manual,cpp-reference}.

\subsection{Ringkasan Kategori Tipe (Gambaran Singkat)}
\begin{table}[H]
  \centering
  \caption{Kategori tipe dan contoh (ukuran tipikal; dapat bervariasi)}
  \begin{tabular}{@{}llll@{}}
    \toprule
    Kategori & Pascal & C & C++ \\
    \midrule
    Integer & \texttt{integer}, \texttt{longint} & \texttt{int}, \texttt{long} & \texttt{int}, \texttt{long} \\
    Pecahan & \texttt{real}, \texttt{double} & \texttt{float}, \texttt{double} & \texttt{float}, \texttt{double} \\
    Karakter & \texttt{char} & \texttt{char} & \texttt{char}, \texttt{char8\_t} (C++20) \\
    Boolean & \texttt{boolean} & (\texttt{\_Bool}/\texttt{bool} di C99) & \texttt{bool} \\
    String & \texttt{string} & \texttt{char[]} (array) & \texttt{std::string} \\
    \bottomrule
  \end{tabular}
  \\Sumber: \parencite{free-pascal-docs,iso-c-draft-n1570,cpp-fundamental-types}
\end{table}

\subsection{Batasan Daya Tampung Tipe Data}
\begin{table}[H]
  \centering
  \caption{Ukuran dan rentang nilai tipe data (implementasi umum)}
  \begin{tabular}{@{}lllll@{}}
    \toprule
    Tipe & Bahasa & Ukuran & Rentang Nilai & Presisi \\
    \midrule
    \texttt{integer} & Pascal & 2 byte & $-32{,}768$ s.d. $32{,}767$ & - \\
    \texttt{longint} & Pascal & 4 byte & $-2{,}147{,}483{,}648$ s.d. $2{,}147{,}483{,}647$ & - \\
    \texttt{real} & Pascal & 6 byte & $2.9 \times 10^{-39}$ s.d. $1.7 \times 10^{38}$ & 6--7 digit \\
    \texttt{char} & Pascal & 1 byte & ASCII (0--255) & - \\
    \texttt{boolean} & Pascal & 1 byte & \texttt{true}, \texttt{false} & - \\
    \midrule
    \texttt{short} & C/C++ & 2 byte & $-32{,}768$ s.d. $32{,}767$ & - \\
    \texttt{int} & C/C++ & 4 byte & $-2{,}147{,}483{,}648$ s.d. $2{,}147{,}483{,}647$ & - \\
    \texttt{long} & C/C++ & 4--8 byte & bergantung platform & - \\
    \texttt{float} & C/C++ & 4 byte & $\pm 3.4 \times 10^{\pm 38}$ & 6--7 digit \\
    \texttt{double} & C/C++ & 8 byte & $\pm 1.7 \times 10^{\pm 308}$ & 15--16 digit \\
    \texttt{char} & C/C++ & 1 byte & $-128$ s.d. $127$ (signed) & - \\
    \texttt{bool} & C++ & 1 byte & \texttt{true}, \texttt{false} & - \\
    \bottomrule
  \end{tabular}
  \\Catatan: Ukuran dan rentang dapat bervariasi bergantung implementasi compiler dan arsitektur.
  \\Sumber: \parencite{free-pascal-docs,iso-c-draft-n1570,cpp-fundamental-types}
\end{table}

\section{Deklarasi \& Inisialisasi Variabel}
Deklarasi menetapkan nama, tipe, cakupan; inisialisasi awal mencegah nilai sampah (khususnya di C). C++ mendukung inisialisasi terbraket yang lebih aman, sedangkan Pascal memiliki bagian \texttt{var} yang eksplisit \parencite{pascal-tutorial-wikibooks,gnu-c-manual,cpp-reference}.

\subsection{Contoh Deklarasi dan Inisialisasi}
\begin{lstlisting}[language=Pascal, caption={Deklarasi dan inisialisasi di Pascal}]
program Vars;
var
  count: integer = 0;
  ratio: double = 0.5;
  letter: char = 'A';
  flag: boolean = true;
begin
  Writeln(count, ' ', ratio:0:2, ' ', letter, ' ', flag);
end.
\end{lstlisting}

\begin{lstlisting}[language=C, caption={Deklarasi dan inisialisasi di C}]
#include <stdio.h>
int main(void) {
  int count = 0;
  double ratio = 0.5;
  char letter = 'A';
  _Bool flag = 1; // C99
  printf("%d %.2f %c %s\n", count, ratio, letter, flag?"true":"false");
  return 0;
}
\end{lstlisting}

\begin{lstlisting}[language=C++, caption={Deklarasi dan inisialisasi di C++}]
#include <iostream>
using namespace std;

int main() {
  int count{0};
  double ratio{0.5};
  char letter{'A'};
  bool flag{true};
  cout << count << ' ' << ratio << ' ' << letter << ' ' << boolalpha << flag << '\n';
}
\end{lstlisting}

Disiplin penamaan yang konsisten meningkatkan keterbacaan. Bila lingkup variabel terlalu panjang, pecahlah fungsi menjadi unit lebih kecil. Gunakan \texttt{const} (atau setara) untuk nilai yang tidak boleh berubah.

Pada C/C++, pahami durasi penyimpanan (otomatis, statik, dinamis) untuk mengelola siklus hidup. Deklarasi dekat titik penggunaan menurunkan kompleksitas. Pada Pascal, bagian \texttt{var} menjaga struktur yang mudah diaudit \parencite{free-pascal-docs,gnu-c-manual}.

\section{Input / Output Dasar}
Masukan/keluaran (I/O) memungkinkan program berinteraksi dengan pengguna dan berkas. Pascal memakai \texttt{Read/Readln} dan \texttt{Write/Writeln}; C memakai \texttt{scanf/printf}; C++ memakai \texttt{std::cin/std::cout} dan manipulator \texttt{\textless iomanip\textgreater} \parencite{w3pascal-io,gnu-c-manual,cplusplus-io,cpp-iomanip}.

Pisahkan keluaran kesalahan (\texttt{stderr}) dari keluaran normal untuk memudahkan otomasi. Pada C, gunakan \texttt{snprintf} untuk mencegah luapan buffer. Pada C++, gunakan \texttt{std::setw}, \texttt{std::setprecision} untuk keluaran yang rapi \parencite{gnu-c-manual,cpp-reference,cpp-iomanip}.

Untuk masukan yang tangguh, baca satu baris lalu parsing eksplisit agar validasi lebih terkendali. Pada C/C++, gabungkan \texttt{fgets}/\texttt{std::getline} dengan pemeriksaan konversi numerik. Dokumentasi standar merinci format dan status stream untuk diagnosa \parencite{iso-c-draft-n1570,cpp-reference}.


\begin{table}[H]
  \centering
  \small
  \caption{Instruksi Input/Output pada Pascal, C, dan C++}
  \begin{tabular}{@{}llll@{}}
    \toprule
    Operasi & Pascal & C & C++ \\
    \midrule
    Input standar & \texttt{Read()}, \texttt{Readln()} & \texttt{scanf()}, \texttt{fgets()} & \texttt{cin >>}, \texttt{getline()} \\
    Output standar & \texttt{Write()}, \texttt{Writeln()} & \texttt{printf()} & \texttt{cout <<} \\
    Output error & - & \texttt{fprintf(stderr)} & \texttt{cerr <<} \\
    Format output & \texttt{var:width:decimal} & \texttt{printf("\%format")} & manipulator \texttt{iomanip} \\
    Input string & \texttt{Readln(str)} & \texttt{fgets(...)} & \texttt{getline(...)} \\
    \bottomrule
  \end{tabular}
  \\\parencite{w3pascal-io,gnu-c-manual,cplusplus-io,cpp-iomanip}
\end{table}

\subsection{Diagram Arus I/O}
\begin{figure}[H]
  \centering
  \begin{tikzpicture}[node distance=1.6cm, >=Stealth]
    \tikzstyle{box}=[rectangle, draw, rounded corners, align=center, minimum width=3.1cm, minimum height=1cm]
    \node[box] (kbd) {Keyboard/\newline Berkas Masuk};
    \node[box, right=of kbd] (prog) {Program};
    \node[box, right=of prog] (term) {Terminal/\newline Berkas Keluar};
    \draw[->] (kbd) -- node[above]{input} (prog);
    \draw[->] (prog) -- node[above]{output} (term);
  \end{tikzpicture}
  \caption{Arus I/O sederhana antara pengguna/berkas dan program}
  \label{fig:io-flow}
\end{figure}

\subsection{Contoh I/O Angka dan Pemformatan}
\begin{lstlisting}[language=Pascal, caption={Baca integer dan format keluaran (Pascal)}]
program IOFmt;

var
  n: integer;
begin
  Write('Masukkan angka: ');
  Readln(n);
  Writeln('Kuadrat = ', n * n);
  Writeln('Pi ~ ', 3.14159:0:2);  // 2 digit desimal
end.
\end{lstlisting}

\begin{lstlisting}[language=C, caption={Baca integer dan format keluaran (C)}]
#include <stdio.h>
int main(void) {
  int n;
  printf("Masukkan angka: ");
  if (scanf("%d", &n) == 1) {
    printf("Kuadrat = %d\n", n*n);
    printf("Pi ~ %.2f\n", 3.14159);
  } else {
    fprintf(stderr, "Input tidak valid\n");
  }
  return 0;
}
\end{lstlisting}

\begin{lstlisting}[language=C++, caption={Baca integer dan format keluaran (C++)}]
#include <iostream>
#include <iomanip>
using namespace std;

int main() {
  int n;
  cout << "Masukkan angka: ";
  if (cin >> n) {
    cout << "Kuadrat = " << n*n << "\n";
    cout << fixed << setprecision(2) << "Pi ~ " << 3.14159 << "\n";
  } else {
    cerr << "Input tidak valid\n";
  }
}
\end{lstlisting}

\subsection{Contoh I/O String (Program Interaktif)}
\begin{lstlisting}[language=Pascal, caption={Input nama pada Pascal}]
program Greet;

var
  name: string;
begin
  Write('Nama Anda? ');
  Readln(name);
  Writeln('Halo, ', name, '!');
end.
\end{lstlisting}

\begin{lstlisting}[language=C, caption={Input nama pada C}]
#include <stdio.h>
int main(void) {
  char name[64];
  printf("Nama Anda? ");
  if (fgets(name, sizeof name, stdin)) {
    printf("Halo, %s", name); // fgets menyertakan newline
  }
  return 0;
}
\end{lstlisting}

\begin{lstlisting}[language=C++, caption={Input nama pada C++}]
#include <iostream>
#include <string>
using namespace std;

int main() {
  string name;
  cout << "Nama Anda? ";
  getline(cin, name);
  cout << "Halo, " << name << "!\n";
}
\end{lstlisting}

\subsection{Ringkasan Pemformatan}
\begin{table}[H]
  \centering
  \caption{Ringkasan pemformatan angka}
  \begin{tabular}{@{}lll@{}}
    \toprule
    Bahasa & Contoh & Keterangan \\
    \midrule
    Pascal & \texttt{real:0:2} & 2 digit desimal \\
    C & \texttt{"\%.2f"} & 2 digit desimal (\texttt{printf}) \\
    C++ & \texttt{std::fixed}, \texttt{std::setprecision(2)} & 2 digit desimal \\
    \bottomrule
  \end{tabular}
  \\\parencite{w3pascal-io,gnu-c-manual,cpp-iomanip}
\end{table}
\end{document}
