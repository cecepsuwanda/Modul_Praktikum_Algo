\documentclass[../main.tex]{subfiles}
\begin{document}
\chapter{Struktur Kontrol Kondisional}
\section{Pernyataan \texttt{if} / \texttt{else}}
Pernyataan \texttt{if} / \texttt{else} mengeksekusi cabang berdasarkan hasil kondisi boolean. Di Pascal, kondisi bertipe \texttt{boolean}; di C/C++, nilai bukan nol dianggap benar \parencite{pascal-tutorial-wikibooks,gnu-c-manual,cpp-reference}.

Gunakan blok yang jelas (\texttt{begin\ldots end} atau \{\}) dan pertimbangkan pola \emph{early return} untuk mengurangi nesting. Uji semua cabang agar regresi mudah terdeteksi.

\subsection{Contoh Lintas Bahasa}
\begin{lstlisting}[language=Pascal, caption={if/else di Pascal}]
if n mod 2 = 0 then Writeln('Genap') else Writeln('Ganjil');
\end{lstlisting}
\begin{lstlisting}[language=C, caption={if/else di C}]
if (n % 2 == 0) printf("Genap\n"); else printf("Ganjil\n");
\end{lstlisting}
\begin{lstlisting}[language=C++, caption={if dengan initializer di C++17}]
if (int rem = n % 2; rem == 0) std::cout << "Genap\n"; else std::cout << "Ganjil\n";
\end{lstlisting}

\subsection{Pola Early Return}
\begin{lstlisting}[language=C]
int process(const char* path) {
  if (!path) return -1;           // validasi awal
  FILE* f = fopen(path, "r");
  if (!f) return -2;              // gagal buka
  // ... proses ...
  fclose(f);
  return 0;
}
\end{lstlisting}

\section{Pernyataan \texttt{case} / \texttt{switch}}
Konstruksi multi-cabang merangkum pilihan diskret. Pascal: \texttt{case\ldots of} (mendukung rentang); C/C++: \texttt{switch} dengan \texttt{case} konstanta integral. Sertakan klausa default untuk ketangguhan \parencite{pascal-tutorial-wikibooks,gnu-c-manual,cpp-reference,cpp-switch,cpp-enum-class}.

\begin{lstlisting}[language=Pascal, caption={case of dengan rentang}]
case score of
  90..100: Writeln('A');
  80..89:  Writeln('B');
  70..79:  Writeln('C');
  else     Writeln('D/E');
end;
\end{lstlisting}

\begin{lstlisting}[language=C, caption={switch di C dengan break}]
switch (score/10) {
  case 10: case 9: printf("A\n"); break;
  case 8: printf("B\n"); break;
  case 7: printf("C\n"); break;
  default: printf("D/E\n");
}
\end{lstlisting}

\begin{lstlisting}[language=C++, caption={switch modern dengan enum class dan [[fallthrough]]}]
enum class Grade { A, B, C, Other };
Grade fromScore(int s){
  switch (s/10) {
    case 10: [[fallthrough]];
    case 9:  return Grade::A;
    case 8:  return Grade::B;
    case 7:  return Grade::C;
    default: return Grade::Other;
  }
}
\end{lstlisting}

Untuk logika yang terus bertambah, pertimbangkan tabel keputusan atau pemetaan fungsi ketimbang rantai \texttt{if/else} panjang.

\subsection{Diagram Alur Keputusan}
\begin{figure}[h]
  \centering
  \begin{tikzpicture}[node distance=1.7cm, >=Stealth]
    \tikzstyle{b}=[rectangle, draw, rounded corners, align=center, minimum width=3.0cm, minimum height=1cm]
    \node[b] (start) {Mulai};
    \node[b, below=of start] (cond) {n genap?};
    \node[b, below left=1.2cm and 1.6cm of cond] (yes) {Cetak "Genap"};
    \node[b, below right=1.2cm and 1.6cm of cond] (no) {Cetak "Ganjil"};
    \node[b, below=2.2cm of cond] (end) {Selesai};
    \draw[->] (start) -- (cond);
    \draw[->] (cond) -- node[left]{Ya} (yes);
    \draw[->] (cond) -- node[right]{Tidak} (no);
    \draw[->] (yes) |- (end);
    \draw[->] (no)  |- (end);
  \end{tikzpicture}
  \caption{Alur keputusan sederhana}
\end{figure}

\subsection{Catatan Eksekusi (OnlineGDB, Lazarus, Code::Blocks)}
\begin{itemize}
  \item \textbf{OnlineGDB}: \url{https://www.onlinegdb.com/} \textrightarrow{} pilih Pascal/C/C++, tempel contoh, Run.
  \item \textbf{Lazarus (Pascal)}: Console Application, tempel contoh Pascal, Run.
  \item \textbf{Code::Blocks (C/C++)}: Console application, pilih C/C++, tempel ke \texttt{main.c}/\texttt{main.cpp}, Build \& Run.
\end{itemize}
\end{document}
