% ============================================================
% Preamble untuk Buku Ajar OBE - Algoritma dan Pemrograman
% ============================================================

% --- Encoding dan Bahasa ---
\usepackage[utf8]{inputenc}
\usepackage[T1]{fontenc}
\usepackage{textcomp}
\usepackage[indonesian]{babel}
\usepackage{lmodern}

% --- Layout Halaman ---
\usepackage{geometry}
\geometry{a4paper, margin=2.5cm, headheight=15pt}
\raggedbottom
\tolerance=2500
\emergencystretch=2em

% --- Grafik dan Tabel ---
\usepackage{graphicx}
\usepackage{booktabs}
\usepackage{longtable}
\usepackage{array}
\usepackage{multirow}
\usepackage{xcolor}
\usepackage{amssymb}

% --- TikZ untuk Diagram (Flowchart, dll.) ---
\usepackage{tikz}
\usetikzlibrary{shapes,arrows,positioning,calc,shapes.geometric}
\tikzset{
  flowstart/.style={
    ellipse,
    draw=black,
    fill=green!10,
    minimum width=2.5cm,
    minimum height=1cm,
    text centered
  },
  flowprocess/.style={
    rectangle,
    draw=black,
    fill=blue!10,
    minimum width=2.5cm,
    minimum height=1cm,
    text centered
  },
  flowdecision/.style={
    diamond,
    draw=black,
    fill=yellow!10,
  },
  flowio/.style={
    trapezium,
    trapezium stretches=true,
    trapezium left angle=70,
    trapezium right angle=110,
    draw=black,
    fill=cyan!10,
    minimum width=2.5cm,
    minimum height=1cm,
    text centered
  },
  arrow/.style={
    thick,
    ->,
    >=stealth
  }
}


% --- List dan Enumerasi ---
\usepackage{enumitem}

% --- Hyperlinks ---
\usepackage{xurl}
\usepackage{hyperref}
\hypersetup{
  colorlinks=true,
  linkcolor=blue,
  urlcolor=blue,
  citecolor=blue,
  bookmarksnumbered=true,
  pdfstartview=FitH
}
\usepackage{bookmark}

% --- Subfiles untuk Modular Structure ---
\usepackage{subfiles}

% --- Header dan Footer ---
\usepackage{fancyhdr}
\pagestyle{fancy}
\fancyhf{}
\fancyhead[LE,RO]{\thepage}
\fancyhead[RE]{\leftmark}
\fancyhead[LO]{\rightmark}
\renewcommand{\headrulewidth}{0.4pt}

% --- Code Listings untuk C ---
\usepackage{listings}
\usepackage{xcolor}

% Definisi warna untuk syntax highlighting
\definecolor{cstring}{rgb}{0.6,0,0}
\definecolor{ccomment}{rgb}{0.25,0.5,0.35}
\definecolor{ckeyword}{rgb}{0.5,0,0.35}
\definecolor{cpreprocessor}{rgb}{0.25,0.35,0.75}
\definecolor{codegray}{rgb}{0.5,0.5,0.5}
\definecolor{codebg}{rgb}{0.95,0.95,0.95}

% Style untuk C code
\lstdefinestyle{cstyle}{
  language=C,
  basicstyle=\ttfamily\small,
  keywordstyle=\color{ckeyword}\bfseries,
  stringstyle=\color{cstring},
  commentstyle=\color{ccomment}\itshape,
  morekeywords={printf,scanf,sizeof},
  numbers=left,
  numberstyle=\tiny\color{codegray},
  stepnumber=1,
  numbersep=10pt,
  backgroundcolor=\color{codebg},
  showspaces=false,
  showstringspaces=false,
  showtabs=false,
  frame=single,
  tabsize=2,
  captionpos=b,
  breaklines=true,
  breakatwhitespace=false,
  escapeinside={(*@}{@*)},
  xleftmargin=15pt,
  xrightmargin=5pt
}

\renewcommand{\lstlistlistingname}{Daftar Kode Program}
\renewcommand{\lstlistingname}{Kode Program}
\lstset{style=cstyle}

% Environment untuk C code dengan caption
\lstnewenvironment{ccode}[1][]{
  \lstset{style=cstyle, #1}
}{}

% --- Boxes untuk Highlight ---
\usepackage{tcolorbox}
\tcbuselibrary{skins,breakable}

% Box untuk konsep penting
\newtcolorbox{konsep}{
  colback=blue!5!white,
  colframe=blue!75!black,
  fonttitle=\bfseries,
  title=Konsep Penting,
  breakable
}

% Box untuk catatan
\newtcolorbox{catatan}{
  colback=yellow!5!white,
  colframe=yellow!75!black,
  fonttitle=\bfseries,
  title=Catatan,
  breakable
}

% Box untuk contoh
\newtcolorbox{contoh}{
  colback=green!5!white,
  colframe=green!75!black,
  fonttitle=\bfseries,
  title=Contoh,
  breakable
}

% Box untuk peringatan (alertbox)
\newtcolorbox{alertbox}[1]{
  colback=red!5!white,
  colframe=red!75!black,
  fonttitle=\bfseries,
  title=#1,
  breakable
}

% Box untuk tip
\newtcolorbox{tip}{
  colback=teal!5!white,
  colframe=teal!75!black,
  fonttitle=\bfseries,
  title=Tip,
  breakable
}

% ============================================================
% Custom Environments untuk OBE
% ============================================================

% --- Environment untuk Sub-CPMK ---
\newenvironment{subcpmk}{
  \vspace{0.5cm}
  \noindent\colorbox{blue!10}{\parbox{\dimexpr\textwidth-2\fboxsep}{
    \textbf{\large Sub-CPMK yang Dicakup dalam Bab Ini:}
  }}
  \vspace{0.3cm}
  \begin{itemize}[leftmargin=*, itemsep=5pt]
}{
  \end{itemize}
  \vspace{0.5cm}
}

% --- Environment untuk Aktivitas Pembelajaran ---
\newenvironment{aktivitas}{
  \vspace{0.5cm}
  \noindent\colorbox{green!10}{\parbox{\dimexpr\textwidth-2\fboxsep}{
    \textbf{\large Aktivitas Pembelajaran}
  }}
  \vspace{0.3cm}
  \begin{enumerate}[leftmargin=*, itemsep=8pt]
}{
  \end{enumerate}
  \vspace{0.5cm}
}

% --- Environment untuk Latihan ---
\newenvironment{latihan}{
  \vspace{0.5cm}
  \noindent\colorbox{orange!10}{\parbox{\dimexpr\textwidth-2\fboxsep}{
    \textbf{\large Latihan dan Refleksi}
  }}
  \vspace{0.3cm}
  \begin{enumerate}[leftmargin=*, itemsep=8pt]
}{
  \end{enumerate}
  \vspace{0.5cm}
}

% --- Environment untuk Asesmen ---
\newenvironment{asesmen}{
  \vspace{0.5cm}
  \noindent\colorbox{red!10}{\parbox{\dimexpr\textwidth-2\fboxsep}{
    \textbf{\large Asesmen (Evaluasi Kinerja)}
  }}
  \vspace{0.3cm}
}{
  \vspace{0.5cm}
}

% --- Environment untuk Checklist Kompetensi ---
\newenvironment{checklist}{
  \vspace{0.5cm}
  \noindent\colorbox{purple!10}{\parbox{\dimexpr\textwidth-2\fboxsep}{
    \textbf{\large Checklist Pencapaian Kompetensi}
  }}
  \vspace{0.3cm}
  \noindent\textit{Centang item berikut setelah Anda yakin telah menguasainya:}
  \vspace{0.2cm}
  \begin{itemize}[leftmargin=*, itemsep=5pt, label=$\square$]
}{
  \end{itemize}
  \vspace{0.5cm}
}

% --- Environment untuk Rangkuman ---
\newenvironment{rangkuman}{
  \vspace{0.5cm}
  \noindent\colorbox{gray!10}{\parbox{\dimexpr\textwidth-2\fboxsep}{
    \textbf{\large Rangkuman}
  }}
  \vspace{0.3cm}
}{
  \vspace{0.5cm}
}

% ============================================================
% Custom Commands
% ============================================================

% Command untuk code inline
\newcommand{\code}[1]{\texttt{#1}}

% Command untuk keyword C
\newcommand{\keyword}[1]{\textcolor{ckeyword}{\texttt{\textbf{#1}}}}

% Command untuk nama fungsi C
\newcommand{\func}[1]{\texttt{#1()}}

% ============================================================
% Pengaturan Tambahan
% ============================================================

% Spacing untuk paragraf
\setlength{\parskip}{0.5em}
\setlength{\parindent}{0pt}

% Depth untuk table of contents
\setcounter{tocdepth}{2}
\setcounter{secnumdepth}{3}
