\frontmatter

% ============================================================
% Halaman Sampul
% ============================================================
\begin{titlepage}
  \centering
  \vspace*{1cm}
  {\LARGE\bfseries Algoritma dan Pemrograman\par}
  \vspace{0.5cm}
  {\large Buku Ajar Berbasis Outcome-Based Education (OBE)\par}
  \vspace{1cm}
  {\large Mata Kuliah: Algoritma dan Pemrograman\par}
  {\large Kode: TIF-101\par}
  {\large Bahasa Pemrograman: C\par}
  \vspace{2cm}
  {\large Program Studi Teknik Informatika\par}
  {\large Fakultas Teknik\par}
  {\large Universitas Lorem Ipsum\par}
  \vfill
  {\large 2026\par}
\end{titlepage}

\cleardoublepage

% ============================================================
% Kata Pengantar
% ============================================================
\chapter*{Kata Pengantar}
\addcontentsline{toc}{chapter}{Kata Pengantar}

Puji syukur kehadirat Tuhan Yang Maha Esa atas terselesaikannya buku ajar \textit{Algoritma dan Pemrograman} ini. Buku ini disusun dengan pendekatan \textbf{Outcome-Based Education (OBE)}, yang berfokus pada pencapaian kompetensi terukur sesuai dengan Capaian Pembelajaran Lulusan (CPL) dan Capaian Pembelajaran Mata Kuliah (CPMK).

Algoritma dan Pemrograman merupakan mata kuliah fundamental yang harus dikuasai oleh setiap mahasiswa Teknik Informatika. Mata kuliah ini tidak hanya mengajarkan cara menulis kode dalam bahasa C, tetapi juga cara berpikir algoritmik dalam merancang solusi permasalahan komputasi secara sistematis.

Buku ini dirancang untuk mendukung pembelajaran mahasiswa semester 1 dengan pendekatan student-centered learning. Setiap bab dilengkapi dengan:
\begin{itemize}
  \item Sub-CPMK yang jelas dan terukur
  \item Materi pokok dengan contoh kode C yang lengkap
  \item Aktivitas pembelajaran yang mendorong eksplorasi mandiri
  \item Latihan dan refleksi untuk penguatan pemahaman
  \item Asesmen untuk mengukur pencapaian kompetensi
  \item Checklist kompetensi untuk self-assessment
\end{itemize}

Kami berharap buku ini dapat menjadi panduan yang efektif dalam perjalanan pembelajaran Anda menguasai Algoritma dan Pemrograman.

\vspace{1cm}
\begin{flushright}
Penyusun\\
Dr. Lorem Ipsum, M.Kom.
\end{flushright}

\cleardoublepage

% ============================================================
% Cara Menggunakan Buku Ini
% ============================================================
\chapter*{Cara Menggunakan Buku Ini}
\addcontentsline{toc}{chapter}{Cara Menggunakan Buku Ini}

Buku ajar ini dirancang dengan pendekatan OBE untuk memaksimalkan pencapaian pembelajaran Anda. Berikut panduan penggunaan buku ini:

\section*{Struktur Buku}

\textbf{Bab I: Pendahuluan dan Orientasi}\\
Memperkenalkan tujuan buku, keterkaitan dengan RPS, dan konteks kurikulum OBE.

\textbf{Bab II: Landasan Teori}\\
Menyajikan fondasi teoretis algoritma dan pemrograman yang menjadi basis pembelajaran seluruh bab berikutnya.

\textbf{Bab III-XVI: Unit Materi Inti}\\
Setiap bab mencakup satu topik utama Algoritma dan Pemrograman (flowchart, pseudocode, variabel, percabangan, perulangan, fungsi, array, string, struct, pengurutan, pencarian) dengan struktur lengkap: Sub-CPMK, materi, aktivitas, latihan, asesmen, dan checklist.

\textbf{Bab XVII: Evaluasi dan Integrasi}\\
Berisi asesmen komprehensif dan panduan refleksi untuk mengukur pencapaian kompetensi secara menyeluruh.

\textbf{Lampiran}\\
Menyediakan rubrik penilaian, template laporan, modul Praktikum 1, glosarium istilah algoritma dan C, serta referensi tambahan.

\section*{Komponen dalam Setiap Bab}

\begin{enumerate}
  \item \textbf{Sub-CPMK}: Baca dengan seksama untuk memahami kompetensi yang harus dicapai
  \item \textbf{Materi Pokok}: Pelajari dengan cermat, jalankan semua contoh kode
  \item \textbf{Aktivitas Pembelajaran}: Lakukan secara mandiri atau berkelompok
  \item \textbf{Latihan}: Kerjakan untuk menguji pemahaman Anda
  \item \textbf{Asesmen}: Gunakan untuk mengukur pencapaian Sub-CPMK
  \item \textbf{Checklist}: Centang setelah yakin menguasai setiap indikator
\end{enumerate}

\section*{Tips Belajar Efektif}

\begin{itemize}
  \item Jangan hanya membaca, praktikkan semua contoh kode C
  \item Gunakan compiler C (GCC, MinGW, atau IDE seperti Code::Blocks, Dev-C++) untuk eksperimen
  \item Kerjakan latihan sebelum melihat solusi
  \item Diskusikan konsep yang sulit dengan teman atau dosen
  \item Manfaatkan checklist untuk self-assessment berkala
  \item Kerjakan proyek mini untuk mengintegrasikan konsep yang dipelajari
\end{itemize}

\cleardoublepage

% ============================================================
% Identitas Mata Kuliah
% ============================================================
\chapter*{Identitas Mata Kuliah}
\addcontentsline{toc}{chapter}{Identitas Mata Kuliah}

\begin{tabular}{ll}
  Nama Program Studi & : Teknik Informatika \\
  Nama Mata Kuliah & : Algoritma dan Pemrograman \\
  Kode Mata Kuliah & : TIF-101 \\
  Semester & : 1 (Satu) \\
  SKS / Bobot Kredit & : 3 SKS (2 Teori, 1 Praktikum) \\
  Bahasa Pemrograman & : C \\
  Dosen Pengampu & : Dr. Lorem Ipsum, M.Kom. \\
  Tanggal Penyusunan & : 12 Februari 2026 \\
\end{tabular}

\vspace{1cm}

\section*{Capaian Pembelajaran Lulusan (CPL)}

CPL yang dibebankan pada mata kuliah ini mencakup kompetensi lulusan dalam aspek pengetahuan, keterampilan, dan sikap:

\begin{enumerate}
  \item \textbf{CPL-1 (Pengetahuan):} Menguasai konsep teoretis bidang pengetahuan tertentu secara umum dan konsep teoretis bagian khusus dalam bidang pengetahuan tersebut secara mendalam, serta mampu memformulasikan penyelesaian masalah prosedural.
  
  \item \textbf{CPL-2 (Keterampilan Umum):} Mampu menerapkan pemikiran logis, kritis, sistematis, dan inovatif dalam konteks pengembangan atau implementasi ilmu pengetahuan dan teknologi yang memperhatikan dan menerapkan nilai humaniora.
  
  \item \textbf{CPL-3 (Keterampilan Khusus):} Mampu merancang dan mengimplementasikan algoritma serta program komputer sederhana menggunakan bahasa pemrograman prosedural untuk menyelesaikan permasalahan komputasi.
  
  \item \textbf{CPL-4 (Sikap):} Menunjukkan sikap bertanggung jawab atas pekerjaan di bidang keahliannya secara mandiri dan mampu bekerja sama dalam tim.
\end{enumerate}

\section*{Capaian Pembelajaran Mata Kuliah (CPMK)}

Kemampuan atau kompetensi spesifik yang diharapkan mahasiswa kuasai setelah menyelesaikan mata kuliah:

\begin{enumerate}
  \item \textbf{CPMK-1:} Mahasiswa mampu memahami dan menjelaskan konsep algoritma, flowchart, dan pseudocode beserta karakteristiknya.
  
  \item \textbf{CPMK-2:} Mahasiswa mampu merancang algoritma untuk menyelesaikan masalah sederhana menggunakan flowchart dan pseudocode.
  
  \item \textbf{CPMK-3:} Mahasiswa mampu mengimplementasikan algoritma dalam bahasa C meliputi variabel, tipe data, I/O, operator, percabangan, dan perulangan.
  
  \item \textbf{CPMK-4:} Mahasiswa mampu menggunakan fungsi/prosedur, array, dan struktur data dasar dalam bahasa C.
  
  \item \textbf{CPMK-5:} Mahasiswa mampu menerapkan algoritma pengurutan dan pencarian sederhana serta mengimplementasikannya dalam program.
\end{enumerate}

\section*{Matriks Keterkaitan CPL dan CPMK}

\begin{table}[htbp]
\centering
\begin{tabular}{|c|c|c|>{\raggedright\arraybackslash}p{5cm}|}
  \hline
  \textbf{CPL} & \textbf{CPMK} & \textbf{Kontribusi} & \textbf{Keterangan} \\
  \hline
  CPL-1 & CPMK-1 & Tinggi & Penguasaan konsep algoritma dan pemrograman \\
  \hline
  CPL-1 & CPMK-2 & Tinggi & Kemampuan merancang algoritma \\
  \hline
  CPL-2 & CPMK-2, CPMK-3 & Tinggi & Penerapan pemikiran sistematis dalam pemrograman \\
  \hline
  CPL-2 & CPMK-4, CPMK-5 & Sedang & Pemikiran kritis dalam implementasi \\
  \hline
  CPL-3 & CPMK-2, CPMK-3 & Tinggi & Perancangan dan implementasi program \\
  \hline
  CPL-3 & CPMK-4, CPMK-5 & Tinggi & Penggunaan struktur data dan algoritma \\
  \hline
  CPL-4 & CPMK-3, CPMK-4 & Sedang & Tanggung jawab dalam implementasi \\
  \hline
  CPL-4 & CPMK-5 & Sedang & Kerja sama dalam proyek \\
  \hline
\end{tabular}
\caption{Matriks Keterkaitan CPL dan CPMK}
\end{table}

\cleardoublepage

% ============================================================
% Daftar Isi
% ============================================================
\phantomsection
\addcontentsline{toc}{chapter}{Daftar Isi}
\tableofcontents
\cleardoublepage

\clearpage
\phantomsection
\addcontentsline{toc}{chapter}{Daftar Gambar}
\listoffigures
\cleardoublepage

\phantomsection
\addcontentsline{toc}{chapter}{Daftar Tabel}
\listoftables
\cleardoublepage

\phantomsection
\addcontentsline{toc}{chapter}{Daftar Kode Program}
\lstlistoflistings
\cleardoublepage

