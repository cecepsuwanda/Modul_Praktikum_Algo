\chapter*{Lampiran}
\addcontentsline{toc}{chapter}{Lampiran}

% ============================================================
% Lampiran A: Rubrik Penilaian
% ============================================================
\section*{Lampiran A: Rubrik Penilaian Tugas Praktik}

\begin{table}[htbp]
\centering
\small
\begin{tabular}{|>{\raggedright\arraybackslash}p{2.8cm}|>{\raggedright\arraybackslash}p{3.6cm}|>{\raggedright\arraybackslash}p{3.6cm}|>{\raggedright\arraybackslash}p{3.6cm}|}
\hline
\textbf{Kriteria} & \textbf{Sangat Baik} & \textbf{Baik} & \textbf{Perlu Perbaikan} \\
\hline
Desain Algoritma & Flowchart/pseudo-code jelas, logika benar & Desain cukup jelas & Desain belum tepat \\
\hline
Implementasi Kode C & Kode benar, kompilasi sukses & Ada minor error & Banyak kesalahan \\
\hline
Modularitas & Fungsi digunakan dengan tepat & Sebagian modular & Monolitik \\
\hline
Keterbacaan Kode & Nama variabel jelas, indentasi rapi & Cukup terbaca & Sulit dibaca \\
\hline
Ketepatan Output & Output sesuai spesifikasi & Minor deviation & Output salah \\
\hline
\end{tabular}
\caption{Rubrik Penilaian Tugas Praktik Algoritma dan Pemrograman}
\end{table}

\subsection*{Pemetaan Rubrik ke CPMK}

Kriteria pada rubrik di atas terhubung ke CPMK sebagai berikut:

\begin{table}[htbp]
\centering
\small
\begin{tabular}{|>{\raggedright\arraybackslash}p{3.2cm}|c|>{\raggedright\arraybackslash}p{5cm}|}
\hline
\textbf{Kriteria Rubrik} & \textbf{CPMK} & \textbf{Indikator} \\
\hline
Desain Algoritma & CPMK-1, CPMK-2 & Flowchart/pseudocode jelas dan benar \\
\hline
Implementasi Kode C & CPMK-3 & Variabel, I/O, percabangan, perulangan benar \\
\hline
Modularitas & CPMK-4 & Penggunaan fungsi dan struktur data \\
\hline
Keterbacaan Kode & CPMK-3, CPMK-4 & Penamaan, indentasi, dokumentasi \\
\hline
Ketepatan Output & CPMK-2 s.d. CPMK-5 & Hasil sesuai spesifikasi dan algoritma \\
\hline
\end{tabular}
\caption{Pemetaan Kriteria Rubrik ke CPMK}
\end{table}

% ============================================================
% Lampiran B: Contoh Template Laporan
% ============================================================
\section*{Lampiran B: Contoh Template Laporan Tugas}
\begin{enumerate}
  \item \textbf{Judul dan Identitas Mahasiswa}
  \begin{itemize}
    \item Nama lengkap dan NIM
    \item Kelas dan kelompok (jika ada)
    \item Judul tugas/proyek
  \end{itemize}
  \item \textbf{Deskripsi Masalah}
  \begin{itemize}
    \item Ringkasan requirements
    \item Input dan output yang diharapkan
    \item Batasan dan asumsi
  \end{itemize}
  \item \textbf{Flowchart dan/atau Pseudocode}
  \begin{itemize}
    \item Desain algoritma lengkap
    \item Penjelasan setiap langkah
  \end{itemize}
  \item \textbf{Implementasi (cuplikan kode C penting)}
  \begin{itemize}
    \item Struktur data yang digunakan
    \item Fungsi-fungsi utama
    \item Bagian kode yang krusial
  \end{itemize}
  \item \textbf{Hasil Pengujian}
  \begin{itemize}
    \item Screenshot output program
    \item Data uji yang digunakan
    \item Analisis hasil pengujian
  \end{itemize}
  \item \textbf{Refleksi dan Kesimpulan}
  \begin{itemize}
    \item Pembelajaran yang didapat
    \item Kesulitan yang dihadapi
    \item Saran perbaikan
  \end{itemize}
\end{enumerate}

% ============================================================
% Lampiran H: Contoh Rubrik Refleksi dan Portofolio
% ============================================================
\section*{Lampiran H: Contoh Rubrik Refleksi dan Pemetaan ke CPMK}

\subsection*{Contoh Rubrik Refleksi (per Bab)}

Refleksi mahasiswa per bab dapat dinilai dengan rubrik berikut. Rubrik ini memetakan indikator ke CPMK yang relevan.

\begin{table}[htbp]
\centering
\small
\begin{tabular}{|>{\raggedright\arraybackslash}p{2.8cm}|>{\raggedright\arraybackslash}p{2.8cm}|>{\raggedright\arraybackslash}p{2.8cm}|>{\raggedright\arraybackslash}p{2.2cm}|}
\hline
\textbf{Kriteria} & \textbf{Baik (3)} & \textbf{Cukup (2)} & \textbf{Kurang (1)} \\
\hline
Pemahaman konsep & Menjelaskan dengan tepat dan contoh jelas & Menjelaskan dengan benar, contoh kurang & Pemahaman samar atau salah \\
\hline
Keterkaitan dengan CPMK & Menyebutkan Sub-CPMK yang tercapai & Menyebutkan tanpa rincian & Tidak mengaitkan ke CPMK \\
\hline
Kesulitan dan solusi & Mengidentifikasi kesulitan dan langkah perbaikan & Mengidentifikasi kesulitan saja & Tidak reflektif \\
\hline
\end{tabular}
\caption{Rubrik Refleksi per Bab}
\end{table}

\subsection*{Contoh Portofolio dan Pemetaan ke CPMK}

Portofolio dapat berisi: (1) hasil tugas per bab (flowchart, kode C, laporan), (2) refleksi per bab, (3) proyek integrasi. Pemetaan artefak ke CPMK:

\begin{itemize}
  \item \textbf{CPMK-1:} Ringkasan konsep algoritma, flowchart, pseudocode; jawaban latihan teori Bab 2--4.
  \item \textbf{CPMK-2:} Flowchart dan pseudocode dari tugas Bab 3--4 dan proyek.
  \item \textbf{CPMK-3:} Kode C untuk variabel, operator, percabangan, perulangan (Bab 5--10).
  \item \textbf{CPMK-4:} Kode C untuk fungsi, array, string, struct (Bab 11--15).
  \item \textbf{CPMK-5:} Implementasi sorting dan searching (Bab 16) serta integrasi dalam proyek.
\end{itemize}

% ============================================================
% Lampiran C: Glosarium
% ============================================================
\section*{Lampiran C: Glosarium Istilah Algoritma dan C}
\begin{itemize}
  \item \textbf{Algoritma}: Urutan langkah logis dan terdefinisi untuk menyelesaikan masalah
  \item \textbf{Flowchart}: Representasi visual algoritma dengan simbol standar
  \item \textbf{Pseudocode}: Deskripsi algoritma dalam bahasa manusia menyerupai kode
  \item \textbf{Variabel}: Tempat penyimpanan data di memori dengan nama dan tipe
  \item \textbf{Tipe Data}: Klasifikasi data (int, float, char, dll)
  \item \textbf{Array}: Kumpulan elemen bertipe sama yang disimpan berurutan
  \item \textbf{Struct}: Tipe data bentukan yang menggabungkan beberapa variabel
  \item \textbf{Fungsi}: Blok kode yang dapat dipanggil berulang dengan nama
  \item \textbf{Percabangan}: Pengambilan keputusan berdasarkan kondisi (if, switch)
  \item \textbf{Perulangan}: Pengulangan eksekusi blok kode (for, while, do-while)
  \item \textbf{Sorting}: Proses mengurutkan elemen array
  \item \textbf{Searching}: Proses mencari elemen dalam array
  \item \textbf{Kompleksitas}: Ukuran efisiensi algoritma (Big O notation)
  \item \textbf{OBE}: Outcome-Based Education, pendidikan berbasis capaian
  \item \textbf{CPL}: Capaian Pembelajaran Lulusan
  \item \textbf{CPMK}: Capaian Pembelajaran Mata Kuliah
\end{itemize}

% ============================================================
% Lampiran D: Cheat Sheet Sintaks Bahasa C
% ============================================================
\section*{Lampiran D: Cheat Sheet Sintaks Bahasa C}

\subsection*{Tipe Data Dasar}
\begin{table}[htbp]
\centering
\footnotesize
\begin{tabular}{|l|l|l|}
\hline
\textbf{Tipe} & \textbf{Format} & \textbf{Ukuran} \\
\hline
int & \%d, \%i & 4 byte \\
\hline
float & \%f & 4 byte \\
\hline
double & \%lf & 8 byte \\
\hline
char & \%c & 1 byte \\
\hline
string (char[]) & \%s & - \\
\hline
\end{tabular}
\end{table}

\subsection*{Input/Output}
\begin{lstlisting}[language=C]
// Output
printf("Teks: %d", variabel_int);
printf("Float: %.2f", variabel_float);
printf("Karakter: %c", variabel_char);

// Input
scanf("%d", &variabel_int);
scanf("%f", &variabel_float);
scanf("%s", variabel_string);
scanf(" %c", &variabel_char); // spasi untuk skip whitespace
\end{lstlisting}

\subsection*{Struktur Kontrol}
\begin{lstlisting}[language=C]
// If-Else
if (kondisi) {
    // blok jika benar
} else if (kondisi_lain) {
    // blok jika kondisi lain benar
} else {
    // blok jika semua salah
}

// Switch-Case
switch (variabel) {
    case nilai1:
        // aksi untuk nilai1
        break;
    case nilai2:
        // aksi untuk nilai2
        break;
    default:
        // aksi default
        break;
}

// For Loop
for (inisialisasi; kondisi; increment) {
    // blok yang diulang
}

// While Loop
while (kondisi) {
    // blok yang diulang
}

// Do-While Loop
do {
    // blok yang diulang
} while (kondisi);
\end{lstlisting}

\subsection*{Array dan Struct}
\begin{lstlisting}[language=C]
// Array
int angka[10];                    // deklarasi
angka[0] = 100;                 // assignment
int nilai[] = {90, 85, 78};      // inisialisasi

// Struct
typedef struct {
    int nim;
    char nama[50];
    float ipk;
} Mahasiswa;

Mahasiswa mhs1;                   // deklarasi
mhs1.nim = 12345;                 // assignment
strcpy(mhs1.nama, "Budi");       // string assignment
\end{lstlisting}

% ============================================================
% Lampiran E: Modul Praktikum 1 (Pertemuan 7)
% ============================================================
\section*{Lampiran E: Modul Praktikum 1 -- Dasar Pemrograman}

Praktikum 1 dilaksanakan pada pertemuan 7 sesuai RPS. Tujuan: menguasai Sub-CPMK 3.1 dan 3.2 (struktur seleksi dan perulangan). Lakukan aktivitas berikut secara berurutan:

\begin{enumerate}
  \item \textbf{Bab III}: Buat flowchart untuk menentukan bilangan ganjil/genap dan konversi suhu Celsius ke Fahrenheit. Implementasikan ke program C.
  \item \textbf{Bab IV}: Tulis pseudocode untuk rata-rata tiga bilangan, konversikan ke C.
  \item \textbf{Bab V}: Buat program membaca nama dan umur; program konversi suhu Celsius ke Fahrenheit.
  \item \textbf{Bab VI}: Buat program kalkulator sederhana dengan operator +, -, *, /.
  \item \textbf{Bab VII}: Buat program menentukan bilangan ganjil/genap dan grade nilai (A--E) dari input skor.
\end{enumerate}

Susun laporan menggunakan template Lampiran B. Asisten laboratorium akan menilai berdasarkan rubrik Lampiran A.

% ============================================================
% Lampiran F: Referensi Tambahan
% ============================================================
\section*{Lampiran F: Referensi Tambahan}

\subsection*{Online Resources}
\begin{itemize}
  \item \textbf{cppreference.com}: \url{https://en.cppreference.com/w/c} - Referensi lengkap bahasa C
  \item \textbf{GeeksforGeeks}: \url{https://www.geeksforgeeks.org/c/} - Tutorial dan contoh kode
  \item \textbf{SPADA Indonesia}: \url{https://lmsspada.kemdiktisaintek.go.id/} - Materi pembelajaran resmi
  \item \textbf{GCC Documentation}: \url{https://gcc.gnu.org/onlinedocs/} - Dokumentasi compiler
  \item \textbf{Learn-C.org}: \url{https://www.learn-c.org/} - Tutorial interaktif C
\end{itemize}

\subsection*{Buku Referensi}
\begin{itemize}
  \item Kernighan, B.W., \& Ritchie, D.M. (1988). \textit{The C Programming Language}. Prentice Hall.
  \item Deitel, H.M., \& Deitel, P.J. (2016). \textit{C How to Program}. Pearson.
  \item Cormen, T.H., et al. (2009). \textit{Introduction to Algorithms}. MIT Press.
\end{itemize}

\subsection*{Alat Pengembangan}
\begin{itemize}
  \item \textbf{Compiler}: GCC/MinGW, Clang, Visual Studio C++
  \item \textbf{IDE}: Code::Blocks, Dev-C++, Visual Studio Code
  \item \textbf{Debugger}: GDB, integrated debugger dalam IDE
  \item \textbf{Version Control}: Git, GitHub Desktop
\end{itemize}

% ============================================================
% Lampiran G: Soal Latihan Tambahan
% ============================================================
\section*{Lampiran G: Soal Latihan Tambahan}

\subsection*{Level Dasar}
\begin{enumerate}
  \item Buat program menghitung luas persegi panjang
  \item Buat program konversi kilometer ke mil
  \item Buat program menampilkan bilangan genap 1-100
\end{enumerate}

\subsection*{Level Menengah}
\begin{enumerate}
  \item Buat program kalkulator scientific (+, -, *, /, \%, pangkat)
  \item Buat program menentukan tahun kabisat
  \item Buat program menghitung faktorial
\end{enumerate}

\subsection*{Level Lanjutan}
\begin{enumerate}
  \item Implementasi insertion sort dan merge sort
  \item Buat program game tebak angka
  \item Buat program database sederhana dengan file I/O
\end{enumerate}
