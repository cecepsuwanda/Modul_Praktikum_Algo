\documentclass[../main.tex]{subfiles}
\ifSubfilesClassLoaded{\setcounter{chapter}{12}}{}
\begin{document}

\chapter{Array Dua Dimensi dan Matriks}

\begin{subcpmk}
  \item Sub-CPMK 4.2: Menggunakan array dua dimensi dan operasi matriks
\end{subcpmk}

\noindent\textbf{Materi Pokok:} Array 2D dan representasi row-major; operasi matriks: penjumlahan, transpose, perkalian \cite{ref7,ref2}.

\section{Array Dua Dimensi}

Array 2D merepresentasikan matriks: baris dan kolom. Deklarasi: \code{tipe nama[baris][kolom];} misalnya \code{int mat[3][4];}. Akses: \code{mat[i][j]} untuk baris $i$ kolom $j$. Dalam memori, elemen disimpan secara row-major (baris demi baris).

Inisialisasi: \code{int m[2][3] = \{\{1,2,3\},\{4,5,6\}\};}. Loop bersarang untuk traversing: loop luar baris, loop dalam kolom. Array 2D sebagai parameter fungsi: tentukan jumlah kolom, misalnya \code{func(int arr[][10])}.

\section{Operasi Matriks Sederhana}

Operasi matriks dasar: penjumlahan (elemen seindeks dijumlahkan), pengurangan, transpose (baris jadi kolom, kolom jadi baris). Untuk perkalian matriks: baris A dikali kolom B; hasil matriks berukuran (baris A x kolom B). Contoh penjumlahan: C[i][j] = A[i][j] + B[i][j] untuk semua i, j. Transpose: B[j][i] = A[i][j]. Perhatikan dimensi matriks untuk operasi yang valid (penjumlahan butuh ukuran sama).


\begin{aktivitas}
  \item Buat program penjumlahan dua matriks.
  \item Buat program transpose matriks.
\end{aktivitas}

\begin{latihan}
  \item Jelaskan representasi array 2D dalam memori (row-major)!
  \item Buat program perkalian matriks 2x2!
  \item \textbf{Refleksi}: Kapan Anda memilih array 2D daripada beberapa array 1D? Berikan contoh masalah yang cocok dengan array 2D.
\end{latihan}

\begin{asesmen}
\textbf{Instrumen untuk Sub-CPMK 4.2}: Buat program C yang membaca dua matriks 2x2, menghitung penjumlahan dan perkalian keduanya, lalu menampilkan hasil. Gunakan nested loop untuk perkalian matriks.
\end{asesmen}

\begin{checklist}
  \item Saya dapat menggunakan array 2D
  \item Saya dapat melakukan operasi matriks sederhana
\end{checklist}

\begin{rangkuman}
Array 2D untuk matriks. Akses mat[i][j]. Operasi: penjumlahan, transpose, perkalian.
\end{rangkuman}

\ifSubfilesClassLoaded{
  \renewcommand{\bibname}{Daftar Pustaka}
  \bibliographystyle{plain}
  \bibliography{../references}
}{}
\end{document}
