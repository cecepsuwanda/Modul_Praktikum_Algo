\section{Tinjauan Pencapaian Kompetensi}

Tinjauan ini membantu Anda memastikan kesiapan sebelum UTS/UAS. Pastikan Anda menguasai kompetensi berikut sesuai CPMK:

\begin{itemize}
  \item \textbf{CPMK-1:} Konsep algoritma dan enam karakteristiknya; definisi dan fungsi flowchart serta pseudocode.
  \item \textbf{CPMK-2:} Merancang algoritma dengan flowchart dan pseudocode untuk masalah sederhana.
  \item \textbf{CPMK-3:} Variabel, tipe data, I/O (\code{printf}/\code{scanf}); operator; percabangan if-else dan switch; perulangan for, while, do-while.
  \item \textbf{CPMK-4:} Fungsi (parameter, return, scope); array 1D dan 2D; string dan \code{string.h}; struct serta array of struct.
  \item \textbf{CPMK-5:} Bubble sort, selection sort, linear search, binary search; pemahaman kompleksitas sederhana.
\end{itemize}

Gunakan \textbf{checklist} di setiap bab untuk self-assessment. Jika ada indikator yang belum Anda centang, pelajari kembali materi bab terkait dan kerjakan latihan. Konsultasikan dengan dosen jika ada CPMK yang belum tercapai.
