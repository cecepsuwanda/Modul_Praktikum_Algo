\section{Panduan Proyek Integrasi (Pertemuan 15)}

Sesuai RPS, pertemuan 15 ditujukan untuk integrasi materi, proyek, dan review. Proyek integrasi mengukur pencapaian CPMK-2 hingga CPMK-5 dengan menggabungkan semua konsep yang telah dipelajari: algoritma, flowchart, pseudocode, variabel, struktur kontrol, fungsi, array, struct, pengurutan, dan pencarian.

\subsection{Studi Kasus 1: Manajemen Data Mahasiswa}

\textbf{Deskripsi:} Buat program yang mengelola data mahasiswa dengan fitur lengkap.

\textbf{Spesifikasi Requirements:}
\begin{itemize}
  \item \textbf{Data Mahasiswa:} NIM (int), nama (string), IPK (float), umur (int)
  \item \textbf{Fitur Input:} Tambah data mahasiswa baru
  \item \textbf{Fitur Tampil:} Tampilkan semua data mahasiswa
  \item \textbf{Fitur Pencarian:} Cari mahasiswa berdasarkan NIM
  \item \textbf{Fitur Analisis:} Hitung rata-rata IPK semua mahasiswa
  \item \textbf{Fitur Pengurutan:} Urutkan data berdasarkan nama atau IPK
\end{itemize}

\textbf{Teknis Implementasi:}
\begin{itemize}
  \item Gunakan \code{struct Mahasiswa} untuk merepresentasikan data
  \item Gunakan array \code{Mahasiswa kelas[50]} untuk menyimpan data
  \item Buat fungsi terpisah untuk setiap operasi
  \item Implementasikan menu interaktif dengan \code{switch-case}
  \item Gunakan bubble sort atau selection sort untuk pengurutan
  \item Gunakan linear search untuk pencarian
\end{itemize}

\textbf{Contoh Struktur Program:}
\begin{lstlisting}[language=C]
#include <stdio.h>
#include <string.h>

typedef struct {
    int nim;
    char nama[50];
    float ipk;
    int umur;
} Mahasiswa;

// Fungsi-fungsi yang dibutuhkan
void tambahMahasiswa(Mahasiswa kelas[], int *jumlah);
void tampilkanSemua(Mahasiswa kelas[], int jumlah);
void cariMahasiswa(Mahasiswa kelas[], int jumlah);
void hitungRataIPK(Mahasiswa kelas[], int jumlah);
void urutkanBerdasarkanIPK(Mahasiswa kelas[], int jumlah);
void urutkanBerdasarkanNama(Mahasiswa kelas[], int jumlah);

int main() {
    Mahasiswa kelas[50];
    int jumlah = 0;
    int pilihan;
    
    do {
        // Menu utama
        printf("\n=== MENU MANAJEMEN MAHASISWA ===\n");
        printf("1. Tambah Mahasiswa\n");
        printf("2. Tampilkan Semua Mahasiswa\n");
        printf("3. Cari Mahasiswa\n");
        printf("4. Hitung Rata-rata IPK\n");
        printf("5. Urutkan berdasarkan IPK\n");
        printf("6. Urutkan berdasarkan Nama\n");
        printf("0. Keluar\n");
        printf("Pilihan: ");
        scanf("%d", &pilihan);
        
        switch(pilihan) {
            case 1: tambahMahasiswa(kelas, &jumlah); break;
            case 2: tampilkanSemua(kelas, jumlah); break;
            case 3: cariMahasiswa(kelas, jumlah); break;
            case 4: hitungRataIPK(kelas, jumlah); break;
            case 5: urutkanBerdasarkanIPK(kelas, jumlah); break;
            case 6: urutkanBerdasarkanNama(kelas, jumlah); break;
        }
    } while (pilihan != 0);
    
    return 0;
}
\end{lstlisting}

\subsection{Studi Kasus 2: Sistem Inventory Barang}

\textbf{Deskripsi:} Buat program inventory untuk mengelola data barang.

\textbf{Spesifikasi Requirements:}
\begin{itemize}
  \item \textbf{Data Barang:} Kode (int), nama (string), jumlah (int), harga (float)
  \item \textbf{Fitur CRUD:} Create, Read, Update, Delete barang
  \item \textbf{Fitur Pencarian:} Cari berdasarkan kode atau nama
  \item \textbf{Fitur Laporan:} Hitung total nilai inventory
  \item \textbf{Fitur Sorting:} Urutkan berdasarkan nama atau harga
\end{itemize}

\subsection{Rubrik Penilaian Proyek}

\begin{table}[htbp]
\centering
\small
\begin{tabular}{|>{\raggedright\arraybackslash}p{3cm}|>{\raggedright\arraybackslash}p{2cm}|>{\raggedright\arraybackslash}p{7cm}|}
\hline
\textbf{Kriteria} & \textbf{Bobot} & \textbf{Deskripsi Penilaian} \\
\hline
Desain Algoritma & 20\% & 
\begin{itemize}
  \item Flowchart lengkap dan benar (10\%)
  \item Pseudocode terstruktur (10\%)
\end{itemize} \\
\hline
Implementasi Kode & 40\% & 
\begin{itemize}
  \item Kode benar dan kompilasi sukses (15\%)
  \item Modularitas dan fungsi yang baik (15\%)
  \item Error handling yang memadai (10\%)
\end{itemize} \\
\hline
Struktur Data & 25\% & 
\begin{itemize}
  \item Penggunaan struct yang tepat (10\%)
  \item Pengelolaan array dengan benar (10\%)
  \item Manipulasi string yang proper (5\%)
\end{itemize} \\
\hline
Integrasi Algoritma & 15\% & 
\begin{itemize}
  \item Implementasi sorting yang benar (8\%)
  \item Implementasi searching yang efisien (7\%)
\end{itemize} \\
\hline
\end{tabular}
\caption{Rubrik Penilaian Proyek Integrasi}
\end{table}

\subsection{Tahap Pengembangan Proyek}

\begin{enumerate}
  \item \textbf{Analisis (Pekan 1):} Pahami requirements, buat flowchart
  \item \textbf{Desain (Pekan 2):} Buat pseudocode, definisikan struct dan fungsi
  \item \textbf{Implementasi (Pekan 3-4):} Koding fitur-fitur utama
  \item \textbf{Integrasi (Pekan 5):} Gabungkan semua fitur, testing
  \item \textbf{Dokumentasi (Pekan 6):} Buat laporan dan presentasi
\end{enumerate}

\subsection{Format Pengumpulan}

\begin{itemize}
  \item \textbf{Source Code:} File .c dan .h (jika ada)
  \item \textbf{Executable:} File .exe (hasil kompilasi)
  \item \textbf{Dokumentasi:} Laporan PDF (maks 10 halaman)
  \item \textbf{Flowchart:} Gambar flowchart (PNG/JPG)
  \item \textbf{Demo:} Video demo program (opsional, nilai tambah)
\end{itemize}
