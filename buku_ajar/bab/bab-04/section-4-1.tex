\section{Definisi dan Notasi Pseudocode}

Pseudocode adalah deskripsi algoritma dalam bahasa manusia yang menyerupai kode program namun tidak terikat pada sintaks bahasa pemrograman tertentu \cite{pseudocode_wikipedia}. Kata "pseudocode" berasal dari "pseudo" (semu) dan "code" (kode), yang berarti kode semu atau tiruan kode program. Pseudocode pertama kali digunakan pada tahun 1950-an sebagai alat untuk mendokumentasikan algoritma sebelum implementasi dalam bahasa assembly.

\subsection{Fungsi Pseudocode}

Pseudocode memiliki beberapa fungsi penting dalam pengembangan perangkat lunak:

\begin{enumerate}
  \item \textbf{Perancangan Logika:} Memungkinkan fokus pada algoritma tanpa detail sintaks
  \item \textbf{Dokumentasi:} Membuat dokumentasi algoritma yang mudah dipahami
  \item \textbf{Komunikasi:} Memfasilitasi komunikasi tim tanpa terikat bahasa pemrograman
  \item \textbf{Validasi:} Memungkinkan pengecekan logika sebelum implementasi
  \item \textbf{Pembelajaran:} Membantu pemula memahami konsep pemrograman
\end{enumerate}

\subsection{Standar Notasi Pseudocode}

Berikut adalah standar notasi pseudocode yang umum digunakan:

\begin{table}[htbp]
\centering
\footnotesize
\begin{tabular}{|>{\raggedright\arraybackslash}p{2.6cm}|>{\raggedright\arraybackslash}p{4.25cm}|>{\raggedright\arraybackslash}p{4.05cm}|}
\hline
\textbf{Kategori} & \textbf{Notasi Pseudocode} & \textbf{Contoh} \\
\hline
Struktur & MULAI, SELESAI & MULAI ... SELESAI \\
\hline
Input/Output & BACA, CETAK, TULIS & BACA nilai, CETAK ``Hello'' \\
\hline
Assignment & := atau = & nilai := 10, x = y + 1 \\
\hline
Percabangan & JIKA...MAKA...JIKA TIDAK & JIKA x > 0 MAKA CETAK ``Positif'' \\
\hline
Perulangan & UNTUK...DARI...\allowbreak SAMPAI & UNTUK i := 1 DARI 1 SAMPAI 10 \\
\hline
Perulangan & SELAMA...LAKUKAN & SELAMA $x > 0$ LAKUKAN $x := x - 1$ \\
\hline
Prosedur & PROSEDUR nama() & PROSEDUR hitungRata() \\
\hline
Fungsi & FUNGSI nama() & FUNGSI maks(a, b) \\
\hline
Komentar & // atau /* ... */ & // Ini komentar \\
\hline
\end{tabular}
\caption{Standar Notasi Pseudocode}
\end{table}

\subsection{Aturan Penulisan Pseudocode}

\begin{itemize}
  \item \textbf{Konsistensi:} Gunakan notasi yang sama di seluruh pseudocode
  \item \textbf{Indentasi:} Gunakan indentasi untuk menunjukkan blok kode
  \item \textbf{Bahasa Jelas:} Gunakan bahasa Indonesia atau Inggris yang jelas
  \item \textbf{Struktur Logis:} Ikuti alur logika yang sistematis
  \item \textbf{Detail Tepat:} Berikan detail cukup namun tidak terlalu teknis
\end{itemize}

\subsection{Perbandingan Pseudocode dan Flowchart}

\begin{table}[htbp]
\centering
\small
\begin{tabular}{|>{\raggedright\arraybackslash}p{5cm}|>{\raggedright\arraybackslash}p{5cm}|}
\hline
\textbf{Pseudocode} & \textbf{Flowchart} \\
\hline
Teks-based, linear & Visual, grafis \\
\hline
Lebih detail implementasi & Lebih fokus pada alur logika \\
\hline
Mudah diketik & Membutuhkan alat gambar \\
\hline
Baik untuk algoritma kompleks & Baik untuk presentasi \\
\hline
Struktur hierarki jelas & Alur visual langsung \\
\hline
\end{tabular}
\caption{Perbandingan Pseudocode dan Flowchart}
\end{table}

Pseudocode memudahkan perancangan logika algoritma sebelum implementasi ke bahasa pemrograman seperti C. Dengan pseudocode, programmer dapat fokus pada alur logika tanpa khawatir tentang detail sintaks.
