\section{Contoh Pseudocode dan Konversi ke Bahasa C}

Berikut adalah contoh-contoh pseudocode lengkap dengan konversinya ke bahasa C untuk memahami hubungan antara desain algoritma dan implementasi kode.

\subsection{Contoh 1: Program Hello World}

\textbf{Pseudocode:}
\begin{lstlisting}[language=]
MULAI
    CETAK "Hello, World"
SELESAI
\end{lstlisting}

\textbf{Konversi ke C:}
\begin{lstlisting}[language=C, caption={Konversi Hello World ke C}]
#include <stdio.h>

int main() {
    printf("Hello, World\n");
    return 0;
}
\end{lstlisting}

\textbf{Analisis Konversi:}
\begin{itemize}
  \item MULAI → \code{int main() \{}
  \item CETAK → \code{printf()}
  \item SELESAI → \code{return 0; \}}
\end{itemize}

\subsection{Contoh 2: Membaca Input dan Menampilkan Output}

\textbf{Pseudocode:}
\begin{lstlisting}[language=]
MULAI
    BACA nama
    CETAK "Halo, ", nama
SELESAI
\end{lstlisting}

\textbf{Konversi ke C:}
\begin{lstlisting}[language=C, caption={Konversi Input/Output ke C}]
#include <stdio.h>

int main() {
    char nama[50];
    
    printf("Masukkan nama: ");
    scanf("%s", nama);
    printf("Halo, %s\n", nama);
    
    return 0;
}
\end{lstlisting}

\textbf{Analisis Konversi:}
\begin{itemize}
  \item BACA nama → \code{scanf("\%s", nama)}
  \item CETAK → \code{printf()}
  \item Perlu deklarasi variabel: \code{char nama[50];}
\end{itemize}

\subsection{Contoh 3: Percabangan If-Else}

\textbf{Pseudocode:}
\begin{lstlisting}[language=]
MULAI
    BACA nilai
    JIKA nilai > 60 MAKA
        CETAK "Lulus"
    JIKA TIDAK
        CETAK "Tidak Lulus"
SELESAI
\end{lstlisting}

\textbf{Konversi ke C:}
\begin{lstlisting}[language=C, caption={Konversi Percabangan ke C}]
#include <stdio.h>

int main() {
    int nilai;
    
    printf("Masukkan nilai: ");
    scanf("%d", &nilai);
    
    if (nilai > 60) {
        printf("Lulus\n");
    } else {
        printf("Tidak Lulus\n");
    }
    
    return 0;
}
\end{lstlisting}

\subsection{Contoh 4: Perulangan For}

\textbf{Pseudocode:}
\begin{lstlisting}[language=]
MULAI
    UNTUK i := 1 DARI 1 SAMPAI 5 LAKUKAN
        CETAK i
SELESAI
\end{lstlisting}

\textbf{Konversi ke C:}
\begin{lstlisting}[language=C, caption={Konversi Perulangan ke C}]
#include <stdio.h>

int main() {
    int i;
    
    for (i = 1; i <= 5; i++) {
        printf("%d\n", i);
    }
    
    return 0;
}
\end{lstlisting}

\subsection{Contoh 5: Menghitung Rata-rata}

\textbf{Pseudocode:}
\begin{lstlisting}[language=]
MULAI
    BACA bil1, bil2, bil3
    jumlah := bil1 + bil2 + bil3
    rata := jumlah / 3
    CETAK "Rata-rata: ", rata
SELESAI
\end{lstlisting}

\textbf{Konversi ke C:}
\begin{lstlisting}[language=C, caption={Konversi Perhitungan Rata-rata ke C}]
#include <stdio.h>

int main() {
    float bil1, bil2, bil3, jumlah, rata;
    
    printf("Masukkan tiga bilangan: ");
    scanf("%f %f %f", &bil1, &bil2, &bil3);
    
    jumlah = bil1 + bil2 + bil3;
    rata = jumlah / 3;
    
    printf("Rata-rata: %.2f\n", rata);
    
    return 0;
}
\end{lstlisting}

\subsection{Tabel Pemetaan Konversi}

\begin{table}[htbp]
\centering
\small
\begin{tabular}{|>{\raggedright\arraybackslash}p{3cm}|>{\raggedright\arraybackslash}p{4cm}|}
\hline
\textbf{Pseudocode} & \textbf{Bahasa C} \\
\hline
MULAI & \code{int main() \{} \\
\hline
SELESAI & \code{return 0; \}} \\
\hline
BACA variabel & \code{scanf("\%format", \&variabel);} \\
\hline
CETAK teks & \code{printf("teks\textbackslash n");} \\
\hline
variabel := nilai & \code{variabel = nilai;} \\
\hline
JIKA kondisi MAKA & \code{if (kondisi) \{} \\
\hline
JIKA TIDAK & \code{\} else \{} \\
\hline
UNTUK i := 1 SAMPAI n & \code{for (i = 1; i <= n; i++) \{} \\
\hline
SELAMA kondisi & \code{while (kondisi) \{} \\
\hline
LAKUKAN & \code{// Isi loop} \\
\hline
\end{tabular}
\caption{Pemetaan Konversi Pseudocode ke Bahasa C}
\end{table}

Konversi pseudocode ke C dilakukan dengan memetakan setiap notasi pseudocode ke sintaks yang sesuai dalam bahasa C, ditambah dengan deklarasi variabel dan header yang diperlukan.
