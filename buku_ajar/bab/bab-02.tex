\documentclass[../main.tex]{subfiles}
\ifSubfilesClassLoaded{\setcounter{chapter}{1}}{}
\begin{document}

\chapter{Landasan Teori dan Konsep Dasar Algoritma}

\begin{subcpmk}
  \item Sub-CPMK 1.1: Menjelaskan definisi dan karakteristik algoritma (input, output, definiteness, finiteness, effectiveness, determinism)
\end{subcpmk}

\noindent\textit{Bab ini juga menyajikan materi pengantar pemrograman prosedural dan bahasa C sebagai fondasi untuk CPMK-3.}

% ============================================================
% MATERI POKOK
% ============================================================
\section{Sejarah dan Konsep Algoritma}

Kata \textit{algoritma} berasal dari nama matematikawan Persia abad ke-9, Abu Abdullah Muhammad bin Musa al-Khwarizmi (780-850 M). Al-Khwarizmi adalah ilmuwan di House of Wisdom (Bayt al-Hikmah) di Baghdad yang menulis kitab \textit{Al-Jabr wa'l-Muqabala} (Buku tentang Penyempurnaan dan Penyeimbangan) pada tahun 825 M. Kitab ini memperkenalkan metode sistematis untuk menyelesaikan persamaan aljabar linear dan kuadrat. Namanya dilatinkan menjadi "Algoritmi" oleh para penerjemah Eropa dan kemudian berkembang menjadi "algorithm" dalam bahasa Inggris.

\subsection{Kontribusi Ilmiah al-Khwarizmi}

Al-Khwarizmi memberikan kontribusi fundamental dalam beberapa bidang:

\begin{itemize}
  \item \textbf{Matematika:} Memperkenalkan sistem desimal posisional dan konsep nol dari India ke dunia Islam dan Eropa
  \item \textbf{Aljabar:} Mengembangkan metode sistematis untuk menyelesaikan persamaan dengan langkah-langkah terstruktur
  \item \textbf{Astronomi:} Membuat tabel astronomi yang akurat untuk navigasi dan penentuan waktu shalat
  \item \textbf{Geografi:} Memperbaiki karya Ptolemy dan membuat peta dunia yang lebih akurat
\end{itemize}

Kontribusinya sangat fundamental dalam matematika komputasi karena mengenalkan pendekatan langkah-demi-langkah (step-by-step) dalam pemecahan masalah, yang menjadi cikal bakal konsep algoritma modern.

Dalam ilmu komputer modern, \textbf{algoritma} didefinisikan sebagai urutan langkah-langkah logis, terdefinisi dengan jelas, dan terbatas untuk menyelesaikan suatu masalah atau mencapai tujuan tertentu \cite{gfg_algorithm_intro}. Algoritma menjadi fondasi fundamental dalam pemrograman karena program komputer pada hakikatnya adalah implementasi algoritma dalam bahasa pemrograman. Setiap program yang kita tulis adalah representasi dari algoritma yang dirancang untuk menyelesaikan masalah spesifik.

\subsection{Perkembangan Historis Algoritma}

Sejarah algoritma tidak terlepas dari perkembangan pemrograman komputer:

\begin{itemize}
  \item \textbf{Era Pra-Komputer (1940-an):} Konsep algoritma sudah ada dalam matematika, namun implementasi masih manual
  \item \textbf{Era Machine Language (1950-an):} Pemrograman dilakukan dengan bahasa mesin dan assembly, algoritma sangat sederhana
  \item \textbf{Era Bahasa Tingkat Tinggi (1960-1970-an):} FORTRAN, COBOL, C, dan Pascal berkembang dengan pendekatan terstruktur
  \item \textbf{Era Modern (1980-an-sekarang):} Algoritma kompleks untuk AI, machine learning, big data
\end{itemize}

Bahasa C yang dikembangkan oleh Dennis Ritchie di Bell Labs pada 1972 menjadi bahasa prosedural yang sangat berpengaruh dan menjadi fondasi bagi banyak bahasa pemrograman modern seperti C++, Java, dan Python.

\subsection{Pentingnya Algoritma dalam Pemrograman}

Pemahaman algoritma penting karena beberapa alasan:

\begin{enumerate}
  \item \textbf{Efisiensi:} Algoritma yang baik menyelesaikan masalah dengan penggunaan sumber daya minimal
  \item \textbf{Kebenaran:} Memastikan solusi yang dihasilkan selalu benar untuk semua kasus input
  \item \textbf{Skalabilitas:} Algoritma yang efisien dapat menangani data dalam jumlah besar
  \item \textbf{Pemeliharaan:} Algoritma yang jelas mudah dipahami dan dimodifikasi
\end{enumerate}

Dalam mata kuliah Algoritma dan Pemrograman, mahasiswa belajar merancang algoritma menggunakan flowchart dan pseudocode sebelum mengimplementasikannya dalam bahasa C. Pendekatan ini memastikan mahasiswa memahami logika solusi sebelum terjun ke implementasi teknis.

\section{Pemrograman Prosedural dan Paradigma Lain}

\textbf{Pemrograman prosedural} adalah paradigma yang berfokus pada prosedur atau fungsi yang beroperasi pada data \cite{procedural_programming_wikipedia}. Program terdiri dari serangkaian instruksi yang dieksekusi secara berurutan, dengan data dan fungsi dapat dipisahkan. Bahasa C adalah contoh utama bahasa pemrograman prosedural yang masih digunakan luas hingga saat ini. Mata kuliah Algoritma dan Pemrograman menggunakan pendekatan prosedural karena memungkinkan fokus pada logika algoritmik tanpa kompleksitas paradigma lain.

Selain prosedural, terdapat paradigma pemrograman lain seperti pemrograman berorientasi objek (OOP) yang berfokus pada objek dan class, serta pemrograman fungsional yang menekankan evaluasi fungsi. Pemrograman prosedural cocok untuk program yang relatif kecil, tugas komputasi langsung, dan pembelajar pemrograman pemula. Pemahaman prosedural menjadi fondasi penting sebelum mempelajari paradigma yang lebih kompleks seperti OOP di semester berikutnya.

\begin{table}[htbp]
\centering
\begin{tabular}{|>{\raggedright\arraybackslash}p{3cm}|>{\raggedright\arraybackslash}p{5cm}|}
\hline
\textbf{Paradigma} & \textbf{Karakteristik Utama} \\
\hline
Prosedural & Fungsi dan data terpisah, alur top-down, bahasa C, Pascal \\
\hline
Berorientasi Objek & Class dan objek, enkapsulasi, bahasa Java, C++, Python \\
\hline
Fungsional & Fungsi sebagai nilai, bahasa Haskell, Lisp \\
\hline
\end{tabular}
\caption{Perbandingan Paradigma Pemrograman}
\end{table}

\section{Pengantar Bahasa C}

Bahasa C dikembangkan oleh Dennis Ritchie di Bell Labs pada tahun 1972 untuk sistem operasi UNIX. C menjadi salah satu bahasa pemrograman paling berpengaruh karena efisiensinya, portabilitasnya, dan kemampuannya melakukan manipulasi level rendah \cite{ref7}. Banyak bahasa pemrograman modern seperti C++, Java, dan Python terpengaruh oleh sintaks dan konsep C. Dalam mata kuliah ini, C dipilih karena kesederhanaannya dan kesesuaiannya untuk mempelajari dasar-dasar algoritma dan pemrograman prosedural.

Program C terdiri dari fungsi-fungsi, dengan fungsi \code{main()} sebagai titik masuk eksekusi. C mendukung tipe data dasar seperti \keyword{int}, \keyword{float}, \keyword{char}, dan \keyword{double}, serta struktur kontrol seperti percabangan dan perulangan. Untuk mengompilasi program C, dibutuhkan compiler seperti GCC (GNU Compiler Collection) atau MinGW di Windows. Setelah dikompilasi, program C menghasilkan file executable yang dapat dijalankan langsung oleh sistem operasi.

\begin{contoh}
Contoh program C minimal yang menampilkan teks ke layar:

\begin{lstlisting}[language=C, caption={Program Hello World dalam C}]
#include <stdio.h>

int main() {
    printf("Hello, World!\n");
    return 0;
}
\end{lstlisting}

Baris \code{\#include <stdio.h>} menyertakan header untuk fungsi input-output. Fungsi \code{printf} menampilkan teks ke layar, dan \code{return 0} mengindikasikan program berakhir sukses.
\end{contoh}

\section{Karakteristik Algoritma}

Berdasarkan standar ilmu komputer modern dan referensi dari GeeksforGeeks, sebuah algoritma harus memiliki enam karakteristik fundamental untuk dapat dikategorikan sebagai algoritma yang baik dan benar \cite{gfg_algorithm_intro}. Karakteristik ini menjadi acuan dalam mengevaluasi kualitas algoritma yang dirancang.

\subsection{Enam Karakteristik Utama Algoritma}

\begin{enumerate}
  \item \textbf{Input (Masukan)}
  \begin{itemize}
    \item Algoritma harus memiliki nol atau lebih input yang terdefinisi dengan jelas
    \item Input harus spesifik dan dapat diukur
    \item Contoh: algoritma pengurutan memerlukan array sebagai input
  \end{itemize}
  
  \item \textbf{Output (Keluaran)}
  \begin{itemize}
    \item Algoritma harus menghasilkan minimal satu output
    \item Output harus sesuai dengan tujuan algoritma
    \item Contoh: array yang sudah terurut, nilai maksimum, hasil perhitungan
  \end{itemize}
  
  \item \textbf{Definiteness (Kejelasan)}
  \begin{itemize}
    \item Setiap langkah harus didefinisikan dengan jelas dan tidak ambigu
    \item Tidak boleh ada interpretasi ganda untuk setiap instruksi
    \item Contoh yang salah: "tambahkan sedikit garam" → tidak jelas
    \item Contoh yang benar: "tambahkan 1 sendok teh garam" → jelas
  \end{itemize}
  
  \item \textbf{Finiteness (Keterbatasan)}
  \begin{itemize}
    \item Algoritma harus berakhir setelah jumlah langkah terbatas
    \item Tidak boleh mengandung infinite loop tanpa kondisi berhenti
    \item Setiap eksekusi harus selesai dalam waktu wajar
  \end{itemize}
  
  \item \textbf{Effectiveness (Efektivitas)}
  \begin{itemize}
    \item Setiap langkah harus dapat dilakukan dengan sumber daya yang tersedia
    \item Instruksi harus dasar dan dapat dieksekusi dalam waktu terbatas
    \item Tidak boleh memerlukan operasi yang tidak mungkin dilakukan
  \end{itemize}
  
  \item \textbf{Determinism (Deterministik)}
  \begin{itemize}
    \item Untuk input yang sama, algoritma harus selalu menghasilkan output yang sama
    \item Tidak ada elemen random atau kebetulan dalam proses
    \item Hasil dapat diprediksi dan direproduksi
  \end{itemize}
\end{enumerate}

\subsection{Tabel Perbandingan Karakteristik}

\begin{table}[htbp]
\centering
\footnotesize
\begin{tabular}{|>{\raggedright\arraybackslash}p{2.4cm}|>{\raggedright\arraybackslash}p{5.5cm}|>{\raggedright\arraybackslash}p{3.95cm}|}
\hline
\textbf{Karakteristik} & \textbf{Deskripsi} & \textbf{Contoh Penerapan} \\
\hline
Input & Data masukan yang diperlukan & Array angka untuk diurutkan \\
\hline
Output & Hasil yang dihasilkan & Array yang sudah terurut \\
\hline
Definiteness & Langkah jelas tidak ambigu & ``Tukar jika A > B'' \\
\hline
Finiteness & Berakhir dalam waktu terbatas & Loop berhenti setelah n-1 iterasi \\
\hline
Effectiveness & Langkah dapat dilakukan & Operasi perbandingan dan pertukaran \\
\hline
Determinism & Hasil konsisten & Input [3,1,2] selalu menghasilkan [1,2,3] \\
\hline
\end{tabular}
\caption{Enam Karakteristik Algoritma yang Baik}
\end{table}

\begin{konsep}
Ringkasan enam karakteristik algoritma:
\begin{enumerate}
  \item \textbf{Input}: Algoritma menerima masukan yang terdefinisi (nol atau lebih)
  \item \textbf{Output}: Algoritma menghasilkan minimal satu keluaran
  \item \textbf{Definiteness}: Setiap langkah jelas dan tidak ambigu
  \item \textbf{Finiteness}: Algoritma berakhir dalam langkah terbatas
  \item \textbf{Effectiveness}: Setiap langkah dapat dilaksanakan secara praktis
  \item \textbf{Determinism}: Input yang sama selalu menghasilkan output yang sama
\end{enumerate}
\end{konsep}

\section{Peta Konsep dan Contoh Algoritma}

\subsection{Peta Konsep Algoritma dan Pemrograman}

Peta konsep mata kuliah Algoritma dan Pemrograman menunjukkan alur pembelajaran dari fondasi hingga penerapan. Dimulai dari pemahaman konsep algoritma dan karakteristiknya, mahasiswa belajar merancang solusi menggunakan flowchart dan pseudocode. Selanjutnya, mahasiswa mempelajari elemen dasar bahasa C meliputi variabel, tipe data, I/O, dan operator.

\begin{center}
\begin{tikzpicture}[node distance=0.8cm]
  \node[flowstart] (a) {Algoritma dan Pemrograman};
  \node[flowprocess, below=of a] (b) {Flowchart \& Pseudocode};
  \node[flowprocess, below=of b] (c) {Variabel, Tipe Data, I/O};
  \node[flowprocess, below=of c] (d) {Percabangan \& Perulangan};
  \node[flowprocess, below=of d] (e) {Fungsi, Array, String};
  \node[flowprocess, below=of e] (f) {Struct, Pengurutan, Pencarian};
  \draw[arrow] (a) -- (b);
  \draw[arrow] (b) -- (c);
  \draw[arrow] (c) -- (d);
  \draw[arrow] (d) -- (e);
  \draw[arrow] (e) -- (f);
\end{tikzpicture}
\end{center}

Struktur percabangan (if-else, switch) dan perulangan (for, while, do-while) memungkinkan pengambilan keputusan dan pengulangan operasi. Fungsi dan prosedur mendukung modularisasi program. Array dan string digunakan untuk mengelola kumpulan data. Struct memungkinkan pembuatan tipe data bentukan. Algoritma pengurutan dan pencarian menerapkan semua konsep yang telah dipelajari untuk menyelesaikan masalah komputasi nyata.

\subsection{Contoh Algoritma Lengkap}

\textbf{Algoritma Menemukan Nilai Maksimum dalam Array:}

\begin{enumerate}
  \item \textbf{Input:} Array A dengan n elemen bilangan bulat
  \item \textbf{Proses:}
  \begin{enumerate}
    \item Inisialisasi: maks ← A[0]
    \item Untuk i ← 1 hingga n-1:
    \begin{itemize}
      \item Jika A[i] > maks maka maks ← A[i]
    \end{itemize}
  \end{enumerate}
  \item \textbf{Output:} Nilai maksimum (maks)
\end{enumerate}

\textbf{Analisis 6 Karakteristik:}
\begin{itemize}
  \item \textbf{Input:} Array A dengan n elemen (terdefinisi jelas)
  \item \textbf{Output:} Satu nilai maksimum (spesifik)
  \item \textbf{Definiteness:} Langkah-langkah jelas (inisialisasi, loop, perbandingan, assignment)
  \item \textbf{Finiteness:} Loop berakhir setelah n-1 iterasi
  \item \textbf{Effectiveness:} Operasi perbandingan dan assignment dapat dilakukan
  \item \textbf{Determinism:} Input [3,1,2,5,4] selalu menghasilkan 5
\end{itemize}

\textbf{Implementasi dalam Bahasa C:}
\begin{lstlisting}[language=C, caption={Implementasi Algoritma Nilai Maksimum}]
#include <stdio.h>

int findMaximum(int arr[], int n) {
    int maks = arr[0];  // Inisialisasi
    
    for (int i = 1; i < n; i++) {
        if (arr[i] > maks) {
            maks = arr[i];  // Update jika ditemukan nilai lebih besar
        }
    }
    
    return maks;  // Output
}

int main() {
    int data[] = {3, 1, 2, 5, 4};
    int n = sizeof(data) / sizeof(data[0]);
    
    printf("Nilai maksimum: %d\n", findMaximum(data, n));
    return 0;
}
\end{lstlisting}

Contoh ini menunjukkan bagaimana algoritma yang dirancang dengan baik memenuhi semua karakteristik fundamental dan dapat diimplementasikan dalam bahasa C dengan hasil yang dapat diprediksi.


% ============================================================
% AKTIVITAS PEMBELAJARAN
% ============================================================

\begin{aktivitas}
  \item \textbf{Diskusi Kelompok}: Identifikasi 5 contoh algoritma dalam kehidupan sehari-hari (misalnya: resep masakan, petunjuk penggunaan ATM) dan analisis apakah memenuhi enam karakteristik algoritma.
  
  \item \textbf{Analisis Kode}: Jalankan program Hello World dalam C, modifikasi untuk menampilkan nama Anda, lalu identifikasi fungsi \code{printf} dan \code{main}.
  
  \item \textbf{Perbandingan Paradigma}: Cari informasi tentang perbedaan pemrograman prosedural dan OOP. Buat ringkasan singkat kapan masing-masing cocok digunakan.
  
  \item \textbf{Instalasi Compiler}: Install GCC atau MinGW di komputer Anda, lalu kompilasi dan jalankan program C sederhana.
\end{aktivitas}

% ============================================================
% LATIHAN DAN REFLEKSI
% ============================================================

\begin{latihan}
  \item Jelaskan definisi algoritma menurut pemahaman Anda! Berikan dua contoh algoritma dari kehidupan sehari-hari.
  
  \item Sebutkan dan jelaskan enam karakteristik algoritma (input, output, definiteness, finiteness, effectiveness, determinism)!
  
  \item Mengapa bahasa C cocok untuk mempelajari dasar-dasar algoritma dan pemrograman?
  
  \item Jelaskan perbedaan utama antara pemrograman prosedural dan pemrograman berorientasi objek!
  
  \item Buat program C minimal yang menampilkan nama dan NIM Anda. Jelaskan fungsi setiap baris kode.
  
  \item \textbf{Refleksi}: Bagaimana pemahaman Anda tentang algoritma sebelum dan sesudah mempelajari bab ini? Konsep mana yang paling sulit dipahami?
\end{latihan}

% ============================================================
% ASESMEN
% ============================================================

\begin{asesmen}
\textbf{Instrumen Penilaian untuk Sub-CPMK 1.1}

\textbf{A. Pilihan Ganda}

\begin{enumerate}
  \item Manakah yang BUKAN merupakan karakteristik algoritma?
  \begin{enumerate}
    \item Input
    \item Output
    \item Abstraksi
    \item Finiteness
  \end{enumerate}
  
  \item Bahasa C dikembangkan oleh:
  \begin{enumerate}
    \item James Gosling
    \item Dennis Ritchie
    \item Bjarne Stroustrup
    \item Guido van Rossum
  \end{enumerate}
  
  \item Karakteristik algoritma yang menyatakan bahwa setiap langkah harus jelas dan tidak ambigu disebut:
  \begin{enumerate}
    \item Input
    \item Definiteness
    \item Effectiveness
    \item Output
  \end{enumerate}
\end{enumerate}

\textbf{B. Essay}

\begin{enumerate}
  \item Jelaskan dengan kata-kata Anda sendiri apa yang dimaksud dengan algoritma dan berikan 2 contoh algoritma dari kehidupan sehari-hari.
  
  \item Sebutkan dan jelaskan keenam karakteristik algoritma yang baik. Berikan contoh langkah yang memenuhi dan tidak memenuhi karakteristik definiteness dan determinism.
\end{enumerate}

\textbf{Rubrik Penilaian}: Lihat Lampiran A
\end{asesmen}

% ============================================================
% CHECKLIST KOMPETENSI
% ============================================================

\begin{checklist}
  \item Saya dapat menjelaskan definisi dan sejarah algoritma
  \item Saya dapat menyebutkan dan menjelaskan enam karakteristik algoritma
  \item Saya memahami perbedaan pemrograman prosedural dan paradigma lain
  \item Saya dapat menulis dan menjalankan program C minimal (Hello World)
  \item Saya memahami peta konsep mata kuliah Algoritma dan Pemrograman
\end{checklist}

% ============================================================
% RANGKUMAN
% ============================================================

\begin{rangkuman}
Bab ini membahas landasan teori Algoritma dan Pemrograman, termasuk sejarah dan konsep algoritma, pemrograman prosedural, pengantar bahasa C, serta karakteristik algoritma.

\textbf{Poin Kunci:}
\begin{itemize}
  \item Algoritma adalah urutan langkah logis yang terdefinisi untuk menyelesaikan masalah
  \item Enam karakteristik algoritma: Input, Output, Definiteness, Finiteness, Effectiveness, Determinism
  \item Bahasa C adalah bahasa prosedural yang cocok untuk mempelajari dasar pemrograman
  \item Pemrograman prosedural fokus pada fungsi dan data, cocok untuk program sederhana
  \item Peta konsep menunjukkan alur dari algoritma, flowchart, variabel, percabangan, perulangan, hingga pengurutan dan pencarian
\end{itemize}

\textbf{Kata Kunci}: Algoritma, Pemrograman Prosedural, Bahasa C, Flowchart, Pseudocode
\end{rangkuman}

\ifSubfilesClassLoaded{
  \renewcommand{\bibname}{Daftar Pustaka}
  \bibliographystyle{plain}
  \bibliography{../references}
}{}
\end{document}
