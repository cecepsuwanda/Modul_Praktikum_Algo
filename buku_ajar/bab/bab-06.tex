\documentclass[../main.tex]{subfiles}
\ifSubfilesClassLoaded{\setcounter{chapter}{5}}{}
\begin{document}

\chapter{Operator dalam C}

\begin{subcpmk}
  \item Sub-CPMK 3.1: Menggunakan operator aritmatika, perbandingan, logika, bitwise, dan penugasan secara tepat dalam menyelesaikan masalah sederhana.
\end{subcpmk}

\noindent\textbf{Materi Pokok:} 
\begin{itemize}
    \item Operator Aritmatika dan Increment/Decrement
    \item Operator Relasional dan Logika
    \item Operator Bitwise dan Penugasan
    \item Prioritas Operator (Precedence) dan Asosiativitas
\end{itemize}

\noindent\textit{Referensi: \cite{ref7, control_flow_cppref, operator_precedence_c, ref2}}

\section{Operator Aritmatika dan Increment/Decrement}

Operator adalah simbol khusus yang memberitahu kompilator untuk melakukan operasi matematika atau logika tertentu. Bahasa C kaya akan operator internal.

\subsection{Operator Aritmatika}
Operator aritmatika digunakan untuk melakukan operasi matematika dasar.

\begin{table}[ht]
\centering
\footnotesize
\caption{Operator Aritmatika dalam C}
\label{tab:aritmatika}
\begin{tabular}{|c|p{2.2cm}|p{2.2cm}|p{3.8cm}|}
\hline
\textbf{Operator} & \textbf{Nama} & \textbf{Contoh} & \textbf{Keterangan} \\ \hline
\code{+} & Penjumlahan & \code{a + b} & Menjumlahkan dua operand \\ \hline
\code{-} & Pengurangan & \code{a - b} & Mengurangkan operand kedua dari pertama \\ \hline
\code{*} & Perkalian & \code{a * b} & Mengalikan dua operand \\ \hline
\code{/} & Pembagian & \code{a / b} & Membagi operand pertama dengan kedua \\ \hline
\code{\%} & Modulus (Sisa Bagi) & \code{a \% b} & Sisa hasil bagi integer \\ \hline
\end{tabular}
\end{table}

\noindent\textbf{Catatan Penting:}
\begin{enumerate}
    \item \textbf{Pembagian Integer}: Jika kedua operand adalah integer, hasilnya adalah integer (dibulatkan ke bawah/truncate). Contoh: \code{7 / 2} menghasilkan \code{3}, bukan \code{3.5}. Agar mendapatkan hasil desimal, salah satu operand harus bertipe \textit{floating point} (misal: \code{7.0 / 2}).
    \item \textbf{Modulus}: Operator \code{\%} hanya bekerja pada tipe data integer. \code{7 \% 2} bernilai \code{1}. Pada C modern, tanda hasil modulus mengikuti operand pertama (misal \code{-7 \% 3} hasilnya \code{-1}).
\end{enumerate}

\subsection{Operator Increment dan Decrement}
C memiliki operator unik untuk menambah atau mengurangi nilai variabel sebesar 1, yaitu \code{++} dan \code{--}. Operator ini bisa diletakkan sebelum variabel (\textit{prefix}) atau sesudah variabel (\textit{postfix}).

\begin{itemize}
    \item \textbf{Prefix (\code{++a})}: Nilai variabel diubah \textit{terlebih dahulu}, lalu hasilnya digunakan dalam ekspresi.
    \item \textbf{Postfix (\code{a++})}: Nilai variabel \textit{saat ini} digunakan dulu dalam ekspresi, baru kemudian nilainya diubah.
\end{itemize}

\begin{lstlisting}[language=C, caption=Perbedaan Prefix dan Postfix]
#include <stdio.h>

int main() {
    int a = 5, b = 5;
    int hasil_a, hasil_b;

    // Prefix increment
    hasil_a = ++a; 
    // a bertambah jadi 6 dulu, lalu nilai 6 dimasukkan ke hasil_a
    printf("a: %d, hasil_a: %d\n", a, hasil_a); // Output: 6, 6

    // Postfix increment
    hasil_b = b++;
    // nilai b (5) dimasukkan ke hasil_b dulu, baru b bertambah jadi 6
    printf("b: %d, hasil_b: %d\n", b, hasil_b); // Output: 6, 5
    
    return 0;
}
\end{lstlisting}

\subsection{Operator Unary}
Operator unary hanya memerlukan satu operand. Selain increment/decrement, ada:
\begin{itemize}
    \item \code{+} (Unary Plus): Menandakan nilai positif (jarang ditulis eksplisit).
    \item \code{-} (Unary Minus): Mengnegasikan nilai (mengubah positif jadi negatif, dan sebaliknya).
\end{itemize}

Operator aritmatika C: \code{+}, \code{-}, \code{*}, \code{/}, \code{\%} (modulo). Prioritas mengikuti aturan matematika. Operator perbandingan: \code{==}, \code{!=}, \code{<}, \code{>}, \code{<=}, \code{>=}; menghasilkan nilai 1 (true) atau 0 (false). Operator penugasan: \code{=} serta bentuk majemuk \code{+=}, \code{-=}, \code{*=}, \code{/=}. Hati-hati dengan \code{=} vs \code{==}; kesalahan umum menggunakan \code{=} pada kondisi.

\section{Operator Relasional dan Logika}

Operator relasional dan logika adalah fondasi dari pengambilan keputusan (percabangan) dan perulangan dalam pemrograman.

\subsection{Operator Relasional (Perbandingan)}
Operator ini membandingkan dua nilai dan menghasilkan nilai kebenaran. Di C (sebelum C99), nilai kebenaran direpresentasikan dengan integer: \textbf{0 untuk False} dan \textbf{1 (atau bukan nol) untuk True}.

\begin{table}[h]
\centering
\caption{Operator Relasional}
\label{tab:relasional}
\begin{tabular}{|c|l|l|}
\hline
\textbf{Operator} & \textbf{Deskripsi} & \textbf{Contoh (A=10, B=20)} \\ \hline
\code{==} & Sama dengan & \code{(A == B)} $\rightarrow$ False (0) \\ \hline
\code{!=} & Tidak sama dengan & \code{(A != B)} $\rightarrow$ True (1) \\ \hline
\code{>} & Lebih besar dari & \code{(A > B)} $\rightarrow$ False (0) \\ \hline
\code{<} & Lebih kecil dari & \code{(A < B)} $\rightarrow$ True (1) \\ \hline
\code{>=} & Lebih besar atau sama dengan & \code{(A >= B)} $\rightarrow$ False (0) \\ \hline
\code{<=} & Lebih kecil atau sama dengan & \code{(A <= B)} $\rightarrow$ True (1) \\ \hline
\end{tabular}
\end{table}

\begin{alertbox}{Peringatan}
Jangan tertukar antara operator penugasan \code{=} (assignment) dengan operator perbandingan \code{==} (equality). Menulis \code{if (x = 5)} adalah \textbf{valid} secara sintaks tapi seringkali berupa \textit{bug logic}, karena akan melakukan assignment nilai 5 ke x, yang nilainya (5) dianggap True.
\end{alertbox}

\subsubsection{Membandingkan Floating Point}
Hati-hati saat membandingkan tipe \code{float} atau \code{double} menggunakan \code{==} karena masalah presisi. Sebaiknya gunakan selisih absolut dengan nilai toleransi (epsilon).
\begin{lstlisting}[language=C]
// Kurang aman
if (f == 3.14) { ... } 

// Lebih aman
if (fabs(f - 3.14) < 0.00001) { ... }
\end{lstlisting}

\subsection{Operator Logika}
Digunakan untuk menggabungkan beberapa ekspresi relasional.

\begin{table}[ht]
\centering
\caption{Tabel Kebenaran Operator Logika}
\label{tab:logika}
\begin{tabular}{|c|c|c|c|c|}
\hline
\textbf{A} & \textbf{B} & \textbf{A \&\& B (AND)} & \textbf{A || B (OR)} & \textbf{!A (NOT)} \\ \hline
0 & 0 & 0 & 0 & 1 \\ \hline
0 & 1 & 0 & 1 & 1 \\ \hline
1 & 0 & 0 & 1 & 0 \\ \hline
1 & 1 & 1 & 1 & 0 \\ \hline
\end{tabular}
\end{table}

\subsubsection{Short-Circuit Evaluation}
C menggunakan evaluasi \textit{short-circuit} untuk efisiensi:
\begin{enumerate}
    \item \textbf{AND (\&\&)}: Jika operand kiri \textbf{False} (0), operand kanan \textbf{tidak dievaluasi} karena hasilnya pasti False.
    \item \textbf{OR (||)}: Jika operand kiri \textbf{True} (1), operand kanan \textbf{tidak dievaluasi} karena hasilnya pasti True.
\end{enumerate}

Contoh keamanan menggunakan short-circuit:
\begin{lstlisting}[language=C]
// Mencegah pembagian dengan nol
if (x != 0 && (100 / x) > 10) {
    // Jika x == 0, bagian (100 / x) tidak akan dieksekusi 
    // sehingga error division by zero terhindarkan.
}
\end{lstlisting}

Operator logika: \code{\&\&} (AND), \code{||} (OR), \code{!} (NOT). AND bernilai true hanya jika kedua operand true. OR bernilai true jika salah satu operand true. NOT membalik nilai kebenaran. Operator logika sering digunakan dalam kondisi percabangan dan perulangan. Contoh: \code{if (x > 0 \&\& x < 10)} memeriksa apakah x berada di antara 0 dan 10. Short-circuit evaluation: jika operand pertama menentukan hasil, operand kedua tidak dievaluasi.

\section{Operator Bitwise, Penugasan, dan Lainnya}

Selain operator matematika dan logika standar, C menyediakan operator yang bekerja pada tingkat bit dan memori, yang menjadi salah satu kekuatan utama bahasa C.

\subsection{Operator Bitwise}
Operator bitwise bekerja pada level representasi biner dari integer. Operator ini tidak dapat digunakan pada tipe \code{float}, \code{double}, atau \code{long double}.

\begin{table}[h]
\centering
\caption{Operator Bitwise}
\label{tab:bitwise}
\begin{tabular}{|c|l|l|}
\hline
\textbf{Op} & \textbf{Nama} & \textbf{Deskripsi} \\ \hline
\code{\&} & AND & Bit bernilai 1 jika kedua bit operand bernilai 1 \\ \hline
\code{|} & OR & Bit bernilai 1 jika salah satu bit operand bernilai 1 \\ \hline
\code{\^} & XOR (Exclusive OR) & Bit bernilai 1 jika bit operand berbeda nilainya \\ \hline
\code{\textasciitilde} & NOT (Complement) & Membalik nilai bit (0 jadi 1, 1 jadi 0) \\ \hline
\code{<<} & Left Shift & Menggeser bit ke kiri (mengalikan dengan $2^n$) \\ \hline
\code{>>} & Right Shift & Menggeser bit ke kanan (membagi dengan $2^n$) \\ \hline
\end{tabular}
\end{table}

\textbf{Contoh:} Misalkan A = 60 (\code{0011 1100}) dan B = 13 (\code{0000 1101}).
\begin{itemize}
    \item \code{A \& B} = 12 (\code{0000 1100})
    \item \code{A | B} = 61 (\code{0011 1101})
    \item \code{A \^ B} = 49 (\code{0011 0001})
    \item \code{A << 2} = 240 (\code{1111 0000})
\end{itemize}

\subsection{Operator Penugasan Majemuk (Compound Assignment)}
C memungkinkan kita menggabungkan operasi aritmatika/bitwise dengan penugasan untuk penulisan kode yang lebih ringkas.

\begin{table}[h]
\centering
\caption{Operator Penugasan Majemuk}
\label{tab:assignment}
\begin{tabular}{|l|l|}
\hline
\textbf{Penulisan Singkat} & \textbf{Ekivalen Dengan} \\ \hline
\code{a += b} & \code{a = a + b} \\ \hline
\code{a -= b} & \code{a = a - b} \\ \hline
\code{a *= b} & \code{a = a * b} \\ \hline
\code{a /= b} & \code{a = a / b} \\ \hline
\code{a \%= b} & \code{a = a \% b} \\ \hline
\code{a <<= b} & \code{a = a << b} \\ \hline
\end{tabular}
\end{table}

\subsection{Operator Lainnya}

\subsubsection{Operator Ternary (Conditional Operator)}
Ini adalah satu-satunya operator C yang mengambil tiga operand. Bentuknya:
\begin{center}
\code{kondisi ? ekspresi\_jika\_true : ekspresi\_jika\_false}
\end{center}

Contoh:
\begin{lstlisting}[language=C]
int a = 10, b = 20;
int max;

// Menggunakan if-else biasa
if (a > b) max = a;
else max = b;

// Menggunakan ternary (lebih ringkas)
max = (a > b) ? a : b;
\end{lstlisting}

\subsubsection{Operator \code{sizeof}}
Operator unary yang mengembalikan ukuran dalam \textit{bytes} dari tipe data atau variabel.
\begin{lstlisting}[language=C]
printf("Ukuran int: %lu bytes\n", sizeof(int)); // Biasanya 4
\end{lstlisting}

\subsubsection{Operator Koma (\code{,})}
Digunakan untuk memisahkan dua atau lebih ekspresi yang dievaluasi berurutan. Nilai ekspresi keseluruhan adalah nilai dari ekspresi paling kanan. Sering dijumpai dalam loop \code{for}.
\begin{lstlisting}[language=C]
int x, y;
y = (x = 5, x + 10); 
// x diisi 5, lalu hitung x+10 (15). y akan bernilai 15.
\end{lstlisting}

\section{Prioritas Operator dan Studi Kasus}

Dalam sebuah ekspresi kompleks yang memiliki banyak operator, urutan eksekusi ditentukan oleh \textbf{Prioritas Operator} (\textit{Operator Precedence}) dan \textbf{Asosiativitas}.

\subsection{Tabel Prioritas Operator}
Tabel berikut menunjukkan urutan operator dari prioritas tertinggi (dievaluasi duluan) hingga terendah.

\begin{table}[ht]
\centering
\footnotesize
\caption{Prioritas Operator (Disederhanakan)}
\label{tab:precedence}
\begin{tabular}{|c|p{3.25cm}|p{2.45cm}|p{2.2cm}|}
\hline
\textbf{Level} & \textbf{Operator} & \textbf{Deskripsi} & \textbf{Asos.} \\ \hline
1 & \code{()} \code{[]} \code{->} \code{.} & Function call, Array subscript, Member & Kiri ke Kanan \\ \hline
2 & \code{!} \code{\textasciitilde} \code{++} \code{--} \code{+} \code{-} \code{*} \code{\&} \code{sizeof} & Unary operators & \textbf{Kanan ke Kiri} \\ \hline
3 & \code{*} \code{/} \code{\%} & Multiplicative & Kiri ke Kanan \\ \hline
4 & \code{+} \code{-} & Additive & Kiri ke Kanan \\ \hline
5 & \code{<<} \code{>>} & Shift & Kiri ke Kanan \\ \hline
6 & \code{<} \code{<=} \code{>} \code{>=} & Relational & Kiri ke Kanan \\ \hline
7 & \code{==} \code{!=} & Equality & Kiri ke Kanan \\ \hline
8 & \code{\&} & Bitwise AND & Kiri ke Kanan \\ \hline
9 & \code{\^} & Bitwise XOR & Kiri ke Kanan \\ \hline
10 & \code{|} & Bitwise OR & Kiri ke Kanan \\ \hline
11 & \code{\&\&} & Logical AND & Kiri ke Kanan \\ \hline
12 & \code{||} & Logical OR & Kiri ke Kanan \\ \hline
13 & \code{? :} & Ternary & \textbf{Kanan ke Kiri} \\ \hline
14 & \code{=} \code{+=} \code{-=} dll & Assignment & \textbf{Kanan ke Kiri} \\ \hline
15 & \code{,} & Comma & Kiri ke Kanan \\ \hline
\end{tabular}
\end{table}

\subsection{Studi Kasus Evaluasi Ekspresi}

Mari kita bedah cara C mengevaluasi ekspresi berikut:
\begin{center}
\code{hasil = 10 + 5 * 2 > 15 \&\& 4 \% 2 == 0;}
\end{center}

\noindent\textbf{Langkah Evaluasi:}
\begin{enumerate}
    \item \textbf{Aritmatika Tertinggi (*) dan (\%):}
    \begin{itemize}
        \item \code{5 * 2} menjadi \code{10}.
        \item \code{4 \% 2} menjadi \code{0}.
        \item Ekspresi kini: \code{10 + 10 > 15 \&\& 0 == 0}
    \end{itemize}
    
    \item \textbf{Aritmatika (+) dan (-):}
    \begin{itemize}
        \item \code{10 + 10} menjadi \code{20}.
        \item Ekspresi kini: \code{20 > 15 \&\& 0 == 0}
    \end{itemize}
    
    \item \textbf{Relasional (>, <, >=, <=):}
    \begin{itemize}
        \item \code{20 > 15} bernilai True (\code{1}).
        \item Ekspresi kini: \code{1 \&\& 0 == 0}
    \end{itemize}
    
    \item \textbf{Equality (==, !=):}
    \begin{itemize}
        \item \code{0 == 0} bernilai True (\code{1}).
        \item Sisa ekspresi: \code{1 \&\& 1}
    \end{itemize}
    
    \item \textbf{Logika AND (\&\&):}
    \begin{itemize}
        \item \code{1 \&\& 1} bernilai True (\code{1}).
    \end{itemize}
    
    \item \textbf{Assignment (=):}
    \begin{itemize}
        \item Nilai \code{1} disimpan ke variabel \code{hasil}.
    \end{itemize}
\end{enumerate}

\begin{alertbox}{Kesalahan Umum: Chaining Relational}
Ekspresi matematika $5 < x < 10$ tidak bisa ditulis mentah-mentah di C sebagai \code{5 < x < 10}.
\begin{itemize}
    \item C akan mengevaluasi \code{(5 < x)} terlebih dahulu, menghasilkan 0 atau 1.
    \item Kemudian hasil tersebut dibandingkan dengan 10: \code{(0 atau 1) < 10}, yang mana \textbf{selalu True}.
    \item \textbf{Solusi:} Gunakan operator logika: \code{5 < x \&\& x < 10}.
\end{itemize}
\end{alertbox}


% ============================================================
% AKTIVITAS PEMBELAJARAN
% ============================================================
\begin{aktivitas}
  \item \textbf{Eksperimen Increment}: Buat program kecil untuk membandingkan \code{a = ++b} dan \code{a = b++}. Tampilkan nilai variabels sebelum dan sesudah operasi.
  \item \textbf{Bitwise Magic}: Gunakan operator bitwise \code{\&} untuk memeriksa apakah sebuah bilangan ganjil atau genap (hint: cek bit terakhir).
  \item \textbf{Short-Circuit Demo}: Buat kondisi \code{if} dengan dua ekspresi, di mana ekspresi kedua memiliki *side effect* (misal print teks atau increment). Amati apakah ekspresi kedua dijalankan jika ekspresi pertama sudah menentukan hasil (misal \code{FALSE \&\& ...}).
\end{aktivitas}

% ============================================================
% LATIHAN DAN REFLEKSI
% ============================================================
\begin{latihan}
  \item Jelaskan perbedaan antara \code{x++} (postfix) dan \code{++x} (prefix) ketika digunakan dalam sebuah ekspresi!
  \item Apa hasil dari ekspresi \code{10 \% 3} dan \code{10 \% -3}? Jelaskan perilaku operator modulus pada bilangan negatif di C.
  \item Ubah kondisi berikut menjadi bentuk operator ternary: 
  \begin{lstlisting}[language=C]
if (nilai >= 60) status = 1;
else status = 0;
  \end{lstlisting}
  \item Diberikan \code{int a = 5; int b = 10;}. Berapa nilai \code{a} dan \code{b} setelah ekspresi: \code{int hasil = (a++ > 5) || (b++ > 10);}? Jelaskan mengapa \code{b} berubah atau tidak berubah.
  \item \textbf{Refleksi}: Mengapa memahami prioritas operator (precedence) itu penting? Berikan contoh bug yang mungkin muncul jika kita mengabaikan prioritas operator.
\end{latihan}

% ============================================================
% ASESMEN
% ============================================================
\begin{asesmen}
\textbf{Instrumen untuk Sub-CPMK 3.1}

\noindent\textbf{Soal 1 (Aritmatika \& Logika):}
Buat program C yang menerima input 3 sisi segitiga (a, b, c). Program harus menentukan:
\begin{enumerate}
    \item Apakah ketiga sisi dapat membentuk segitiga yang valid? (Syarat: $a+b>c$ DAN $a+c>b$ DAN $b+c>a$)
    \item Jika valid, tentukan jenisnya: Sama Sisi, Sama Kaki, atau Sembarang.
\end{enumerate}
Gunakan operator logika dan relasional yang tepat.

\noindent\textbf{Soal 2 (Bitwise):}
Buat program yang menerima input bilangan bulat, lalu menggunakan operator geser (\code{<<} atau \code{>>}) untuk mengalikan bilangan tersebut dengan 4 dan membaginya dengan 2. Bandingkan hasilnya dengan operator aritmatika biasa.

\noindent\textbf{Soal 3 (Evaluasi Ekspresi):}
Evaluasi ekspresi berikut secara manual, lalu buat program C untuk memverifikasi jawaban Anda:
\code{result = 10 + 5 * 2 > 15 \&\& 4 \% 2 == 0;}
Jelaskan urutan pengerjaannya berdasarkan prioritas operator.
\end{asesmen}

% ============================================================
% CHECKLIST KOMPETENSI
% ============================================================
\begin{checklist}
  \item Saya memahami perbedaan prefix dan postfix increment/decrement
  \item Saya dapat menggunakan operator aritmatika, relasional, dan logika dengan benar
  \item Saya memahami konsep \textit{short-circuit evaluation}
  \item Saya dapat menggunakan operator bitwise dasar (\&, |, \^{}, \textasciitilde, <<, >>)
  \item Saya dapat menggunakan operator ternary sebagai alternatif if-else sederhana
  \item Saya memahami urutan evaluasi berdasarkan tabel prioritas operator
\end{checklist}

% ============================================================
% RANGKUMAN
% ============================================================
\begin{rangkuman}
Bab ini mengupas tuntas berbagai operator dalam bahasa C:
\begin{itemize}
    \item \textbf{Aritmatika}: Melakukan perhitungan matematika (+, -, *, /, \%). Perhatikan pembagian integer dan modulus.
    \item \textbf{Increment/Decrement}: Menambah/mengurang 1 nilai variabal. Prefix (\code{++i}) mengubah nilai sebelum digunakan, Postfix (\code{i++}) menggunakan nilai lama dulu baru mengubah.
    \item \textbf{Relasional \& Logika}: Membandingkan nilai dan menggabungkan kondisi (\&\&, ||, !). Menggunakan \textit{short-circuit evaluation} untuk efisiensi dan keamanan.
    \item \textbf{Bitwise}: Memanipulasi data pada level bit, sangat berguna untuk pemrograman sistem atau optimasi tingkat rendah.
    \item \textbf{Assignment Majemuk}: Menyingkat penulisan operasi update (\code{+=}, \code{*=}).
    \item \textbf{Ternary}: \code{kondisi ? nilai\_true : nilai\_false} untuk percabangan inline.
    \item \textbf{Precedence}: Aturan yang menentukan operator mana yang dieksekusi lebih dulu dalam ekspresi kompleks.
\end{itemize}
\end{rangkuman}

\ifSubfilesClassLoaded{
  \renewcommand{\bibname}{Daftar Pustaka}
  \bibliographystyle{plain}
  \bibliography{../references}
}{}
\end{document}
\ifSubfilesClassLoaded{\setcounter{chapter}{5}}{}
\begin{document}

\chapter{Operator dalam C}

\begin{subcpmk}
  \item Sub-CPMK 3.1: Menggunakan operator aritmatika, perbandingan, logika, dan penugasan
\end{subcpmk}

\noindent\textbf{Materi Pokok:} Operator aritmatika, perbandingan, logika, dan penugasan dalam C; prioritas operator; short-circuit evaluation \cite{ref7}.

\section{Operator Aritmatika dan Increment/Decrement}

Operator adalah simbol khusus yang memberitahu kompilator untuk melakukan operasi matematika atau logika tertentu. Bahasa C kaya akan operator internal.

\subsection{Operator Aritmatika}
Operator aritmatika digunakan untuk melakukan operasi matematika dasar.

\begin{table}[ht]
\centering
\footnotesize
\caption{Operator Aritmatika dalam C}
\label{tab:aritmatika}
\begin{tabular}{|c|p{2.2cm}|p{2.2cm}|p{3.8cm}|}
\hline
\textbf{Operator} & \textbf{Nama} & \textbf{Contoh} & \textbf{Keterangan} \\ \hline
\code{+} & Penjumlahan & \code{a + b} & Menjumlahkan dua operand \\ \hline
\code{-} & Pengurangan & \code{a - b} & Mengurangkan operand kedua dari pertama \\ \hline
\code{*} & Perkalian & \code{a * b} & Mengalikan dua operand \\ \hline
\code{/} & Pembagian & \code{a / b} & Membagi operand pertama dengan kedua \\ \hline
\code{\%} & Modulus (Sisa Bagi) & \code{a \% b} & Sisa hasil bagi integer \\ \hline
\end{tabular}
\end{table}

\noindent\textbf{Catatan Penting:}
\begin{enumerate}
    \item \textbf{Pembagian Integer}: Jika kedua operand adalah integer, hasilnya adalah integer (dibulatkan ke bawah/truncate). Contoh: \code{7 / 2} menghasilkan \code{3}, bukan \code{3.5}. Agar mendapatkan hasil desimal, salah satu operand harus bertipe \textit{floating point} (misal: \code{7.0 / 2}).
    \item \textbf{Modulus}: Operator \code{\%} hanya bekerja pada tipe data integer. \code{7 \% 2} bernilai \code{1}. Pada C modern, tanda hasil modulus mengikuti operand pertama (misal \code{-7 \% 3} hasilnya \code{-1}).
\end{enumerate}

\subsection{Operator Increment dan Decrement}
C memiliki operator unik untuk menambah atau mengurangi nilai variabel sebesar 1, yaitu \code{++} dan \code{--}. Operator ini bisa diletakkan sebelum variabel (\textit{prefix}) atau sesudah variabel (\textit{postfix}).

\begin{itemize}
    \item \textbf{Prefix (\code{++a})}: Nilai variabel diubah \textit{terlebih dahulu}, lalu hasilnya digunakan dalam ekspresi.
    \item \textbf{Postfix (\code{a++})}: Nilai variabel \textit{saat ini} digunakan dulu dalam ekspresi, baru kemudian nilainya diubah.
\end{itemize}

\begin{lstlisting}[language=C, caption=Perbedaan Prefix dan Postfix]
#include <stdio.h>

int main() {
    int a = 5, b = 5;
    int hasil_a, hasil_b;

    // Prefix increment
    hasil_a = ++a; 
    // a bertambah jadi 6 dulu, lalu nilai 6 dimasukkan ke hasil_a
    printf("a: %d, hasil_a: %d\n", a, hasil_a); // Output: 6, 6

    // Postfix increment
    hasil_b = b++;
    // nilai b (5) dimasukkan ke hasil_b dulu, baru b bertambah jadi 6
    printf("b: %d, hasil_b: %d\n", b, hasil_b); // Output: 6, 5
    
    return 0;
}
\end{lstlisting}

\subsection{Operator Unary}
Operator unary hanya memerlukan satu operand. Selain increment/decrement, ada:
\begin{itemize}
    \item \code{+} (Unary Plus): Menandakan nilai positif (jarang ditulis eksplisit).
    \item \code{-} (Unary Minus): Mengnegasikan nilai (mengubah positif jadi negatif, dan sebaliknya).
\end{itemize}

Operator aritmatika C: \code{+}, \code{-}, \code{*}, \code{/}, \code{\%} (modulo). Prioritas mengikuti aturan matematika. Operator perbandingan: \code{==}, \code{!=}, \code{<}, \code{>}, \code{<=}, \code{>=}; menghasilkan nilai 1 (true) atau 0 (false). Operator penugasan: \code{=} serta bentuk majemuk \code{+=}, \code{-=}, \code{*=}, \code{/=}. Hati-hati dengan \code{=} vs \code{==}; kesalahan umum menggunakan \code{=} pada kondisi.

\section{Operator Relasional dan Logika}

Operator relasional dan logika adalah fondasi dari pengambilan keputusan (percabangan) dan perulangan dalam pemrograman.

\subsection{Operator Relasional (Perbandingan)}
Operator ini membandingkan dua nilai dan menghasilkan nilai kebenaran. Di C (sebelum C99), nilai kebenaran direpresentasikan dengan integer: \textbf{0 untuk False} dan \textbf{1 (atau bukan nol) untuk True}.

\begin{table}[h]
\centering
\caption{Operator Relasional}
\label{tab:relasional}
\begin{tabular}{|c|l|l|}
\hline
\textbf{Operator} & \textbf{Deskripsi} & \textbf{Contoh (A=10, B=20)} \\ \hline
\code{==} & Sama dengan & \code{(A == B)} $\rightarrow$ False (0) \\ \hline
\code{!=} & Tidak sama dengan & \code{(A != B)} $\rightarrow$ True (1) \\ \hline
\code{>} & Lebih besar dari & \code{(A > B)} $\rightarrow$ False (0) \\ \hline
\code{<} & Lebih kecil dari & \code{(A < B)} $\rightarrow$ True (1) \\ \hline
\code{>=} & Lebih besar atau sama dengan & \code{(A >= B)} $\rightarrow$ False (0) \\ \hline
\code{<=} & Lebih kecil atau sama dengan & \code{(A <= B)} $\rightarrow$ True (1) \\ \hline
\end{tabular}
\end{table}

\begin{alertbox}{Peringatan}
Jangan tertukar antara operator penugasan \code{=} (assignment) dengan operator perbandingan \code{==} (equality). Menulis \code{if (x = 5)} adalah \textbf{valid} secara sintaks tapi seringkali berupa \textit{bug logic}, karena akan melakukan assignment nilai 5 ke x, yang nilainya (5) dianggap True.
\end{alertbox}

\subsubsection{Membandingkan Floating Point}
Hati-hati saat membandingkan tipe \code{float} atau \code{double} menggunakan \code{==} karena masalah presisi. Sebaiknya gunakan selisih absolut dengan nilai toleransi (epsilon).
\begin{lstlisting}[language=C]
// Kurang aman
if (f == 3.14) { ... } 

// Lebih aman
if (fabs(f - 3.14) < 0.00001) { ... }
\end{lstlisting}

\subsection{Operator Logika}
Digunakan untuk menggabungkan beberapa ekspresi relasional.

\begin{table}[ht]
\centering
\caption{Tabel Kebenaran Operator Logika}
\label{tab:logika}
\begin{tabular}{|c|c|c|c|c|}
\hline
\textbf{A} & \textbf{B} & \textbf{A \&\& B (AND)} & \textbf{A || B (OR)} & \textbf{!A (NOT)} \\ \hline
0 & 0 & 0 & 0 & 1 \\ \hline
0 & 1 & 0 & 1 & 1 \\ \hline
1 & 0 & 0 & 1 & 0 \\ \hline
1 & 1 & 1 & 1 & 0 \\ \hline
\end{tabular}
\end{table}

\subsubsection{Short-Circuit Evaluation}
C menggunakan evaluasi \textit{short-circuit} untuk efisiensi:
\begin{enumerate}
    \item \textbf{AND (\&\&)}: Jika operand kiri \textbf{False} (0), operand kanan \textbf{tidak dievaluasi} karena hasilnya pasti False.
    \item \textbf{OR (||)}: Jika operand kiri \textbf{True} (1), operand kanan \textbf{tidak dievaluasi} karena hasilnya pasti True.
\end{enumerate}

Contoh keamanan menggunakan short-circuit:
\begin{lstlisting}[language=C]
// Mencegah pembagian dengan nol
if (x != 0 && (100 / x) > 10) {
    // Jika x == 0, bagian (100 / x) tidak akan dieksekusi 
    // sehingga error division by zero terhindarkan.
}
\end{lstlisting}

Operator logika: \code{\&\&} (AND), \code{||} (OR), \code{!} (NOT). AND bernilai true hanya jika kedua operand true. OR bernilai true jika salah satu operand true. NOT membalik nilai kebenaran. Operator logika sering digunakan dalam kondisi percabangan dan perulangan. Contoh: \code{if (x > 0 \&\& x < 10)} memeriksa apakah x berada di antara 0 dan 10. Short-circuit evaluation: jika operand pertama menentukan hasil, operand kedua tidak dievaluasi.


\begin{aktivitas}
  \item Buat program kalkulator sederhana dengan operator +, -, *, /.
  \item Eksperimen dengan operator logika pada kondisi if.
\end{aktivitas}

\begin{latihan}
  \item Apa perbedaan \code{=} dan \code{==}?
  \item Jelaskan perilaku short-circuit pada \code{\&\&} dan \code{||}!
  \item \textbf{Refleksi}: Kapan Anda memilih operator logika \code{\&\&} daripada \code{||}? Berikan contoh kasus dari program yang pernah Anda buat.
\end{latihan}

\begin{asesmen}
\textbf{Instrumen untuk Sub-CPMK 3.1}: Buat program C yang membaca dua bilangan dan satu operator (+, -, *, /), lalu menampilkan hasil operasi. Gunakan operator aritmatika dan validasi pembagi nol. Jelaskan penggunaan operator yang Anda pakai.
\end{asesmen}

\begin{checklist}
  \item Saya dapat menggunakan operator aritmatika dan perbandingan
  \item Saya dapat menggunakan operator logika
\end{checklist}

\begin{rangkuman}
Operator C meliputi aritmatika (+ - * / \%), perbandingan (== != < >), logika (\&\& || !), dan penugasan (= += -=).
\end{rangkuman}

\ifSubfilesClassLoaded{
  \renewcommand{\bibname}{Daftar Pustaka}
  \bibliographystyle{plain}
  \bibliography{../references}
}{}
\end{document}
