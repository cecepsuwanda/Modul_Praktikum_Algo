\documentclass[../main.tex]{subfiles}
\ifSubfilesClassLoaded{\setcounter{chapter}{13}}{}
\begin{document}

\chapter{String dalam C}

\begin{subcpmk}
  \item Sub-CPMK 4.2: Menggunakan string dan manipulasi teks dalam C
\end{subcpmk}

\noindent\textbf{Materi Pokok:} String sebagai array \code{char} dengan null terminator; fungsi \code{strlen}, \code{strcpy}, \code{strcmp}, \code{strcat} dari \code{string.h}.\cite{ref7,string_cppref}

\section{String dalam C}

String dalam C adalah array karakter yang diakhiri null character \code{'\textbackslash0'}. Deklarasi: \code{char str[50];} atau \code{char str[] = "Hello";}. String literal menggunakan tanda kutip ganda. Fungsi \code{scanf("\%s", str)} membaca kata (tanpa spasi); \code{gets} atau \code{fgets} untuk baris lengkap. Header \code{string.h} menyediakan \code{strlen}, \code{strcpy}, \code{strcmp}, \code{strcat}. Perhatikan ukuran array cukup untuk menyimpan string termasuk \code{'\textbackslash0'}.

\section{Manipulasi String}

Fungsi \code{strlen(str)} mengembalikan panjang string tanpa null. \code{strcpy(dest, src)} menyalin src ke dest. \code{strcmp(s1, s2)} membandingkan: return 0 jika sama, <0 jika s1<s2, >0 jika s1>s2. \code{strcat(dest, src)} menggabungkan src ke akhir dest. Pastikan buffer dest cukup besar. Untuk input aman gunakan \code{fgets(str, size, stdin)}. Manipulasi manual: loop melalui indeks hingga \code{'\textbackslash0'}. Contoh: membalik string, menghitung jumlah karakter tertentu.


\begin{aktivitas}
  \item Buat program mengecek palindrom.
  \item Buat program menghitung jumlah kata dalam kalimat.
\end{aktivitas}

\begin{latihan}
  \item Mengapa string C diakhiri '\textbackslash0'?
  \item Jelaskan perbedaan strcpy dan strcmp!
  \item \textbf{Refleksi}: Apa risiko jika lupa mengalokasikan ruang cukup untuk string (termasuk \textbackslash0) atau lupa null terminator? Berikan contoh.
\end{latihan}

\begin{asesmen}
\textbf{Instrumen untuk Sub-CPMK 4.2}: Buat program C yang membaca sebuah kata, lalu mengecek apakah kata tersebut palindrom (dibaca sama dari kiri dan kanan). Gunakan loop dan bandingkan karakter, atau gunakan fungsi dari \code{string.h}.
\end{asesmen}

\begin{checklist}
  \item Saya dapat menggunakan string dan fungsi string.h
  \item Saya dapat memanipulasi string
\end{checklist}

\begin{rangkuman}
String = array char + null terminator. Fungsi: strlen, strcpy, strcmp, strcat.
\end{rangkuman}

\ifSubfilesClassLoaded{
  \renewcommand{\bibname}{Daftar Pustaka}
  \bibliographystyle{plain}
  \bibliography{../references}
}{}
\end{document}
