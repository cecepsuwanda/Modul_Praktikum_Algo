\documentclass[../main.tex]{subfiles}
\ifSubfilesClassLoaded{\setcounter{chapter}{3}}{}
\begin{document}

\chapter{Pseudocode dan Notasi Algoritma}

\begin{subcpmk}
  \item Sub-CPMK 1.2: Membuat pseudocode untuk masalah sederhana dan mengkonversinya ke bahasa C
\end{subcpmk}

\noindent\textbf{Materi Pokok:} Konsep dan notasi pseudocode, pemetaan pseudocode ke struktur kontrol C, serta contoh konversi sederhana \cite{pseudocode_wikipedia,ref7}.

\section{Definisi dan Notasi Pseudocode}

Pseudocode adalah deskripsi algoritma dalam bahasa manusia yang menyerupai kode program namun tidak terikat pada sintaks bahasa pemrograman tertentu \cite{pseudocode_wikipedia}. Kata "pseudocode" berasal dari "pseudo" (semu) dan "code" (kode), yang berarti kode semu atau tiruan kode program. Pseudocode pertama kali digunakan pada tahun 1950-an sebagai alat untuk mendokumentasikan algoritma sebelum implementasi dalam bahasa assembly.

\subsection{Fungsi Pseudocode}

Pseudocode memiliki beberapa fungsi penting dalam pengembangan perangkat lunak:

\begin{enumerate}
  \item \textbf{Perancangan Logika:} Memungkinkan fokus pada algoritma tanpa detail sintaks
  \item \textbf{Dokumentasi:} Membuat dokumentasi algoritma yang mudah dipahami
  \item \textbf{Komunikasi:} Memfasilitasi komunikasi tim tanpa terikat bahasa pemrograman
  \item \textbf{Validasi:} Memungkinkan pengecekan logika sebelum implementasi
  \item \textbf{Pembelajaran:} Membantu pemula memahami konsep pemrograman
\end{enumerate}

\subsection{Standar Notasi Pseudocode}

Berikut adalah standar notasi pseudocode yang umum digunakan:

\begin{table}[htbp]
\centering
\footnotesize
\begin{tabular}{|>{\raggedright\arraybackslash}p{2.6cm}|>{\raggedright\arraybackslash}p{4.25cm}|>{\raggedright\arraybackslash}p{4.05cm}|}
\hline
\textbf{Kategori} & \textbf{Notasi Pseudocode} & \textbf{Contoh} \\
\hline
Struktur & MULAI, SELESAI & MULAI ... SELESAI \\
\hline
Input/Output & BACA, CETAK, TULIS & BACA nilai, CETAK ``Hello'' \\
\hline
Assignment & := atau = & nilai := 10, x = y + 1 \\
\hline
Percabangan & JIKA...MAKA...JIKA TIDAK & JIKA x > 0 MAKA CETAK ``Positif'' \\
\hline
Perulangan & UNTUK...DARI...\allowbreak SAMPAI & UNTUK i := 1 DARI 1 SAMPAI 10 \\
\hline
Perulangan & SELAMA...LAKUKAN & SELAMA $x > 0$ LAKUKAN $x := x - 1$ \\
\hline
Prosedur & PROSEDUR nama() & PROSEDUR hitungRata() \\
\hline
Fungsi & FUNGSI nama() & FUNGSI maks(a, b) \\
\hline
Komentar & // atau /* ... */ & // Ini komentar \\
\hline
\end{tabular}
\caption{Standar Notasi Pseudocode}
\end{table}

\subsection{Aturan Penulisan Pseudocode}

\begin{itemize}
  \item \textbf{Konsistensi:} Gunakan notasi yang sama di seluruh pseudocode
  \item \textbf{Indentasi:} Gunakan indentasi untuk menunjukkan blok kode
  \item \textbf{Bahasa Jelas:} Gunakan bahasa Indonesia atau Inggris yang jelas
  \item \textbf{Struktur Logis:} Ikuti alur logika yang sistematis
  \item \textbf{Detail Tepat:} Berikan detail cukup namun tidak terlalu teknis
\end{itemize}

\subsection{Perbandingan Pseudocode dan Flowchart}

\begin{table}[htbp]
\centering
\small
\begin{tabular}{|>{\raggedright\arraybackslash}p{5cm}|>{\raggedright\arraybackslash}p{5cm}|}
\hline
\textbf{Pseudocode} & \textbf{Flowchart} \\
\hline
Teks-based, linear & Visual, grafis \\
\hline
Lebih detail implementasi & Lebih fokus pada alur logika \\
\hline
Mudah diketik & Membutuhkan alat gambar \\
\hline
Baik untuk algoritma kompleks & Baik untuk presentasi \\
\hline
Struktur hierarki jelas & Alur visual langsung \\
\hline
\end{tabular}
\caption{Perbandingan Pseudocode dan Flowchart}
\end{table}

Pseudocode memudahkan perancangan logika algoritma sebelum implementasi ke bahasa pemrograman seperti C. Dengan pseudocode, programmer dapat fokus pada alur logika tanpa khawatir tentang detail sintaks.

\section{Contoh Pseudocode dan Konversi ke Bahasa C}

Berikut adalah contoh-contoh pseudocode lengkap dengan konversinya ke bahasa C untuk memahami hubungan antara desain algoritma dan implementasi kode.

\subsection{Contoh 1: Program Hello World}

\textbf{Pseudocode:}
\begin{lstlisting}[language=]
MULAI
    CETAK "Hello, World"
SELESAI
\end{lstlisting}

\textbf{Konversi ke C:}
\begin{lstlisting}[language=C, caption={Konversi Hello World ke C}]
#include <stdio.h>

int main() {
    printf("Hello, World\n");
    return 0;
}
\end{lstlisting}

\textbf{Analisis Konversi:}
\begin{itemize}
  \item MULAI → \code{int main() \{}
  \item CETAK → \code{printf()}
  \item SELESAI → \code{return 0; \}}
\end{itemize}

\subsection{Contoh 2: Membaca Input dan Menampilkan Output}

\textbf{Pseudocode:}
\begin{lstlisting}[language=]
MULAI
    BACA nama
    CETAK "Halo, ", nama
SELESAI
\end{lstlisting}

\textbf{Konversi ke C:}
\begin{lstlisting}[language=C, caption={Konversi Input/Output ke C}]
#include <stdio.h>

int main() {
    char nama[50];
    
    printf("Masukkan nama: ");
    scanf("%s", nama);
    printf("Halo, %s\n", nama);
    
    return 0;
}
\end{lstlisting}

\textbf{Analisis Konversi:}
\begin{itemize}
  \item BACA nama → \code{scanf("\%s", nama)}
  \item CETAK → \code{printf()}
  \item Perlu deklarasi variabel: \code{char nama[50];}
\end{itemize}

\subsection{Contoh 3: Percabangan If-Else}

\textbf{Pseudocode:}
\begin{lstlisting}[language=]
MULAI
    BACA nilai
    JIKA nilai > 60 MAKA
        CETAK "Lulus"
    JIKA TIDAK
        CETAK "Tidak Lulus"
SELESAI
\end{lstlisting}

\textbf{Konversi ke C:}
\begin{lstlisting}[language=C, caption={Konversi Percabangan ke C}]
#include <stdio.h>

int main() {
    int nilai;
    
    printf("Masukkan nilai: ");
    scanf("%d", &nilai);
    
    if (nilai > 60) {
        printf("Lulus\n");
    } else {
        printf("Tidak Lulus\n");
    }
    
    return 0;
}
\end{lstlisting}

\subsection{Contoh 4: Perulangan For}

\textbf{Pseudocode:}
\begin{lstlisting}[language=]
MULAI
    UNTUK i := 1 DARI 1 SAMPAI 5 LAKUKAN
        CETAK i
SELESAI
\end{lstlisting}

\textbf{Konversi ke C:}
\begin{lstlisting}[language=C, caption={Konversi Perulangan ke C}]
#include <stdio.h>

int main() {
    int i;
    
    for (i = 1; i <= 5; i++) {
        printf("%d\n", i);
    }
    
    return 0;
}
\end{lstlisting}

\subsection{Contoh 5: Menghitung Rata-rata}

\textbf{Pseudocode:}
\begin{lstlisting}[language=]
MULAI
    BACA bil1, bil2, bil3
    jumlah := bil1 + bil2 + bil3
    rata := jumlah / 3
    CETAK "Rata-rata: ", rata
SELESAI
\end{lstlisting}

\textbf{Konversi ke C:}
\begin{lstlisting}[language=C, caption={Konversi Perhitungan Rata-rata ke C}]
#include <stdio.h>

int main() {
    float bil1, bil2, bil3, jumlah, rata;
    
    printf("Masukkan tiga bilangan: ");
    scanf("%f %f %f", &bil1, &bil2, &bil3);
    
    jumlah = bil1 + bil2 + bil3;
    rata = jumlah / 3;
    
    printf("Rata-rata: %.2f\n", rata);
    
    return 0;
}
\end{lstlisting}

\subsection{Tabel Pemetaan Konversi}

\begin{table}[htbp]
\centering
\small
\begin{tabular}{|>{\raggedright\arraybackslash}p{3cm}|>{\raggedright\arraybackslash}p{4cm}|}
\hline
\textbf{Pseudocode} & \textbf{Bahasa C} \\
\hline
MULAI & \code{int main() \{} \\
\hline
SELESAI & \code{return 0; \}} \\
\hline
BACA variabel & \code{scanf("\%format", \&variabel);} \\
\hline
CETAK teks & \code{printf("teks\textbackslash n");} \\
\hline
variabel := nilai & \code{variabel = nilai;} \\
\hline
JIKA kondisi MAKA & \code{if (kondisi) \{} \\
\hline
JIKA TIDAK & \code{\} else \{} \\
\hline
UNTUK i := 1 SAMPAI n & \code{for (i = 1; i <= n; i++) \{} \\
\hline
SELAMA kondisi & \code{while (kondisi) \{} \\
\hline
LAKUKAN & \code{// Isi loop} \\
\hline
\end{tabular}
\caption{Pemetaan Konversi Pseudocode ke Bahasa C}
\end{table}

Konversi pseudocode ke C dilakukan dengan memetakan setiap notasi pseudocode ke sintaks yang sesuai dalam bahasa C, ditambah dengan deklarasi variabel dan header yang diperlukan.


\begin{aktivitas}
  \item Tulis pseudocode untuk menghitung rata-rata tiga bilangan, lalu konversikan ke C.
  \item Bandingkan flowchart dan pseudocode untuk masalah yang sama. Mana yang lebih mudah dibuat?
\end{aktivitas}

\begin{latihan}
  \item Jelaskan perbedaan flowchart dan pseudocode!
  \item Tulis pseudocode untuk menentukan bilangan terbesar dari tiga input!
  \item \textbf{Refleksi}: Bagian mana dari konversi pseudocode ke C yang paling menantang bagi Anda? Jelaskan alasannya.
\end{latihan}

\begin{asesmen}
\textbf{Instrumen untuk Sub-CPMK 1.2}: Buat pseudocode dan implementasi C untuk menghitung luas segitiga dari input alas dan tinggi, lalu jelaskan pemetaan setiap langkah pseudocode ke kode C.
\end{asesmen}

\begin{checklist}
  \item Saya dapat menulis pseudocode untuk masalah sederhana
  \item Saya dapat mengkonversi pseudocode ke kode C
\end{checklist}

\begin{rangkuman}
Pseudocode adalah deskripsi algoritma dalam bahasa manusia yang memudahkan perancangan sebelum implementasi ke C.
\end{rangkuman}

\ifSubfilesClassLoaded{
  \renewcommand{\bibname}{Daftar Pustaka}
  \bibliographystyle{plain}
  \bibliography{../references}
}{}
\end{document}
