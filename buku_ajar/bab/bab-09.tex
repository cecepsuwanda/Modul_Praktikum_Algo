\documentclass[../main.tex]{subfiles}
\ifSubfilesClassLoaded{\setcounter{chapter}{8}}{}
\begin{document}

\chapter{Struktur Perulangan (for, while, do-while)}

\begin{subcpmk}
  \item Sub-CPMK 2.1, 3.2: Merancang dan mengimplementasikan struktur perulangan dalam C
\end{subcpmk}

\noindent\textbf{Materi Pokok:} Loop \code{for}, \code{while}, \code{do-while}; inisialisasi, kondisi, dan increment; kapan menggunakan masing-masing \cite{ref7,control_flow_cppref}.

\section{Perulangan for dan while}

Perulangan \code{for}: \code{for (inisialisasi; kondisi; update) \{ ... \}}. Inisialisasi dijalankan sekali di awal; kondisi dicek setiap iterasi; update dijalankan setelah tiap iterasi. Perulangan \code{while}: \code{while (kondisi) \{ ... \}}; tubuh loop dieksekusi selama kondisi benar. For cocok ketika jumlah iterasi diketahui; while cocok ketika kondisi berhenti tidak pasti. Pastikan ada mekanisme agar loop dapat berhenti untuk menghindari infinite loop.

\section{Perulangan do-while}

Perulangan \code{do-while}: \code{do \{ ... \} while (kondisi);}. Berbeda dengan while, tubuh loop di \code{do-while} dieksekusi minimal sekali karena kondisi dicek di akhir. Berguna untuk input validasi: minta input berulang sampai valid. Sintaks: perhatikan titik koma setelah \code{while (kondisi)}. Bandingkan: while bisa tidak pernah dieksekusi; do-while selalu minimal satu kali.


\begin{aktivitas}
  \item Buat program menampilkan bilangan 1 sampai 10 dengan for.
  \item Buat program menghitung jumlah digit bilangan dengan while.
\end{aktivitas}

\begin{latihan}
  \item Jelaskan perbedaan for, while, dan do-while!
  \item Buat program menampilkan tabel perkalian!
  \item \textbf{Refleksi}: Untuk tugas \"tampilkan bilangan 1 sampai N\", mengapa for sering lebih nyaman daripada while? Kapan Anda justru memilih while?
\end{latihan}

\begin{asesmen}
\textbf{Instrumen untuk Sub-CPMK 2.1, 3.2}: Buat program C yang membaca bilangan bulat positif N, lalu menampilkan jumlah bilangan dari 1 sampai N (1+2+...+N) menggunakan loop. Implementasikan dengan \code{for} dan bandingkan dengan versi \code{while}.
\end{asesmen}

\begin{checklist}
  \item Saya dapat menggunakan for, while, do-while
  \item Saya memahami kapan menggunakan masing-masing
\end{checklist}

\begin{rangkuman}
For untuk iterasi dengan jumlah pasti; while untuk kondisi berhenti; do-while minimal satu eksekusi.
\end{rangkuman}

\ifSubfilesClassLoaded{
  \renewcommand{\bibname}{Daftar Pustaka}
  \bibliographystyle{plain}
  \bibliography{../references}
}{}
\end{document}
