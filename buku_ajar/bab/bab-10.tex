\documentclass[../main.tex]{subfiles}
\ifSubfilesClassLoaded{\setcounter{chapter}{9}}{}
\begin{document}

\chapter{Perulangan Bersarang}

\begin{subcpmk}
  \item Sub-CPMK 3.2: Mengimplementasikan perulangan bersarang
\end{subcpmk}

\noindent\textbf{Materi Pokok:} Perulangan bersarang (nested loop); alur eksekusi; penerapan untuk pola, tabel, dan matriks \cite{ref7,control_flow_cppref}.

\section{Konsep Perulangan Bersarang}

Perulangan bersarang (nested loop) adalah perulangan di dalam perulangan. Loop luar mengontrol baris; loop dalam mengontrol kolom. Setiap iterasi loop luar, loop dalam dijalankan lengkap. Contoh: menampilkan pola persegi atau segitiga dengan karakter. Perhatikan indeks dan kondisi masing-masing loop agar tidak konflik. Kompleksitas waktu bertambah: dua loop bersarang masing-masing n iterasi menghasilkan n² operasi.

\section{Penerapan Perulangan Bersarang}

Aplikasi: tabel perkalian (baris i, kolom j, tampilkan i*j), pola bintang (segitiga, persegi), operasi matriks (perkalian, transpose), array dua dimensi. Contoh pola segitiga: loop luar i dari 0 ke n, loop dalam j dari 0 ke i, cetak bintang. Untuk array 2D, loop bersarang untuk mengakses setiap elemen. Pahami urutan eksekusi: loop dalam selesai penuh sebelum loop luar increment.


\begin{aktivitas}
  \item Buat program menampilkan tabel perkalian 10x10.
  \item Buat program menampilkan pola segitiga bintang.
\end{aktivitas}

\begin{latihan}
  \item Jelaskan alur eksekusi perulangan bersarang!
  \item Buat program menampilkan pola piramida angka!
  \item \textbf{Refleksi}: Saat membuat pola segitiga atau tabel, kesalahan apa yang paling sering Anda buat (indeks, kondisi)? Bagaimana Anda mendebugnya?
\end{latihan}

\begin{asesmen}
\textbf{Instrumen untuk Sub-CPMK 3.2}: Buat program C yang membaca tinggi N, lalu menampilkan pola segitiga siku-siku terbalik dari bintang (tinggi N baris). Gunakan nested loop. Tuliskan pseudocode sebelum implementasi.
\end{asesmen}

\begin{checklist}
  \item Saya dapat membuat perulangan bersarang
  \item Saya dapat menerapkannya untuk pola dan tabel
\end{checklist}

\begin{rangkuman}
Perulangan bersarang: loop di dalam loop. Digunakan untuk pola, tabel, dan operasi matriks.
\end{rangkuman}

\ifSubfilesClassLoaded{
  \renewcommand{\bibname}{Daftar Pustaka}
  \bibliographystyle{plain}
  \bibliography{../references}
}{}
\end{document}
