\section{Perulangan do-while}

Perulangan \code{do-while}: \code{do \{ ... \} while (kondisi);}. Berbeda dengan while, tubuh loop di \code{do-while} dieksekusi minimal sekali karena kondisi dicek di akhir. Berguna untuk input validasi: minta input berulang sampai valid. Sintaks: perhatikan titik koma setelah \code{while (kondisi)}. Bandingkan: while bisa tidak pernah dieksekusi; do-while selalu minimal satu kali.
