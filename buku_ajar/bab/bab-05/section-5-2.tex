\section{Tipe Data Dasar}

Tipe data dasar C menentukan jenis data yang dapat disimpan dalam variabel, ukuran memori yang digunakan, dan rentang nilai yang mungkin. Pemilihan tipe data yang tepat sangat penting untuk efisiensi memori dan akurasi perhitungan.

\subsection{Tipe Data Integer}

Tipe data integer digunakan untuk menyimpan bilangan bulat (tanpa desimal):

\begin{table}[h]
\centering
\small
\begin{tabular}{|>{\raggedright\arraybackslash}p{2.5cm}|>{\raggedright\arraybackslash}p{2cm}|>{\raggedright\arraybackslash}p{4cm}|>{\raggedright\arraybackslash}p{4cm}|}
\hline
\textbf{Tipe Data} & \textbf{Ukuran} & \textbf{Rentang Nilai} & \textbf{Contoh Penggunaan} \\
\hline
\code{short int} & 2 byte & -32,768 hingga 32,767 & Usia, jumlah kecil \\
\hline
\code{unsigned short} & 2 byte & 0 hingga 65,535 & ID positif \\
\hline
\code{int} & 4 byte & -2,147,483,648 hingga 2,147,483,647 & Counter, umum \\
\hline
\code{unsigned int} & 4 byte & 0 hingga 4,294,967,295 & ID, populasi \\
\hline
\code{long int} & 4/8 byte & Sama seperti int & Kompatibilitas \\
\hline
\code{long long} & 8 byte & $\pm$9.2 x 10$^{18}$ & Bilangan sangat besar \\
\hline
\end{tabular}
\caption{Tipe Data Integer dalam C}
\end{table}

\subsection{Tipe Data Floating Point}

Tipe data floating point digunakan untuk menyimpan bilangan desimal:

\begin{table}[h]
\centering
\small
\begin{tabular}{|>{\raggedright\arraybackslash}p{2.5cm}|>{\raggedright\arraybackslash}p{2cm}|>{\raggedright\arraybackslash}p{4cm}|>{\raggedright\arraybackslash}p{4cm}|}
\hline
\textbf{Tipe Data} & \textbf{Ukuran} & \textbf{Presisi} & \textbf{Contoh Penggunaan} \\
\hline
\code{float} & 4 byte & 6-7 digit desimal & Suhu, persentase \\
\hline
\code{double} & 8 byte & 15-16 digit desimal & IPK, nilai akurat \\
\hline
\code{long double} & 10/16 byte & 19-21 digit desimal & Ilmiah, presisi tinggi \\
\hline
\end{tabular}
\caption{Tipe Data Floating Point dalam C}
\end{table}

\subsection{Tipe Data Karakter}

\begin{itemize}
  \item \textbf{\code{char}:} Menyimpan satu karakter (1 byte)
  \item \textbf{Rentang:} -128 hingga 127 (signed) atau 0 hingga 255 (unsigned)
  \item \textbf{Penggunaan:} Huruf, angka, simbol, atau nilai ASCII
\end{itemize}

\textbf{Contoh:}
\begin{lstlisting}[language=C]
char grade = 'A';        // Karakter A
char newline = '\n';      // Karakter newline
char nilai_ascii = 65;     // Sama dengan 'A'
\end{lstlisting}

\subsection{Modifier Tipe Data}

Modifier dapat digunakan untuk mengubah sifat tipe data dasar:

\begin{table}[h]
\centering
\small
\begin{tabular}{|>{\raggedright\arraybackslash}p{3cm}|>{\raggedright\arraybackslash}p{7cm}|}
\hline
\textbf{Modifier} & \textbf{Pengaruh} \\
\hline
\code{signed} & Default, dapat menyimpan nilai positif dan negatif \\
\hline
\code{unsigned} & Hanya nilai non-negatif, rentang positif dua kali lipat \\
\hline
\code{short} & Mengurangi ukuran memori (biasanya setengah dari int) \\
\hline
\code{long} & Meningkatkan ukuran memori (minimal sama dengan int) \\
\hline
\end{tabular}
\caption{Modifier Tipe Data dalam C}
\end{table}

\subsection{Memilih Tipe Data yang Tepat}

\textbf{Guidelines pemilihan tipe data:}

\begin{itemize}
  \item \textbf{Usia, jumlah item:} \code{int} atau \code{unsigned int}
  \item \textbf{Uang, nilai akademik:} \code{double} (hindari float untuk akurasi)
  \item \textbf{Suhu, persentase:} \code{float} (cukup untuk presisi biasa)
  \item \textbf{Huruf, status:} \code{char}
  \item \textbf{Bilangan sangat besar:} \code{long long}
  \item \textbf{ID yang selalu positif:} \code{unsigned int}
\end{itemize}

\textbf{Contoh implementasi:}
\begin{lstlisting}[language=C]
#include <stdio.h>

int main() {
    // Tipe data yang tepat untuk berbagai kebutuhan
    int umur = 20;                    // Umur mahasiswa
    double ipk = 3.75;                 // IPK (perlu presisi)
    float suhu = 36.5f;               // Suhu tubuh
    char grade = 'A';                  // Nilai huruf
    unsigned int npm = 2024001234;      // Nomor pokok mahasiswa
    
    printf("Umur: %d tahun\n", umur);
    printf("IPK: %.2f\n", ipk);
    printf("Suhu: %.1f derajat C\n", suhu);
    printf("Grade: %c\n", grade);
    printf("NPM: %u\n", npm);
    
    return 0;
}
\end{lstlisting}
