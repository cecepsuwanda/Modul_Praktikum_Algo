\section{Variabel dan Deklarasi dalam C}

Variabel adalah tempat penyimpanan data di memori yang memiliki nama dan tipe data \cite{ref7}. Dalam pemrograman, variabel berfungsi seperti wadah yang dapat menampung nilai-nilai yang berubah selama eksekusi program. Setiap variabel dalam C memiliki alamat memori unik yang digunakan untuk menyimpan dan mengakses data.

\subsection{Deklarasi Variabel}

Dalam C, variabel harus dideklarasikan sebelum digunakan dengan sintaks umum:
\begin{lstlisting}[language=C]
tipe_data nama_variabel;
\end{lstlisting}

\textbf{Contoh Deklarasi:}
\begin{lstlisting}[language=C]
int umur;                    // Variabel integer untuk umur
float tinggi;                // Variabel float untuk tinggi badan
char grade;                  // Variabel karakter untuk nilai
double gpa;                  // Variabel double untuk IPK
\end{lstlisting}

\subsection{Inisialisasi Variabel}

Variabel dapat diinisialisasi saat deklarasi atau diberi nilai kemudian:

\begin{lstlisting}[language=C]
// Inisialisasi saat deklarasi
int x = 10;
float pi = 3.14159;
char huruf = 'A';

// Inisialisasi terpisah
int y;
y = 20;  // Assignment setelah deklarasi
\end{lstlisting}

\subsection{Aturan Penamaan Variabel}

Nama variabel (identifier) harus mengikuti aturan berikut:

\begin{itemize}
  \item \textbf{Karakter yang diizinkan:} Huruf (a-z, A-Z), angka (0-9), dan underscore (\_)
  \item \textbf{Karakter pertama:} Harus huruf atau underscore, tidak boleh angka
  \item \textbf{Panjang maksimal:} Biasanya 31 karakter (standar C)
  \item \textbf{Case sensitive:} \code{nilai} berbeda dengan \code{Nilai}
  \item \textbf{Reserved words:} Tidak boleh menggunakan kata kunci C seperti \code{int}, \code{if}, \code{while}
\end{itemize}

\textbf{Contoh nama variabel yang valid:}
\begin{lstlisting}[language=C]
int umurMahasiswa;
float nilai_rata_rata;
char grade;
int _counter;
\end{lstlisting}

\textbf{Contoh nama variabel yang tidak valid:}
\begin{lstlisting}[language=C]
int 2umur;          // Tidak boleh diawali angka
float nilai-rata;    // Tidak boleh menggunakan strip (-)
char class;          // Tidak boleh reserved word
\end{lstlisting}

\subsection{Scope Variabel}

Scope menentukan di mana variabel dapat diakses dalam program:

\begin{table}[htbp]
\centering
\small
\begin{tabular}{|>{\raggedright\arraybackslash}p{3cm}|>{\raggedright\arraybackslash}p{7cm}|}
\hline
\textbf{Jenis Scope} & \textbf{Keterangan} \\
\hline
Local & Dideklarasikan di dalam fungsi, hanya dapat diakses di fungsi tersebut \\
\hline
Global & Dideklarasikan di luar semua fungsi, dapat diakses di seluruh program \\
\hline
Block & Dideklarasikan di dalam blok \{\}, hanya dapat diakses di blok tersebut \\
\hline
\end{tabular}
\caption{Jenis Scope Variabel dalam C}
\end{table}

\textbf{Contoh Scope:}
\begin{lstlisting}[language=C]
#include <stdio.h>

int globalVar = 100;  // Variabel global

int main() {
    int localVar = 50;  // Variabel local
    
    {
        int blockVar = 25;  // Variabel block
        printf("Block: %d\n", blockVar);  // Bisa diakses
    }
    
    printf("Local: %d\n", localVar);    // Bisa diakses
    printf("Global: %d\n", globalVar);   // Bisa diakses
    // printf("Block: %d\n", blockVar);  // Error: tidak bisa diakses
    
    return 0;
}
\end{lstlisting}
