\section{Pemrograman Prosedural dan Paradigma Lain}

\textbf{Pemrograman prosedural} adalah paradigma yang berfokus pada prosedur atau fungsi yang beroperasi pada data \cite{procedural_programming_wikipedia}. Program terdiri dari serangkaian instruksi yang dieksekusi secara berurutan, dengan data dan fungsi dapat dipisahkan. Bahasa C adalah contoh utama bahasa pemrograman prosedural yang masih digunakan luas hingga saat ini. Mata kuliah Algoritma dan Pemrograman menggunakan pendekatan prosedural karena memungkinkan fokus pada logika algoritmik tanpa kompleksitas paradigma lain.

Selain prosedural, terdapat paradigma pemrograman lain seperti pemrograman berorientasi objek (OOP) yang berfokus pada objek dan class, serta pemrograman fungsional yang menekankan evaluasi fungsi. Pemrograman prosedural cocok untuk program yang relatif kecil, tugas komputasi langsung, dan pembelajar pemrograman pemula. Pemahaman prosedural menjadi fondasi penting sebelum mempelajari paradigma yang lebih kompleks seperti OOP di semester berikutnya.

\begin{table}[htbp]
\centering
\begin{tabular}{|>{\raggedright\arraybackslash}p{3cm}|>{\raggedright\arraybackslash}p{5cm}|}
\hline
\textbf{Paradigma} & \textbf{Karakteristik Utama} \\
\hline
Prosedural & Fungsi dan data terpisah, alur top-down, bahasa C, Pascal \\
\hline
Berorientasi Objek & Class dan objek, enkapsulasi, bahasa Java, C++, Python \\
\hline
Fungsional & Fungsi sebagai nilai, bahasa Haskell, Lisp \\
\hline
\end{tabular}
\caption{Perbandingan Paradigma Pemrograman}
\end{table}
