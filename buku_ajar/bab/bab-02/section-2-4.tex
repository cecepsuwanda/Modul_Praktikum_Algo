\section{Karakteristik Algoritma}

Berdasarkan standar ilmu komputer modern dan referensi dari GeeksforGeeks, sebuah algoritma harus memiliki enam karakteristik fundamental untuk dapat dikategorikan sebagai algoritma yang baik dan benar \cite{gfg_algorithm_intro}. Karakteristik ini menjadi acuan dalam mengevaluasi kualitas algoritma yang dirancang.

\subsection{Enam Karakteristik Utama Algoritma}

\begin{enumerate}
  \item \textbf{Input (Masukan)}
  \begin{itemize}
    \item Algoritma harus memiliki nol atau lebih input yang terdefinisi dengan jelas
    \item Input harus spesifik dan dapat diukur
    \item Contoh: algoritma pengurutan memerlukan array sebagai input
  \end{itemize}
  
  \item \textbf{Output (Keluaran)}
  \begin{itemize}
    \item Algoritma harus menghasilkan minimal satu output
    \item Output harus sesuai dengan tujuan algoritma
    \item Contoh: array yang sudah terurut, nilai maksimum, hasil perhitungan
  \end{itemize}
  
  \item \textbf{Definiteness (Kejelasan)}
  \begin{itemize}
    \item Setiap langkah harus didefinisikan dengan jelas dan tidak ambigu
    \item Tidak boleh ada interpretasi ganda untuk setiap instruksi
    \item Contoh yang salah: "tambahkan sedikit garam" → tidak jelas
    \item Contoh yang benar: "tambahkan 1 sendok teh garam" → jelas
  \end{itemize}
  
  \item \textbf{Finiteness (Keterbatasan)}
  \begin{itemize}
    \item Algoritma harus berakhir setelah jumlah langkah terbatas
    \item Tidak boleh mengandung infinite loop tanpa kondisi berhenti
    \item Setiap eksekusi harus selesai dalam waktu wajar
  \end{itemize}
  
  \item \textbf{Effectiveness (Efektivitas)}
  \begin{itemize}
    \item Setiap langkah harus dapat dilakukan dengan sumber daya yang tersedia
    \item Instruksi harus dasar dan dapat dieksekusi dalam waktu terbatas
    \item Tidak boleh memerlukan operasi yang tidak mungkin dilakukan
  \end{itemize}
  
  \item \textbf{Determinism (Deterministik)}
  \begin{itemize}
    \item Untuk input yang sama, algoritma harus selalu menghasilkan output yang sama
    \item Tidak ada elemen random atau kebetulan dalam proses
    \item Hasil dapat diprediksi dan direproduksi
  \end{itemize}
\end{enumerate}

\subsection{Tabel Perbandingan Karakteristik}

\begin{table}[htbp]
\centering
\footnotesize
\begin{tabular}{|>{\raggedright\arraybackslash}p{2.4cm}|>{\raggedright\arraybackslash}p{5.5cm}|>{\raggedright\arraybackslash}p{3.95cm}|}
\hline
\textbf{Karakteristik} & \textbf{Deskripsi} & \textbf{Contoh Penerapan} \\
\hline
Input & Data masukan yang diperlukan & Array angka untuk diurutkan \\
\hline
Output & Hasil yang dihasilkan & Array yang sudah terurut \\
\hline
Definiteness & Langkah jelas tidak ambigu & ``Tukar jika A > B'' \\
\hline
Finiteness & Berakhir dalam waktu terbatas & Loop berhenti setelah n-1 iterasi \\
\hline
Effectiveness & Langkah dapat dilakukan & Operasi perbandingan dan pertukaran \\
\hline
Determinism & Hasil konsisten & Input [3,1,2] selalu menghasilkan [1,2,3] \\
\hline
\end{tabular}
\caption{Enam Karakteristik Algoritma yang Baik}
\end{table}

\begin{konsep}
Ringkasan enam karakteristik algoritma:
\begin{enumerate}
  \item \textbf{Input}: Algoritma menerima masukan yang terdefinisi (nol atau lebih)
  \item \textbf{Output}: Algoritma menghasilkan minimal satu keluaran
  \item \textbf{Definiteness}: Setiap langkah jelas dan tidak ambigu
  \item \textbf{Finiteness}: Algoritma berakhir dalam langkah terbatas
  \item \textbf{Effectiveness}: Setiap langkah dapat dilaksanakan secara praktis
  \item \textbf{Determinism}: Input yang sama selalu menghasilkan output yang sama
\end{enumerate}
\end{konsep}
