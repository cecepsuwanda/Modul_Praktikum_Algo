\section{Sejarah dan Konsep Algoritma}

Kata \textit{algoritma} berasal dari nama matematikawan Persia abad ke-9, Abu Abdullah Muhammad bin Musa al-Khwarizmi (780-850 M). Al-Khwarizmi adalah ilmuwan di House of Wisdom (Bayt al-Hikmah) di Baghdad yang menulis kitab \textit{Al-Jabr wa'l-Muqabala} (Buku tentang Penyempurnaan dan Penyeimbangan) pada tahun 825 M. Kitab ini memperkenalkan metode sistematis untuk menyelesaikan persamaan aljabar linear dan kuadrat. Namanya dilatinkan menjadi "Algoritmi" oleh para penerjemah Eropa dan kemudian berkembang menjadi "algorithm" dalam bahasa Inggris.

\subsection{Kontribusi Ilmiah al-Khwarizmi}

Al-Khwarizmi memberikan kontribusi fundamental dalam beberapa bidang:

\begin{itemize}
  \item \textbf{Matematika:} Memperkenalkan sistem desimal posisional dan konsep nol dari India ke dunia Islam dan Eropa
  \item \textbf{Aljabar:} Mengembangkan metode sistematis untuk menyelesaikan persamaan dengan langkah-langkah terstruktur
  \item \textbf{Astronomi:} Membuat tabel astronomi yang akurat untuk navigasi dan penentuan waktu shalat
  \item \textbf{Geografi:} Memperbaiki karya Ptolemy dan membuat peta dunia yang lebih akurat
\end{itemize}

Kontribusinya sangat fundamental dalam matematika komputasi karena mengenalkan pendekatan langkah-demi-langkah (step-by-step) dalam pemecahan masalah, yang menjadi cikal bakal konsep algoritma modern.

Dalam ilmu komputer modern, \textbf{algoritma} didefinisikan sebagai urutan langkah-langkah logis, terdefinisi dengan jelas, dan terbatas untuk menyelesaikan suatu masalah atau mencapai tujuan tertentu \cite{gfg_algorithm_intro}. Algoritma menjadi fondasi fundamental dalam pemrograman karena program komputer pada hakikatnya adalah implementasi algoritma dalam bahasa pemrograman. Setiap program yang kita tulis adalah representasi dari algoritma yang dirancang untuk menyelesaikan masalah spesifik.

\subsection{Perkembangan Historis Algoritma}

Sejarah algoritma tidak terlepas dari perkembangan pemrograman komputer:

\begin{itemize}
  \item \textbf{Era Pra-Komputer (1940-an):} Konsep algoritma sudah ada dalam matematika, namun implementasi masih manual
  \item \textbf{Era Machine Language (1950-an):} Pemrograman dilakukan dengan bahasa mesin dan assembly, algoritma sangat sederhana
  \item \textbf{Era Bahasa Tingkat Tinggi (1960-1970-an):} FORTRAN, COBOL, C, dan Pascal berkembang dengan pendekatan terstruktur
  \item \textbf{Era Modern (1980-an-sekarang):} Algoritma kompleks untuk AI, machine learning, big data
\end{itemize}

Bahasa C yang dikembangkan oleh Dennis Ritchie di Bell Labs pada 1972 menjadi bahasa prosedural yang sangat berpengaruh dan menjadi fondasi bagi banyak bahasa pemrograman modern seperti C++, Java, dan Python.

\subsection{Pentingnya Algoritma dalam Pemrograman}

Pemahaman algoritma penting karena beberapa alasan:

\begin{enumerate}
  \item \textbf{Efisiensi:} Algoritma yang baik menyelesaikan masalah dengan penggunaan sumber daya minimal
  \item \textbf{Kebenaran:} Memastikan solusi yang dihasilkan selalu benar untuk semua kasus input
  \item \textbf{Skalabilitas:} Algoritma yang efisien dapat menangani data dalam jumlah besar
  \item \textbf{Pemeliharaan:} Algoritma yang jelas mudah dipahami dan dimodifikasi
\end{enumerate}

Dalam mata kuliah Algoritma dan Pemrograman, mahasiswa belajar merancang algoritma menggunakan flowchart dan pseudocode sebelum mengimplementasikannya dalam bahasa C. Pendekatan ini memastikan mahasiswa memahami logika solusi sebelum terjun ke implementasi teknis.
