\documentclass[../main.tex]{subfiles}
\ifSubfilesClassLoaded{\setcounter{chapter}{11}}{}
\begin{document}

\chapter{Array Satu Dimensi}

\begin{subcpmk}
  \item Sub-CPMK 4.2: Menggunakan array satu dimensi dalam program
\end{subcpmk}

\noindent\textbf{Materi Pokok:} Deklarasi dan inisialisasi array 1D; indeks dari 0; operasi dasar: input, output, penjumlahan, pencarian \cite{ref7,ref2}.

\section{Deklarasi dan Akses Array Satu Dimensi}

Array adalah kumpulan elemen bertipe sama yang disimpan berurutan dalam memori. Deklarasi: \code{tipe nama[ukuran];} misalnya \code{int arr[10];}. Indeks dimulai dari 0; elemen terakhir berindeks ukuran-1. Akses elemen: \code{arr[i]}. Inisialisasi: \code{int arr[] = \{1, 2, 3\};} ukuran otomatis 3. Array tidak mengecek batas indeks; akses di luar range menyebabkan undefined behavior. Array diteruskan ke fungsi dengan nama (pointer ke elemen pertama).

\section{Operasi Dasar pada Array}

Operasi dasar: input (loop dengan scanf), output (loop dengan printf), penjumlahan elemen, pencarian nilai (linear search), mencari nilai maks/min. Contoh input: \code{for (i=0; i<n; i++) scanf("\%d", \&arr[i]);}. Ukuran array bisa konstan atau variabel (C99: Variable Length Array). Array tidak menyimpan ukuran; programmer harus melacak jumlah elemen valid. Operasi pengurutan dan pencarian akan dibahas di bab terpisah.


\begin{aktivitas}
  \item Buat program membaca n bilangan, menyimpan di array, lalu menampilkan rata-rata.
  \item Buat program mencari nilai terbesar dalam array.
\end{aktivitas}

\begin{latihan}
  \item Mengapa indeks array dimulai dari 0?
  \item Buat program membalik urutan elemen array!
  \item \textbf{Refleksi}: Kesalahan apa yang sering terjadi saat mengakses array (off-by-one, out of bounds)? Bagaimana Anda menghindarinya?
\end{latihan}

\begin{asesmen}
\textbf{Instrumen untuk Sub-CPMK 4.2}: Buat program C yang membaca N bilangan ke dalam array, lalu menampilkan nilai terbesar, terkecil, dan rata-ratanya. Validasi N (misalnya 1--100). Jelaskan penggunaan indeks dalam loop Anda.
\end{asesmen}

\begin{checklist}
  \item Saya dapat mendeklarasikan dan mengakses array 1D
  \item Saya dapat melakukan operasi dasar pada array
\end{checklist}

\begin{rangkuman}
Array 1D: kumpulan elemen bertipe sama. Indeks dari 0. Operasi: input, output, penjumlahan, pencarian.
\end{rangkuman}

\ifSubfilesClassLoaded{
  \renewcommand{\bibname}{Daftar Pustaka}
  \bibliographystyle{plain}
  \bibliography{../references}
}{}
\end{document}
