\documentclass[../main.tex]{subfiles}
\ifSubfilesClassLoaded{\setcounter{chapter}{0}}{}
\begin{document}

\chapter{Pendahuluan dan Orientasi Buku}

\section{Tujuan Buku Ajar}

Buku ajar ini dirancang sebagai panduan komprehensif untuk menguasai Algoritma dan Pemrograman menggunakan bahasa C. Fokus utama buku ini adalah pada pemahaman konsep algoritma, kemampuan merancang solusi, dan implementasi program komputer secara prosedural. Tujuan spesifik buku ini adalah memberikan landasan yang kokoh bagi mahasiswa Teknik Informatika sebelum mempelajari mata kuliah pemrograman lanjutan.

Setelah mempelajari buku ini secara menyeluruh, mahasiswa diharapkan mampu memahami dan menjelaskan konsep algoritma, flowchart, serta pseudocode beserta karakteristiknya. Mahasiswa juga diharapkan mampu merancang algoritma untuk menyelesaikan masalah sederhana menggunakan flowchart dan pseudocode. Selain itu, mahasiswa harus mampu mengimplementasikan algoritma dalam bahasa C meliputi variabel, tipe data, I/O, operator, percabangan, dan perulangan.

Mahasiswa diharapkan mampu menggunakan fungsi, prosedur, array, dan struktur data dasar dalam bahasa C. Terakhir, mahasiswa harus mampu menerapkan algoritma pengurutan dan pencarian sederhana serta mengimplementasikannya dalam program. Semua tujuan tersebut dirancang untuk memfasilitasi pencapaian CPL dan CPMK yang telah ditetapkan dalam RPS mata kuliah Algoritma dan Pemrograman.

\section[Keterkaitan Buku Ajar dengan RPS Berbasis OBE]{Keterkaitan Buku Ajar dengan RPS\\ Berbasis OBE}

Buku ajar ini dirancang selaras dengan Rencana Pembelajaran Semester (RPS) mata kuliah Algoritma dan Pemrograman yang berbasis OBE. Keterkaitan ini diwujudkan melalui pemetaan eksplisit setiap bab ke Sub-CPMK yang berkontribusi pada pencapaian CPMK dan CPL program studi. Struktur ini memastikan bahwa materi pembelajaran fokus pada pencapaian kompetensi terukur sesuai dengan matriks 16 pertemuan dalam silabus.

\subsection{Alignment dengan CPL dan CPMK}

Setiap bab dalam buku ini dipetakan secara eksplisit ke Sub-CPMK yang berkontribusi pada pencapaian CPMK dan CPL. Materi pembelajaran dirancang untuk mendukung pengembangan keterampilan merancang algoritma dan mengimplementasikan program dalam bahasa C. Asesmen dalam setiap bab mengukur pencapaian kompetensi secara objektif sesuai indikator yang telah ditetapkan.

\subsection{Integrasi Metode Pembelajaran}

Buku ini mengintegrasikan berbagai metode pembelajaran yang tercantum dalam RPS, antara lain ceramah interaktif, Problem-Based Learning (PBL), praktikum terbimbing, serta diskusi dan studi kasus. Aktivitas pembelajaran mendorong mahasiswa untuk mempraktikkan coding di laboratorium dan menyelesaikan tugas individu maupun proyek kecil. Komponen asesmen sejalan dengan sistem penilaian RPS meliputi tugas individu, kuis, praktikum, UTS, tugas proyek, dan UAS.

\section{Petunjuk Penggunaan Buku Ajar}

\subsection{Untuk Mahasiswa}

Sebelum perkuliahan, bacalah Sub-CPMK di awal bab untuk memahami target pembelajaran yang harus dicapai. Pelajari materi pokok dengan seksama dan jalankan semua contoh kode C yang diberikan menggunakan compiler seperti GCC atau MinGW. Catat pertanyaan atau konsep yang belum dipahami untuk didiskusikan di kelas.

Selama perkuliahan, diskusikan konsep yang sulit dengan dosen dan teman secara aktif. Kerjakan aktivitas pembelajaran dan praktikum dengan sungguh-sungguh di laboratorium. Manfaatkan kesempatan untuk bertanya dan berpartisipasi dalam diskusi kelompok. Setelah perkuliahan, kerjakan latihan dan refleksi, lakukan asesmen mandiri, serta centang checklist kompetensi yang telah dikuasai. Kerjakan proyek mini untuk memperdalam pemahaman dan mengintegrasikan konsep yang dipelajari.

\subsection{Untuk Dosen}

Buku ini dapat digunakan sebagai bahan ajar utama untuk perkuliahan Algoritma dan Pemrograman semester 1. Buku ini menyediakan sumber latihan dan tugas yang selaras dengan CPMK serta panduan untuk merancang aktivitas pembelajaran. Dosen dapat memanfaatkan rubrik asesmen dalam buku untuk mengukur pencapaian CPMK mahasiswa secara konsisten.

\section{Konteks Kurikulum OBE}

\subsection{Apa itu Outcome-Based Education?}

\textbf{Outcome-Based Education (OBE)} adalah pendekatan pembelajaran yang berfokus pada pencapaian hasil (outcomes) yang terukur \cite{obe_wikipedia}. Dalam OBE, proses pembelajaran dirancang secara sistematis untuk memastikan mahasiswa mencapai kompetensi yang telah ditetapkan. Prinsip utama OBE meliputi clarity of focus, designing down dari outcomes yang diinginkan, high expectations, dan expanded opportunity bagi semua mahasiswa.

\subsection{Implementasi OBE dalam Buku Ini}

Buku ini mengimplementasikan OBE melalui Sub-CPMK eksplisit di setiap bab, materi yang disusun dari dasar ke lanjut secara sistematis, serta aktivitas beragam berupa latihan, studi kasus, dan praktikum. Asesmen terukur dengan rubrik penilaian yang jelas disertai checklist untuk self-assessment memungkinkan mahasiswa memantau pencapaian kompetensi secara berkelanjutan. Implementasi tersebut dirancang agar pembelajaran di tingkat mikro (per bab) mendukung pencapaian kompetensi di tingkat makro (lulusan program studi).

\section{Peta Konsep Algoritma dan Pemrograman}

Mata kuliah Algoritma dan Pemrograman mencakup 17 bab yang saling terkait secara berurutan. Tabel berikut memetakan bab buku ajar ke 16 pertemuan dalam RPS:

\begin{table}[htbp]
\centering
\small
\begin{tabular}{|c|>{\raggedright\arraybackslash}p{6.5cm}|>{\raggedright\arraybackslash}p{3.5cm}|}
\hline
\textbf{Pert.} & \textbf{Materi Silabus} & \textbf{Bab Buku Ajar} \\
\hline
1 & Pengenalan algoritma, flowchart, pseudocode & Bab II, III \\
\hline
2 & Variabel, tipe data, I/O, operator & Bab V, VI \\
\hline
3 & Percabangan if-else & Bab VII \\
\hline
4 & Percabangan switch-case & Bab VIII \\
\hline
5 & Perulangan for, while, do-while & Bab IX \\
\hline
6 & Perulangan bersarang & Bab X \\
\hline
7 & Praktikum 1 (dasar pemrograman) & Aktivitas Bab III--VII \\
\hline
8 & Fungsi dan prosedur & Bab XI \\
\hline
9 & UTS & -- \\
\hline
10 & Array satu dimensi & Bab XII \\
\hline
11 & Array dua dimensi, matriks & Bab XIII \\
\hline
12 & String dan manipulasi teks & Bab XIV \\
\hline
13 & Struct/record, pengurutan dan pencarian & Bab XV, XVI \\
\hline
14 & Algoritma pengurutan dan pencarian lanjutan & Bab XVI \\
\hline
15 & Integrasi, proyek, review materi & Bab XVII \\
\hline
16 & UAS & Bab XVII \\
\hline
\end{tabular}
\caption{Pemetaan Bab Buku Ajar ke 16 Pertemuan RPS}
\end{table}

Bab II menyajikan landasan teori dan konsep dasar algoritma serta pengantar bahasa C. Bab III dan IV membahas pengenalan algoritma, flowchart, dan pseudocode sebagai fondasi merancang solusi. Bab V dan VI membahas variabel, tipe data, I/O, serta operator dalam bahasa C.

Bab VII dan VIII membahas struktur percabangan (if-else dan switch-case) untuk pengambilan keputusan dalam program. Bab IX dan X membahas struktur perulangan (for, while, do-while) serta perulangan bersarang. Bab XI membahas fungsi dan prosedur untuk modularisasi program. Bab XII, XIII, dan XIV membahas array satu dimensi, array dua dimensi, serta string dalam C.

Bab XV membahas struct dan tipe data bentukan untuk merepresentasikan data kompleks. Bab XVI membahas algoritma pengurutan dan pencarian sebagai penerapan konsep yang telah dipelajari. Bab XVII berisi evaluasi, refleksi, dan integrasi kompetensi untuk mengukur pencapaian pembelajaran secara menyeluruh. Alur pembelajaran dirancang secara progresif dari konsep dasar hingga penerapan algoritma pada masalah nyata.


% ============================================================
% Rangkuman Bab
% ============================================================
\begin{rangkuman}
Bab ini memperkenalkan tujuan buku ajar, keterkaitan dengan RPS berbasis OBE, petunjuk penggunaan, dan konteks kurikulum OBE. Pemahaman yang baik tentang struktur dan pendekatan buku ini akan membantu Anda memaksimalkan pembelajaran Algoritma dan Pemrograman.

\textbf{Poin Kunci:}
\begin{itemize}
  \item Buku ini dirancang dengan pendekatan OBE yang fokus pada pencapaian kompetensi terukur
  \item Setiap bab dipetakan ke Sub-CPMK yang berkontribusi pada CPL
  \item Gunakan komponen OBE (Sub-CPMK, aktivitas, latihan, asesmen, checklist) secara optimal
  \item Pembelajaran Algoritma dan Pemrograman tersusun sistematis dari flowchart hingga algoritma pengurutan dan pencarian
\end{itemize}
\end{rangkuman}

\ifSubfilesClassLoaded{
  \renewcommand{\bibname}{Daftar Pustaka}
  \bibliographystyle{plain}
  \bibliography{../references}
}{}
\end{document}
