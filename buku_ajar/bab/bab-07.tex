\documentclass[../main.tex]{subfiles}
\ifSubfilesClassLoaded{\setcounter{chapter}{6}}{}
\begin{document}

\chapter{Struktur Percabangan (Selection)}

\begin{subcpmk}
  \item Sub-CPMK 2.1: Merancang algoritma dengan struktur seleksi yang tepat (if, if-else, nested if).
  \item Sub-CPMK 3.1: Mengimplementasikan struktur percabangan dalam bahasa C untuk menyelesaikan masalah pengambilan keputusan.
\end{subcpmk}

\noindent\textbf{Materi Pokok:} 
\begin{itemize}
    \item Konsep Percabangan: Single Selection, Two-Way Selection, Multi-Way Selection
    \item Struktur \code{if} dan \code{if-else}
    \item Struktur \code{if-else if} (Ladder)
    \item Percabangan Bersarang (\textit{Nested If})
    \item Studi Kasus dan Kesalahan Umum (\textit{Common Pitfalls})
\end{itemize}

\noindent\textit{Referensi: \cite{ref7, control_flow_cppref, ref2}}

\section{Konsep Percabangan dan Struktur Dasar}

Percabangan (branching) atau seleksi adalah salah satu elemen dasar algoritma yang memungkinkan program untuk melakukan tindakan yang berbeda berdasarkan kondisi tertentu. Tanpa percabangan, program hanya akan berjalan lurus (\textit{sequential}) dari baris pertama hingga terakhir.

\subsection{Alur Logika Seleksi}
Dalam flowchart, keputusan digambarkan dengan simbol \textit{Diamond} (Belah Ketupat). Simbol ini memiliki satu input aliran dan minimal dua output aliran (biasanya \textbf{True/Yes} dan \textbf{False/No}) \cite{flowchart_wikipedia}.

\begin{figure}[h]
\centering
\begin{tikzpicture}[node distance=1.5cm]
  \node[flowstart] (start) {Mulai};
  \node[flowdecision, below=of start] (cond) {Kondisi Benar?};
  \node[flowprocess, below=of cond] (stepA) {Langkah A};
  \node[flowprocess, below=2.5cm of cond] (stepB) {Lanjut ke Langkah B};

  \draw[arrow] (start) -- (cond);
  \draw[arrow] (cond) -- node[left] {Ya} (stepA);
  
  % Jalur Tidak (False) - Mengelilingi blok True
  \draw[arrow] (cond.east) -- ++(1.5,0) |- (stepB.east) node[near start, above] {Tidak};
  
  \draw[arrow] (stepA) -- (stepB);
\end{tikzpicture}
\caption{Flowchart Logika Single Selection (if)}
\label{fig:flowchart_if}
\end{figure}

\subsection{Statement \code{if} (Single Selection)}
Struktur \code{if} digunakan untuk menjalankan blok kode hanya jika kondisi bernilai \textbf{True} (non-zero). Jika kondisi \textbf{False} (0), blok kode dilewati.

\textbf{Sintaks:}
\begin{lstlisting}[language=C]
if (kondisi) {
    // Pernyataan yang dijalankan jika kondisi True
    pernyataan;
}
\end{lstlisting}

\textbf{Contoh:} Program menghitung nilai mutlak.
\begin{lstlisting}[language=C]
int angka = -5;
if (angka < 0) {
    angka = -angka; // Mengubah negatif menjadi positif
}
printf("Nilai mutlak: %d", angka); // Output: 5
\end{lstlisting}

\subsection{Statement \code{if-else} (Two-Way Selection)}
Struktur ini memberikan alternatif. Jika kondisi True, jalankan Blok A. Jika False, jalankan Blok B. Salah satu blok \textbf{pasti} dijalankan.

\textbf{Sintaks:}
\begin{lstlisting}[language=C]
if (kondisi) {
    // Blok A (Jika True)
} else {
    // Blok B (Jika False)
}
\end{lstlisting}

\textbf{Contoh:} Menentukan kelulusan.
\begin{lstlisting}[language=C]
if (nilai >= 60) {
    printf("Selamat, Anda Lulus!\n");
} else {
    printf("Maaf, Anda Harus Mengulang.\n");
}
\end{lstlisting}

\subsection{Blok Kode dan Scope}
Dalam C, blok kode ditandai dengan kurung kurawal \code{\{ \}}. Meskipun C mengizinkan penghilangan kurung kurawal jika blok hanya terdiri dari satu baris pernyataan, sangat disarankan untuk \textbf{selalu menggunakan kurung kurawal}.

\begin{lstlisting}[language=C]
// TIDAK DISARANKAN (Rawan Error)
if (x > 0)
    printf("Positif");
    x++; // Baris ini SELALU dijalankan, tidak terpengaruh if!

// DISARANKAN (Aman)
if (x > 0) {
    printf("Positif");
    x++; // Baris ini hanya jalan jika x > 0
}
\end{lstlisting}

Percabangan memungkinkan program mengambil keputusan berdasarkan kondisi. Sintaks \code{if}: \code{if (kondisi) \{ pernyataan; \}}. Jika kondisi benar (non-nol), blok pernyataan dieksekusi. Sintaks \code{if-else}: \code{if (kondisi) \{ ... \} else \{ ... \}}; jika kondisi benar blok if dieksekusi, jika salah blok else dieksekusi. Blok dapat berisi satu atau lebih pernyataan; untuk satu pernyataan kurung kurawal bisa dihilangkan tapi tidak disarankan untuk kejelasan.

\section{Percabangan Bertingkat dan Bersarang}

Seringkali keputusan yang diambil tidak cukup hanya ya/tidak, atau memiliki kondisi prasyarat. Untuk itu kita menggunakan percabangan bertingkat atau bersarang.

\subsection{Struktur \code{if-else if} (Multi-Way Selection)}
Digunakan ketika ada lebih dari dua kemungkinan kondisi yang harus diuji secara berurutan. Ini sering disebut sebagai \"The if-else-if Ladder\".

\textbf{Prinsip Kerja:}
\begin{enumerate}
    \item Kondisi dievaluasi dari atas ke bawah.
    \item Begitu ditemukan kondisi yang \textbf{True}, blok kodenya dieksekusi, dan sisa struktur dilewati (diabaikan).
    \item Bagian \code{else} terakhir bersifat opsional, berfungsi sebagai \"block catch-all\" jika tidak ada kondisi yang terpenuhi.
\end{enumerate}

\begin{figure}[h]
\centering
\begin{tikzpicture}[node distance=1.2cm]
  \node[flowstart] (start) {Mulai};
  \node[flowdecision, below=of start] (cond1) {Kondisi 1?};
  \node[flowprocess, right=of cond1] (aksi1) {Aksi 1};
  
  \node[flowdecision, below=of cond1] (cond2) {Kondisi 2?};
  \node[flowprocess, right=of cond2] (aksi2) {Aksi 2};
  
  \node[flowprocess, below=of cond2] (aksiElse) {Aksi Else};
  \node[flowstart, below=1cm of aksiElse] (end) {Selesai};

  \draw[arrow] (start) -- (cond1);
  \draw[arrow] (cond1) -- node[above] {Ya} (aksi1);
  \draw[arrow] (cond1) -- node[left] {Tidak} (cond2);
  
  \draw[arrow] (cond2) -- node[above] {Ya} (aksi2);
  \draw[arrow] (cond2) -- node[left] {Tidak} (aksiElse);
  
  % Connecting paths to End
  \draw[arrow] (aksi1) |- (end);
  \draw[arrow] (aksi2) |- (end);
  \draw[arrow] (aksiElse) -- (end);
\end{tikzpicture}
\caption{Flowchart Bertingkat (Ladder If)}
\label{fig:ladder_if}
\end{figure}

\textbf{Contoh:} Konversi Nilai Angka ke Huruf.
\begin{lstlisting}[language=C]
if (nilai >= 85) {
    printf("Grade A");
} else if (nilai >= 70) {
    printf("Grade B");
} else if (nilai >= 55) {
    printf("Grade C");
} else if (nilai >= 40) {
    printf("Grade D");
} else {
    printf("Grade E");
}
\end{lstlisting}

\begin{alertbox}{Pentingnya Urutan}
Pada struktur tangga (\textit{ladder}), urutan kondisi sangat krusial. Jika kita membalik urutannya:
\code{if (nilai >= 40) ... else if (nilai >= 85) ...}
Maka input nilai 90 akan masuk ke blok pertama (\code{>= 40} bernilai True), dan mencetak Grade D, yang mana salah. \textbf{Pastikan urutan kondisi logis (misal dari terbesar ke terkecil).}
\end{alertbox}

\subsection{Percabangan Bersarang (\textit{Nested if})}
Kita bisa menempatkan struktur \code{if} di dalam blok \code{if} lainnya. Ini digunakan untuk logika berlapis atau prasyarat.

\textbf{Sintaks:}
\begin{lstlisting}[language=C]
if (kondisi1) {
    // Dieksekusi jika kondisi1 True
    if (kondisi2) {
        // Dieksekusi jika kondisi1 True DAN kondisi2 True
    } else {
        // Dieksekusi jika kondisi1 True DAN kondisi2 False
    }
} else {
    // Dieksekusi jika kondisi1 False
}
\end{lstlisting}

\textbf{Contoh:} Logika Login Sederhana.
\begin{lstlisting}[language=C]
if (username_valid) {
    if (password_valid) {
        printf("Login Berhasil!");
        if (is_admin) {
            printf("Selamat Datang Admin.");
        }
    } else {
        printf("Password Salah!");
    }
} else {
    printf("Username tidak ditemukan.");
}
\end{lstlisting}

\subsubsection{Tips Mengelola Nesting}
Nesting yang terlalu dalam (\textit{Deep Nesting}) membuat kode sulit dibaca dan dipahami (\textit{Spaghetti Code}).
\begin{itemize}
    \item Gunakan operator logika (\code{\&\&}) untuk menggabungkan kondisi jika memungkinkan.
    \item Gunakan \"Early Return\" atau \"Guard Clause\" (akan dibahas di bab Fungsi).
\end{itemize}

Untuk kondisi bertingkat gunakan \code{if-else if-else}: \code{if (k1) {...} else if (k2) {...} else {...}}. Kondisi dievaluasi berurutan; blok pertama yang memenuhi kondisi akan dieksekusi. Percabangan bersarang: if di dalam if. Indentasi yang baik penting untuk keterbacaan. Contoh: menentukan grade nilai (A, B, C, D, E) berdasarkan rentang skor menggunakan if-else if.

\section{Kesalahan Umum dan Praktik Terbaik}

Dalam menggunakan struktur percabangan, programmer pemula seringkali terjebak dalam beberapa kesalahan umum yang dapat menyebabkan \textit{logic error} yang sulit dilacak.

\subsection{Masalah \textit{Dangling Else}}
Masalah ini muncul ketika ada `if` bersarang namun jumlah `else` lebih sedikit daripada `if`, dan tidak menggunakan kurung kurawal. Compiler C akan memasangkan `else` dengan `if` \textbf{terdekat sebelumnya}.

\textbf{Kode Membingungkan (Tanpa Kurung Kurawal):}
\begin{lstlisting}[language=C]
if (a > 0)
    if (b > 0)
        printf("A dan B positif");
else
    printf("???"); // Else ini milik siapa?
\end{lstlisting}

Secara indentasi, seolah-olah `else` milik `if (a > 0)`. Namun bagi Compiler C, `else` tersebut milik `if (b > 0)`. Jika `a = -5`, tidak ada output yang muncul (padahal mungkin kita mengharapkan blok else jalan).

\textbf{Solusi:} Selalu gunakan kurung kurawal!
\begin{lstlisting}[language=C]
if (a > 0) {
    if (b > 0) {
        printf("A dan B positif");
    }
} else {
    printf("A negatif atau nol");
}
\end{lstlisting}

\subsection{Assignment di dalam Kondisi}
Kesalahan menggunakan `=` (penugasan) alih-alih `==` (perbandingan).

\begin{lstlisting}[language=C]
// SALAH - Bug Fatal
if (skor = 100) { // Nilai 100 di-assign ke skor. 100 != 0, maka dianggap True.
    printf("Sempurna!"); // Selalu tercetak, berapapun skor awalnya.
}

// BENAR
if (skor == 100) {
    printf("Sempurna!");
}
\end{lstlisting}

\begin{tip}
Beberapa programmer menggunakan gaya Yoda Condition: \code{if (100 == skor)} untuk mencegah error ini. Jika tidak sengaja menulis \code{if (100 = skor)}, compiler akan error karena literal angka tidak bisa di-assign.
\end{tip}

\subsection{Titik Koma (Semicolon) yang Salah Tempat}
Menaruh titik koma tepat setelah `if` akan mengakhiri statement `if` tersebut tanpa menjalankan blok apa-apa.

\begin{lstlisting}[language=C]
// SALAH
if (nilai > 60); // Titik koma ini mengakhiri if!
{
    printf("Lulus"); // Blok ini jadi tidak terikat if, SELALU dijalankan.
}
\end{lstlisting}

\subsection{Membandingkan Floating Point}
Seperti dibahas di Bab 6, hindari menggunakan `==` untuk `float` atau `double`.
\begin{lstlisting}[language=C]
float x = 1.0 / 3.0;
if (x * 3.0 == 1.0) // Mungkin False karena presisi
\end{lstlisting}
Gunakan toleransi epsilon \code{fabs(a - b) < 0.00001}.

\section{Studi Kasus Percabangan Kompleks}

Bagian ini menyajikan studi kasus implementasi logika bisnis yang lebih kompleks menggunakan gabungan operator logika dan struktur percabangan.

\subsection{Kasus 1: Menentukan Tahun Kabisat}
Dulu kita diajarkan \"Tahun kabisat adalah tahun yang habis dibagi 4\". Namun aturan Gregorian Calendar sebenarnya:
\begin{enumerate}
    \item Jika tahun habis dibagi 400, maka \textbf{Kabisat}.
    \item Jika tidak habis dibagi 400 tetapi habis dibagi 100, maka \textbf{Bukan Kabisat}.
    \item Jika tidak habis dibagi 100 tetapi habis dibagi 4, maka \textbf{Kabisat}.
    \item Sisanya \textbf{Bukan Kabisat}.
\end{enumerate}

\textbf{Implementasi Nested If:}
\begin{lstlisting}[language=C]
if (tahun % 400 == 0) {
    printf("Kabisat");
} else if (tahun % 100 == 0) {
    printf("Bukan Kabisat");
} else if (tahun % 4 == 0) {
    printf("Kabisat");
} else {
    printf("Bukan Kabisat");
}
\end{lstlisting}

\textbf{Implementasi Single Logic (Efisiensi):}
\begin{lstlisting}[language=C]
// Logika: (Habis 400) ATAU (Habis 4 DAN Tidak Habis 100)
if ((tahun % 400 == 0) || ((tahun % 4 == 0) && (tahun % 100 != 0))) {
    printf("Kabisat");
} else {
    printf("Bukan Kabisat");
}
\end{lstlisting}

\begin{figure}[h]
\centering
\begin{tikzpicture}[node distance=1.5cm]
  \node[flowstart] (start) {Mulai};
  \node[flowio, below=of start] (input) {Input Tahun};
  
  % Logic flow for leap year
  \node[flowdecision, below=of input] (d400) { \% 400 == 0?};
  \node[flowdecision, below=1cm of d400] (d100) { \% 100 == 0?};
  \node[flowdecision, below=1cm of d100] (d4) { \% 4 == 0?};
  
  \node[flowprocess, right=2cm of d400] (kabisat) {Cetak "Kabisat"};
  \node[flowprocess, right=2cm of d4] (bukan) {Cetak "Bukan"};
  
  \node[flowstart, below=of bukan] (end) {Selesai};

  % Edges
  \draw[arrow] (start) -- (input);
  \draw[arrow] (input) -- (d400);
  
  % 400 check
  \draw[arrow] (d400) -- node[above] {Ya} (kabisat);
  \draw[arrow] (d400) -- node[right] {Tidak} (d100);
  
  % 100 check
  % Manually routing to avoid crossing labels bad
  \draw[arrow] (d100) -| node[near start, above] {Ya} (bukan); 
  \draw[arrow] (d100) -- node[right] {Tidak} (d4);
  
  % 4 check
  \draw[arrow] (d4) -| node[near start, below] {Ya} (kabisat);
  \draw[arrow] (d4) -| node[near start, above] {Tidak} (bukan);

  % Connecting to End
  \draw[arrow] (kabisat) |- (end);
  \draw[arrow] (bukan) -- (end);

\end{tikzpicture}
\caption{Flowchart Logika Tahun Kabisat}
\label{fig:flowchart_kabisat}
\end{figure}

\subsection{Kasus 2: Validasi Segitiga}
Diberikan tiga panjang sisi $a, b, c$. Tentukan apakah bisa membentuk segitiga, dan jika ya, jenisnya.
Syarat Segitiga Valid (Triangle Inequality): $a+b>c$ DAN $a+c>b$ DAN $b+c>a$.

\begin{lstlisting}[language=C]
if (a + b > c && a + c > b && b + c > a) {
    // Segitiga Valid
    if (a == b && b == c) {
        printf("Segitiga Sama Sisi");
    } else if (a == b || a == c || b == c) {
        printf("Segitiga Sama Kaki");
    } else {
        // Cek Siku-Siku (Pythagoras)
        // Anggap c sisi terpanjang (perlu sorting idealnya)
        if (a*a + b*b == c*c || a*a + c*c == b*b || b*b + c*c == a*a)
             printf("Segitiga Siku-Siku");
        else
             printf("Segitiga Sembarang");
    }
} else {
    printf("Bukan Segitiga (Sisi tidak valid)");
}
\end{lstlisting}

\subsection{Kasus 3: Menu Program Sederhana}
Sebelum mengenal `switch-case`, kita bisa membuat menu dengan `if-else if`.

\begin{lstlisting}[language=C]
printf("Menu:\n1. Nasi Goreng\n2. Mie Goreng\n3. Jus Jeruk\n");
printf("Plihan Anda: ");
scanf("%d", &pilihan);

if (pilihan == 1) {
    harga = 15000;
    printf("Anda memesan Nasi Goreng. Harga: %d", harga);
} else if (pilihan == 2) {
    harga = 12000;
    printf("Anda memesan Mie Goreng. Harga: %d", harga);
} else if (pilihan == 3) {
    harga = 5000;
    printf("Anda memesan Jus Jeruk. Harga: %d", harga);
} else {
    printf("Menu tidak tersedia!");
}
\end{lstlisting}


% ============================================================
% AKTIVITAS PEMBELAJARAN
% ============================================================
\begin{aktivitas}
  \item \textbf{Analisis Logika}: Diberikan flowchart "Menentukan Bilangan Terbesar dari 3 Angka", tuliskan pseudocode dan implementasinya dalam C.
  \item \textbf{Debugging}: Cari kesalahan pada kode berikut dan perbaiki:
  \begin{lstlisting}[language=C]
if (nilai >= 70);
    printf("Lulus\n");
else
    printf("Tidak Lulus\n");
  \end{lstlisting}
  \item \textbf{Refactoring}: Ubah serangkaian \code{if} tunggal menjadi struktur \code{if-else if} yang lebih efisien untuk menentukan kategori usia (Anak, Remaja, Dewasa, Lansia).
\end{aktivitas}

% ============================================================
% LATIHAN DAN REFLEKSI
% ============================================================
\begin{latihan}
  \item Jelaskan perbedaan mendasar antara \textit{Two-Way Selection} (\code{if-else}) dan \textit{Multi-Way Selection} (\code{if-else if})!
  \item Apa itu \textit{Dangling Else}? Bagaimana cara menghindarinya?
  \item Buatlah tabel kebenaran untuk logika tahun kabisat (Kabisat jika: habis dibagi 400, ATAU (habis dibagi 4 TAPI tidak habis dibagi 100)).
  \item \textbf{Refleksi}: Mengapa indentasi (penulisan kode yang menjorok) sangat penting dalam \textit{Nested If}, meskipun compiler C tidak mempedulikannya?
\end{latihan}

% ============================================================
% ASESMEN
% ============================================================
\begin{asesmen}
\textbf{Instrumen untuk Sub-CPMK 2.1, 3.1}

\noindent\textbf{Soal 1 (Nested If - Kategori Nilai):}
Buat program yang membaca Nilai Angka (0-100) dan Kehadiran (0-100\%). Mahasiswa dinyatakan LULUS jika Nilai Angka $\ge$ 60 DAN Kehadiran $\ge$ 80\%.
\begin{itemize}
    \item Jika Lulus, tentukan Grade: A ($\ge$ 85), B ($\ge$ 70), C (sisanya).
    \item Jika Tidak Lulus, berikan alasannya: "Nilai Kurang", "Kehadiran Kurang", atau "Keduanya Kurang".
\end{itemize}

\noindent\textbf{Soal 2 (Validasi Segitiga):}
Baca 3 buah bilangan bulat positif. Tentukan apakah ketiga bilangan tersebut bisa membentuk segitiga. Jika ya, tentukan apakah segitiga Siku-siku (gunakan Pythagoras $a^2 + b^2 = c^2$), Sama Kaki, atau Sama Sisi.

\noindent\textbf{Soal 3 (Analisis Kompleksitas):}
Diberikan 3 kondisi independen. Manakah yang lebih efisien: menggunakan 3 buah \code{if} terpisah atau menggunakan \code{if-else if}? Jelaskan alasannya.
\end{asesmen}

% ============================================================
% CHECKLIST KOMPETENSI
% ============================================================
\begin{checklist}
  \item Saya memahami konsep blok kode (\code{\{\}})
  \item Saya dapat menggunakan struktur \code{if} dan \code{if-else} dengan benar
  \item Saya dapat merancang logika bertingkat menggunakan \code{if-else if}
  \item Saya mampu mengelola kompleksitas pada \textit{nested if}
  \item Saya mengetahui bahaya \textit{dangling else} dan assignment dalam kondisi
\end{checklist}

% ============================================================
% RANGKUMAN
% ============================================================
\begin{rangkuman}
Bab ini membahas struktur kontrol seleksi yang memungkinkan program \"berpikir\" dan mengambil keputusan:
\begin{itemize}
    \item \code{if}: Seleksi tunggal, hanya dijalankan jika kondisi True.
    \item \code{if-else}: Seleksi dua arah, memilih satu dari dua blok kode yang harus dijalankan.
    \item \code{if-else if}: Seleksi banyak arah (bertingkat), memeriksa kondisi secara berurutan sampai ada yang memenuhi.
    \item \code{Nested if}: Struktur percabangan di dalam percabangan lain untuk logika yang lebih spesifik/kompleks.
    \item \textbf{Best Practice}: Selalu gunakan kurung kurawal \code{\{\}} untuk setiap blok \code{if} atau \code{else} guna mencegah ambiguitas dan bug logis.
\end{itemize}
\end{rangkuman}

\ifSubfilesClassLoaded{
  \renewcommand{\bibname}{Daftar Pustaka}
  \bibliographystyle{plain}
  \bibliography{../references}
}{}
\end{document}
\ifSubfilesClassLoaded{\setcounter{chapter}{6}}{}
\begin{document}

\chapter{Struktur Percabangan (if-else)}

\begin{subcpmk}
  \item Sub-CPMK 2.1, 3.1: Merancang dan mengimplementasikan struktur seleksi if-else dalam C
\end{subcpmk}

\noindent\textbf{Materi Pokok:} Struktur \code{if}, \code{if-else}, \code{if-else if}; percabangan bersarang; penggabungan kondisi dengan operator logika \cite{ref7,control_flow_cppref}.

\section{Konsep Percabangan dan Struktur Dasar}

Percabangan (branching) atau seleksi adalah salah satu elemen dasar algoritma yang memungkinkan program untuk melakukan tindakan yang berbeda berdasarkan kondisi tertentu. Tanpa percabangan, program hanya akan berjalan lurus (\textit{sequential}) dari baris pertama hingga terakhir.

\subsection{Alur Logika Seleksi}
Dalam flowchart, keputusan digambarkan dengan simbol \textit{Diamond} (Belah Ketupat). Simbol ini memiliki satu input aliran dan minimal dua output aliran (biasanya \textbf{True/Yes} dan \textbf{False/No}) \cite{flowchart_wikipedia}.

\begin{figure}[h]
\centering
\begin{tikzpicture}[node distance=1.5cm]
  \node[flowstart] (start) {Mulai};
  \node[flowdecision, below=of start] (cond) {Kondisi Benar?};
  \node[flowprocess, below=of cond] (stepA) {Langkah A};
  \node[flowprocess, below=2.5cm of cond] (stepB) {Lanjut ke Langkah B};

  \draw[arrow] (start) -- (cond);
  \draw[arrow] (cond) -- node[left] {Ya} (stepA);
  
  % Jalur Tidak (False) - Mengelilingi blok True
  \draw[arrow] (cond.east) -- ++(1.5,0) |- (stepB.east) node[near start, above] {Tidak};
  
  \draw[arrow] (stepA) -- (stepB);
\end{tikzpicture}
\caption{Flowchart Logika Single Selection (if)}
\label{fig:flowchart_if}
\end{figure}

\subsection{Statement \code{if} (Single Selection)}
Struktur \code{if} digunakan untuk menjalankan blok kode hanya jika kondisi bernilai \textbf{True} (non-zero). Jika kondisi \textbf{False} (0), blok kode dilewati.

\textbf{Sintaks:}
\begin{lstlisting}[language=C]
if (kondisi) {
    // Pernyataan yang dijalankan jika kondisi True
    pernyataan;
}
\end{lstlisting}

\textbf{Contoh:} Program menghitung nilai mutlak.
\begin{lstlisting}[language=C]
int angka = -5;
if (angka < 0) {
    angka = -angka; // Mengubah negatif menjadi positif
}
printf("Nilai mutlak: %d", angka); // Output: 5
\end{lstlisting}

\subsection{Statement \code{if-else} (Two-Way Selection)}
Struktur ini memberikan alternatif. Jika kondisi True, jalankan Blok A. Jika False, jalankan Blok B. Salah satu blok \textbf{pasti} dijalankan.

\textbf{Sintaks:}
\begin{lstlisting}[language=C]
if (kondisi) {
    // Blok A (Jika True)
} else {
    // Blok B (Jika False)
}
\end{lstlisting}

\textbf{Contoh:} Menentukan kelulusan.
\begin{lstlisting}[language=C]
if (nilai >= 60) {
    printf("Selamat, Anda Lulus!\n");
} else {
    printf("Maaf, Anda Harus Mengulang.\n");
}
\end{lstlisting}

\subsection{Blok Kode dan Scope}
Dalam C, blok kode ditandai dengan kurung kurawal \code{\{ \}}. Meskipun C mengizinkan penghilangan kurung kurawal jika blok hanya terdiri dari satu baris pernyataan, sangat disarankan untuk \textbf{selalu menggunakan kurung kurawal}.

\begin{lstlisting}[language=C]
// TIDAK DISARANKAN (Rawan Error)
if (x > 0)
    printf("Positif");
    x++; // Baris ini SELALU dijalankan, tidak terpengaruh if!

// DISARANKAN (Aman)
if (x > 0) {
    printf("Positif");
    x++; // Baris ini hanya jalan jika x > 0
}
\end{lstlisting}

Percabangan memungkinkan program mengambil keputusan berdasarkan kondisi. Sintaks \code{if}: \code{if (kondisi) \{ pernyataan; \}}. Jika kondisi benar (non-nol), blok pernyataan dieksekusi. Sintaks \code{if-else}: \code{if (kondisi) \{ ... \} else \{ ... \}}; jika kondisi benar blok if dieksekusi, jika salah blok else dieksekusi. Blok dapat berisi satu atau lebih pernyataan; untuk satu pernyataan kurung kurawal bisa dihilangkan tapi tidak disarankan untuk kejelasan.

\section{Percabangan Bertingkat dan Bersarang}

Seringkali keputusan yang diambil tidak cukup hanya ya/tidak, atau memiliki kondisi prasyarat. Untuk itu kita menggunakan percabangan bertingkat atau bersarang.

\subsection{Struktur \code{if-else if} (Multi-Way Selection)}
Digunakan ketika ada lebih dari dua kemungkinan kondisi yang harus diuji secara berurutan. Ini sering disebut sebagai \"The if-else-if Ladder\".

\textbf{Prinsip Kerja:}
\begin{enumerate}
    \item Kondisi dievaluasi dari atas ke bawah.
    \item Begitu ditemukan kondisi yang \textbf{True}, blok kodenya dieksekusi, dan sisa struktur dilewati (diabaikan).
    \item Bagian \code{else} terakhir bersifat opsional, berfungsi sebagai \"block catch-all\" jika tidak ada kondisi yang terpenuhi.
\end{enumerate}

\begin{figure}[h]
\centering
\begin{tikzpicture}[node distance=1.2cm]
  \node[flowstart] (start) {Mulai};
  \node[flowdecision, below=of start] (cond1) {Kondisi 1?};
  \node[flowprocess, right=of cond1] (aksi1) {Aksi 1};
  
  \node[flowdecision, below=of cond1] (cond2) {Kondisi 2?};
  \node[flowprocess, right=of cond2] (aksi2) {Aksi 2};
  
  \node[flowprocess, below=of cond2] (aksiElse) {Aksi Else};
  \node[flowstart, below=1cm of aksiElse] (end) {Selesai};

  \draw[arrow] (start) -- (cond1);
  \draw[arrow] (cond1) -- node[above] {Ya} (aksi1);
  \draw[arrow] (cond1) -- node[left] {Tidak} (cond2);
  
  \draw[arrow] (cond2) -- node[above] {Ya} (aksi2);
  \draw[arrow] (cond2) -- node[left] {Tidak} (aksiElse);
  
  % Connecting paths to End
  \draw[arrow] (aksi1) |- (end);
  \draw[arrow] (aksi2) |- (end);
  \draw[arrow] (aksiElse) -- (end);
\end{tikzpicture}
\caption{Flowchart Bertingkat (Ladder If)}
\label{fig:ladder_if}
\end{figure}

\textbf{Contoh:} Konversi Nilai Angka ke Huruf.
\begin{lstlisting}[language=C]
if (nilai >= 85) {
    printf("Grade A");
} else if (nilai >= 70) {
    printf("Grade B");
} else if (nilai >= 55) {
    printf("Grade C");
} else if (nilai >= 40) {
    printf("Grade D");
} else {
    printf("Grade E");
}
\end{lstlisting}

\begin{alertbox}{Pentingnya Urutan}
Pada struktur tangga (\textit{ladder}), urutan kondisi sangat krusial. Jika kita membalik urutannya:
\code{if (nilai >= 40) ... else if (nilai >= 85) ...}
Maka input nilai 90 akan masuk ke blok pertama (\code{>= 40} bernilai True), dan mencetak Grade D, yang mana salah. \textbf{Pastikan urutan kondisi logis (misal dari terbesar ke terkecil).}
\end{alertbox}

\subsection{Percabangan Bersarang (\textit{Nested if})}
Kita bisa menempatkan struktur \code{if} di dalam blok \code{if} lainnya. Ini digunakan untuk logika berlapis atau prasyarat.

\textbf{Sintaks:}
\begin{lstlisting}[language=C]
if (kondisi1) {
    // Dieksekusi jika kondisi1 True
    if (kondisi2) {
        // Dieksekusi jika kondisi1 True DAN kondisi2 True
    } else {
        // Dieksekusi jika kondisi1 True DAN kondisi2 False
    }
} else {
    // Dieksekusi jika kondisi1 False
}
\end{lstlisting}

\textbf{Contoh:} Logika Login Sederhana.
\begin{lstlisting}[language=C]
if (username_valid) {
    if (password_valid) {
        printf("Login Berhasil!");
        if (is_admin) {
            printf("Selamat Datang Admin.");
        }
    } else {
        printf("Password Salah!");
    }
} else {
    printf("Username tidak ditemukan.");
}
\end{lstlisting}

\subsubsection{Tips Mengelola Nesting}
Nesting yang terlalu dalam (\textit{Deep Nesting}) membuat kode sulit dibaca dan dipahami (\textit{Spaghetti Code}).
\begin{itemize}
    \item Gunakan operator logika (\code{\&\&}) untuk menggabungkan kondisi jika memungkinkan.
    \item Gunakan \"Early Return\" atau \"Guard Clause\" (akan dibahas di bab Fungsi).
\end{itemize}

Untuk kondisi bertingkat gunakan \code{if-else if-else}: \code{if (k1) {...} else if (k2) {...} else {...}}. Kondisi dievaluasi berurutan; blok pertama yang memenuhi kondisi akan dieksekusi. Percabangan bersarang: if di dalam if. Indentasi yang baik penting untuk keterbacaan. Contoh: menentukan grade nilai (A, B, C, D, E) berdasarkan rentang skor menggunakan if-else if.


\begin{aktivitas}
  \item Buat program menentukan bilangan ganjil/genap.
  \item Buat program menentukan grade nilai (A-E) dari input skor.
\end{aktivitas}

\begin{latihan}
  \item Jelaskan perbedaan if, if-else, dan if-else if!
  \item Buat program mengecek tahun kabisat!
  \item \textbf{Refleksi}: Dalam kehidupan sehari-hari, berikan satu contoh keputusan yang mirip dengan \code{if-else if} bertingkat. Bagaimana Anda memastikan urutan kondisi sudah tepat?
\end{latihan}

\begin{asesmen}
\textbf{Instrumen untuk Sub-CPMK 2.1, 3.1}: Buat program C yang membaca nilai ujian (0--100) dan menampilkan grade (A/B/C/D/E) serta pesan lulus/tidak. Gunakan struktur if-else if. Buat juga flowchart untuk algoritma tersebut.
\end{asesmen}

\begin{checklist}
  \item Saya dapat menggunakan if, if-else, if-else if
  \item Saya dapat membuat percabangan bersarang
\end{checklist}

\begin{rangkuman}
Percabangan if-else memungkinkan pengambilan keputusan berdasarkan kondisi. Gunakan if-else if untuk kondisi bertingkat.
\end{rangkuman}

\ifSubfilesClassLoaded{
  \renewcommand{\bibname}{Daftar Pustaka}
  \bibliographystyle{plain}
  \bibliography{../references}
}{}
\end{document}
