\documentclass[../main.tex]{subfiles}
\ifSubfilesClassLoaded{\setcounter{chapter}{4}}{}
\begin{document}

\chapter{Variabel, Tipe Data, dan I/O dalam C}

\begin{subcpmk}
  \item Sub-CPMK 3.1 (dasar): Mendeklarasikan variabel, menggunakan tipe data, serta \code{printf} dan \code{scanf}
\end{subcpmk}

\noindent\textbf{Materi Pokok:} Konsep variabel, tipe data dasar, format I/O dengan \code{printf}/\code{scanf}, serta casting tipe data \cite{ref7}.

\section{Variabel dan Deklarasi dalam C}

Variabel adalah tempat penyimpanan data di memori yang memiliki nama dan tipe data \cite{ref7}. Dalam pemrograman, variabel berfungsi seperti wadah yang dapat menampung nilai-nilai yang berubah selama eksekusi program. Setiap variabel dalam C memiliki alamat memori unik yang digunakan untuk menyimpan dan mengakses data.

\subsection{Deklarasi Variabel}

Dalam C, variabel harus dideklarasikan sebelum digunakan dengan sintaks umum:
\begin{lstlisting}[language=C]
tipe_data nama_variabel;
\end{lstlisting}

\textbf{Contoh Deklarasi:}
\begin{lstlisting}[language=C]
int umur;                    // Variabel integer untuk umur
float tinggi;                // Variabel float untuk tinggi badan
char grade;                  // Variabel karakter untuk nilai
double gpa;                  // Variabel double untuk IPK
\end{lstlisting}

\subsection{Inisialisasi Variabel}

Variabel dapat diinisialisasi saat deklarasi atau diberi nilai kemudian:

\begin{lstlisting}[language=C]
// Inisialisasi saat deklarasi
int x = 10;
float pi = 3.14159;
char huruf = 'A';

// Inisialisasi terpisah
int y;
y = 20;  // Assignment setelah deklarasi
\end{lstlisting}

\subsection{Aturan Penamaan Variabel}

Nama variabel (identifier) harus mengikuti aturan berikut:

\begin{itemize}
  \item \textbf{Karakter yang diizinkan:} Huruf (a-z, A-Z), angka (0-9), dan underscore (\_)
  \item \textbf{Karakter pertama:} Harus huruf atau underscore, tidak boleh angka
  \item \textbf{Panjang maksimal:} Biasanya 31 karakter (standar C)
  \item \textbf{Case sensitive:} \code{nilai} berbeda dengan \code{Nilai}
  \item \textbf{Reserved words:} Tidak boleh menggunakan kata kunci C seperti \code{int}, \code{if}, \code{while}
\end{itemize}

\textbf{Contoh nama variabel yang valid:}
\begin{lstlisting}[language=C]
int umurMahasiswa;
float nilai_rata_rata;
char grade;
int _counter;
\end{lstlisting}

\textbf{Contoh nama variabel yang tidak valid:}
\begin{lstlisting}[language=C]
int 2umur;          // Tidak boleh diawali angka
float nilai-rata;    // Tidak boleh menggunakan strip (-)
char class;          // Tidak boleh reserved word
\end{lstlisting}

\subsection{Scope Variabel}

Scope menentukan di mana variabel dapat diakses dalam program:

\begin{table}[htbp]
\centering
\small
\begin{tabular}{|>{\raggedright\arraybackslash}p{3cm}|>{\raggedright\arraybackslash}p{7cm}|}
\hline
\textbf{Jenis Scope} & \textbf{Keterangan} \\
\hline
Local & Dideklarasikan di dalam fungsi, hanya dapat diakses di fungsi tersebut \\
\hline
Global & Dideklarasikan di luar semua fungsi, dapat diakses di seluruh program \\
\hline
Block & Dideklarasikan di dalam blok \{\}, hanya dapat diakses di blok tersebut \\
\hline
\end{tabular}
\caption{Jenis Scope Variabel dalam C}
\end{table}

\textbf{Contoh Scope:}
\begin{lstlisting}[language=C]
#include <stdio.h>

int globalVar = 100;  // Variabel global

int main() {
    int localVar = 50;  // Variabel local
    
    {
        int blockVar = 25;  // Variabel block
        printf("Block: %d\n", blockVar);  // Bisa diakses
    }
    
    printf("Local: %d\n", localVar);    // Bisa diakses
    printf("Global: %d\n", globalVar);   // Bisa diakses
    // printf("Block: %d\n", blockVar);  // Error: tidak bisa diakses
    
    return 0;
}
\end{lstlisting}

\section{Tipe Data Dasar}

Tipe data dasar C menentukan jenis data yang dapat disimpan dalam variabel, ukuran memori yang digunakan, dan rentang nilai yang mungkin. Pemilihan tipe data yang tepat sangat penting untuk efisiensi memori dan akurasi perhitungan.

\subsection{Tipe Data Integer}

Tipe data integer digunakan untuk menyimpan bilangan bulat (tanpa desimal):

\begin{table}[h]
\centering
\small
\begin{tabular}{|>{\raggedright\arraybackslash}p{2.5cm}|>{\raggedright\arraybackslash}p{2cm}|>{\raggedright\arraybackslash}p{4cm}|>{\raggedright\arraybackslash}p{4cm}|}
\hline
\textbf{Tipe Data} & \textbf{Ukuran} & \textbf{Rentang Nilai} & \textbf{Contoh Penggunaan} \\
\hline
\code{short int} & 2 byte & -32,768 hingga 32,767 & Usia, jumlah kecil \\
\hline
\code{unsigned short} & 2 byte & 0 hingga 65,535 & ID positif \\
\hline
\code{int} & 4 byte & -2,147,483,648 hingga 2,147,483,647 & Counter, umum \\
\hline
\code{unsigned int} & 4 byte & 0 hingga 4,294,967,295 & ID, populasi \\
\hline
\code{long int} & 4/8 byte & Sama seperti int & Kompatibilitas \\
\hline
\code{long long} & 8 byte & $\pm$9.2 x 10$^{18}$ & Bilangan sangat besar \\
\hline
\end{tabular}
\caption{Tipe Data Integer dalam C}
\end{table}

\subsection{Tipe Data Floating Point}

Tipe data floating point digunakan untuk menyimpan bilangan desimal:

\begin{table}[h]
\centering
\small
\begin{tabular}{|>{\raggedright\arraybackslash}p{2.5cm}|>{\raggedright\arraybackslash}p{2cm}|>{\raggedright\arraybackslash}p{4cm}|>{\raggedright\arraybackslash}p{4cm}|}
\hline
\textbf{Tipe Data} & \textbf{Ukuran} & \textbf{Presisi} & \textbf{Contoh Penggunaan} \\
\hline
\code{float} & 4 byte & 6-7 digit desimal & Suhu, persentase \\
\hline
\code{double} & 8 byte & 15-16 digit desimal & IPK, nilai akurat \\
\hline
\code{long double} & 10/16 byte & 19-21 digit desimal & Ilmiah, presisi tinggi \\
\hline
\end{tabular}
\caption{Tipe Data Floating Point dalam C}
\end{table}

\subsection{Tipe Data Karakter}

\begin{itemize}
  \item \textbf{\code{char}:} Menyimpan satu karakter (1 byte)
  \item \textbf{Rentang:} -128 hingga 127 (signed) atau 0 hingga 255 (unsigned)
  \item \textbf{Penggunaan:} Huruf, angka, simbol, atau nilai ASCII
\end{itemize}

\textbf{Contoh:}
\begin{lstlisting}[language=C]
char grade = 'A';        // Karakter A
char newline = '\n';      // Karakter newline
char nilai_ascii = 65;     // Sama dengan 'A'
\end{lstlisting}

\subsection{Modifier Tipe Data}

Modifier dapat digunakan untuk mengubah sifat tipe data dasar:

\begin{table}[h]
\centering
\small
\begin{tabular}{|>{\raggedright\arraybackslash}p{3cm}|>{\raggedright\arraybackslash}p{7cm}|}
\hline
\textbf{Modifier} & \textbf{Pengaruh} \\
\hline
\code{signed} & Default, dapat menyimpan nilai positif dan negatif \\
\hline
\code{unsigned} & Hanya nilai non-negatif, rentang positif dua kali lipat \\
\hline
\code{short} & Mengurangi ukuran memori (biasanya setengah dari int) \\
\hline
\code{long} & Meningkatkan ukuran memori (minimal sama dengan int) \\
\hline
\end{tabular}
\caption{Modifier Tipe Data dalam C}
\end{table}

\subsection{Memilih Tipe Data yang Tepat}

\textbf{Guidelines pemilihan tipe data:}

\begin{itemize}
  \item \textbf{Usia, jumlah item:} \code{int} atau \code{unsigned int}
  \item \textbf{Uang, nilai akademik:} \code{double} (hindari float untuk akurasi)
  \item \textbf{Suhu, persentase:} \code{float} (cukup untuk presisi biasa)
  \item \textbf{Huruf, status:} \code{char}
  \item \textbf{Bilangan sangat besar:} \code{long long}
  \item \textbf{ID yang selalu positif:} \code{unsigned int}
\end{itemize}

\textbf{Contoh implementasi:}
\begin{lstlisting}[language=C]
#include <stdio.h>

int main() {
    // Tipe data yang tepat untuk berbagai kebutuhan
    int umur = 20;                    // Umur mahasiswa
    double ipk = 3.75;                 // IPK (perlu presisi)
    float suhu = 36.5f;               // Suhu tubuh
    char grade = 'A';                  // Nilai huruf
    unsigned int npm = 2024001234;      // Nomor pokok mahasiswa
    
    printf("Umur: %d tahun\n", umur);
    printf("IPK: %.2f\n", ipk);
    printf("Suhu: %.1f derajat C\n", suhu);
    printf("Grade: %c\n", grade);
    printf("NPM: %u\n", npm);
    
    return 0;
}
\end{lstlisting}

\section{Input dan Output}

Fungsi \code{printf} digunakan untuk menampilkan output ke layar; format specifier seperti \code{\%d} (integer), \code{\%f} (float), \code{\%c} (char), \code{\%s} (string). Fungsi \code{scanf} digunakan untuk membaca input; memerlukan alamat variabel dengan operator \code{\&}. Contoh: \code{scanf("\%d", \&x);} membaca integer ke variabel \code{x}. Header \code{stdio.h} harus disertakan untuk kedua fungsi.

\section{Type Casting}

Type casting mengubah tipe data suatu nilai ke tipe lain. Cast eksplisit: \code{(tipe) ekspresi}, misalnya \code{(float) 5 / 2} menghasilkan 2.5. Cast implisit terjadi saat tipe berbeda dalam operasi; nilai tipe lebih kecil dinaikkan ke tipe lebih besar.

Perhatikan kehilangan presisi saat casting dari float ke int. Contoh: \code{int a = (int) 3.14;} menghasilkan \code{a = 3}.


\begin{aktivitas}
  \item Buat program yang membaca nama dan umur, lalu menampilkannya.
  \item Buat program konversi suhu Celsius ke Fahrenheit.
\end{aktivitas}

\begin{latihan}
  \item Jelaskan perbedaan \code{int}, \code{float}, dan \code{double}!
  \item Mengapa \code{scanf} memerlukan \code{\&} sebelum nama variabel?
  \item \textbf{Refleksi}: Bagaimana Anda memilih tipe data yang paling tepat untuk sebuah masalah?
\end{latihan}

\begin{asesmen}
\textbf{Instrumen untuk Sub-CPMK 3.1}: Buat program C yang membaca nama, usia, dan tinggi badan, lalu menampilkan kembali dengan format rapi menggunakan \code{printf}. Jelaskan alasan pemilihan tipe data dan format specifier.
\end{asesmen}

\begin{checklist}
  \item Saya dapat mendeklarasikan variabel dengan tipe data yang tepat
  \item Saya dapat menggunakan \code{printf} dan \code{scanf}
  \item Saya memahami type casting
\end{checklist}

\begin{rangkuman}
Variabel menyimpan data; tipe data dasar C meliputi int, float, double, char. \code{printf} dan \code{scanf} digunakan untuk I/O.
\end{rangkuman}

\ifSubfilesClassLoaded{
  \renewcommand{\bibname}{Daftar Pustaka}
  \bibliographystyle{plain}
  \bibliography{../references}
}{}
\end{document}
