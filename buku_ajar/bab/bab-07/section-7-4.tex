\section{Studi Kasus Percabangan Kompleks}

Bagian ini menyajikan studi kasus implementasi logika bisnis yang lebih kompleks menggunakan gabungan operator logika dan struktur percabangan.

\subsection{Kasus 1: Menentukan Tahun Kabisat}
Dulu kita diajarkan \"Tahun kabisat adalah tahun yang habis dibagi 4\". Namun aturan Gregorian Calendar sebenarnya:
\begin{enumerate}
    \item Jika tahun habis dibagi 400, maka \textbf{Kabisat}.
    \item Jika tidak habis dibagi 400 tetapi habis dibagi 100, maka \textbf{Bukan Kabisat}.
    \item Jika tidak habis dibagi 100 tetapi habis dibagi 4, maka \textbf{Kabisat}.
    \item Sisanya \textbf{Bukan Kabisat}.
\end{enumerate}

\textbf{Implementasi Nested If:}
\begin{lstlisting}[language=C]
if (tahun % 400 == 0) {
    printf("Kabisat");
} else if (tahun % 100 == 0) {
    printf("Bukan Kabisat");
} else if (tahun % 4 == 0) {
    printf("Kabisat");
} else {
    printf("Bukan Kabisat");
}
\end{lstlisting}

\textbf{Implementasi Single Logic (Efisiensi):}
\begin{lstlisting}[language=C]
// Logika: (Habis 400) ATAU (Habis 4 DAN Tidak Habis 100)
if ((tahun % 400 == 0) || ((tahun % 4 == 0) && (tahun % 100 != 0))) {
    printf("Kabisat");
} else {
    printf("Bukan Kabisat");
}
\end{lstlisting}

\begin{figure}[h]
\centering
\begin{tikzpicture}[node distance=1.5cm]
  \node[flowstart] (start) {Mulai};
  \node[flowio, below=of start] (input) {Input Tahun};
  
  % Logic flow for leap year
  \node[flowdecision, below=of input] (d400) { \% 400 == 0?};
  \node[flowdecision, below=1cm of d400] (d100) { \% 100 == 0?};
  \node[flowdecision, below=1cm of d100] (d4) { \% 4 == 0?};
  
  \node[flowprocess, right=2cm of d400] (kabisat) {Cetak "Kabisat"};
  \node[flowprocess, right=2cm of d4] (bukan) {Cetak "Bukan"};
  
  \node[flowstart, below=of bukan] (end) {Selesai};

  % Edges
  \draw[arrow] (start) -- (input);
  \draw[arrow] (input) -- (d400);
  
  % 400 check
  \draw[arrow] (d400) -- node[above] {Ya} (kabisat);
  \draw[arrow] (d400) -- node[right] {Tidak} (d100);
  
  % 100 check
  % Manually routing to avoid crossing labels bad
  \draw[arrow] (d100) -| node[near start, above] {Ya} (bukan); 
  \draw[arrow] (d100) -- node[right] {Tidak} (d4);
  
  % 4 check
  \draw[arrow] (d4) -| node[near start, below] {Ya} (kabisat);
  \draw[arrow] (d4) -| node[near start, above] {Tidak} (bukan);

  % Connecting to End
  \draw[arrow] (kabisat) |- (end);
  \draw[arrow] (bukan) -- (end);

\end{tikzpicture}
\caption{Flowchart Logika Tahun Kabisat}
\label{fig:flowchart_kabisat}
\end{figure}

\subsection{Kasus 2: Validasi Segitiga}
Diberikan tiga panjang sisi $a, b, c$. Tentukan apakah bisa membentuk segitiga, dan jika ya, jenisnya.
Syarat Segitiga Valid (Triangle Inequality): $a+b>c$ DAN $a+c>b$ DAN $b+c>a$.

\begin{lstlisting}[language=C]
if (a + b > c && a + c > b && b + c > a) {
    // Segitiga Valid
    if (a == b && b == c) {
        printf("Segitiga Sama Sisi");
    } else if (a == b || a == c || b == c) {
        printf("Segitiga Sama Kaki");
    } else {
        // Cek Siku-Siku (Pythagoras)
        // Anggap c sisi terpanjang (perlu sorting idealnya)
        if (a*a + b*b == c*c || a*a + c*c == b*b || b*b + c*c == a*a)
             printf("Segitiga Siku-Siku");
        else
             printf("Segitiga Sembarang");
    }
} else {
    printf("Bukan Segitiga (Sisi tidak valid)");
}
\end{lstlisting}

\subsection{Kasus 3: Menu Program Sederhana}
Sebelum mengenal `switch-case`, kita bisa membuat menu dengan `if-else if`.

\begin{lstlisting}[language=C]
printf("Menu:\n1. Nasi Goreng\n2. Mie Goreng\n3. Jus Jeruk\n");
printf("Plihan Anda: ");
scanf("%d", &pilihan);

if (pilihan == 1) {
    harga = 15000;
    printf("Anda memesan Nasi Goreng. Harga: %d", harga);
} else if (pilihan == 2) {
    harga = 12000;
    printf("Anda memesan Mie Goreng. Harga: %d", harga);
} else if (pilihan == 3) {
    harga = 5000;
    printf("Anda memesan Jus Jeruk. Harga: %d", harga);
} else {
    printf("Menu tidak tersedia!");
}
\end{lstlisting}
