\section{Kesalahan Umum dan Praktik Terbaik}

Dalam menggunakan struktur percabangan, programmer pemula seringkali terjebak dalam beberapa kesalahan umum yang dapat menyebabkan \textit{logic error} yang sulit dilacak.

\subsection{Masalah \textit{Dangling Else}}
Masalah ini muncul ketika ada `if` bersarang namun jumlah `else` lebih sedikit daripada `if`, dan tidak menggunakan kurung kurawal. Compiler C akan memasangkan `else` dengan `if` \textbf{terdekat sebelumnya}.

\textbf{Kode Membingungkan (Tanpa Kurung Kurawal):}
\begin{lstlisting}[language=C]
if (a > 0)
    if (b > 0)
        printf("A dan B positif");
else
    printf("???"); // Else ini milik siapa?
\end{lstlisting}

Secara indentasi, seolah-olah `else` milik `if (a > 0)`. Namun bagi Compiler C, `else` tersebut milik `if (b > 0)`. Jika `a = -5`, tidak ada output yang muncul (padahal mungkin kita mengharapkan blok else jalan).

\textbf{Solusi:} Selalu gunakan kurung kurawal!
\begin{lstlisting}[language=C]
if (a > 0) {
    if (b > 0) {
        printf("A dan B positif");
    }
} else {
    printf("A negatif atau nol");
}
\end{lstlisting}

\subsection{Assignment di dalam Kondisi}
Kesalahan menggunakan `=` (penugasan) alih-alih `==` (perbandingan).

\begin{lstlisting}[language=C]
// SALAH - Bug Fatal
if (skor = 100) { // Nilai 100 di-assign ke skor. 100 != 0, maka dianggap True.
    printf("Sempurna!"); // Selalu tercetak, berapapun skor awalnya.
}

// BENAR
if (skor == 100) {
    printf("Sempurna!");
}
\end{lstlisting}

\begin{tip}
Beberapa programmer menggunakan gaya Yoda Condition: \code{if (100 == skor)} untuk mencegah error ini. Jika tidak sengaja menulis \code{if (100 = skor)}, compiler akan error karena literal angka tidak bisa di-assign.
\end{tip}

\subsection{Titik Koma (Semicolon) yang Salah Tempat}
Menaruh titik koma tepat setelah `if` akan mengakhiri statement `if` tersebut tanpa menjalankan blok apa-apa.

\begin{lstlisting}[language=C]
// SALAH
if (nilai > 60); // Titik koma ini mengakhiri if!
{
    printf("Lulus"); // Blok ini jadi tidak terikat if, SELALU dijalankan.
}
\end{lstlisting}

\subsection{Membandingkan Floating Point}
Seperti dibahas di Bab 6, hindari menggunakan `==` untuk `float` atau `double`.
\begin{lstlisting}[language=C]
float x = 1.0 / 3.0;
if (x * 3.0 == 1.0) // Mungkin False karena presisi
\end{lstlisting}
Gunakan toleransi epsilon \code{fabs(a - b) < 0.00001}.
