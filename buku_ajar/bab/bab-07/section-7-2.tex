\section{Percabangan Bertingkat dan Bersarang}

Seringkali keputusan yang diambil tidak cukup hanya ya/tidak, atau memiliki kondisi prasyarat. Untuk itu kita menggunakan percabangan bertingkat atau bersarang.

\subsection{Struktur \code{if-else if} (Multi-Way Selection)}
Digunakan ketika ada lebih dari dua kemungkinan kondisi yang harus diuji secara berurutan. Ini sering disebut sebagai \"The if-else-if Ladder\".

\textbf{Prinsip Kerja:}
\begin{enumerate}
    \item Kondisi dievaluasi dari atas ke bawah.
    \item Begitu ditemukan kondisi yang \textbf{True}, blok kodenya dieksekusi, dan sisa struktur dilewati (diabaikan).
    \item Bagian \code{else} terakhir bersifat opsional, berfungsi sebagai \"block catch-all\" jika tidak ada kondisi yang terpenuhi.
\end{enumerate}

\begin{figure}[h]
\centering
\begin{tikzpicture}[node distance=1.2cm]
  \node[flowstart] (start) {Mulai};
  \node[flowdecision, below=of start] (cond1) {Kondisi 1?};
  \node[flowprocess, right=of cond1] (aksi1) {Aksi 1};
  
  \node[flowdecision, below=of cond1] (cond2) {Kondisi 2?};
  \node[flowprocess, right=of cond2] (aksi2) {Aksi 2};
  
  \node[flowprocess, below=of cond2] (aksiElse) {Aksi Else};
  \node[flowstart, below=1cm of aksiElse] (end) {Selesai};

  \draw[arrow] (start) -- (cond1);
  \draw[arrow] (cond1) -- node[above] {Ya} (aksi1);
  \draw[arrow] (cond1) -- node[left] {Tidak} (cond2);
  
  \draw[arrow] (cond2) -- node[above] {Ya} (aksi2);
  \draw[arrow] (cond2) -- node[left] {Tidak} (aksiElse);
  
  % Connecting paths to End
  \draw[arrow] (aksi1) |- (end);
  \draw[arrow] (aksi2) |- (end);
  \draw[arrow] (aksiElse) -- (end);
\end{tikzpicture}
\caption{Flowchart Bertingkat (Ladder If)}
\label{fig:ladder_if}
\end{figure}

\textbf{Contoh:} Konversi Nilai Angka ke Huruf.
\begin{lstlisting}[language=C]
if (nilai >= 85) {
    printf("Grade A");
} else if (nilai >= 70) {
    printf("Grade B");
} else if (nilai >= 55) {
    printf("Grade C");
} else if (nilai >= 40) {
    printf("Grade D");
} else {
    printf("Grade E");
}
\end{lstlisting}

\begin{alertbox}{Pentingnya Urutan}
Pada struktur tangga (\textit{ladder}), urutan kondisi sangat krusial. Jika kita membalik urutannya:
\code{if (nilai >= 40) ... else if (nilai >= 85) ...}
Maka input nilai 90 akan masuk ke blok pertama (\code{>= 40} bernilai True), dan mencetak Grade D, yang mana salah. \textbf{Pastikan urutan kondisi logis (misal dari terbesar ke terkecil).}
\end{alertbox}

\subsection{Percabangan Bersarang (\textit{Nested if})}
Kita bisa menempatkan struktur \code{if} di dalam blok \code{if} lainnya. Ini digunakan untuk logika berlapis atau prasyarat.

\textbf{Sintaks:}
\begin{lstlisting}[language=C]
if (kondisi1) {
    // Dieksekusi jika kondisi1 True
    if (kondisi2) {
        // Dieksekusi jika kondisi1 True DAN kondisi2 True
    } else {
        // Dieksekusi jika kondisi1 True DAN kondisi2 False
    }
} else {
    // Dieksekusi jika kondisi1 False
}
\end{lstlisting}

\textbf{Contoh:} Logika Login Sederhana.
\begin{lstlisting}[language=C]
if (username_valid) {
    if (password_valid) {
        printf("Login Berhasil!");
        if (is_admin) {
            printf("Selamat Datang Admin.");
        }
    } else {
        printf("Password Salah!");
    }
} else {
    printf("Username tidak ditemukan.");
}
\end{lstlisting}

\subsubsection{Tips Mengelola Nesting}
Nesting yang terlalu dalam (\textit{Deep Nesting}) membuat kode sulit dibaca dan dipahami (\textit{Spaghetti Code}).
\begin{itemize}
    \item Gunakan operator logika (\code{\&\&}) untuk menggabungkan kondisi jika memungkinkan.
    \item Gunakan \"Early Return\" atau \"Guard Clause\" (akan dibahas di bab Fungsi).
\end{itemize}

Untuk kondisi bertingkat gunakan \code{if-else if-else}: \code{if (k1) {...} else if (k2) {...} else {...}}. Kondisi dievaluasi berurutan; blok pertama yang memenuhi kondisi akan dieksekusi. Percabangan bersarang: if di dalam if. Indentasi yang baik penting untuk keterbacaan. Contoh: menentukan grade nilai (A, B, C, D, E) berdasarkan rentang skor menggunakan if-else if.
