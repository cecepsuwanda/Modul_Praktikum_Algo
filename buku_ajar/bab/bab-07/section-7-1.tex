\section{Konsep Percabangan dan Struktur Dasar}

Percabangan (branching) atau seleksi adalah salah satu elemen dasar algoritma yang memungkinkan program untuk melakukan tindakan yang berbeda berdasarkan kondisi tertentu. Tanpa percabangan, program hanya akan berjalan lurus (\textit{sequential}) dari baris pertama hingga terakhir.

\subsection{Alur Logika Seleksi}
Dalam flowchart, keputusan digambarkan dengan simbol \textit{Diamond} (Belah Ketupat). Simbol ini memiliki satu input aliran dan minimal dua output aliran (biasanya \textbf{True/Yes} dan \textbf{False/No}) \cite{flowchart_wikipedia}.

\begin{figure}[h]
\centering
\begin{tikzpicture}[node distance=1.5cm]
  \node[flowstart] (start) {Mulai};
  \node[flowdecision, below=of start] (cond) {Kondisi Benar?};
  \node[flowprocess, below=of cond] (stepA) {Langkah A};
  \node[flowprocess, below=2.5cm of cond] (stepB) {Lanjut ke Langkah B};

  \draw[arrow] (start) -- (cond);
  \draw[arrow] (cond) -- node[left] {Ya} (stepA);
  
  % Jalur Tidak (False) - Mengelilingi blok True
  \draw[arrow] (cond.east) -- ++(1.5,0) |- (stepB.east) node[near start, above] {Tidak};
  
  \draw[arrow] (stepA) -- (stepB);
\end{tikzpicture}
\caption{Flowchart Logika Single Selection (if)}
\label{fig:flowchart_if}
\end{figure}

\subsection{Statement \code{if} (Single Selection)}
Struktur \code{if} digunakan untuk menjalankan blok kode hanya jika kondisi bernilai \textbf{True} (non-zero). Jika kondisi \textbf{False} (0), blok kode dilewati.

\textbf{Sintaks:}
\begin{lstlisting}[language=C]
if (kondisi) {
    // Pernyataan yang dijalankan jika kondisi True
    pernyataan;
}
\end{lstlisting}

\textbf{Contoh:} Program menghitung nilai mutlak.
\begin{lstlisting}[language=C]
int angka = -5;
if (angka < 0) {
    angka = -angka; // Mengubah negatif menjadi positif
}
printf("Nilai mutlak: %d", angka); // Output: 5
\end{lstlisting}

\subsection{Statement \code{if-else} (Two-Way Selection)}
Struktur ini memberikan alternatif. Jika kondisi True, jalankan Blok A. Jika False, jalankan Blok B. Salah satu blok \textbf{pasti} dijalankan.

\textbf{Sintaks:}
\begin{lstlisting}[language=C]
if (kondisi) {
    // Blok A (Jika True)
} else {
    // Blok B (Jika False)
}
\end{lstlisting}

\textbf{Contoh:} Menentukan kelulusan.
\begin{lstlisting}[language=C]
if (nilai >= 60) {
    printf("Selamat, Anda Lulus!\n");
} else {
    printf("Maaf, Anda Harus Mengulang.\n");
}
\end{lstlisting}

\subsection{Blok Kode dan Scope}
Dalam C, blok kode ditandai dengan kurung kurawal \code{\{ \}}. Meskipun C mengizinkan penghilangan kurung kurawal jika blok hanya terdiri dari satu baris pernyataan, sangat disarankan untuk \textbf{selalu menggunakan kurung kurawal}.

\begin{lstlisting}[language=C]
// TIDAK DISARANKAN (Rawan Error)
if (x > 0)
    printf("Positif");
    x++; // Baris ini SELALU dijalankan, tidak terpengaruh if!

// DISARANKAN (Aman)
if (x > 0) {
    printf("Positif");
    x++; // Baris ini hanya jalan jika x > 0
}
\end{lstlisting}

Percabangan memungkinkan program mengambil keputusan berdasarkan kondisi. Sintaks \code{if}: \code{if (kondisi) \{ pernyataan; \}}. Jika kondisi benar (non-nol), blok pernyataan dieksekusi. Sintaks \code{if-else}: \code{if (kondisi) \{ ... \} else \{ ... \}}; jika kondisi benar blok if dieksekusi, jika salah blok else dieksekusi. Blok dapat berisi satu atau lebih pernyataan; untuk satu pernyataan kurung kurawal bisa dihilangkan tapi tidak disarankan untuk kejelasan.
