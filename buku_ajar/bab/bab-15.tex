\documentclass[../main.tex]{subfiles}
\ifSubfilesClassLoaded{\setcounter{chapter}{14}}{}
\begin{document}

\chapter{Struct dan Tipe Data Bentukan}

\begin{subcpmk}
  \item Sub-CPMK 4.2: Menggunakan struct/tipe data bentukan dan array of struct dalam program
\end{subcpmk}

\noindent\textbf{Materi Pokok:} Definisi \code{struct}, akses anggota, \code{typedef}; array of struct untuk kumpulan data terstruktur \cite{ref7,ref2}.

\section{Definisi Struct}

Struct (structure) adalah tipe data bentukan yang menggabungkan beberapa variabel dengan tipe berbeda dalam satu unit. Struct memungkinkan programmer merepresentasikan entitas dunia nyata (misalnya Mahasiswa dengan NIM, nama, IPK) dalam satu variabel terstruktur. Struct adalah salah satu fitur fundamental dalam C untuk mengorganisir data yang kompleks.

\subsection{Deklarasi Struct}

Sintaks dasar deklarasi struct:
\begin{lstlisting}[language=C]
struct NamaStruct {
    tipe_data field1;
    tipe_data field2;
    // ... field lainnya
};
\end{lstlisting}

\textbf{Contoh dasar - Struct Mahasiswa:}
\begin{lstlisting}[language=C]
struct Mahasiswa {
    int nim;
    char nama[50];
    float ipk;
    int umur;
};
\end{lstlisting}

\subsection{Definisi Variabel Struct}

Setelah struct dideklarasikan, kita dapat membuat variabel dari struct tersebut:

\begin{lstlisting}[language=C]
// Metode 1: Deklarasi terpisah
struct Mahasiswa mhs1;

// Metode 2: Deklarasi dengan inisialisasi
struct Mahasiswa mhs2 = {12345, "Budi Santoso", 3.75, 20};

// Metode 3: Menggunakan typedef (lebih praktis)
typedef struct {
    int nim;
    char nama[50];
    float ipk;
    int umur;
} Mahasiswa;

// Setelah typedef, deklarasi lebih sederhana
Mahasiswa mhs3;
\end{lstlisting}

\subsection{Akses Anggota Struct}

Anggota struct diakses menggunakan operator dot (.):

\begin{lstlisting}[language=C]
#include <stdio.h>
#include <string.h>

typedef struct {
    int nim;
    char nama[50];
    float ipk;
    int umur;
} Mahasiswa;

int main() {
    Mahasiswa mhs;
    
    // Mengisi nilai ke anggota struct
    mhs.nim = 2024001234;
    strcpy(mhs.nama, "Ahmad Fauzi");
    mhs.ipk = 3.85;
    mhs.umur = 21;
    
    // Mengakses dan menampilkan nilai
    printf("NIM: %d\n", mhs.nim);
    printf("Nama: %s\n", mhs.nama);
    printf("IPK: %.2f\n", mhs.ipk);
    printf("Umur: %d tahun\n", mhs.umur);
    
    return 0;
}
\end{lstlisting}

\subsection{Nested Struct}

Struct dapat berisi struct lain sebagai anggotanya:

\begin{lstlisting}[language=C]
#include <stdio.h>
#include <string.h>

// Struct untuk tanggal
typedef struct {
    int hari;
    int bulan;
    int tahun;
} Tanggal;

// Struct untuk alamat
typedef struct {
    char jalan[100];
    char kota[50];
    char provinsi[50];
    int kodePos;
} Alamat;

// Struct mahasiswa yang menggunakan struct lain
typedef struct {
    int nim;
    char nama[50];
    Tanggal tanggalLahir;
    Alamat alamat;
    float ipk;
} MahasiswaLengkap;

int main() {
    MahasiswaLengkap mhs;
    
    // Mengisi data mahasiswa
    mhs.nim = 2024001234;
    strcpy(mhs.nama, "Siti Nurhaliza");
    
    // Mengisi data tanggal lahir
    mhs.tanggalLahir.hari = 15;
    mhs.tanggalLahir.bulan = 8;
    mhs.tanggalLahir.tahun = 2002;
    
    // Mengisi data alamat
    strcpy(mhs.alamat.jalan, "Jl. Merdeka No. 123");
    strcpy(mhs.alamat.kota, "Jakarta");
    strcpy(mhs.alamat.provinsi, "DKI Jakarta");
    mhs.alamat.kodePos = 12345;
    
    mhs.ipk = 3.90;
    
    // Menampilkan data
    printf("Data Mahasiswa:\n");
    printf("NIM: %d\n", mhs.nim);
    printf("Nama: %s\n", mhs.nama);
    printf("Tanggal Lahir: %d/%d/%d\n", 
           mhs.tanggalLahir.hari, 
           mhs.tanggalLahir.bulan, 
           mhs.tanggalLahir.tahun);
    printf("Alamat: %s, %s, %s %d\n", 
           mhs.alamat.jalan, 
           mhs.alamat.kota, 
           mhs.alamat.provinsi, 
           mhs.alamat.kodePos);
    printf("IPK: %.2f\n", mhs.ipk);
    
    return 0;
}
\end{lstlisting}

\subsection{Array of Struct}

Struct dapat dibuat dalam bentuk array untuk mengelola banyak entitas:

\begin{lstlisting}[language=C]
#include <stdio.h>
#include <string.h>

typedef struct {
    int nim;
    char nama[50];
    float ipk;
} Mahasiswa;

int main() {
    // Array of struct
    Mahasiswa kelas[5];
    int i;
    
    // Mengisi data untuk 5 mahasiswa
    for (i = 0; i < 5; i++) {
        printf("Mahasiswa ke-%d:\n", i + 1);
        printf("NIM: ");
        scanf("%d", &kelas[i].nim);
        printf("Nama: ");
        scanf("%s", kelas[i].nama);
        printf("IPK: ");
        scanf("%f", &kelas[i].ipk);
        printf("\n");
    }
    
    // Menampilkan semua data
    printf("\nData Semua Mahasiswa:\n");
    for (i = 0; i < 5; i++) {
        printf("%d. %s - NIM: %d - IPK: %.2f\n", 
               i + 1, kelas[i].nama, kelas[i].nim, kelas[i].ipk);
    }
    
    return 0;
}
\end{lstlisting}

\subsection{Keuntungan Menggunakan Struct}

\begin{itemize}
  \item \textbf{Organisasi Data:} Mengelompokkan data terkait dalam satu unit
  \item \textbf{Readability:} Kode lebih mudah dibaca dan dimengerti
  \item \textbf{Maintainability:} Modifikasi data menjadi lebih terstruktur
  \item \textbf{Real-world Modeling:} Mudah memodelkan entitas dunia nyata
  \item \textbf{Parameter Passing:} Memudahkan passing data kompleks ke fungsi
\end{itemize}

Struct adalah fondasi untuk pemrograman berorientasi objek dan sangat penting dalam pengembangan aplikasi yang mengelola data kompleks.

\section{Array of Struct}

Array dapat menyimpan elemen bertipe struct. Deklarasi: \code{struct Mahasiswa mhs[100];}. Akses: \code{mhs[i].nim}, \code{mhs[i].nama}. Berguna untuk mengelola data banyak entitas (daftar mahasiswa, barang, dll). Input/output dengan loop.

Struct dapat diteruskan ke fungsi pass by value atau pass by reference dengan pointer. Typedef: \code{typedef struct Mahasiswa Mhs;} lalu \code{Mhs mhs[100];}.


\begin{aktivitas}
  \item Buat struct Buku (judul, pengarang, tahun), lalu buat array of struct.
  \item Buat program mengelola data mahasiswa dengan struct.
\end{aktivitas}

\begin{latihan}
  \item Jelaskan kegunaan struct dan kapan menggunakannya!
  \item Bagaimana akses anggota struct dalam array of struct?
  \item \textbf{Refleksi}: Kapan Anda memilih struct daripada beberapa array terpisah (misalnya array NIM, array nama)? Apa keuntungannya?
\end{latihan}

\begin{asesmen}
\textbf{Instrumen untuk Sub-CPMK 4.2}: Buat program C dengan struct \code{Mahasiswa} (NIM, nama, nilai). Simpan data 3 mahasiswa dalam array of struct. Tampilkan data dan hitung rata-rata nilai. Jelaskan cara akses \code{arr[i].nilai}.
\end{asesmen}

\begin{checklist}
  \item Saya dapat mendefinisikan struct
  \item Saya dapat menggunakan array of struct
\end{checklist}

\begin{rangkuman}
Struct menggabungkan variabel beda tipe. Array of struct untuk data banyak entitas.
\end{rangkuman}

\ifSubfilesClassLoaded{
  \renewcommand{\bibname}{Daftar Pustaka}
  \bibliographystyle{plain}
  \bibliography{../references}
}{}
\end{document}
