\section{Prioritas Operator dan Studi Kasus}

Dalam sebuah ekspresi kompleks yang memiliki banyak operator, urutan eksekusi ditentukan oleh \textbf{Prioritas Operator} (\textit{Operator Precedence}) dan \textbf{Asosiativitas}.

\subsection{Tabel Prioritas Operator}
Tabel berikut menunjukkan urutan operator dari prioritas tertinggi (dievaluasi duluan) hingga terendah.

\begin{table}[ht]
\centering
\footnotesize
\caption{Prioritas Operator (Disederhanakan)}
\label{tab:precedence}
\begin{tabular}{|c|p{3.25cm}|p{2.45cm}|p{2.2cm}|}
\hline
\textbf{Level} & \textbf{Operator} & \textbf{Deskripsi} & \textbf{Asos.} \\ \hline
1 & \code{()} \code{[]} \code{->} \code{.} & Function call, Array subscript, Member & Kiri ke Kanan \\ \hline
2 & \code{!} \code{\textasciitilde} \code{++} \code{--} \code{+} \code{-} \code{*} \code{\&} \code{sizeof} & Unary operators & \textbf{Kanan ke Kiri} \\ \hline
3 & \code{*} \code{/} \code{\%} & Multiplicative & Kiri ke Kanan \\ \hline
4 & \code{+} \code{-} & Additive & Kiri ke Kanan \\ \hline
5 & \code{<<} \code{>>} & Shift & Kiri ke Kanan \\ \hline
6 & \code{<} \code{<=} \code{>} \code{>=} & Relational & Kiri ke Kanan \\ \hline
7 & \code{==} \code{!=} & Equality & Kiri ke Kanan \\ \hline
8 & \code{\&} & Bitwise AND & Kiri ke Kanan \\ \hline
9 & \code{\^} & Bitwise XOR & Kiri ke Kanan \\ \hline
10 & \code{|} & Bitwise OR & Kiri ke Kanan \\ \hline
11 & \code{\&\&} & Logical AND & Kiri ke Kanan \\ \hline
12 & \code{||} & Logical OR & Kiri ke Kanan \\ \hline
13 & \code{? :} & Ternary & \textbf{Kanan ke Kiri} \\ \hline
14 & \code{=} \code{+=} \code{-=} dll & Assignment & \textbf{Kanan ke Kiri} \\ \hline
15 & \code{,} & Comma & Kiri ke Kanan \\ \hline
\end{tabular}
\end{table}

\subsection{Studi Kasus Evaluasi Ekspresi}

Mari kita bedah cara C mengevaluasi ekspresi berikut:
\begin{center}
\code{hasil = 10 + 5 * 2 > 15 \&\& 4 \% 2 == 0;}
\end{center}

\noindent\textbf{Langkah Evaluasi:}
\begin{enumerate}
    \item \textbf{Aritmatika Tertinggi (*) dan (\%):}
    \begin{itemize}
        \item \code{5 * 2} menjadi \code{10}.
        \item \code{4 \% 2} menjadi \code{0}.
        \item Ekspresi kini: \code{10 + 10 > 15 \&\& 0 == 0}
    \end{itemize}
    
    \item \textbf{Aritmatika (+) dan (-):}
    \begin{itemize}
        \item \code{10 + 10} menjadi \code{20}.
        \item Ekspresi kini: \code{20 > 15 \&\& 0 == 0}
    \end{itemize}
    
    \item \textbf{Relasional (>, <, >=, <=):}
    \begin{itemize}
        \item \code{20 > 15} bernilai True (\code{1}).
        \item Ekspresi kini: \code{1 \&\& 0 == 0}
    \end{itemize}
    
    \item \textbf{Equality (==, !=):}
    \begin{itemize}
        \item \code{0 == 0} bernilai True (\code{1}).
        \item Sisa ekspresi: \code{1 \&\& 1}
    \end{itemize}
    
    \item \textbf{Logika AND (\&\&):}
    \begin{itemize}
        \item \code{1 \&\& 1} bernilai True (\code{1}).
    \end{itemize}
    
    \item \textbf{Assignment (=):}
    \begin{itemize}
        \item Nilai \code{1} disimpan ke variabel \code{hasil}.
    \end{itemize}
\end{enumerate}

\begin{alertbox}{Kesalahan Umum: Chaining Relational}
Ekspresi matematika $5 < x < 10$ tidak bisa ditulis mentah-mentah di C sebagai \code{5 < x < 10}.
\begin{itemize}
    \item C akan mengevaluasi \code{(5 < x)} terlebih dahulu, menghasilkan 0 atau 1.
    \item Kemudian hasil tersebut dibandingkan dengan 10: \code{(0 atau 1) < 10}, yang mana \textbf{selalu True}.
    \item \textbf{Solusi:} Gunakan operator logika: \code{5 < x \&\& x < 10}.
\end{itemize}
\end{alertbox}
