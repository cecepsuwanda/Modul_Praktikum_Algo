\section{Operator Relasional dan Logika}

Operator relasional dan logika adalah fondasi dari pengambilan keputusan (percabangan) dan perulangan dalam pemrograman.

\subsection{Operator Relasional (Perbandingan)}
Operator ini membandingkan dua nilai dan menghasilkan nilai kebenaran. Di C (sebelum C99), nilai kebenaran direpresentasikan dengan integer: \textbf{0 untuk False} dan \textbf{1 (atau bukan nol) untuk True}.

\begin{table}[h]
\centering
\caption{Operator Relasional}
\label{tab:relasional}
\begin{tabular}{|c|l|l|}
\hline
\textbf{Operator} & \textbf{Deskripsi} & \textbf{Contoh (A=10, B=20)} \\ \hline
\code{==} & Sama dengan & \code{(A == B)} $\rightarrow$ False (0) \\ \hline
\code{!=} & Tidak sama dengan & \code{(A != B)} $\rightarrow$ True (1) \\ \hline
\code{>} & Lebih besar dari & \code{(A > B)} $\rightarrow$ False (0) \\ \hline
\code{<} & Lebih kecil dari & \code{(A < B)} $\rightarrow$ True (1) \\ \hline
\code{>=} & Lebih besar atau sama dengan & \code{(A >= B)} $\rightarrow$ False (0) \\ \hline
\code{<=} & Lebih kecil atau sama dengan & \code{(A <= B)} $\rightarrow$ True (1) \\ \hline
\end{tabular}
\end{table}

\begin{alertbox}{Peringatan}
Jangan tertukar antara operator penugasan \code{=} (assignment) dengan operator perbandingan \code{==} (equality). Menulis \code{if (x = 5)} adalah \textbf{valid} secara sintaks tapi seringkali berupa \textit{bug logic}, karena akan melakukan assignment nilai 5 ke x, yang nilainya (5) dianggap True.
\end{alertbox}

\subsubsection{Membandingkan Floating Point}
Hati-hati saat membandingkan tipe \code{float} atau \code{double} menggunakan \code{==} karena masalah presisi. Sebaiknya gunakan selisih absolut dengan nilai toleransi (epsilon).
\begin{lstlisting}[language=C]
// Kurang aman
if (f == 3.14) { ... } 

// Lebih aman
if (fabs(f - 3.14) < 0.00001) { ... }
\end{lstlisting}

\subsection{Operator Logika}
Digunakan untuk menggabungkan beberapa ekspresi relasional.

\begin{table}[ht]
\centering
\caption{Tabel Kebenaran Operator Logika}
\label{tab:logika}
\begin{tabular}{|c|c|c|c|c|}
\hline
\textbf{A} & \textbf{B} & \textbf{A \&\& B (AND)} & \textbf{A || B (OR)} & \textbf{!A (NOT)} \\ \hline
0 & 0 & 0 & 0 & 1 \\ \hline
0 & 1 & 0 & 1 & 1 \\ \hline
1 & 0 & 0 & 1 & 0 \\ \hline
1 & 1 & 1 & 1 & 0 \\ \hline
\end{tabular}
\end{table}

\subsubsection{Short-Circuit Evaluation}
C menggunakan evaluasi \textit{short-circuit} untuk efisiensi:
\begin{enumerate}
    \item \textbf{AND (\&\&)}: Jika operand kiri \textbf{False} (0), operand kanan \textbf{tidak dievaluasi} karena hasilnya pasti False.
    \item \textbf{OR (||)}: Jika operand kiri \textbf{True} (1), operand kanan \textbf{tidak dievaluasi} karena hasilnya pasti True.
\end{enumerate}

Contoh keamanan menggunakan short-circuit:
\begin{lstlisting}[language=C]
// Mencegah pembagian dengan nol
if (x != 0 && (100 / x) > 10) {
    // Jika x == 0, bagian (100 / x) tidak akan dieksekusi 
    // sehingga error division by zero terhindarkan.
}
\end{lstlisting}

Operator logika: \code{\&\&} (AND), \code{||} (OR), \code{!} (NOT). AND bernilai true hanya jika kedua operand true. OR bernilai true jika salah satu operand true. NOT membalik nilai kebenaran. Operator logika sering digunakan dalam kondisi percabangan dan perulangan. Contoh: \code{if (x > 0 \&\& x < 10)} memeriksa apakah x berada di antara 0 dan 10. Short-circuit evaluation: jika operand pertama menentukan hasil, operand kedua tidak dievaluasi.
