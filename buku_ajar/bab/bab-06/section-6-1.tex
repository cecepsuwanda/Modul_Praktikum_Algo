\section{Operator Aritmatika dan Increment/Decrement}

Operator adalah simbol khusus yang memberitahu kompilator untuk melakukan operasi matematika atau logika tertentu. Bahasa C kaya akan operator internal.

\subsection{Operator Aritmatika}
Operator aritmatika digunakan untuk melakukan operasi matematika dasar.

\begin{table}[ht]
\centering
\footnotesize
\caption{Operator Aritmatika dalam C}
\label{tab:aritmatika}
\begin{tabular}{|c|p{2.2cm}|p{2.2cm}|p{3.8cm}|}
\hline
\textbf{Operator} & \textbf{Nama} & \textbf{Contoh} & \textbf{Keterangan} \\ \hline
\code{+} & Penjumlahan & \code{a + b} & Menjumlahkan dua operand \\ \hline
\code{-} & Pengurangan & \code{a - b} & Mengurangkan operand kedua dari pertama \\ \hline
\code{*} & Perkalian & \code{a * b} & Mengalikan dua operand \\ \hline
\code{/} & Pembagian & \code{a / b} & Membagi operand pertama dengan kedua \\ \hline
\code{\%} & Modulus (Sisa Bagi) & \code{a \% b} & Sisa hasil bagi integer \\ \hline
\end{tabular}
\end{table}

\noindent\textbf{Catatan Penting:}
\begin{enumerate}
    \item \textbf{Pembagian Integer}: Jika kedua operand adalah integer, hasilnya adalah integer (dibulatkan ke bawah/truncate). Contoh: \code{7 / 2} menghasilkan \code{3}, bukan \code{3.5}. Agar mendapatkan hasil desimal, salah satu operand harus bertipe \textit{floating point} (misal: \code{7.0 / 2}).
    \item \textbf{Modulus}: Operator \code{\%} hanya bekerja pada tipe data integer. \code{7 \% 2} bernilai \code{1}. Pada C modern, tanda hasil modulus mengikuti operand pertama (misal \code{-7 \% 3} hasilnya \code{-1}).
\end{enumerate}

\subsection{Operator Increment dan Decrement}
C memiliki operator unik untuk menambah atau mengurangi nilai variabel sebesar 1, yaitu \code{++} dan \code{--}. Operator ini bisa diletakkan sebelum variabel (\textit{prefix}) atau sesudah variabel (\textit{postfix}).

\begin{itemize}
    \item \textbf{Prefix (\code{++a})}: Nilai variabel diubah \textit{terlebih dahulu}, lalu hasilnya digunakan dalam ekspresi.
    \item \textbf{Postfix (\code{a++})}: Nilai variabel \textit{saat ini} digunakan dulu dalam ekspresi, baru kemudian nilainya diubah.
\end{itemize}

\begin{lstlisting}[language=C, caption=Perbedaan Prefix dan Postfix]
#include <stdio.h>

int main() {
    int a = 5, b = 5;
    int hasil_a, hasil_b;

    // Prefix increment
    hasil_a = ++a; 
    // a bertambah jadi 6 dulu, lalu nilai 6 dimasukkan ke hasil_a
    printf("a: %d, hasil_a: %d\n", a, hasil_a); // Output: 6, 6

    // Postfix increment
    hasil_b = b++;
    // nilai b (5) dimasukkan ke hasil_b dulu, baru b bertambah jadi 6
    printf("b: %d, hasil_b: %d\n", b, hasil_b); // Output: 6, 5
    
    return 0;
}
\end{lstlisting}

\subsection{Operator Unary}
Operator unary hanya memerlukan satu operand. Selain increment/decrement, ada:
\begin{itemize}
    \item \code{+} (Unary Plus): Menandakan nilai positif (jarang ditulis eksplisit).
    \item \code{-} (Unary Minus): Mengnegasikan nilai (mengubah positif jadi negatif, dan sebaliknya).
\end{itemize}

Operator aritmatika C: \code{+}, \code{-}, \code{*}, \code{/}, \code{\%} (modulo). Prioritas mengikuti aturan matematika. Operator perbandingan: \code{==}, \code{!=}, \code{<}, \code{>}, \code{<=}, \code{>=}; menghasilkan nilai 1 (true) atau 0 (false). Operator penugasan: \code{=} serta bentuk majemuk \code{+=}, \code{-=}, \code{*=}, \code{/=}. Hati-hati dengan \code{=} vs \code{==}; kesalahan umum menggunakan \code{=} pada kondisi.
