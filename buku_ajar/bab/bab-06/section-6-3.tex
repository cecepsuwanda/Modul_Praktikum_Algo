\section{Operator Bitwise, Penugasan, dan Lainnya}

Selain operator matematika dan logika standar, C menyediakan operator yang bekerja pada tingkat bit dan memori, yang menjadi salah satu kekuatan utama bahasa C.

\subsection{Operator Bitwise}
Operator bitwise bekerja pada level representasi biner dari integer. Operator ini tidak dapat digunakan pada tipe \code{float}, \code{double}, atau \code{long double}.

\begin{table}[h]
\centering
\caption{Operator Bitwise}
\label{tab:bitwise}
\begin{tabular}{|c|l|l|}
\hline
\textbf{Op} & \textbf{Nama} & \textbf{Deskripsi} \\ \hline
\code{\&} & AND & Bit bernilai 1 jika kedua bit operand bernilai 1 \\ \hline
\code{|} & OR & Bit bernilai 1 jika salah satu bit operand bernilai 1 \\ \hline
\code{\^} & XOR (Exclusive OR) & Bit bernilai 1 jika bit operand berbeda nilainya \\ \hline
\code{\textasciitilde} & NOT (Complement) & Membalik nilai bit (0 jadi 1, 1 jadi 0) \\ \hline
\code{<<} & Left Shift & Menggeser bit ke kiri (mengalikan dengan $2^n$) \\ \hline
\code{>>} & Right Shift & Menggeser bit ke kanan (membagi dengan $2^n$) \\ \hline
\end{tabular}
\end{table}

\textbf{Contoh:} Misalkan A = 60 (\code{0011 1100}) dan B = 13 (\code{0000 1101}).
\begin{itemize}
    \item \code{A \& B} = 12 (\code{0000 1100})
    \item \code{A | B} = 61 (\code{0011 1101})
    \item \code{A \^ B} = 49 (\code{0011 0001})
    \item \code{A << 2} = 240 (\code{1111 0000})
\end{itemize}

\subsection{Operator Penugasan Majemuk (Compound Assignment)}
C memungkinkan kita menggabungkan operasi aritmatika/bitwise dengan penugasan untuk penulisan kode yang lebih ringkas.

\begin{table}[h]
\centering
\caption{Operator Penugasan Majemuk}
\label{tab:assignment}
\begin{tabular}{|l|l|}
\hline
\textbf{Penulisan Singkat} & \textbf{Ekivalen Dengan} \\ \hline
\code{a += b} & \code{a = a + b} \\ \hline
\code{a -= b} & \code{a = a - b} \\ \hline
\code{a *= b} & \code{a = a * b} \\ \hline
\code{a /= b} & \code{a = a / b} \\ \hline
\code{a \%= b} & \code{a = a \% b} \\ \hline
\code{a <<= b} & \code{a = a << b} \\ \hline
\end{tabular}
\end{table}

\subsection{Operator Lainnya}

\subsubsection{Operator Ternary (Conditional Operator)}
Ini adalah satu-satunya operator C yang mengambil tiga operand. Bentuknya:
\begin{center}
\code{kondisi ? ekspresi\_jika\_true : ekspresi\_jika\_false}
\end{center}

Contoh:
\begin{lstlisting}[language=C]
int a = 10, b = 20;
int max;

// Menggunakan if-else biasa
if (a > b) max = a;
else max = b;

// Menggunakan ternary (lebih ringkas)
max = (a > b) ? a : b;
\end{lstlisting}

\subsubsection{Operator \code{sizeof}}
Operator unary yang mengembalikan ukuran dalam \textit{bytes} dari tipe data atau variabel.
\begin{lstlisting}[language=C]
printf("Ukuran int: %lu bytes\n", sizeof(int)); // Biasanya 4
\end{lstlisting}

\subsubsection{Operator Koma (\code{,})}
Digunakan untuk memisahkan dua atau lebih ekspresi yang dievaluasi berurutan. Nilai ekspresi keseluruhan adalah nilai dari ekspresi paling kanan. Sering dijumpai dalam loop \code{for}.
\begin{lstlisting}[language=C]
int x, y;
y = (x = 5, x + 10); 
// x diisi 5, lalu hitung x+10 (15). y akan bernilai 15.
\end{lstlisting}
