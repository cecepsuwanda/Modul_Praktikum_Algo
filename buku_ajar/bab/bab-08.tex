\documentclass[../main.tex]{subfiles}
\ifSubfilesClassLoaded{\setcounter{chapter}{7}}{}
\begin{document}

\chapter{Struktur Percabangan (switch-case)}

\begin{subcpmk}
  \item Sub-CPMK 3.1: Mengimplementasikan struktur seleksi switch-case dalam C
\end{subcpmk}

\noindent\textbf{Materi Pokok:} Sintaks \code{switch-case}, peran \code{break} dan \code{default}; kapan memilih switch daripada if-else \cite{ref7,control_flow_cppref}.

\section{Sintaks switch-case}

Struktur \code{switch-case} digunakan untuk percabangan berdasarkan nilai ekspresi yang dibandingkan dengan konstanta. Sintaks: \code{switch (ekspresi) \{ case nilai1: ... break; case nilai2: ... break; default: ... \}}. Ekspresi harus bertipe integer atau char. Setiap \code{case} diikuti nilai konstanta dan titik dua. \code{break} mencegah fall-through ke case berikutnya. \code{default} menangani nilai yang tidak cocok dengan case mana pun.

\section{Perbandingan switch dan if-else}

Switch cocok ketika membandingkan satu variabel dengan banyak nilai konstanta diskret; kode lebih ringkas. If-else lebih fleksibel untuk kondisi kompleks, range nilai, atau ekspresi logika. Switch tidak bisa membandingkan string atau float secara langsung. Fall-through di switch (tanpa break) kadang disengaja untuk beberapa case yang menjalankan aksi sama. Pilih switch untuk menu pilihan angka/huruf; if-else untuk kondisi umum.


\begin{aktivitas}
  \item Buat program kalkulator dengan menu (1.Tambah 2.Kurang 3.Kali 4.Bagi) menggunakan switch.
  \item Buat program konversi angka ke nama hari (1=Senin, 2=Selasa, ...).
\end{aktivitas}

\begin{latihan}
  \item Kapan sebaiknya menggunakan switch daripada if-else?
  \item Apa fungsi break dalam switch? Apa yang terjadi jika dihilangkan?
  \item \textbf{Refleksi}: Kapan Anda memilih menu dengan switch-case dan kapan dengan if-else? Jelaskan pertimbangan Anda.
\end{latihan}

\begin{asesmen}
\textbf{Instrumen untuk Sub-CPMK 3.1}: Buat program C dengan menu (1--4) untuk operasi geometri: 1) luas persegi, 2) luas lingkaran, 3) luas segitiga, 4) keluar. Gunakan switch-case. Jelaskan mengapa switch tepat untuk kasus ini.
\end{asesmen}

\begin{checklist}
  \item Saya dapat menggunakan switch-case
  \item Saya mengetahui kapan menggunakan switch vs if-else
\end{checklist}

\begin{rangkuman}
Switch-case untuk percabangan berdasarkan nilai diskret. Gunakan break untuk mencegah fall-through; default untuk nilai tidak cocok.
\end{rangkuman}

\ifSubfilesClassLoaded{
  \renewcommand{\bibname}{Daftar Pustaka}
  \bibliographystyle{plain}
  \bibliography{../references}
}{}
\end{document}
