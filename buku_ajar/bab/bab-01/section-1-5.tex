\section{Peta Konsep Algoritma dan Pemrograman}

Mata kuliah Algoritma dan Pemrograman mencakup 17 bab yang saling terkait secara berurutan. Tabel berikut memetakan bab buku ajar ke 16 pertemuan dalam RPS:

\begin{table}[htbp]
\centering
\small
\begin{tabular}{|c|>{\raggedright\arraybackslash}p{6.5cm}|>{\raggedright\arraybackslash}p{3.5cm}|}
\hline
\textbf{Pert.} & \textbf{Materi Silabus} & \textbf{Bab Buku Ajar} \\
\hline
1 & Pengenalan algoritma, flowchart, pseudocode & Bab II, III \\
\hline
2 & Variabel, tipe data, I/O, operator & Bab V, VI \\
\hline
3 & Percabangan if-else & Bab VII \\
\hline
4 & Percabangan switch-case & Bab VIII \\
\hline
5 & Perulangan for, while, do-while & Bab IX \\
\hline
6 & Perulangan bersarang & Bab X \\
\hline
7 & Praktikum 1 (dasar pemrograman) & Aktivitas Bab III--VII \\
\hline
8 & Fungsi dan prosedur & Bab XI \\
\hline
9 & UTS & -- \\
\hline
10 & Array satu dimensi & Bab XII \\
\hline
11 & Array dua dimensi, matriks & Bab XIII \\
\hline
12 & String dan manipulasi teks & Bab XIV \\
\hline
13 & Struct/record, pengurutan dan pencarian & Bab XV, XVI \\
\hline
14 & Algoritma pengurutan dan pencarian lanjutan & Bab XVI \\
\hline
15 & Integrasi, proyek, review materi & Bab XVII \\
\hline
16 & UAS & Bab XVII \\
\hline
\end{tabular}
\caption{Pemetaan Bab Buku Ajar ke 16 Pertemuan RPS}
\end{table}

Bab II menyajikan landasan teori dan konsep dasar algoritma serta pengantar bahasa C. Bab III dan IV membahas pengenalan algoritma, flowchart, dan pseudocode sebagai fondasi merancang solusi. Bab V dan VI membahas variabel, tipe data, I/O, serta operator dalam bahasa C.

Bab VII dan VIII membahas struktur percabangan (if-else dan switch-case) untuk pengambilan keputusan dalam program. Bab IX dan X membahas struktur perulangan (for, while, do-while) serta perulangan bersarang. Bab XI membahas fungsi dan prosedur untuk modularisasi program. Bab XII, XIII, dan XIV membahas array satu dimensi, array dua dimensi, serta string dalam C.

Bab XV membahas struct dan tipe data bentukan untuk merepresentasikan data kompleks. Bab XVI membahas algoritma pengurutan dan pencarian sebagai penerapan konsep yang telah dipelajari. Bab XVII berisi evaluasi, refleksi, dan integrasi kompetensi untuk mengukur pencapaian pembelajaran secara menyeluruh. Alur pembelajaran dirancang secara progresif dari konsep dasar hingga penerapan algoritma pada masalah nyata.
