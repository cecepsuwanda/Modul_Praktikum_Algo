\section{Konteks Kurikulum OBE}

\subsection{Apa itu Outcome-Based Education?}

\textbf{Outcome-Based Education (OBE)} adalah pendekatan pembelajaran yang berfokus pada pencapaian hasil (outcomes) yang terukur \cite{obe_wikipedia}. Dalam OBE, proses pembelajaran dirancang secara sistematis untuk memastikan mahasiswa mencapai kompetensi yang telah ditetapkan. Prinsip utama OBE meliputi clarity of focus, designing down dari outcomes yang diinginkan, high expectations, dan expanded opportunity bagi semua mahasiswa.

\subsection{Implementasi OBE dalam Buku Ini}

Buku ini mengimplementasikan OBE melalui Sub-CPMK eksplisit di setiap bab, materi yang disusun dari dasar ke lanjut secara sistematis, serta aktivitas beragam berupa latihan, studi kasus, dan praktikum. Asesmen terukur dengan rubrik penilaian yang jelas disertai checklist untuk self-assessment memungkinkan mahasiswa memantau pencapaian kompetensi secara berkelanjutan. Implementasi tersebut dirancang agar pembelajaran di tingkat mikro (per bab) mendukung pencapaian kompetensi di tingkat makro (lulusan program studi).
