\section{Tujuan Buku Ajar}

Buku ajar ini dirancang sebagai panduan komprehensif untuk menguasai Algoritma dan Pemrograman menggunakan bahasa C. Fokus utama buku ini adalah pada pemahaman konsep algoritma, kemampuan merancang solusi, dan implementasi program komputer secara prosedural. Tujuan spesifik buku ini adalah memberikan landasan yang kokoh bagi mahasiswa Teknik Informatika sebelum mempelajari mata kuliah pemrograman lanjutan.

Setelah mempelajari buku ini secara menyeluruh, mahasiswa diharapkan mampu memahami dan menjelaskan konsep algoritma, flowchart, serta pseudocode beserta karakteristiknya. Mahasiswa juga diharapkan mampu merancang algoritma untuk menyelesaikan masalah sederhana menggunakan flowchart dan pseudocode. Selain itu, mahasiswa harus mampu mengimplementasikan algoritma dalam bahasa C meliputi variabel, tipe data, I/O, operator, percabangan, dan perulangan.

Mahasiswa diharapkan mampu menggunakan fungsi, prosedur, array, dan struktur data dasar dalam bahasa C. Terakhir, mahasiswa harus mampu menerapkan algoritma pengurutan dan pencarian sederhana serta mengimplementasikannya dalam program. Semua tujuan tersebut dirancang untuk memfasilitasi pencapaian CPL dan CPMK yang telah ditetapkan dalam RPS mata kuliah Algoritma dan Pemrograman.
