\section[Keterkaitan Buku Ajar dengan RPS Berbasis OBE]{Keterkaitan Buku Ajar dengan RPS\\ Berbasis OBE}

Buku ajar ini dirancang selaras dengan Rencana Pembelajaran Semester (RPS) mata kuliah Algoritma dan Pemrograman yang berbasis OBE. Keterkaitan ini diwujudkan melalui pemetaan eksplisit setiap bab ke Sub-CPMK yang berkontribusi pada pencapaian CPMK dan CPL program studi. Struktur ini memastikan bahwa materi pembelajaran fokus pada pencapaian kompetensi terukur sesuai dengan matriks 16 pertemuan dalam silabus.

\subsection{Alignment dengan CPL dan CPMK}

Setiap bab dalam buku ini dipetakan secara eksplisit ke Sub-CPMK yang berkontribusi pada pencapaian CPMK dan CPL. Materi pembelajaran dirancang untuk mendukung pengembangan keterampilan merancang algoritma dan mengimplementasikan program dalam bahasa C. Asesmen dalam setiap bab mengukur pencapaian kompetensi secara objektif sesuai indikator yang telah ditetapkan.

\subsection{Integrasi Metode Pembelajaran}

Buku ini mengintegrasikan berbagai metode pembelajaran yang tercantum dalam RPS, antara lain ceramah interaktif, Problem-Based Learning (PBL), praktikum terbimbing, serta diskusi dan studi kasus. Aktivitas pembelajaran mendorong mahasiswa untuk mempraktikkan coding di laboratorium dan menyelesaikan tugas individu maupun proyek kecil. Komponen asesmen sejalan dengan sistem penilaian RPS meliputi tugas individu, kuis, praktikum, UTS, tugas proyek, dan UAS.
