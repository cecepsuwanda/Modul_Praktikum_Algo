\section{Petunjuk Penggunaan Buku Ajar}

\subsection{Untuk Mahasiswa}

Sebelum perkuliahan, bacalah Sub-CPMK di awal bab untuk memahami target pembelajaran yang harus dicapai. Pelajari materi pokok dengan seksama dan jalankan semua contoh kode C yang diberikan menggunakan compiler seperti GCC atau MinGW. Catat pertanyaan atau konsep yang belum dipahami untuk didiskusikan di kelas.

Selama perkuliahan, diskusikan konsep yang sulit dengan dosen dan teman secara aktif. Kerjakan aktivitas pembelajaran dan praktikum dengan sungguh-sungguh di laboratorium. Manfaatkan kesempatan untuk bertanya dan berpartisipasi dalam diskusi kelompok. Setelah perkuliahan, kerjakan latihan dan refleksi, lakukan asesmen mandiri, serta centang checklist kompetensi yang telah dikuasai. Kerjakan proyek mini untuk memperdalam pemahaman dan mengintegrasikan konsep yang dipelajari.

\subsection{Untuk Dosen}

Buku ini dapat digunakan sebagai bahan ajar utama untuk perkuliahan Algoritma dan Pemrograman semester 1. Buku ini menyediakan sumber latihan dan tugas yang selaras dengan CPMK serta panduan untuk merancang aktivitas pembelajaran. Dosen dapat memanfaatkan rubrik asesmen dalam buku untuk mengukur pencapaian CPMK mahasiswa secara konsisten.
