\section{Definisi dan Fungsi Flowchart}

Flowchart adalah representasi visual dari algoritma menggunakan simbol-simbol standar yang dihubungkan dengan garis alur \cite{flowchart_wikipedia,flowchart_lucidchart}. Kata "flowchart" berasal dari "flow" (alur) dan "chart" (diagram), yang secara harfiah berarti diagram alur. Flowchart pertama kali diperkenalkan oleh Frank Gilbreth pada tahun 1921 untuk memvisualisasikan proses industri dan kemudian diadopsi dalam pemrograman komputer pada tahun 1960-an.

\subsection{Fungsi Utama Flowchart}

Flowchart memiliki beberapa fungsi penting dalam pengembangan perangkat lunak:

\begin{enumerate}
  \item \textbf{Visualisasi Logika:} Mengubah algoritma abstrak menjadi diagram visual yang mudah dipahami
  \item \textbf{Dokumentasi:} Membuat dokumentasi yang jelas tentang cara kerja program
  \item \textbf{Komunikasi:} Memfasilitasi komunikasi antar tim programmer dan stakeholder
  \item \textbf{Debugging:} Membantu mengidentifikasi kesalahan logika sebelum implementasi
  \item \textbf{Analisis:} Memungkinkan analisis efisiensi dan kompleksitas algoritma
\end{enumerate}

\subsection{Manfaat Flowchart dalam Pembelajaran}

Untuk mahasiswa pemula, flowchart memberikan beberapa keuntungan:

\begin{itemize}
  \item \textbf{Pemahaman Konseptual:} Membantu memahami alur logika tanpa terbebani sintaks bahasa pemrograman
  \item \textbf{Problem Solving:} Melatih berpikir sistematis dalam memecahkan masalah
  \item \textbf{Struktur Berpikir:} Mengembangkan kemampuan berpikir terstruktur dan logis
  \item \textbf{Transisi ke Kode:} Menjadi jembatan antara konsep algoritma dan implementasi kode
\end{itemize}

\subsection{Standar Flowchart}

Flowchart mengikuti standar internasional yang ditetapkan oleh American National Standards Institute (ANSI) pada tahun 1970-an. Standar ini memastikan konsistensi dan universalitas dalam penggunaan simbol flowchart di seluruh dunia. Setiap simbol memiliki makna spesifik yang telah disepakati secara internasional, sehingga flowchart yang dibuat di Indonesia dapat dipahami oleh programmer di Amerika atau Eropa.

Penggunaan flowchart memudahkan komunikasi antar programmer dan pemangku kepentingan dalam memahami logika solusi sebelum implementasi teknis dilakukan.
