\section{Contoh Flowchart: Konversi Suhu}

Berikut contoh flowchart lengkap untuk mengonversi suhu dari Celsius ke Fahrenheit dengan implementasi dalam bahasa C.

\subsection{Algoritma Konversi Suhu}

\textbf{Algoritma:}
\begin{enumerate}
  \item Mulai
  \item Input suhu dalam Celsius
  \item Hitung Fahrenheit = Celsius $\times$ 9/5 + 32
  \item Tampilkan hasil Fahrenheit
  \item Selesai
\end{enumerate}

\subsection{Flowchart Lengkap}

\begin{center}
\begin{tikzpicture}[node distance=0.8cm]
  \node[flowstart] (start) {Mulai};
  \node[flowio, below=of start] (input) {Input Celsius};
  \node[flowprocess, below=of input] (calc) {F = C $\times$ 9/5 + 32};
  \node[flowio, below=of calc] (output) {Tampilkan F};
  \node[flowstart, below=of output] (end) {Selesai};
  \draw[arrow] (start) -- (input);
  \draw[arrow] (input) -- (calc);
  \draw[arrow] (calc) -- (output);
  \draw[arrow] (output) -- (end);
\end{tikzpicture}
\end{center}

\subsection{Implementasi dalam Bahasa C}

\begin{lstlisting}[language=C, caption={Program Konversi Suhu Celsius ke Fahrenheit}]
#include <stdio.h>

int main() {
    float celsius, fahrenheit;
    
    // Input suhu Celsius
    printf("Masukkan suhu dalam Celsius: ");
    scanf("%f", &celsius);
    
    // Konversi ke Fahrenheit
    fahrenheit = celsius * 9.0 / 5.0 + 32;
    
    // Tampilkan hasil
    printf("%.2f Celsius = %.2f Fahrenheit\n", celsius, fahrenheit);
    
    return 0;
}
\end{lstlisting}

\subsection{Analisis Karakteristik Algoritma}

\begin{itemize}
  \item \textbf{Input:} Satu nilai suhu dalam Celsius
  \item \textbf{Output:} Satu nilai suhu dalam Fahrenheit
  \item \textbf{Definiteness:} Langkah-langkah jelas (input, hitung, output)
  \item \textbf{Finiteness:} Algoritma berakhir setelah 4 langkah
  \item \textbf{Effectiveness:} Operasi aritmatika dasar dapat dilakukan
  \item \textbf{Determinism:} Input 25°C selalu menghasilkan 77°F
\end{itemize}

\subsection{Contoh Output Program}

\begin{verbatim}
Masukkan suhu dalam Celsius: 25
25.00 Celsius = 77.00 Fahrenheit

Masukkan suhu dalam Celsius: 0
0.00 Celsius = 32.00 Fahrenheit

Masukkan suhu dalam Celsius: 100
100.00 Celsius = 212.00 Fahrenheit
\end{verbatim}

Contoh ini menunjukkan bagaimana flowchart dapat membantu merancang algoritma sederhana sebelum implementasi dalam bahasa pemrograman.
