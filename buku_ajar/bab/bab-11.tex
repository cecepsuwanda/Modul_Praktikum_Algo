\documentclass[../main.tex]{subfiles}
\ifSubfilesClassLoaded{\setcounter{chapter}{10}}{}
\begin{document}

\chapter{Fungsi dan Prosedur}

\begin{subcpmk}
  \item Sub-CPMK 4.1: Membuat fungsi dengan parameter dan nilai return dalam C
\end{subcpmk}

\noindent\textbf{Materi Pokok:} Definisi dan pemanggilan fungsi; parameter (pass by value) dan return value; scope variabel (lokal vs global) \cite{ref7,ref2}.

\section{Definisi Fungsi}

Fungsi adalah blok kode yang dapat dipanggil berulang kali dengan nama. Fungsi mendukung modularisasi: memecah program menjadi bagian-bagian kecil yang dapat dipelihara. Sintaks definisi: \code{tipe\_return nama(parameter) \{ ... return nilai; \}}. Fungsi dengan tipe \code{void} tidak mengembalikan nilai. Parameter adalah input fungsi; bisa nol atau lebih, dipisah koma. Return mengembalikan nilai dan mengakhiri eksekusi fungsi.

\section{Parameter dan Return Value}

Parameter formal dideklarasikan dalam definisi fungsi; parameter aktual (argumen) diberikan saat pemanggilan. Pass by value: nilai argument disalin ke parameter; perubahan di dalam fungsi tidak mempengaruhi variabel asal. Untuk mengembalikan lebih dari satu nilai atau mengubah variabel asal, gunakan pointer (materi lanjut). Return value harus bertipe sesuai deklarasi fungsi. Contoh: \code{int tambah(int a, int b) \{ return a + b; \}}.

\section{Scope Variabel}

Scope variabel menentukan di mana variabel dapat diakses. Variabel lokal dideklarasikan di dalam fungsi; hanya bisa diakses dalam fungsi itu. Variabel global dideklarasikan di luar semua fungsi; dapat diakses di mana saja. Hindari penggunaan global berlebihan; preferensi variabel lokal untuk modularitas. Lifetime: variabel lokal ada selama eksekusi fungsi; variabel global ada selama program berjalan. Variabel dengan nama sama: lokal menutupi (shadow) global dalam scope-nya.


\begin{aktivitas}
  \item Buat fungsi untuk menghitung faktorial, lalu panggil dari main.
  \item Buat fungsi untuk mengecek bilangan prima.
\end{aktivitas}

\begin{latihan}
  \item Jelaskan perbedaan parameter dan return value!
  \item Apa perbedaan variabel lokal dan global?
  \item \textbf{Refleksi}: Kapan Anda memecah program menjadi beberapa fungsi? Berikan contoh satu program yang menurut Anda lebih baik jika di-modularisasi.
\end{latihan}

\begin{asesmen}
\textbf{Instrumen untuk Sub-CPMK 4.1}: Buat program C dengan fungsi \code{int max3(int a, int b, int c)} yang mengembalikan nilai terbesar dari tiga bilangan. Panggil dari \code{main} dengan input dari user. Tambahkan fungsi \code{void cetakBilangan(int n)} yang menampilkan 1 sampai n.
\end{asesmen}

\begin{checklist}
  \item Saya dapat mendefinisikan dan memanggil fungsi
  \item Saya memahami parameter dan return value
  \item Saya memahami scope variabel
\end{checklist}

\begin{rangkuman}
Fungsi modularisasi program. Parameter sebagai input; return sebagai output. Variabel lokal vs global.
\end{rangkuman}

\ifSubfilesClassLoaded{
  \renewcommand{\bibname}{Daftar Pustaka}
  \bibliographystyle{plain}
  \bibliography{../references}
}{}
\end{document}
