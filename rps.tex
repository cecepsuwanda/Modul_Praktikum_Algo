\documentclass[12pt,a4paper]{book}
\usepackage[utf8]{inputenc}
\usepackage[T1]{fontenc}
\usepackage{lmodern}
\usepackage{geometry}
\usepackage{hyperref}
\usepackage{listings}
\usepackage{color}

\geometry{
  left=3cm,
  right=3cm,
  top=3cm,
  bottom=3cm
}

\definecolor{codegray}{rgb}{0.5,0.5,0.5}
\definecolor{codepurple}{rgb}{0.58,0,0.82}
\definecolor{backcolour}{rgb}{0.95,0.95,0.92}

\lstdefinestyle{mystyle}{
    backgroundcolor=\color{backcolour},
    commentstyle=\color{codegray},
    keywordstyle=\color{blue},
    numberstyle=\tiny\color{codegray},
    stringstyle=\color{codepurple},
    basicstyle=\ttfamily\small,
    breaklines=true,
    frame=single,
    numbers=left,
    numbersep=5pt
}

\lstset{style=mystyle}

\begin{document}

\frontmatter
\title{Buku Praktikum: Pascal, C \& C++}
\author{Dosen Informatika}
\date{\today}
\maketitle
\tableofcontents

\mainmatter

\chapter{Pengantar \& Persiapan Lingkungan}
\section{Sejarah dan Posisi Pascal, C, C++}
\section{Menyiapkan Lingkungan Pengembangan}
\subsection{Compiler / IDE untuk Pascal}
\subsection{Compiler / IDE untuk C / C++}
\section{Hello, World! — Program Pertama}

\chapter{T tipe Data, Variabel, Input / Output}
\section{Tipe Data Dasar}
\section{Deklarasi \& Inisialisasi Variabel}
\section{Input / Output Dasar}
\section{Konversi Tipe dan Operasi Terkait}

\chapter{Operator \& Ekspresi}
\section{Operator Aritmetika, Relasional, Logika}
\section{Operator Bitwise}
\section{Precedence dan Asosiasi}
\section{Operator Overloading (Pendahuluan)}

\chapter{Struktur Kontrol Kondisional}
\section{Pernyataan \texttt{if} / \texttt{else}}
\section{Pernyataan \texttt{case} / \texttt{switch}}

\chapter{Struktur Kontrol Perulangan & Lainnya}
\section{Perulangan: \texttt{for}, \texttt{while}, \texttt{repeat} / \texttt{do–while}}
\section{Penggunaan \texttt{break}, \texttt{continue}}
\section{Operator Ternary / Conditional (C / C++)}

\chapter{Array \& Multidimensi}
\section{Array Satu Dimensi}
\section{Array Multidimensi}
\section{Operasi Dasar (iterasi, traversal)}
\section{Alokasi Dinamis pada Array (pointer ke array, jika relevan)}

\chapter{String / Karakter \& Operasi String}
\section{String sebagai Array Karakter}
\section{Operasi Dasar String (concat, substring, length)}
\section{Pengolahan Karakter, escape sequences}
\section{Fungsi / prosedur string library (jika tersedia)}

\chapter{Struktur Data Dasar \& Pointer Dasar}
\section{Record / \texttt{struct}, Enumerasi, Set}
\section{Pointer & Referensi Dasar}
\section{Pointer ke Record / Struct}
\section{Akses anggota via pointer / referensi}

\chapter{Fungsi \& Prosedur / Metode}
\section{Deklarasi \& Definisi Fungsi / Prosedur}
\section{Parameter — by value / by reference}
\section{Fungsi Rekursif}
\section{Metode \& Fungsi dalam C++}

\chapter{Alokasi Dinamis \& Pointer Lanjutan}
\section{Alokasi Memori Dinamis}
\section{Pointer \& Dereferensi}
\section{Pointer ke Array / Pointer ke Pointer}
\section{Pointer ke Fungsi / Smart Pointers}

\chapter{File \& Operasi I/O Lanjutan}
\section{File Teks \& Biner}
\section{Mode File, Seek, ftell}
\section{Penanganan Kesalahan File}

\chapter{Modul / Unit / Header \& Manajemen Proyek}
\section{Unit dan “uses” (Pascal)}
\section{Header / Implementasi (C / C++)}
\section{Include Guards / \texttt{\#pragma once}}
\section{Organisasi Proyek Modular}

\backmatter
\chapter*{Referensi}
% Daftar sumber: GitHub repo, buku, artikel

\end{document}
